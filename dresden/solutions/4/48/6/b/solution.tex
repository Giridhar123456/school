 Let the equation of plane is 
 \begin{align}
 ax +by +cz +d = 0 \label{eq:solutions/4/48/6/b/eq2.1}
 \end{align}
 
% we can express \eqref{eq:solutions/4/48/6/b/eq2.1} as ,
% \begin{align}
%\myvec{a & b & c}\vec{x} = -d
% \end{align}
Direction ratio of the line \eqref{eq:solutions/4/48/6/b/eq1.1} is given as 
\begin{align}
\vec{D} = \myvec{4\\ 6 \\ -12}
\end{align}
 Now let consider 
\begin{align}
\vec{A} = \myvec{-2 & -3 & 4} 
\vec{A}\vec{D} + d = 0 \\
\implies d = 37
\end{align}
Since plane is passing through the point A (-2, -3, 4 ) and perpendicular to the line 
\eqref{eq:solutions/4/48/6/b/eq1.1}.
Hence equation of the plane is 
\begin{align}
2x + 3y - 6z + 37 = 0 \\
\implies 2x + 3y - 6z = -37 \label{eq:solutions/4/48/6/b/2.7}
\end{align} 
equation \eqref{eq:solutions/4/48/6/b/2.7} can written as :
\begin{align}
\myvec{2 & 3 & -6}\vec{x} = -37\\
\end{align}
For foot perpendicular we need to find the distance between the plane and point P $\left( 0, 3, -2\right)$.\\ 
First we find orthogonal vectors $\vec{m_1}$ and $\vec{m_2}$ to the given normal vector $\vec{n}$. Let, $\vec{m}$ = $\myvec{a\\b\\c}$, then
\begin{align}
\vec{m^T}\vec{n} &= 0\\
\implies\myvec{a&b&c}\myvec{2\\3\\-6} &= 0\\
\implies 2a+3b-6c &= 0\\
\end{align}
{Putting a=1 and b=0 we get,}
\begin{align}
\vec{m_1} &= \myvec{1\\0\\\frac{1}{3} }\\
\end{align}
{Putting a=0 and b=1 we get,}
\begin{align}
\vec{m_2} &= \myvec{0\\1\\ \frac{1}{2}}
\end{align}
Now we solve the equation,
\begin{align}
\vec{M}\vec{x} &= \vec{b}\label{eq:solutions/4/48/6/b/eq1}\\
\end{align}
{Putting values in \eqref{eq:solutions/4/48/6/b/eq1},}
\begin{align}
\myvec{1&0\\0&1\\ \frac{1}{3}& \frac{1}{2}}\vec{x} &= \myvec{0\\3\\-2}\label{eq:solutions/4/48/6/b/eq2}
\end{align}
Now, to solve \eqref{eq:solutions/4/48/6/b/eq2}, we perform Singular Value Decomposition on $\vec{M}$ as follows,
\begin{align}
\vec{M}=\vec{U}\vec{S}\vec{V}^T\label{eq:solutions/4/48/6/b/eqSVD}
\end{align}
Where the columns of $\vec{V}$ are the eigen vectors of $\vec{M}^T\vec{M}$ ,the columns of $\vec{U}$ are the eigen vectors of $\vec{M}\vec{M}^T$ and $\vec{S}$ is diagonal matrix of singular value of eigenvalues of $\vec{M}^T\vec{M}$.
\begin{align}
\vec{M}^T\vec{M}=\myvec{\frac{10}{9}& \frac{1}{6}\\ \frac{1}{6}&\frac{5}{4}}\label{eq:solutions/4/48/6/b/eqMTM}\\
\vec{M}\vec{M}^T=\myvec{1&0& \frac{1}{3}\\0&1& \frac{1}{2}\\ \frac{1}{3}& \frac{1}{2}& \frac{13}{36}}
\end{align}
From \eqref{eq:solutions/4/48/6/b/eq1} putting \eqref{eq:solutions/4/48/6/b/eqSVD} we get,
\begin{align}
\vec{U}\vec{S}\vec{V}^T\vec{x} & = \vec{b}\\
\implies\vec{x} &= \vec{V}\vec{S_+}\vec{U^T}\vec{b}\label{eq:solutions/4/48/6/b/eqX}
\end{align}
Where $\vec{S_+}$ is Moore-Penrose Pseudo-Inverse of $\vec{S}$.Now, calculating eigen value of $\vec{M}\vec{M}^T$,


\begin{align}
\mydet{\vec{M}\vec{M}^T - \lambda\vec{I}} &= 0\\
\implies\myvec{1-\lambda&0& \frac{1}{3} \\0&1-\lambda& \frac{1}{2}\\   \frac{1}{3}& \frac{1}{2}&\frac{13}{36}-\lambda} &=0\\
\implies\lambda (\lambda - 1)(\lambda - \frac{49}{36}) &=0
\end{align}


Hence eigen values of $\vec{M}\vec{M}^T$ are,
\begin{align}
\lambda_1 &= \frac{49}{36}\\
\lambda_2 &= 1\\
\lambda_3 &= 0\\
\end{align}
Hence the eigen vectors of $\vec{M}\vec{M}^T$ are,
\begin{align}
\vec{u_1}=\myvec{\frac{12}{13}\\\frac{18}{13}\\1}\quad
\vec{u_2}=\myvec{\frac{-3}{2}\\1\\0}\quad
\vec{u_3}=\myvec{\frac{-1}{3}\\\frac{-1}{2}\\1}
\end{align}
{Normalizing the eigen vectors we get,}
\begin{align}
\vec{u_1}=\myvec{\frac{12}{7\sqrt{13}}\\\frac{18}{7\sqrt{13}}\\\frac{\sqrt{13}}{7}}\quad
\vec{u_2}=\myvec{\frac{-3}{\sqrt{13}}\\\frac{2}{\sqrt{13}}\\0}\quad
\vec{u_3}=\myvec{\frac{-2}{7}\\\frac{-3}{7}\\\frac{6}{7}}
\end{align}
Hence we obtain $\vec{U}$ of \eqref{eq:solutions/4/48/6/b/eqSVD} as follows,
\begin{align}
\vec{U}=\myvec{\frac{12}{7\sqrt{13}}&\frac{-3}{\sqrt{13}}&\frac{-2}{7}\\\frac{18}{7\sqrt{13}}&\frac{2}{\sqrt{13}}&\frac{-3}{7}\\\frac{\sqrt{13}}{7}&0&\frac{6}{7}}\label{eq:solutions/4/48/6/b/eqU}
\end{align}
After computing the singular values from eigen values $\lambda_1, \lambda_2, \lambda_3$ we get $\vec{S}$ of \eqref{eq:solutions/4/48/6/b/eqSVD} as follows,
\begin{align}
\vec{S}=\myvec{\frac{7}{6}&0\\0&1\\0&0}\label{eq:solutions/4/48/6/b/eqS}
\end{align}
Now, calculating eigen value of $\vec{M}^T\vec{M}$,
\begin{align}
\mydet{\vec{M}^T\vec{M} - \lambda\vec{I}} &= 0\\
\implies\mydet{\frac{5}{4}-\lambda&\frac{1}{6}\\\frac{1}{6}&\frac{10}{9}-\lambda} &=0\\
\implies\lambda^2-\frac{85}{36}\lambda+\frac{49}{36} &=0
\end{align}
Hence eigen values of $\vec{M}^T\vec{M}$ are,
\begin{align}
\lambda_1 = \frac{49}{36}\quad
\lambda_2 = 1
\end{align}
Hence the eigen vectors of $\vec{M}^T\vec{M}$ are,
\begin{align}
\vec{v}_1=\myvec{\frac{2}{3}\\1} \quad
\vec{v}_2=\myvec{\frac{-3}{2}\\1}
\end{align}
Normalizing the eigen vectors,
\begin{align}
\vec{v}_1=\myvec{\frac{2}{\sqrt{13}}\\\frac{3}{\sqrt{13}}} \quad
\vec{v}_2=\myvec{\frac{-3}{\sqrt{13}}\\\frac{2}{\sqrt{13}}}
\end{align}
Hence we obtain $\vec{V}$ of \eqref{eq:solutions/4/48/6/b/eqSVD} as follows,
\begin{align}
\vec{V}=\myvec{\frac{2}{\sqrt{13}}&\frac{-3}{\sqrt{13}}\\\frac{3}{\sqrt{13}}&\frac{2}{\sqrt{13}}}
\end{align}
Finally from \eqref{eq:solutions/4/48/6/b/eqSVD} we get the Singualr Value Decomposition of $\vec{M}$ as follows,
\begin{align}
\vec{M} = \myvec{\frac{12}{7\sqrt{13}}&\frac{-3}{\sqrt{13}}&\frac{-2}{7}\\\frac{18}{7\sqrt{13}}&\frac{2}{\sqrt{13}}&\frac{-3}{7}\\\frac{\sqrt{13}}{7}&0&\frac{6}{7}}\myvec{\frac{7}{6}&0\\0&1\\0&0}\myvec{\frac{2}{\sqrt{13}}&\frac{-3}{\sqrt{13}}\\\frac{3}{\sqrt{13}}&\frac{2}{\sqrt{13}}}^T
\end{align}
Now, Moore-Penrose Pseudo inverse of $\vec{S}$ is given by,
\begin{align}
\vec{S_+} =\myvec{\frac{6}{7}&0&0\\0&1&0}
\end{align}
From \eqref{eq:solutions/4/48/6/b/eqX} we get,
\begin{align}
\vec{U}^T\vec{b}&=\myvec{\frac{4}{\sqrt{13}}\\\frac{6}{\sqrt{13}}\\-3}\\
\vec{S_+}\vec{U}^T\vec{b}&=\myvec{\frac{24}{7\sqrt{13}}\\\frac{6}{\sqrt{13}}}\\
\vec{x} &= \vec{V}\vec{S_+}\vec{U}^T\vec{b} &= \myvec{\frac{-6}{7}\\\frac{12}{7}}\label{eq:solutions/4/48/6/b/eqXSol1}
\end{align}
Verifying the solution of \eqref{eq:solutions/4/48/6/b/eqXSol1} using,
\begin{align}
\vec{M}^T\vec{M}\vec{x} = \vec{M}^T\vec{b}\label{eq:solutions/4/48/6/b/eqVerify}
\end{align}
Evaluating the R.H.S in \eqref{eq:solutions/4/48/6/b/eqVerify} we get,
\begin{align}
\vec{M}^T\vec{M}\vec{x} &= \myvec{\frac{-2}{3}\\ 2}\\
\implies\myvec{\frac{10}{9}&\frac{1}{6}\\\frac{1}{6}&\frac{5}{4}}\vec{x} &= \myvec{\frac{-2}{3}\\2}\label{eq:solutions/4/48/6/b/eqMateq}
\end{align}
Solving the augmented matrix of \eqref{eq:solutions/4/48/6/b/eqMateq} we get,
\begin{align}
\myvec{\frac{10}{9}&\frac{1}{6}& \frac{-2}{3}\\\frac{1}{6}&\frac{5}{4}& 2} &\xleftrightarrow{R_1=\frac{9R_1}{10}}\myvec{1&\frac{3}{20}&\frac{-3}{5}\\\frac{1}{6}&\frac{5}{4}&2}\\
&\xleftrightarrow{R_2=R_2-\frac{R_1}{6}}\myvec{1&\frac{3}{20}&\frac{-3}{5}\\0&\frac{49}{40}&\frac{21}{10}}\\
&\xleftrightarrow{R_2=\frac{40}{49}R_2}\myvec{1&\frac{3}{20}&\frac{-3}{5}\\0&1&\frac{12}{7}}\\
&\xleftrightarrow{R_1=R_1-\frac{3R_2}{20}}\myvec{1&0&\frac{-6}{7}\\0&1&\frac{12}{7}}
\end{align}
Hence, Solution of \eqref{eq:solutions/4/48/6/b/eqVerify} is given by,
\begin{align}
\vec{x}=\myvec{\frac{-6}{7}\\\frac{12}{7}}\label{eq:solutions/4/48/6/b/eqX2}
\end{align}
Comparing results of $\vec{x}$ from \eqref{eq:solutions/4/48/6/b/eqXSol1} and \eqref{eq:solutions/4/48/6/b/eqX2} we conclude that the solution is verified.
