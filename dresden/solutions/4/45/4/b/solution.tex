Equation of plane can be expressed as 
\begin{align}\label{eq:solutions/4/45/4/b/eq1}
	\vec{n}^T\vec{x} = c
\end{align}
Rewriting given equation of plane in \eqref{eq:solutions/4/45/4/b/eq1} form
\begin{align}\label{eq:solutions/4/45/4/b/eq2}
	\myvec{5 & 0 & -2}\myvec{x\\y\\z} = 3
\end{align}
where :
$\vec{n}=\myvec{5\\0\\-2}$, $\vec{x} = \myvec{x\\y\\z}$  and $c=3$\\
We need to represent equation of plane in parametric form,
\begin{equation}\label{eq:solutions/4/45/4/b/eq3}
	\vec{x} = \vec{p} + \lambda_1\vec{q} + \lambda_2\vec{r}
\end{equation}
Here $p$ is any point on plane and $\vec{q}, \vec{r}$ are two vectors parallel to plane and hence $\perp$ to $\vec{n}$. Find two vectors that are $\perp$ to $\vec{n}$
\begin{align}\label{eq:solutions/4/45/4/b/eq4}
	\myvec{5 & 0 &-2}\myvec{a\\b\\c} = 0
\end{align}
Put $a=0$ and $b=1$ in \eqref{eq:solutions/4/45/4/b/eq3}, $\implies c=0$\\
Put $a=1$ and $b=0$ in \eqref{eq:solutions/4/45/4/b/eq3}, $\implies c=\frac{5}{2}$\\
Hence $\vec{q} = \myvec{1\\0\\\frac{5}{2}}, \vec{r} = \myvec{0\\1\\0}$\\
Let us find point $\vec{p}$ on the plane. Put $x=1,y=0$ in \eqref{eq:solutions/4/45/4/b/eq2}, we get $\vec{p} = \myvec{1\\0\\1}$\\
Since given plane is parallel to y-axis, we can use any point $P$ on y-axis to compute shortest distance. 
\begin{equation}\label{eq:solutions/4/45/4/b/eq5}
	\vec{P} = \myvec{0\\0\\0}
\end{equation}
Let $\vec{Q}$ be the point on plane with shortest distance to $\vec{P}$. $\vec{Q}$ can be expressed in \eqref{eq:solutions/4/45/4/b/eq4} form as
\begin{align}\label{eq:solutions/4/45/4/b/eq6}
	\vec{Q} = \myvec{1\\0\\1} + \lambda_1\myvec{1\\0\\\frac{5}{2}} + \lambda_2\myvec{0\\1\\0}
\end{align}
Equation $\vec{P}$ and $\vec{Q}$, and computing pseudo inverse using SVD should give the value of $\lambda_1$ and $\lambda_2$ (since plane and y-axis never intersect pseudo inverse should give the points which are closest)
\begin{align}
	\label{eq:solutions/4/45/4/b/eq7}\myvec{1\\0\\1} + \lambda_1\myvec{1\\0\\\frac{5}{2}} + \lambda_2\myvec{0\\1\\0} &= \myvec{0\\0\\0}\\
	\label{eq:solutions/4/45/4/b/eq8}\lambda_1\myvec{1\\0\\\frac{5}{2}} + \lambda_2\myvec{0\\1\\0} &= \myvec{-1\\0\\-1}\\
	\label{eq:solutions/4/45/4/b/eq9}\myvec{1 & 0\\0 & 1\\\frac{5}{2} & 0} \myvec{\lambda_1 \\ \lambda_2} &=\myvec{-1\\0\\-1}\\
	\label{eq:solutions/4/45/4/b/eq10}\vec{M}\vec{x} &= \vec{b}\\
	\label{eq:solutions/4/45/4/b/eq11}\vec{x} &= \vec{M}^{+}\vec{b}
\end{align}
where $\vec{M} = \myvec{1 & 0\\0 & 1\\\frac{5}{2} & 0}$, $\vec{x}= \myvec{\lambda_1 \\ \lambda_2}$ and $\vec{b}=\myvec{-1\\0\\-1}$\\
\\
Diagonalize $\vec{M}\vec{M}^T$
\begin{align}
	\label{eq:solutions/4/45/4/b/eq12}\vec{M}\vec{M}^T &= \myvec{1 & 0\\0 & 1\\\frac{5}{2} & 0}\myvec{1 & 0 & \frac{5}{2}\\0 & 1 & 0} = \myvec{1 & 0 & \frac{5}{2}\\0 & 1 & 0\\\frac{5}{2} & 0 & \frac{25}{4}}\\
	\label{eq:solutions/4/45/4/b/eq13}&= \myvec{0 & \frac{2}{5} & -\frac{5}{2}\\1 & 0 & 0\\0 & 1 & 1}\myvec{\frac{29}{4} & 0 & 0\\0 & 1 & 0\\0 & 0 & 0}\myvec{0 & 1 & 0\\\frac{2}{5} & 0 & 1\\-\frac{5}{2} & 0 & 1}\\
	&= \vec{U}\Sigma^T\Sigma\vec{U}^T
\end{align}
Verify \eqref{eq:solutions/4/45/4/b/eq13} from,
\begin{lstlisting}
codes/diagonalize1.py
\end{lstlisting}
Diagonalize $\vec{M}^T\vec{M}$
\begin{align}
	\label{eq:solutions/4/45/4/b/e14}\vec{M}^T\vec{M} &= \myvec{1 & 0 & \frac{5}{2}\\0 & 1 & 0}\myvec{1 & 0\\0 & 1\\\frac{5}{2} & 0} = \myvec{\frac{29}{4} & 0\\0 & 1}\\
	\label{eq:solutions/4/45/4/b/eq15}&= \myvec{0 & 1\\1 & 0}\myvec{1 & 0\\0 & \frac{29}{4}}\myvec{0 & 1\\1 & 0}\\
	&= \vec{V}\Sigma^T\Sigma\vec{V}^T
\end{align}
Verify \eqref{eq:solutions/4/45/4/b/eq15} from,
\begin{lstlisting}
codes/diagonalize2.py
\end{lstlisting}
Compute SVD of $\vec{M}$ from \eqref{eq:solutions/4/45/4/b/eq13} and \eqref{eq:solutions/4/45/4/b/eq16},
\begin{align}
	\label{eq:solutions/4/45/4/b/eq16}\vec{M} &= \vec{U}\Sigma\vec{V}^T\\
	\label{eq:solutions/4/45/4/b/eq17}\myvec{1 & 0\\0 & 1\\\frac{5}{2} & 0} &= \myvec{0 & \frac{2}{5} & -\frac{5}{2}\\1 & 0 & 0\\0 & 1 & 1}\myvec{\frac{\sqrt{29}}{2} & 0 \\0 & 1\\0 & 0} \myvec{1 & 0\\0 & 1}\\
	\label{eq:solutions/4/45/4/b/eq18}\vec{M}^{+} &= \vec{V}\Sigma^T\vec{U}^T\\
	\label{eq:solutions/4/45/4/b/eq19}&= \myvec{1 & 0\\0 & 1}\myvec{\frac{\sqrt{29}}{2} & 0 & 0\\0 & 1 & 0}\myvec{0 & 1 & 0\\\frac{2}{5} & 0 & 1\\-\frac{5}{2} & 0 & 1}\\
	\label{eq:solutions/4/45/4/b/eq20}&=\myvec{\frac{4}{29} & 0 & \frac{10}{29}\\0 & 1 & 0}
\end{align}
Verify \eqref{eq:solutions/4/45/4/b/eq20} from,
\begin{lstlisting}
codes/pseudo_inverse.py
\end{lstlisting}
Substitute \eqref{eq:solutions/4/45/4/b/eq20} in \eqref{eq:solutions/4/45/4/b/eq11},
\begin{align}\label{eq:solutions/4/45/4/b/23}
	\vec{x} = \myvec{\frac{4}{29} & 0 & \frac{10}{29}\\0 & 1 & 0}\myvec{-1\\0\\-1} = \myvec{-\frac{14}{29}\\0} = \myvec{\lambda_1 \\ \lambda_2}
\end{align}
Substituting $\lambda_1$, $\lambda_2$ in \eqref{eq:solutions/4/45/4/b/eq6}
\begin{equation}
	\vec{Q} = \myvec{\frac{15}{29}\\0\\-\frac{6}{29}}
\end{equation}
Distance between point $\vec{P}$ and $\vec{Q}$ is
\begin{align}
	\norm{\vec{P}-\vec{Q}} = \sqrt{\left(\frac{15}{29}\right)^2 + 0 + \left(-\frac{6}{29}\right)^2} = \frac{3}{\sqrt{29}}
\end{align}
Hence, distance from y-axis to $5x - 2z - 3 = 0$ is $\frac{3}{\sqrt{29}}$.\\\\
Verifying solution to \eqref{eq:solutions/4/45/4/b/eq10} by least squares method
\begin{align}
	\vec{M}^T(\vec{b} - \vec{M}\vec{x}) &= 0\\
	\label{eq:solutions/4/45/4/b/eq30}\implies \vec{M}^T\vec{M}\vec{x} &= \vec{M}^T\vec{b}
\end{align}
Substituting $\vec{M}, \vec{b}$ from \eqref{eq:solutions/4/45/4/b/eq9} in \eqref{eq:solutions/4/45/4/b/eq30}
\begin{align}
	\myvec{1 & 0 & \frac{5}{2}\\0 & 1 & 0}\myvec{1 & 0\\0 & 1\\\frac{5}{2} & 0}\vec{x} &= \myvec{1 & 0 & \frac{5}{2}\\0 & 1 & 0}\myvec{-1\\0\\-1}\\
	\myvec{\frac{29}{4} & 0\\0 & 1}\myvec{\lambda_1\\\lambda_2} &= \myvec{-\frac{7}{2}\\0}\\
	\implies\frac{29}{4}\lambda_1 &= -\frac{7}{2}\\
	\lambda_1 &= -\frac{7}{2} \times \frac{4}{29} = -\frac{14}{29}\\
	\text{and }\lambda_2 &= 0\\
	\label{eq:solutions/4/45/4/b/31}\vec{x} &= \myvec{-\frac{14}{29}\\0}
\end{align}
Comparing \eqref{eq:solutions/4/45/4/b/23} and \eqref{eq:solutions/4/45/4/b/31} solution is verified.
