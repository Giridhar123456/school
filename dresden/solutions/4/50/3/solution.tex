
 Let the equation of plane is 
 \begin{align}
 ax +by +cz +d = 0 \label{eq:solutions/4/50/3/eq2.1}
 \end{align}
 
% we can express \eqref{eq:solutions/4/50/3/eq2.1} as ,
% \begin{align}
%\myvec{a & b & c}\vec{x} = -d
% \end{align}
  Direction ratio of the line \eqref{eq:solutions/4/50/3/eq1.1} is given as 
  \begin{align}
  \vec{D} = \myvec{-3 \\ 1 \\ 2}
  \end{align}
  
  \begin{align}
  \intertext{ Now let consider } 
  \vec{A} = \myvec{-3 & 4 & -1} 
  \intertext{Since plane is passing through the point A (-3, 4, -1 ) and perpendicular to the line \eqref{eq:solutions/4/50/3/eq1.1}, hence}
  \vec{A}\vec{D} + d = 0 \\
  \implies d = -11
   \end{align}
  Hence equation of the plane is 
  \begin{align}
  -3x + y + 2z - 11 = 0 \\
  \implies -3x + y + 2z = 11 \label{eq:solutions/4/50/3/2.7}
  \end{align} 
  equation \eqref{eq:solutions/4/50/3/2.7} can written as :
  \begin{align}
  \myvec{-3 & 1 & 2}\vec{x} = 11
  \end{align}
  For foot perpendicular we need to find the distance between the plane and point P $\left( -2, 2, 4\right)$.\\ 
First we find orthogonal vectors $\vec{m_1}$ and $\vec{m_2}$ to the given normal vector $\vec{n}$. Let, $\vec{m}$ = $\myvec{a\\b\\c}$, then
\begin{align}
\vec{m^T}\vec{n} &= 0\\
\implies\myvec{a&b&c}\myvec{-3\\1\\2} &= 0\\
\implies -3a+b+2c &= 0\\
\intertext{Putting a=1 and b=0 we get,}
\vec{m_1} &= \myvec{1\\0\\\frac{3}{2} }\\
\intertext{Putting a=0 and b=1 we get,}
\vec{m_2} &= \myvec{0\\1\\ -\frac{1}{2}}
\end{align}

Now we solve the equation,
\begin{align}
\vec{M}\vec{x} &= \vec{b}\label{eq:solutions/4/50/3/eq1}\\
\intertext{Putting values in \eqref{eq:solutions/4/50/3/eq1},}
\myvec{1&0\\0&1\\ \frac{3}{2}& -\frac{1}{2}}\vec{x} &= \myvec{-2\\2\\4}\label{eq:solutions/4/50/3/eq2}
\end{align}
Now, to solve \eqref{eq:solutions/4/50/3/eq2}, we perform Singular Value Decomposition on $\vec{M}$ as follows,
\begin{align}
\vec{M}=\vec{U}\vec{S}\vec{V}^T\label{eq:solutions/4/50/3/eqSVD}
\end{align}
Where the columns of $\vec{V}$ are the eigen vectors of $\vec{M}^T\vec{M}$ ,the columns of $\vec{U}$ are the eigen vectors of $\vec{M}\vec{M}^T$ and $\vec{S}$ is diagonal matrix of singular value of eigenvalues of $\vec{M}^T\vec{M}$.
\begin{align}
\vec{M}^T\vec{M}=\myvec{\frac{13}{4}& -\frac{3}{4}\\ -\frac{3}{4}&\frac{5}{4}}\label{eq:solutions/4/50/3/eqMTM}\\
\vec{M}\vec{M}^T=\myvec{1&0& \frac{3}{2}\\0&1& -\frac{1}{2}\\ \frac{3}{2}&-\frac{1}{2}& \frac{5}{2}}
\end{align}
From \eqref{eq:solutions/4/50/3/eq1} putting \eqref{eq:solutions/4/50/3/eqSVD} we get,
\begin{align}
\vec{U}\vec{S}\vec{V}^T\vec{x} & = \vec{b}\\
\implies\vec{x} &= \vec{V}\vec{S_+}\vec{U^T}\vec{b}\label{eq:solutions/4/50/3/eqX}
\end{align}
Where $\vec{S_+}$ is Moore-Penrose Pseudo-Inverse of $\vec{S}$.Now, calculating eigen value of $\vec{M}\vec{M}^T$,


\begin{align}
\mydet{\vec{M}\vec{M}^T - \lambda\vec{I}} &= 0\\
\implies\myvec{1-\lambda&0& \frac{3}{2} \\0&1-\lambda& -\frac{1}{2}\\   \frac{3}{2}& -\frac{1}{2}&\frac{5}{2}-\lambda} &=0\\
\implies\lambda (\lambda - 1)(\lambda - \frac{7}{2}) &=0
\end{align}


Hence eigen values of $\vec{M}\vec{M}^T$ are,
\begin{align}
\lambda_1 &= \frac{7}{2}\\
\lambda_2 &= 1\\
\lambda_3 &= 0
\end{align}
Hence the eigen vectors of $\vec{M}\vec{M}^T$ are,
\begin{align}
\vec{u}_1=\myvec{\frac{3}{5} \\-\frac{1}{5}\\ 1},
\vec{u}_2=\myvec{\frac{1}{3}\\ 1 \\ 0 },
\vec{u}_3=\myvec{-\frac{3}{2}\\ \frac{1}{2} \\ 1 },
\intertext{Normalizing the eigen vectors we get,}
\vec{u}_1=\myvec{ \frac{3}{\sqrt{35}}\\ - \frac{1}{\sqrt{35}}\\\frac{5}{\sqrt{35}}},
\vec{u}_2=\myvec{\frac{1}{\sqrt{10}}\\\frac{3}{\sqrt{10}}\\ 0},
\vec{u}_3=\myvec{-\frac{3}{\sqrt{14}}\\ \frac{1}{\sqrt{14}}\\ \frac{2}{\sqrt{14}}}
\end{align}
Hence we obtain $\vec{U}$ of \eqref{eq:solutions/4/50/3/eqSVD} as follows,
\begin{align}
\vec{U} = \myvec{ \frac{3}{\sqrt{35}} & \frac{1}{\sqrt{10}}  &  -\frac{3}{\sqrt{14}}  \\  - \frac{1}{\sqrt{35}} & \frac{3}{\sqrt{10}}  &  \frac{1}{\sqrt{14}} \\ \frac{5}{\sqrt{35}} & 0  &  \frac{2}{\sqrt{14}} }\label{eq:solutions/4/50/3/eqU}
\end{align}
After computing the singular values from eigen values $\lambda_1, \lambda_2, \lambda_3$ we get $\vec{S}$ of \eqref{eq:solutions/4/50/3/eqSVD} as follows,
\begin{align}
\vec{S}=\myvec{\sqrt{\frac{7}{2}}&0\\0&1\\0&0}\label{eq:solutions/4/50/3/eqS}
\end{align}
Now, calculating eigen value of $\vec{M}^T\vec{M}$,
\begin{align}
\mydet{\vec{M}^T\vec{M} - \lambda\vec{I}} &= 0\\
\implies\myvec{\frac{13}{4}-\lambda& -\frac{3}{4}\\-\frac{3}{4}&\frac{5}{4}-\lambda} &=0\\
\implies\lambda^2-\frac{9}{2}\lambda+\frac{7}{2} &=0
\end{align}
Hence eigen values of $\vec{M}^T\vec{M}$ are,
\begin{align}
\lambda_4 &= \frac{7}{2}\\
\lambda_5 &= 1
\end{align}
Hence the eigen vectors of $\vec{M}^T\vec{M}$ are,
\begin{align}
\vec{v}_1=\myvec{-3\\1},
\vec{v}_2=\myvec{ \frac{1}{3}\\ 1}
\intertext{Normalizing the eigen vectors we get,}
\vec{v}_1=\myvec{-\frac{3}{\sqrt{10}}\\\frac{1}{\sqrt{10}}},
\vec{v}_2=\myvec{\frac{1}{\sqrt{10}}\\\frac{3}{\sqrt{10}}}
\end{align}
Hence we obtain $\vec{V}$ of \eqref{eq:solutions/4/50/3/eqSVD} as follows,
\begin{align}
\vec{V}=\myvec{-\frac{3}{\sqrt{10}}&\frac{1}{\sqrt{10}}\\ \frac{1}{\sqrt{10}}&\frac{3}{\sqrt{10}}}
\end{align}
Finally from \eqref{eq:solutions/4/50/3/eqSVD} we get the Singualr Value Decomposition of $\vec{M}$ as follows,
\begin{align}
\vec{M} =  \myvec{ \frac{3}{\sqrt{35}} & \frac{1}{\sqrt{10}}  &  -\frac{3}{\sqrt{14}}  \\  - \frac{1}{\sqrt{35}} & \frac{3}{\sqrt{10}}  &  \frac{1}{\sqrt{14}} \\ \frac{5}{\sqrt{35}} & 0  &  \frac{2}{\sqrt{14}} }\myvec{\sqrt{\frac{7}{2}}&0\\0&1\\0&0}\myvec{\frac{3}{\sqrt{10}}&\frac{1}{\sqrt{10}}\\ -\frac{1}{\sqrt{10}}&\frac{3}{\sqrt{10}}}^T
\end{align}
Now, Moore-Penrose Pseudo inverse of $\vec{S}$ is given by,
\begin{align}
\vec{S_+} = \myvec{\sqrt{\frac{2}{7}}&0&0\\0&1&0}
\end{align}
From \eqref{eq:solutions/4/50/3/eqX} we get,
\begin{align}
\vec{U}^T\vec{b}&=\myvec{\frac{12}{\sqrt{35}}\\\frac{2\sqrt{2}}{\sqrt{5}}\\ \frac{8\sqrt{2}}{\sqrt{7}}}\\
\vec{S_+}\vec{U}^T\vec{b}&=\myvec{\frac{12\sqrt{10}}{35}\\\frac{2\sqrt{10}}{5}}\\
\vec{x} = \vec{V}\vec{S_+}\vec{U}^T\vec{b} &= \myvec{\frac{10}{7}\\\frac{6}{7}}\label{eq:solutions/4/50/3/eqXSol1}
\end{align}
Verifying the solution of \eqref{eq:solutions/4/50/3/eqXSol1} using,
\begin{align}
\vec{M}^T\vec{M}\vec{x} = \vec{M}^T\vec{b}\label{eq:solutions/4/50/3/eqVerify}
\end{align}
Evaluating the R.H.S in \eqref{eq:solutions/4/50/3/eqVerify} we get,
\begin{align}
\vec{M}^T\vec{M}\vec{x} &= \myvec{-7\\\frac{1}{2}}\\
\implies\myvec{\frac{13}{4}& -\frac{3}{4}\\ -\frac{3}{4}&\frac{5}{4}}\vec{x} &= \myvec{4\\0}\label{eq:solutions/4/50/3/eqMateq}
\end{align}
Solving the augmented matrix of \eqref{eq:solutions/4/50/3/eqMateq} we get,
\begin{align}
\myvec{\frac{13}{4}&-\frac{3}{4}& 4 \\-\frac{3}{4}&\frac{5}{4}& 0} &\xleftrightarrow{R_1=\frac{4}{13}R_1}\myvec{1&-\frac{3}{13}&\frac{16}{13}\\ -\frac{3}{4}&\frac{5}{4}&0}\\
&\xleftrightarrow{R_2=R_2+\frac{3}{4}R_1}\myvec{1&-\frac{3}{13}&\frac{16}{13}\\0 & \frac{14}{13}&\frac{12}{13}}\\
&\xleftrightarrow{R_2=\frac{13}{14}R_2}\myvec{1& -\frac{3}{13}&\frac{16}{13}\\ 0 & 1 &\frac{6}{7}}\\
&\xleftrightarrow{R_1=R_1+\frac{3}{13}R_1}\myvec{1&0&\frac{10}{7}\\0&1&\frac{6}{7}}
\end{align}
Hence, Solution of \eqref{eq:solutions/4/50/3/eqVerify} is given by,
\begin{align}
\vec{x}=\myvec{\frac{10}{7}\\\frac{6}{7}}\label{eq:solutions/4/50/3/eqX2}
\end{align}
Comparing results of $\vec{x}$ from \eqref{eq:solutions/4/50/3/eqXSol1} and \eqref{eq:solutions/4/50/3/eqX2} we conclude that the solution is verified.















