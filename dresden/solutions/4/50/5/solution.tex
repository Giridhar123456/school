Given a point $\vec{A} = \myvec{3 \\4\\-1}$  and a plane $\myvec{2 & -1 & 2}\vec{x}=5$. We know that the equation of a plane is given by
\begin{align}
	\vec{n^Tx}= c
\end{align}
Hence, normal vector $\vec{n}$ is given by
\begin{align}
	\vec{n} = \myvec{2 \\ -1\\ 2 }
\end{align}
Let $\vec{m_1}$ and $\vec{m_2}$ be two vectors that are normal to normal vector $\vec{n}$.
Let $\vec{m}= \myvec{a\\b\\c}$, then if
\begin{align}
& \vec{n^Tm}= 0 \\
& \myvec{2 & -1 &2}\myvec{a\\b\\c} = 0
\end{align}
Taking $a =1$, $b=0$, we get $c=-1$, and hence
\begin{align}
	\vec{m_1}= \myvec{1\\0\\-1}
\end{align}
Take $a=0$ and $b=1$, we get $c=\frac{1}{2}$, and hence
\begin{align}
	\vec{m_2}= \myvec{0\\1\\\frac{1}{2}}
\end{align}
Since  foot of perpendicular is the point where the plane is met by a line perpendicular to the same plane. So, to get foot of perpendicular, we solve
\begin{align}
\vec{Mx=b} \label{eq:solutions/4/50/5/1}
\end{align}
where
\begin{align}
\vec{M} = \myvec{1 & 0 \\ 0 & 1 \\ -1 & \frac{1}{2}},b = \myvec{3 \\4\\-1}\label{eq:solutions/4/50/5/2}	
\end{align}
To solve \eqref{eq:solutions/4/50/5/1}, we perform singular value decomposition on $\vec{M}$ given as 
\begin{align}
	\vec{M = USV^T }\label{eq:solutions/4/50/5/3}
\end{align}
Substituting the value of $\vec{M}$ from \eqref{eq:solutions/4/50/5/3} in \eqref{eq:solutions/4/50/5/1}, we get
\begin{align}
	&\vec{USV^T}\vec{x} = \vec{b} \\
\implies& \vec{x} = \vec{VS_+U^T}\vec{b}\label{eq:solutions/4/50/5/4}
\end{align}
where, $\vec{S_+}$ is Moore-Pen-rose Pseudo-Inverse of $\vec{S}$. Columns of $\vec{U}$ are eigen-vectors of $\vec{MM^T}$, columns of $\vec{V}$ are eigenvectors of $\vec{M^TM}$ and $\vec{S}$ is diagonal matrix of singular value of eigenvalues of $\vec{M^TM}$. First calculating the eigenvectors corresponding to $\vec{M^TM}$.
\begin{align}
\vec{M^TM} = \myvec{1 & 0 & -1 \\ 0 & 1 & \frac{1}{2} } \myvec{1 & 0 \\ 0 & 1 \\ -1 & \frac{1}{2}} = \myvec{2 & \frac{-1}{2} \\ \frac{-1}{2} & \frac{5}{4}}
\end{align}
Eigen values of $\vec{M^TM}$ can be found out as
\begin{align}
	& \abs{\vec{M^TM-\lambda I}} = 0 \\
	& \abs{\myvec{2-\lambda & \frac{-1}{2} \\ \frac{-1}{2} & \frac{5}{4} -\lambda}} = 0 \\
	& \brak{\frac{5}{4}-\lambda}\brak{2-\lambda} - \frac{1}{4} = 0 \\
	& \brak{\lambda-\frac{9}{4}}\brak{\lambda-1} = 0 
\end{align}
Hence,
\begin{align}
\lambda_1 = \frac{9}{4}, \lambda_2 = 1
\end{align}
Eigen-vector corresponding to $\lambda=\frac{9}{4}$,
\begin{align}
\vec{v_1} = \myvec{2 \\ -1} 
\end{align}
Eigen-vector corresponding to $\lambda = 1$,
\begin{align}
\vec{v_2} = \myvec{1 \\ 2}
\end{align}
Normalizing, the eigen vectors $\vec{v_1}$ and $\vec{v_2}$, we get
\begin{align}
& \vec{v_1} = \frac{1}{\sqrt{5}}\myvec{2 \\ -1} \\
& \vec{v_2} = \frac{1}{\sqrt{5}}\myvec{1 \\ 2} 
\end{align}
Hence,
\begin{align}
	\vec{V} = \frac{1}{\sqrt{5}}\myvec{2 & 1 \\ -1 & 2 } \label{eq:solutions/4/50/5/5}
\end{align}
Now calculating the eigenvectors corresponding to $\vec{MM^T}$
\begin{multline}
\vec{MM^T} = \myvec{1 & 0 \\ 0 & 1 \\ -1 & \frac{1}{2}}\myvec{1 & 0 & -1 \\ 0 & 1 & \frac{1}{2} } \\ =
\myvec{1 & 0 &-1 \\ 0 & 1 & \frac{1}{2} \\ -1 & \frac{1}{2} & \frac{5}{4}}
\end{multline}
Eigen values of $\vec{MM^T}$ can be found out as
\begin{align}
	& \abs{\vec{MM^T-\lambda I}} = 0 \\
	& \abs{\myvec{1-\lambda & 0 & -1 \\ 0 & 1-\lambda & \frac{1}{2} \\ -1 & \frac{1}{2} & \frac{5}{4}-\lambda}} = 0 \\
	& \brak{1-\lambda}\brak{\brak{1-\lambda}\brak{\frac{5}{4}-\lambda}-\frac{1}{4}}
	-1+\lambda = 0 \\
	& \lambda\brak{\lambda-\frac{9}{4}}\brak{\lambda-1} = 0  
\end{align}
Hence,
\begin{align}
\lambda_3 = 0, \lambda_4 = 1, \lambda_5=\frac{9}{4}
\end{align}
Eigen-vector corresponding to $\lambda=0$,
\begin{align}
	\vec{v_3} = \myvec{2 \\ -1 \\ 2}
\end{align}
Eigen-vector corresponding to $\lambda=1$,
\begin{align}
	\vec{v_4} = \myvec{1 \\ 2 \\ 0}
\end{align}
Eigen-vector corresponding to $\lambda=\frac{9}{4}$,
\begin{align}
	\vec{v_5} = \myvec{4 \\ -2 \\ -5}
\end{align}
Normalizing, the eigen vectors $\vec{v_3}$, $\vec{v_4}$ and $\vec{v_5}$, we get
\begin{align}
	& \vec{v_3} = \frac{1}{3}\myvec{2 \\ -1 \\ 2} = \myvec{\frac{2}{3} \\ \frac{-1}{3} \\ \frac{2}{3}} \\
	& \vec{v_4} = \frac{1}{\sqrt{5}}\myvec{1 \\ 2 \\ 0} = \myvec{\frac{1}{\sqrt{5}} \\ \frac{2}{\sqrt{5}} \\ 0} \\ 
	& \vec{v_5} = \frac{1}{3\sqrt{5}}\myvec{4 \\ -2 \\ -5} =  \myvec{\frac{4}{3\sqrt{5}} \\ \frac{-2}{3\sqrt{5}} \\ \frac{-5}{3\sqrt{5}}}
\end{align}
Hence,
\begin{align}
	\vec{U} = \myvec{\frac{4}{3\sqrt{5}} & \frac{1}{\sqrt{5}} & \frac{2}{3} \\ \frac{-2}{3\sqrt{5}}  & \frac{2}{\sqrt{5}} & \frac{-1}{3} \\ \frac{-5}{3\sqrt{5}}  & 0 & \frac{2}{3} } \label{eq:solutions/4/50/5/6}
\end{align}
Now $\vec{S}$ corresponding to eigenvalues $\lambda_5$, $\lambda_4$ and $\lambda_3$ is as follows,
\begin{align}
	\vec{S} = \myvec{\frac{3}{2} & 0 \\ 0 & 1  \\ 0 & 0}\label{eq:solutions/4/50/5/7}
\end{align}
Now, Moore-Pen-Rose Pseudo inverse of $\vec{S}$ is given by,
\begin{align}
	\vec{S_+} = \myvec{\frac{2}{3} & 0 & 0 \\ 0 & 1 & 0}
\end{align}
Hence, we get singular value decomposition of $\vec{M}$ as,
\begin{align}
\vec{M} = \myvec{\frac{4}{3\sqrt{5}} & \frac{1}{\sqrt{5}} & \frac{2}{3} \\ \frac{-2}{3\sqrt{5}}  & \frac{2}{\sqrt{5}} & \frac{-1}{3} \\ \frac{-5}{3\sqrt{5}}  & 0 & \frac{2}{3} } \myvec{\frac{3}{2} & 0 \\ 0 & 1  \\ 0 & 0} \frac{1}{\sqrt{5}}\myvec{2 & -1 \\ 1 & 2 }
\end{align}
Substituting values of \eqref{eq:solutions/4/50/5/2}, \eqref{eq:solutions/4/50/5/5}, \eqref{eq:solutions/4/50/5/6} and \eqref{eq:solutions/4/50/5/7} into \eqref{eq:solutions/4/50/5/4}, we get
\begin{align}
& \vec{U^Tb} =  \myvec{\frac{4}{3\sqrt{5}} & \frac{-2}{3\sqrt{5}} & \frac{-5}{3\sqrt{5}} \\ \frac{1}{\sqrt{5}}   & \frac{2}{\sqrt{5}} & 0 \\ \frac{2}{3} & \frac{-1}{3} & \frac{2}{3}}
\myvec{3 \\4\\-1} \\
& \qquad \qquad\implies \vec{U^Tb} = \myvec{\frac{3}{\sqrt{5}} \\ \frac{11}{\sqrt{5}} \\ 0} 
\end{align}
Now,
\begin{align}
& \vec{VS_+} = \frac{1}{\sqrt{5}}\myvec{2 & 1 \\ -1 & 2 }\myvec{\frac{2}{3} & 0 & 0 \\ 0 & 1 & 0}\\
& \qquad \qquad \implies \vec{VS_+} = \frac{1}{\sqrt{5}}\myvec{\frac{4}{3} & 1 &0 \\ \frac{-2}{3} & 2 & 0 }
\end{align}
Now, by \eqref{eq:solutions/4/50/5/4}, we have
\begin{align}
& \vec{x} = \frac{1}{\sqrt{5}}\myvec{\frac{4}{3} & 1 &0 \\ \frac{-2}{3} & 2 & 0 } 
\myvec{\frac{3}{\sqrt{5}} \\ \frac{11}{\sqrt{5}} \\ 0}  \\
& \qquad \qquad \implies \vec{x} = \myvec{3 \\ 4}
\end{align}
Now, we verify our solution using
\begin{align}
&	\vec{M^TMx = M^Tb} \\
\implies & \myvec{1 & 0 & -1  \\ 0 & 1 & \frac{1}{2} }\myvec{1 & 0 \\ 0 & 1 \\ -1 & \frac{1}{2} }\vec{x} = \myvec{1 & 0 & -1  \\ 0 & 1 & \frac{1}{2} }\myvec{3 \\4\\-1} 
\end{align}
\begin{align}
\implies & \myvec{2 & \frac{-1}{2} \\ \frac{-1}{2} & \frac{5}{4}}\vec{x} =\myvec{4 \\\frac{7}{2}}
\end{align}
Solving the augumented matrix, we get
\begin{align}
& \myvec{2 & \frac{-1}{2} & 4 \\ \frac{-1}{2} & \frac{5}{4} & \frac{7}{2}} 
&\xleftrightarrow[]{r_1 = (1/2)*(r_1) } & \myvec{1 & \frac{-1}{4} & 2 \\ \frac{-1}{2} & \frac{5}{4} & \frac{7}{2}} \\
& \myvec{1 & \frac{-1}{4} & 2 \\ \frac{-1}{2} & \frac{5}{4} & \frac{7}{2}} & \xleftrightarrow[]{r_2 = r_2 + (1/2)*(r_1)} & \myvec{1 & \frac{-1}{4} & 2 \\ 0 & \frac{9}{8} & \frac{9}{2}} \\
&\myvec{1 & \frac{-1}{4} & 2 \\ 0 & \frac{9}{8} & \frac{9}{2}} &\xleftrightarrow[]{r_2 =(8/9)*(r_2)} & \myvec{1 & \frac{-1}{4} & 2 \\ 0 & 1 & 4} \\
& \myvec{1 & \frac{-1}{4} & 2 \\ 0 & 1 & 4} &\xleftrightarrow[]{r_1 = r_1 + (-1/4)*(r_2)} & \myvec{1 & 0 & 3 \\ 0 & 1 & 4}
\end{align}
Thus,
\begin{align}
	\vec{x} = \myvec{3\\4} 
\end{align}
verifying the result from SVD.

Now, we solve for third coordinate of foot of perpendicular by,
\begin{align}
&\vec{n^Tx} = 5 \\
& \myvec{2 & -1 & 2}\myvec{3 \\ -4 \\ z} = 5\\
& z = \frac{-5}{2}
\end{align}
Normalizing z, we get
\begin{align}
	& z = \frac{\brak{\frac{-5}{2}}}{3} \implies z = \frac{-5}{6}
\end{align}
Hence, coordinate of foot of perpendicular is
\begin{align}
	\vec{x} = \myvec{3 \\ 4\\\frac{-5}{6}}
\end{align}
Now, we try to find equation of straight line through $\vec{P}= \myvec{3\\4\\-1}$ and having direction cosines as $\vec{Q}= \myvec{2 \\-1 \\ 2}$
\begin{align}
	L_1 :  \vec{x} = \myvec{3\\4\\-1} + \lambda\myvec{2 \\-1\\2}
\end{align}
