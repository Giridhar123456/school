\renewcommand{\theequation}{\theenumi}
\begin{enumerate}[label=\arabic*.,ref=\thesubsection.\theenumi]
\item What does the equation 
\begin{align}
\vec{x}^T\myvec{1 & 0\\0 & -1}\vec{x}-\myvec{4 & 6}\vec{x}-6=0
\end{align}
become when the origin is moved to the point $\myvec{2\\-3}$?
\\
\solution
Let the balanced version of (\ref{eq:solutions/chem/6ato balance}) be
\begin{align}
    \label{eq:solutions/chem/6abalanced}x_{1}HNO_{3}+ x_{2}Ca(OH)_{2}\to x_{3}Ca(NO_{3})_{2}+ x_{4}H_{2}O
\end{align}

which results in the following equations:
\begin{align}
    (x_{1}+ 2x_{2}-2x_{4}) H= 0\\
    (x_{1}-2x_{3}) N= 0\\
    (3x_{1}+ 2x_{2}-6x_{3}- x_{4}) O=0\\
    (x_{2}-x_{3}) Ca= 0
\end{align}

which can be expressed as
\begin{align}
    x_{1}+ 2x_{2}+ 0.x_{3} -2x_{4} = 0\\
    x_{1}+ 0.x_{2} -2x_{3} +0.x_{4}= 0\\
    3x_{1}+ 2x_{2}-6x_{3}- x_{4} =0\\
    0.x_{1} +x_{2}-x_{3} +0.x_{4}= 0
\end{align}

resulting in the matrix equation
\begin{align}
    \label{eq:solutions/chem/6a matrix}
    \myvec{1 & 2 & 0 & -2\\
           1 & 0 & -2 & 0\\
           3 & 2 & -6 & -1\\
           0 & 1 & -1 & 0}\vec{x}
           =\vec{0}
\end{align}

where,
\begin{align}
   \vec{x}= \myvec{x_{1}\\x_{2}\\x_{3}\\x_{4}}
\end{align}

(\ref{eq:solutions/chem/6a matrix}) can be reduced as follows:
\begin{align}
    \myvec{1 & 2 & 0 & -2\\
           1 & 0 & -2 & 0\\
           3 & 2 & -6 & -1\\
           0 & 1 & -1 & 0}
    \xleftrightarrow[R_{3}\leftarrow \frac{R_3}{3}-R_{1}]{R_{2}\leftarrow R_2- R_1}
    \myvec{1 & 2 & 0 & -2\\
           0 & -2 & -2 & 2\\
           0 & -\frac{4}{3} & -2 & \frac{5}{3}\\
           0 & 1 & -1 & 0}\\
    \xleftrightarrow{R_2 \leftarrow -\frac{R_2}{2}}
    \myvec{1 & 2 & 0 & -2\\
          0 & 1 & 1 & -1\\
          0 & -\frac{4}{3} & -2 & \frac{5}{3}\\
          0 & 1 & -1 & 0}\\
    \xleftrightarrow[R_4 \leftarrow R_4- R_2]{R_3 \leftarrow R_3 + \frac{4}{3}R_2}
    \myvec{1 & 2 & 0 & -2\\
           0 & 1 & 1 & -1\\
           0 & 0 & -\frac{2}{3} & \frac{1}{3}\\
           0 & 0 & -2 & 1}\\
    \xleftrightarrow[R_3 \leftarrow -\frac{3}{2}R_3]{R_1 \leftarrow R_1- 2R_2}
    \myvec{1 & 0 & -2 & 0\\
           0 & 1 & 1 & -1\\
           0 & 0 & 1 & -\frac{1}{2}\\
           0 & 0 & -2 & 1}\\
    \xleftrightarrow{R_4\leftarrow R_4 + 2R_3}
    \myvec{1 & 0 & -2 & 0\\
           0 & 1 & 1 & -1\\
           0 & 0 & 1 & -\frac{1}{2}\\
           0 & 0 & 0 & 0}\\
    \xleftrightarrow[R_2\leftarrow R_2-R_3]{R_1\leftarrow R_1 + 2R_3}
    \myvec{1 & 0 & 0 & -1\\
           0 & 1 & 0 & -\frac{1}{2}\\
           0 & 0 & 1 & -\frac{1}{2}\\
           0 & 0 & 0 & 0}
\end{align}

Thus,
\begin{align}
    x_1=x_4, x_2= \frac{1}{2}x_4, x_3=\frac{1}{2}x_4\\
    \implies \quad\vec{x}= x_4\myvec{1\\ \frac{1}{2}\\ \frac{1}{2}\\1} =\myvec{2\\1\\1\\2}
\end{align} 
by substituting $x_4= 2$.

\hfill\break
%\vspace{5mm} 
Hence, (\ref{eq:solutions/chem/6abalanced}) finally becomes
\begin{align}
    2HNO_{3}+ Ca(OH)_{2}\to Ca(NO_{3})_{2}+ 2H_{2}O
\end{align}

\item To what point must origin be shifted so that
\begin{align}\label{eq:solutions/3/4/2/eq12}
\vec{x}^T\myvec{2 & \frac{-3}{2}\\\frac{-3}{2}& 4}\vec{x}+\myvec{ 10& -19}\vec{x}+23=0
\end{align}
is transformed to 
\begin{align}\label{eq:solutions/3/4/2/eq11}
\vec{x}^T\myvec{2 & \frac{-3}{2}\\\frac{-3}{2}& 4}\vec{x}=1
\end{align}
\\
\solution
Let the balanced version of (\ref{eq:solutions/chem/6ato balance}) be
\begin{align}
    \label{eq:solutions/chem/6abalanced}x_{1}HNO_{3}+ x_{2}Ca(OH)_{2}\to x_{3}Ca(NO_{3})_{2}+ x_{4}H_{2}O
\end{align}

which results in the following equations:
\begin{align}
    (x_{1}+ 2x_{2}-2x_{4}) H= 0\\
    (x_{1}-2x_{3}) N= 0\\
    (3x_{1}+ 2x_{2}-6x_{3}- x_{4}) O=0\\
    (x_{2}-x_{3}) Ca= 0
\end{align}

which can be expressed as
\begin{align}
    x_{1}+ 2x_{2}+ 0.x_{3} -2x_{4} = 0\\
    x_{1}+ 0.x_{2} -2x_{3} +0.x_{4}= 0\\
    3x_{1}+ 2x_{2}-6x_{3}- x_{4} =0\\
    0.x_{1} +x_{2}-x_{3} +0.x_{4}= 0
\end{align}

resulting in the matrix equation
\begin{align}
    \label{eq:solutions/chem/6a matrix}
    \myvec{1 & 2 & 0 & -2\\
           1 & 0 & -2 & 0\\
           3 & 2 & -6 & -1\\
           0 & 1 & -1 & 0}\vec{x}
           =\vec{0}
\end{align}

where,
\begin{align}
   \vec{x}= \myvec{x_{1}\\x_{2}\\x_{3}\\x_{4}}
\end{align}

(\ref{eq:solutions/chem/6a matrix}) can be reduced as follows:
\begin{align}
    \myvec{1 & 2 & 0 & -2\\
           1 & 0 & -2 & 0\\
           3 & 2 & -6 & -1\\
           0 & 1 & -1 & 0}
    \xleftrightarrow[R_{3}\leftarrow \frac{R_3}{3}-R_{1}]{R_{2}\leftarrow R_2- R_1}
    \myvec{1 & 2 & 0 & -2\\
           0 & -2 & -2 & 2\\
           0 & -\frac{4}{3} & -2 & \frac{5}{3}\\
           0 & 1 & -1 & 0}\\
    \xleftrightarrow{R_2 \leftarrow -\frac{R_2}{2}}
    \myvec{1 & 2 & 0 & -2\\
          0 & 1 & 1 & -1\\
          0 & -\frac{4}{3} & -2 & \frac{5}{3}\\
          0 & 1 & -1 & 0}\\
    \xleftrightarrow[R_4 \leftarrow R_4- R_2]{R_3 \leftarrow R_3 + \frac{4}{3}R_2}
    \myvec{1 & 2 & 0 & -2\\
           0 & 1 & 1 & -1\\
           0 & 0 & -\frac{2}{3} & \frac{1}{3}\\
           0 & 0 & -2 & 1}\\
    \xleftrightarrow[R_3 \leftarrow -\frac{3}{2}R_3]{R_1 \leftarrow R_1- 2R_2}
    \myvec{1 & 0 & -2 & 0\\
           0 & 1 & 1 & -1\\
           0 & 0 & 1 & -\frac{1}{2}\\
           0 & 0 & -2 & 1}\\
    \xleftrightarrow{R_4\leftarrow R_4 + 2R_3}
    \myvec{1 & 0 & -2 & 0\\
           0 & 1 & 1 & -1\\
           0 & 0 & 1 & -\frac{1}{2}\\
           0 & 0 & 0 & 0}\\
    \xleftrightarrow[R_2\leftarrow R_2-R_3]{R_1\leftarrow R_1 + 2R_3}
    \myvec{1 & 0 & 0 & -1\\
           0 & 1 & 0 & -\frac{1}{2}\\
           0 & 0 & 1 & -\frac{1}{2}\\
           0 & 0 & 0 & 0}
\end{align}

Thus,
\begin{align}
    x_1=x_4, x_2= \frac{1}{2}x_4, x_3=\frac{1}{2}x_4\\
    \implies \quad\vec{x}= x_4\myvec{1\\ \frac{1}{2}\\ \frac{1}{2}\\1} =\myvec{2\\1\\1\\2}
\end{align} 
by substituting $x_4= 2$.

\hfill\break
%\vspace{5mm} 
Hence, (\ref{eq:solutions/chem/6abalanced}) finally becomes
\begin{align}
    2HNO_{3}+ Ca(OH)_{2}\to Ca(NO_{3})_{2}+ 2H_{2}O
\end{align}

\item Show that the equation
\begin{align}
\vec{x}^T\vec{x}= a^2
\end{align}
remains unaltered by any rotation of the axes.
\item What does the equation
\begin{align}
\vec{x}^T\myvec{1 & \sqrt{3}\\ \sqrt{3} & -1}\vec{x} = 2a^2
\end{align}
become when the axes are turned through $30\degree$?
\item What does the equation
\begin{align}
\vec{x}^T\myvec{1 & -1\\-1 & 1}\vec{x}-4\sqrt{2}a\myvec{1 & 1}\vec{x}=0
\end{align}
become when the axes are turned through $45\degree$?
\item To what point must the origin be moved in order that the equation
\begin{align}
    \vec{x}^T\myvec{1 & 2\\2 & -2}\vec{x}+\myvec{10 & -4}\vec{x}=0 \label{eq:solutions/3/4/6/eq:q1}
\end{align}
may become
\begin{align}
    \vec{x}^T\myvec{1 & 2\\2 & -2}\vec{x}=1\label{eq:solutions/3/4/6/eq:q2}
\end{align}
and through what angle must the axes be turned in order to obtain
\begin{align}
    \vec{x}^T\myvec{p & 0\\0 & q}\vec{x}=1\label{eq:solutions/3/4/6/eq:q3}
\end{align}
\solution
Let the balanced version of (\ref{eq:solutions/chem/6ato balance}) be
\begin{align}
    \label{eq:solutions/chem/6abalanced}x_{1}HNO_{3}+ x_{2}Ca(OH)_{2}\to x_{3}Ca(NO_{3})_{2}+ x_{4}H_{2}O
\end{align}

which results in the following equations:
\begin{align}
    (x_{1}+ 2x_{2}-2x_{4}) H= 0\\
    (x_{1}-2x_{3}) N= 0\\
    (3x_{1}+ 2x_{2}-6x_{3}- x_{4}) O=0\\
    (x_{2}-x_{3}) Ca= 0
\end{align}

which can be expressed as
\begin{align}
    x_{1}+ 2x_{2}+ 0.x_{3} -2x_{4} = 0\\
    x_{1}+ 0.x_{2} -2x_{3} +0.x_{4}= 0\\
    3x_{1}+ 2x_{2}-6x_{3}- x_{4} =0\\
    0.x_{1} +x_{2}-x_{3} +0.x_{4}= 0
\end{align}

resulting in the matrix equation
\begin{align}
    \label{eq:solutions/chem/6a matrix}
    \myvec{1 & 2 & 0 & -2\\
           1 & 0 & -2 & 0\\
           3 & 2 & -6 & -1\\
           0 & 1 & -1 & 0}\vec{x}
           =\vec{0}
\end{align}

where,
\begin{align}
   \vec{x}= \myvec{x_{1}\\x_{2}\\x_{3}\\x_{4}}
\end{align}

(\ref{eq:solutions/chem/6a matrix}) can be reduced as follows:
\begin{align}
    \myvec{1 & 2 & 0 & -2\\
           1 & 0 & -2 & 0\\
           3 & 2 & -6 & -1\\
           0 & 1 & -1 & 0}
    \xleftrightarrow[R_{3}\leftarrow \frac{R_3}{3}-R_{1}]{R_{2}\leftarrow R_2- R_1}
    \myvec{1 & 2 & 0 & -2\\
           0 & -2 & -2 & 2\\
           0 & -\frac{4}{3} & -2 & \frac{5}{3}\\
           0 & 1 & -1 & 0}\\
    \xleftrightarrow{R_2 \leftarrow -\frac{R_2}{2}}
    \myvec{1 & 2 & 0 & -2\\
          0 & 1 & 1 & -1\\
          0 & -\frac{4}{3} & -2 & \frac{5}{3}\\
          0 & 1 & -1 & 0}\\
    \xleftrightarrow[R_4 \leftarrow R_4- R_2]{R_3 \leftarrow R_3 + \frac{4}{3}R_2}
    \myvec{1 & 2 & 0 & -2\\
           0 & 1 & 1 & -1\\
           0 & 0 & -\frac{2}{3} & \frac{1}{3}\\
           0 & 0 & -2 & 1}\\
    \xleftrightarrow[R_3 \leftarrow -\frac{3}{2}R_3]{R_1 \leftarrow R_1- 2R_2}
    \myvec{1 & 0 & -2 & 0\\
           0 & 1 & 1 & -1\\
           0 & 0 & 1 & -\frac{1}{2}\\
           0 & 0 & -2 & 1}\\
    \xleftrightarrow{R_4\leftarrow R_4 + 2R_3}
    \myvec{1 & 0 & -2 & 0\\
           0 & 1 & 1 & -1\\
           0 & 0 & 1 & -\frac{1}{2}\\
           0 & 0 & 0 & 0}\\
    \xleftrightarrow[R_2\leftarrow R_2-R_3]{R_1\leftarrow R_1 + 2R_3}
    \myvec{1 & 0 & 0 & -1\\
           0 & 1 & 0 & -\frac{1}{2}\\
           0 & 0 & 1 & -\frac{1}{2}\\
           0 & 0 & 0 & 0}
\end{align}

Thus,
\begin{align}
    x_1=x_4, x_2= \frac{1}{2}x_4, x_3=\frac{1}{2}x_4\\
    \implies \quad\vec{x}= x_4\myvec{1\\ \frac{1}{2}\\ \frac{1}{2}\\1} =\myvec{2\\1\\1\\2}
\end{align} 
by substituting $x_4= 2$.

\hfill\break
%\vspace{5mm} 
Hence, (\ref{eq:solutions/chem/6abalanced}) finally becomes
\begin{align}
    2HNO_{3}+ Ca(OH)_{2}\to Ca(NO_{3})_{2}+ 2H_{2}O
\end{align}

\item Through what angle must the axes be turned to reduce the equation
\begin{align}
\vec{x}^T\myvec{1 & -1\\-1 & -1}\vec{x}=1
\end{align}
to the form
\begin{align}
\vec{x}^T\myvec{0 & \frac{1}{2}\\ \frac{1}{2} & 0}\vec{x} = c
\end{align}
where $c$ is a constant.
\item Show that, by changing the origin, the equation
\begin{align}
2\vec{x}^T\vec{x}+\myvec{7 & 5}\vec{x} - 13 = 0
\end{align}
can be transformed to 
\begin{align}
8\vec{x}^T\vec{x} = 89
\end{align}
\item Show that, by rotating the axes, the equation
\begin{align}
\vec{x}^T\myvec{3 & \frac{7}{2}\\ \frac{7}{2} & -3}\vec{x}= 1
\end{align}
can be reduced to 
\begin{align}
\sqrt{85}\vec{x}^T\myvec{1 & 0\\ 0 & -1}\vec{x}= 2
\end{align}
\item Show that, by rotating the axes, the equation
\begin{align}
\vec{x}^T\myvec{41 & 12\\ 12 & 34}\vec{x}= 75
\label{eq:solutions/3/4/9/1}
\end{align}
can be reduced to 
\begin{align}
\vec{x}^T\myvec{2 & 0\\ 0 & 1}\vec{x}= 3
\end{align}
\\
\solution
Let the balanced version of (\ref{eq:solutions/chem/6ato balance}) be
\begin{align}
    \label{eq:solutions/chem/6abalanced}x_{1}HNO_{3}+ x_{2}Ca(OH)_{2}\to x_{3}Ca(NO_{3})_{2}+ x_{4}H_{2}O
\end{align}

which results in the following equations:
\begin{align}
    (x_{1}+ 2x_{2}-2x_{4}) H= 0\\
    (x_{1}-2x_{3}) N= 0\\
    (3x_{1}+ 2x_{2}-6x_{3}- x_{4}) O=0\\
    (x_{2}-x_{3}) Ca= 0
\end{align}

which can be expressed as
\begin{align}
    x_{1}+ 2x_{2}+ 0.x_{3} -2x_{4} = 0\\
    x_{1}+ 0.x_{2} -2x_{3} +0.x_{4}= 0\\
    3x_{1}+ 2x_{2}-6x_{3}- x_{4} =0\\
    0.x_{1} +x_{2}-x_{3} +0.x_{4}= 0
\end{align}

resulting in the matrix equation
\begin{align}
    \label{eq:solutions/chem/6a matrix}
    \myvec{1 & 2 & 0 & -2\\
           1 & 0 & -2 & 0\\
           3 & 2 & -6 & -1\\
           0 & 1 & -1 & 0}\vec{x}
           =\vec{0}
\end{align}

where,
\begin{align}
   \vec{x}= \myvec{x_{1}\\x_{2}\\x_{3}\\x_{4}}
\end{align}

(\ref{eq:solutions/chem/6a matrix}) can be reduced as follows:
\begin{align}
    \myvec{1 & 2 & 0 & -2\\
           1 & 0 & -2 & 0\\
           3 & 2 & -6 & -1\\
           0 & 1 & -1 & 0}
    \xleftrightarrow[R_{3}\leftarrow \frac{R_3}{3}-R_{1}]{R_{2}\leftarrow R_2- R_1}
    \myvec{1 & 2 & 0 & -2\\
           0 & -2 & -2 & 2\\
           0 & -\frac{4}{3} & -2 & \frac{5}{3}\\
           0 & 1 & -1 & 0}\\
    \xleftrightarrow{R_2 \leftarrow -\frac{R_2}{2}}
    \myvec{1 & 2 & 0 & -2\\
          0 & 1 & 1 & -1\\
          0 & -\frac{4}{3} & -2 & \frac{5}{3}\\
          0 & 1 & -1 & 0}\\
    \xleftrightarrow[R_4 \leftarrow R_4- R_2]{R_3 \leftarrow R_3 + \frac{4}{3}R_2}
    \myvec{1 & 2 & 0 & -2\\
           0 & 1 & 1 & -1\\
           0 & 0 & -\frac{2}{3} & \frac{1}{3}\\
           0 & 0 & -2 & 1}\\
    \xleftrightarrow[R_3 \leftarrow -\frac{3}{2}R_3]{R_1 \leftarrow R_1- 2R_2}
    \myvec{1 & 0 & -2 & 0\\
           0 & 1 & 1 & -1\\
           0 & 0 & 1 & -\frac{1}{2}\\
           0 & 0 & -2 & 1}\\
    \xleftrightarrow{R_4\leftarrow R_4 + 2R_3}
    \myvec{1 & 0 & -2 & 0\\
           0 & 1 & 1 & -1\\
           0 & 0 & 1 & -\frac{1}{2}\\
           0 & 0 & 0 & 0}\\
    \xleftrightarrow[R_2\leftarrow R_2-R_3]{R_1\leftarrow R_1 + 2R_3}
    \myvec{1 & 0 & 0 & -1\\
           0 & 1 & 0 & -\frac{1}{2}\\
           0 & 0 & 1 & -\frac{1}{2}\\
           0 & 0 & 0 & 0}
\end{align}

Thus,
\begin{align}
    x_1=x_4, x_2= \frac{1}{2}x_4, x_3=\frac{1}{2}x_4\\
    \implies \quad\vec{x}= x_4\myvec{1\\ \frac{1}{2}\\ \frac{1}{2}\\1} =\myvec{2\\1\\1\\2}
\end{align} 
by substituting $x_4= 2$.

\hfill\break
%\vspace{5mm} 
Hence, (\ref{eq:solutions/chem/6abalanced}) finally becomes
\begin{align}
    2HNO_{3}+ Ca(OH)_{2}\to Ca(NO_{3})_{2}+ 2H_{2}O
\end{align}

\item Show that, by a change of origin and the directions of the coordinate axes, the equation
\begin{align}
\vec{x}^T\myvec{5 & 1\\ 1 & 5}\vec{x}-\myvec{14 & 22}\vec{x}+27= 0
\end{align}
can be transformed to
\begin{align}
\vec{x}^T\myvec{3 & 0\\ 0 & 2}\vec{x}= 1
\end{align}
or
\begin{align}
\vec{x}^T\myvec{2 & 0\\ 0 & 3}\vec{x}= 1
\end{align}
\end{enumerate}
