\renewcommand{\theequation}{\theenumi}
\begin{enumerate}[label=\arabic*.,ref=\thesubsection.\theenumi]
\item Find the locus of a point which is equidistant from the points $\myvec{6\\-3}$, $\myvec{-4\\7}$.
\item Find the point on the line
\begin{align}
\myvec{2 & 5}\vec{x}+7=0
\end{align}
which is equidistant from the points $\myvec{2\\-3}$, $\myvec{-4\\1}$.
\item Find the coordinates of the circumcentre of the triangle whose corners are at the points $\myvec{4\\3}$, $\myvec{-1\\2}$, $\myvec{2\\-2}$.
\item Find the equations of the lines through $\myvec{3\\1}$ which are respectively parallel and perpendicular to the line joining the points $\myvec{2\\4}$, $\myvec{5\\-6}$.
\item Find the locus of a point at which the join of the points $\myvec{2\\1}$ and $\myvec{-3\\4}$ subtends a right angle.
\item Find the orthocentre of a triangle whose corners are at the points $\myvec{1\\2}$, $\myvec{-3\\-4}$, $\myvec{6\\2}$.
\item Prove that the line joining the points $\myvec{2\\-1}$, $\myvec{-3\\5}$ makes with the axes a triangle of area $\frac{49}{60}$.
\item $ABCD$ is a parallelogram and the coordinates of $\vec{A}$, $\vec{B}$ and $\vec{C}$ are $\myvec{2\\4}$, $\myvec{1\\2}$ and $\myvec{4\\1}$ Find the coordinates of $\vec{D}$.
\item Find the area of the triangle formed by the lines 
\numberwithin{equation}{enumi}
\begin{align}
\myvec{3&-2}\vec{x}=5
\\ 
\myvec{3 &4}\vec{x}=7
\\
\myvec{0 & 1}\vec{x}  +2 = 0
\end{align}
\item Find the centre of the inscribed circle of the triangle whose sides are
\begin{align}
\myvec{3&-4}\vec{x}&=0
\\
\myvec{ 12&-5}\vec{x}&=0,
\\ 
\myvec{4&3}\vec{x}&=8
\end{align}
\renewcommand{\theequation}{\theenumi}
\item The ends of a diagonal of a square are on the coordinate axes at the points $\myvec{2a\\0}$, $\myvec{0\\a}$.  Find the equations of the sides.
\item The sides of a triangle $ABC$ are
\begin{align}
AB=3, BC = 5, CA = 4
\end{align}
and $\vec{A}, \vec{B}$ are on the axes $OX$, $OY$ respectively, while $AC$ makes an angle $\theta$ with $OX$.  Prove that the locus of $\vec{C}$, as $\theta$ varies, is given by the equation
\begin{align}
\vec{x}^T\myvec{16 & -12\\-12 & 25}\vec{x} = 256
%16x^2-24xy+25y^2 = 256
\end{align}
\item Prove that the locus of a point at which the join of the points $\myvec{a\\0}$ and $\myvec{-a\\0}$ subtends an angle of $45\degree$ is
\begin{align}
\vec{x}^T\vec{x}-2a\myvec{0 & 1}\vec{x} = a^2
%x^2+y^2-2ay = a^2
\end{align}
\numberwithin{equation}{enumi}
\item Prove that the line
\begin{align}
\vec{n}^T\vec{x}+c=0
\end{align}
divides the line joining  the points $\vec{x}_1, \vec{x}_2$ in the ratio
\begin{align}
-\frac{\vec{n}^T\vec{x}_1+c}{\vec{n}^T\vec{x}_2+c}
\end{align}
\item Find the equation of the line joining the point $\vec{x}_1$, to the point of intersection of the lines
\begin{align}
\vec{n}^T\vec{x}+c=0
\\
\vec{n}_1^T\vec{x}+c_1=0
\end{align}
\item Find  the equations of the diagonals of the parallelogram whose sides are
\begin{align}
\vec{n}^T\vec{x}+c&=0
\\
\vec{n}^T\vec{x}+d&=0
\\
\vec{n}_1^T\vec{x}+c_1&=0
\\
\vec{n}_1^T\vec{x}+d_1&=0
\end{align}
%are
%{\small
%\begin{align}
%\brak{c_1-d_1}\brak{ax+by+c}-\brak{c-d}\brak{a_1x+b_1y+c_1} = 0
%\\
%\brak{c_1-d_1}\brak{ax+by+c}+\brak{c-d}\brak{a_1x+b_1y+c_1} = 0
%\\
%\brak{c-d}\brak{c_1-d_1}/\brak{ab_1-a_1b}.
%\end{align}
%}
\renewcommand{\theequation}{\theenumi}
\item Prove that for all values of $k$ the line
\begin{align}
\myvec{2+k &1-2k}\vec{x} + 5 = 0
\end{align}
passes through a fixed point, and find its coordinates.
\item Find the angle between the lines
\begin{align}
\vec{x}^T
\myvec{1 & -\sec \theta\\-\sec\theta & 1} 
\vec{x} = 0
\end{align}
\numberwithin{equation}{enumi}
\item Prove that the pairs of straight lines represented by
\begin{align}
\vec{x}^T
\myvec{1 & \frac{1}{2}\\\frac{1}{2} & 0} 
\vec{x} = 0
\\
\vec{x}^T
\myvec{6 & -\frac{1}{2}\\-\frac{1}{2} & -1} 
\vec{x} = 0
\end{align}
are such that the angles between one pair are equal to the angles between the other pair.
\renewcommand{\theequation}{\theenumi}
\item Find the angles between the lines
\begin{align}
x^3-3x^2y-3xy^2+y^3 = 0
\end{align}
\numberwithin{equation}{enumi}
\item Find the area of the triangle whose sides are given by
\begin{align}
\vec{x}^T
\myvec{1 & -{2}\\-{2} & 3} 
\vec{x} = 0
\\
\myvec{3&4}\vec{x}=7
\end{align}
\renewcommand{\theequation}{\theenumi}
\item Show that the equation 
\begin{align}
    \vec{x}^T\myvec{6&-\frac{1}{2}\\-\frac{1}{2}&-15}\vec{x}+\myvec{-11& 31}\vec{x}-10=0\label{eq:solutions/2/6/22/eq:1}
\end{align}
represents two straight lines,and find the equations of the bisectors of the angles between them.
%
%\item Show that the equation
%\begin{align}
%\vec{x}^T
%\myvec{6 & -\frac{1}{2}\\-\frac{1}{2} & -15} 
%\vec{x} 
%+\myvec{-11&31}\vec{x}-10=0
%\end{align}
%represents two straight lines, and find the equations of the bisectors of the angles between them.
\\
\solution
Let the balanced version of (\ref{eq:solutions/chem/6ato balance}) be
\begin{align}
    \label{eq:solutions/chem/6abalanced}x_{1}HNO_{3}+ x_{2}Ca(OH)_{2}\to x_{3}Ca(NO_{3})_{2}+ x_{4}H_{2}O
\end{align}

which results in the following equations:
\begin{align}
    (x_{1}+ 2x_{2}-2x_{4}) H= 0\\
    (x_{1}-2x_{3}) N= 0\\
    (3x_{1}+ 2x_{2}-6x_{3}- x_{4}) O=0\\
    (x_{2}-x_{3}) Ca= 0
\end{align}

which can be expressed as
\begin{align}
    x_{1}+ 2x_{2}+ 0.x_{3} -2x_{4} = 0\\
    x_{1}+ 0.x_{2} -2x_{3} +0.x_{4}= 0\\
    3x_{1}+ 2x_{2}-6x_{3}- x_{4} =0\\
    0.x_{1} +x_{2}-x_{3} +0.x_{4}= 0
\end{align}

resulting in the matrix equation
\begin{align}
    \label{eq:solutions/chem/6a matrix}
    \myvec{1 & 2 & 0 & -2\\
           1 & 0 & -2 & 0\\
           3 & 2 & -6 & -1\\
           0 & 1 & -1 & 0}\vec{x}
           =\vec{0}
\end{align}

where,
\begin{align}
   \vec{x}= \myvec{x_{1}\\x_{2}\\x_{3}\\x_{4}}
\end{align}

(\ref{eq:solutions/chem/6a matrix}) can be reduced as follows:
\begin{align}
    \myvec{1 & 2 & 0 & -2\\
           1 & 0 & -2 & 0\\
           3 & 2 & -6 & -1\\
           0 & 1 & -1 & 0}
    \xleftrightarrow[R_{3}\leftarrow \frac{R_3}{3}-R_{1}]{R_{2}\leftarrow R_2- R_1}
    \myvec{1 & 2 & 0 & -2\\
           0 & -2 & -2 & 2\\
           0 & -\frac{4}{3} & -2 & \frac{5}{3}\\
           0 & 1 & -1 & 0}\\
    \xleftrightarrow{R_2 \leftarrow -\frac{R_2}{2}}
    \myvec{1 & 2 & 0 & -2\\
          0 & 1 & 1 & -1\\
          0 & -\frac{4}{3} & -2 & \frac{5}{3}\\
          0 & 1 & -1 & 0}\\
    \xleftrightarrow[R_4 \leftarrow R_4- R_2]{R_3 \leftarrow R_3 + \frac{4}{3}R_2}
    \myvec{1 & 2 & 0 & -2\\
           0 & 1 & 1 & -1\\
           0 & 0 & -\frac{2}{3} & \frac{1}{3}\\
           0 & 0 & -2 & 1}\\
    \xleftrightarrow[R_3 \leftarrow -\frac{3}{2}R_3]{R_1 \leftarrow R_1- 2R_2}
    \myvec{1 & 0 & -2 & 0\\
           0 & 1 & 1 & -1\\
           0 & 0 & 1 & -\frac{1}{2}\\
           0 & 0 & -2 & 1}\\
    \xleftrightarrow{R_4\leftarrow R_4 + 2R_3}
    \myvec{1 & 0 & -2 & 0\\
           0 & 1 & 1 & -1\\
           0 & 0 & 1 & -\frac{1}{2}\\
           0 & 0 & 0 & 0}\\
    \xleftrightarrow[R_2\leftarrow R_2-R_3]{R_1\leftarrow R_1 + 2R_3}
    \myvec{1 & 0 & 0 & -1\\
           0 & 1 & 0 & -\frac{1}{2}\\
           0 & 0 & 1 & -\frac{1}{2}\\
           0 & 0 & 0 & 0}
\end{align}

Thus,
\begin{align}
    x_1=x_4, x_2= \frac{1}{2}x_4, x_3=\frac{1}{2}x_4\\
    \implies \quad\vec{x}= x_4\myvec{1\\ \frac{1}{2}\\ \frac{1}{2}\\1} =\myvec{2\\1\\1\\2}
\end{align} 
by substituting $x_4= 2$.

\hfill\break
%\vspace{5mm} 
Hence, (\ref{eq:solutions/chem/6abalanced}) finally becomes
\begin{align}
    2HNO_{3}+ Ca(OH)_{2}\to Ca(NO_{3})_{2}+ 2H_{2}O
\end{align}

\item For what value of $k$ does the equation
\begin{align}
\vec{x}^T
\myvec{12 & \frac{7}{2}\\\frac{7}{2} & k} 
\vec{x} 
+\myvec{13&-1}\vec{x}+3=0
\end{align}
represent two straight lines? What is the angle between them?

\item For what value of $k$ does the equation 
\begin{equation} \label{eq:solutions/2/6/24/eq:1.1}
\vec{x}^T \myvec{6 && k/2 \\ k/2 && -3} \vec{x} + \myvec{4 && 5}\vec{x} -2 = 0
\end{equation}
represent a pair of straight lines?
\\
%\item For what values of $k$ does the equation
%\begin{align}
%\vec{x}^T
%\myvec{6 & \frac{k}{2}\\\frac{k}{2} & -3} 
%\vec{x} 
%+\myvec{4&5}\vec{x}-2=0
%\end{align}
%represent two straight lines?
\solution
Let the balanced version of (\ref{eq:solutions/chem/6ato balance}) be
\begin{align}
    \label{eq:solutions/chem/6abalanced}x_{1}HNO_{3}+ x_{2}Ca(OH)_{2}\to x_{3}Ca(NO_{3})_{2}+ x_{4}H_{2}O
\end{align}

which results in the following equations:
\begin{align}
    (x_{1}+ 2x_{2}-2x_{4}) H= 0\\
    (x_{1}-2x_{3}) N= 0\\
    (3x_{1}+ 2x_{2}-6x_{3}- x_{4}) O=0\\
    (x_{2}-x_{3}) Ca= 0
\end{align}

which can be expressed as
\begin{align}
    x_{1}+ 2x_{2}+ 0.x_{3} -2x_{4} = 0\\
    x_{1}+ 0.x_{2} -2x_{3} +0.x_{4}= 0\\
    3x_{1}+ 2x_{2}-6x_{3}- x_{4} =0\\
    0.x_{1} +x_{2}-x_{3} +0.x_{4}= 0
\end{align}

resulting in the matrix equation
\begin{align}
    \label{eq:solutions/chem/6a matrix}
    \myvec{1 & 2 & 0 & -2\\
           1 & 0 & -2 & 0\\
           3 & 2 & -6 & -1\\
           0 & 1 & -1 & 0}\vec{x}
           =\vec{0}
\end{align}

where,
\begin{align}
   \vec{x}= \myvec{x_{1}\\x_{2}\\x_{3}\\x_{4}}
\end{align}

(\ref{eq:solutions/chem/6a matrix}) can be reduced as follows:
\begin{align}
    \myvec{1 & 2 & 0 & -2\\
           1 & 0 & -2 & 0\\
           3 & 2 & -6 & -1\\
           0 & 1 & -1 & 0}
    \xleftrightarrow[R_{3}\leftarrow \frac{R_3}{3}-R_{1}]{R_{2}\leftarrow R_2- R_1}
    \myvec{1 & 2 & 0 & -2\\
           0 & -2 & -2 & 2\\
           0 & -\frac{4}{3} & -2 & \frac{5}{3}\\
           0 & 1 & -1 & 0}\\
    \xleftrightarrow{R_2 \leftarrow -\frac{R_2}{2}}
    \myvec{1 & 2 & 0 & -2\\
          0 & 1 & 1 & -1\\
          0 & -\frac{4}{3} & -2 & \frac{5}{3}\\
          0 & 1 & -1 & 0}\\
    \xleftrightarrow[R_4 \leftarrow R_4- R_2]{R_3 \leftarrow R_3 + \frac{4}{3}R_2}
    \myvec{1 & 2 & 0 & -2\\
           0 & 1 & 1 & -1\\
           0 & 0 & -\frac{2}{3} & \frac{1}{3}\\
           0 & 0 & -2 & 1}\\
    \xleftrightarrow[R_3 \leftarrow -\frac{3}{2}R_3]{R_1 \leftarrow R_1- 2R_2}
    \myvec{1 & 0 & -2 & 0\\
           0 & 1 & 1 & -1\\
           0 & 0 & 1 & -\frac{1}{2}\\
           0 & 0 & -2 & 1}\\
    \xleftrightarrow{R_4\leftarrow R_4 + 2R_3}
    \myvec{1 & 0 & -2 & 0\\
           0 & 1 & 1 & -1\\
           0 & 0 & 1 & -\frac{1}{2}\\
           0 & 0 & 0 & 0}\\
    \xleftrightarrow[R_2\leftarrow R_2-R_3]{R_1\leftarrow R_1 + 2R_3}
    \myvec{1 & 0 & 0 & -1\\
           0 & 1 & 0 & -\frac{1}{2}\\
           0 & 0 & 1 & -\frac{1}{2}\\
           0 & 0 & 0 & 0}
\end{align}

Thus,
\begin{align}
    x_1=x_4, x_2= \frac{1}{2}x_4, x_3=\frac{1}{2}x_4\\
    \implies \quad\vec{x}= x_4\myvec{1\\ \frac{1}{2}\\ \frac{1}{2}\\1} =\myvec{2\\1\\1\\2}
\end{align} 
by substituting $x_4= 2$.

\hfill\break
%\vspace{5mm} 
Hence, (\ref{eq:solutions/chem/6abalanced}) finally becomes
\begin{align}
    2HNO_{3}+ Ca(OH)_{2}\to Ca(NO_{3})_{2}+ 2H_{2}O
\end{align}

%
\end{enumerate}
