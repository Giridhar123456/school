\renewcommand{\theequation}{\theenumi}
\begin{enumerate}[label=\arabic*.,ref=\thesubsection.\theenumi]
\numberwithin{equation}{enumi}
\item Without drawing a figure, determine whether the points $\myvec{-1\\2}$, $\myvec{0\\0}$, $\myvec{3\\-4}$ lie outside, on the circumference, or inside the circle
\begin{align}
\vec{x}^T\vec{x}+\myvec{-5 & 2}\vec{x}-5 = 0
\end{align}
\solution
Let the balanced version of (\ref{eq:solutions/chem/6ato balance}) be
\begin{align}
    \label{eq:solutions/chem/6abalanced}x_{1}HNO_{3}+ x_{2}Ca(OH)_{2}\to x_{3}Ca(NO_{3})_{2}+ x_{4}H_{2}O
\end{align}

which results in the following equations:
\begin{align}
    (x_{1}+ 2x_{2}-2x_{4}) H= 0\\
    (x_{1}-2x_{3}) N= 0\\
    (3x_{1}+ 2x_{2}-6x_{3}- x_{4}) O=0\\
    (x_{2}-x_{3}) Ca= 0
\end{align}

which can be expressed as
\begin{align}
    x_{1}+ 2x_{2}+ 0.x_{3} -2x_{4} = 0\\
    x_{1}+ 0.x_{2} -2x_{3} +0.x_{4}= 0\\
    3x_{1}+ 2x_{2}-6x_{3}- x_{4} =0\\
    0.x_{1} +x_{2}-x_{3} +0.x_{4}= 0
\end{align}

resulting in the matrix equation
\begin{align}
    \label{eq:solutions/chem/6a matrix}
    \myvec{1 & 2 & 0 & -2\\
           1 & 0 & -2 & 0\\
           3 & 2 & -6 & -1\\
           0 & 1 & -1 & 0}\vec{x}
           =\vec{0}
\end{align}

where,
\begin{align}
   \vec{x}= \myvec{x_{1}\\x_{2}\\x_{3}\\x_{4}}
\end{align}

(\ref{eq:solutions/chem/6a matrix}) can be reduced as follows:
\begin{align}
    \myvec{1 & 2 & 0 & -2\\
           1 & 0 & -2 & 0\\
           3 & 2 & -6 & -1\\
           0 & 1 & -1 & 0}
    \xleftrightarrow[R_{3}\leftarrow \frac{R_3}{3}-R_{1}]{R_{2}\leftarrow R_2- R_1}
    \myvec{1 & 2 & 0 & -2\\
           0 & -2 & -2 & 2\\
           0 & -\frac{4}{3} & -2 & \frac{5}{3}\\
           0 & 1 & -1 & 0}\\
    \xleftrightarrow{R_2 \leftarrow -\frac{R_2}{2}}
    \myvec{1 & 2 & 0 & -2\\
          0 & 1 & 1 & -1\\
          0 & -\frac{4}{3} & -2 & \frac{5}{3}\\
          0 & 1 & -1 & 0}\\
    \xleftrightarrow[R_4 \leftarrow R_4- R_2]{R_3 \leftarrow R_3 + \frac{4}{3}R_2}
    \myvec{1 & 2 & 0 & -2\\
           0 & 1 & 1 & -1\\
           0 & 0 & -\frac{2}{3} & \frac{1}{3}\\
           0 & 0 & -2 & 1}\\
    \xleftrightarrow[R_3 \leftarrow -\frac{3}{2}R_3]{R_1 \leftarrow R_1- 2R_2}
    \myvec{1 & 0 & -2 & 0\\
           0 & 1 & 1 & -1\\
           0 & 0 & 1 & -\frac{1}{2}\\
           0 & 0 & -2 & 1}\\
    \xleftrightarrow{R_4\leftarrow R_4 + 2R_3}
    \myvec{1 & 0 & -2 & 0\\
           0 & 1 & 1 & -1\\
           0 & 0 & 1 & -\frac{1}{2}\\
           0 & 0 & 0 & 0}\\
    \xleftrightarrow[R_2\leftarrow R_2-R_3]{R_1\leftarrow R_1 + 2R_3}
    \myvec{1 & 0 & 0 & -1\\
           0 & 1 & 0 & -\frac{1}{2}\\
           0 & 0 & 1 & -\frac{1}{2}\\
           0 & 0 & 0 & 0}
\end{align}

Thus,
\begin{align}
    x_1=x_4, x_2= \frac{1}{2}x_4, x_3=\frac{1}{2}x_4\\
    \implies \quad\vec{x}= x_4\myvec{1\\ \frac{1}{2}\\ \frac{1}{2}\\1} =\myvec{2\\1\\1\\2}
\end{align} 
by substituting $x_4= 2$.

\hfill\break
%\vspace{5mm} 
Hence, (\ref{eq:solutions/chem/6abalanced}) finally becomes
\begin{align}
    2HNO_{3}+ Ca(OH)_{2}\to Ca(NO_{3})_{2}+ 2H_{2}O
\end{align}

\item Find the points of intersection of the line 
\begin{align}
\myvec{3 &2}\vec{x}=12
\end{align}
and the circle 
\begin{align}
\norm{\vec{x}}^2=13
\end{align}
 and find for what values of $c$ the line 
\begin{align}
\myvec{3 &2}\vec{x}=c
\end{align}
touches the circle.

\item Prove that the line 
\begin{align}
\myvec{3 &2}\vec{x}=30 \label{eq:solutions/4/2/3/given_line_eq}
\end{align}
touches the circle
\begin{align}
\vec{x}^T\vec{x}-\myvec{10 & 2}\vec{x}+13 = 0 \label{eq:solutions/4/2/3/given_circle_eq}
\end{align}
and find the coordinates of the point of contact.
%
\solution
Let the balanced version of (\ref{eq:solutions/chem/6ato balance}) be
\begin{align}
    \label{eq:solutions/chem/6abalanced}x_{1}HNO_{3}+ x_{2}Ca(OH)_{2}\to x_{3}Ca(NO_{3})_{2}+ x_{4}H_{2}O
\end{align}

which results in the following equations:
\begin{align}
    (x_{1}+ 2x_{2}-2x_{4}) H= 0\\
    (x_{1}-2x_{3}) N= 0\\
    (3x_{1}+ 2x_{2}-6x_{3}- x_{4}) O=0\\
    (x_{2}-x_{3}) Ca= 0
\end{align}

which can be expressed as
\begin{align}
    x_{1}+ 2x_{2}+ 0.x_{3} -2x_{4} = 0\\
    x_{1}+ 0.x_{2} -2x_{3} +0.x_{4}= 0\\
    3x_{1}+ 2x_{2}-6x_{3}- x_{4} =0\\
    0.x_{1} +x_{2}-x_{3} +0.x_{4}= 0
\end{align}

resulting in the matrix equation
\begin{align}
    \label{eq:solutions/chem/6a matrix}
    \myvec{1 & 2 & 0 & -2\\
           1 & 0 & -2 & 0\\
           3 & 2 & -6 & -1\\
           0 & 1 & -1 & 0}\vec{x}
           =\vec{0}
\end{align}

where,
\begin{align}
   \vec{x}= \myvec{x_{1}\\x_{2}\\x_{3}\\x_{4}}
\end{align}

(\ref{eq:solutions/chem/6a matrix}) can be reduced as follows:
\begin{align}
    \myvec{1 & 2 & 0 & -2\\
           1 & 0 & -2 & 0\\
           3 & 2 & -6 & -1\\
           0 & 1 & -1 & 0}
    \xleftrightarrow[R_{3}\leftarrow \frac{R_3}{3}-R_{1}]{R_{2}\leftarrow R_2- R_1}
    \myvec{1 & 2 & 0 & -2\\
           0 & -2 & -2 & 2\\
           0 & -\frac{4}{3} & -2 & \frac{5}{3}\\
           0 & 1 & -1 & 0}\\
    \xleftrightarrow{R_2 \leftarrow -\frac{R_2}{2}}
    \myvec{1 & 2 & 0 & -2\\
          0 & 1 & 1 & -1\\
          0 & -\frac{4}{3} & -2 & \frac{5}{3}\\
          0 & 1 & -1 & 0}\\
    \xleftrightarrow[R_4 \leftarrow R_4- R_2]{R_3 \leftarrow R_3 + \frac{4}{3}R_2}
    \myvec{1 & 2 & 0 & -2\\
           0 & 1 & 1 & -1\\
           0 & 0 & -\frac{2}{3} & \frac{1}{3}\\
           0 & 0 & -2 & 1}\\
    \xleftrightarrow[R_3 \leftarrow -\frac{3}{2}R_3]{R_1 \leftarrow R_1- 2R_2}
    \myvec{1 & 0 & -2 & 0\\
           0 & 1 & 1 & -1\\
           0 & 0 & 1 & -\frac{1}{2}\\
           0 & 0 & -2 & 1}\\
    \xleftrightarrow{R_4\leftarrow R_4 + 2R_3}
    \myvec{1 & 0 & -2 & 0\\
           0 & 1 & 1 & -1\\
           0 & 0 & 1 & -\frac{1}{2}\\
           0 & 0 & 0 & 0}\\
    \xleftrightarrow[R_2\leftarrow R_2-R_3]{R_1\leftarrow R_1 + 2R_3}
    \myvec{1 & 0 & 0 & -1\\
           0 & 1 & 0 & -\frac{1}{2}\\
           0 & 0 & 1 & -\frac{1}{2}\\
           0 & 0 & 0 & 0}
\end{align}

Thus,
\begin{align}
    x_1=x_4, x_2= \frac{1}{2}x_4, x_3=\frac{1}{2}x_4\\
    \implies \quad\vec{x}= x_4\myvec{1\\ \frac{1}{2}\\ \frac{1}{2}\\1} =\myvec{2\\1\\1\\2}
\end{align} 
by substituting $x_4= 2$.

\hfill\break
%\vspace{5mm} 
Hence, (\ref{eq:solutions/chem/6abalanced}) finally becomes
\begin{align}
    2HNO_{3}+ Ca(OH)_{2}\to Ca(NO_{3})_{2}+ 2H_{2}O
\end{align}

\item For what values of $m$ does the line 
\begin{align}
\myvec{m &-1}\vec{x}=0
\label{eq:solutions/4/2/4/1.1}
\end{align}
touch the circle
\begin{align}
\vec{x}^T\vec{x}-\myvec{6 & 2}\vec{x}+8 = 0 \label{eq:solutions/4/2/4/1.2}
\end{align}
\solution
Let the balanced version of (\ref{eq:solutions/chem/6ato balance}) be
\begin{align}
    \label{eq:solutions/chem/6abalanced}x_{1}HNO_{3}+ x_{2}Ca(OH)_{2}\to x_{3}Ca(NO_{3})_{2}+ x_{4}H_{2}O
\end{align}

which results in the following equations:
\begin{align}
    (x_{1}+ 2x_{2}-2x_{4}) H= 0\\
    (x_{1}-2x_{3}) N= 0\\
    (3x_{1}+ 2x_{2}-6x_{3}- x_{4}) O=0\\
    (x_{2}-x_{3}) Ca= 0
\end{align}

which can be expressed as
\begin{align}
    x_{1}+ 2x_{2}+ 0.x_{3} -2x_{4} = 0\\
    x_{1}+ 0.x_{2} -2x_{3} +0.x_{4}= 0\\
    3x_{1}+ 2x_{2}-6x_{3}- x_{4} =0\\
    0.x_{1} +x_{2}-x_{3} +0.x_{4}= 0
\end{align}

resulting in the matrix equation
\begin{align}
    \label{eq:solutions/chem/6a matrix}
    \myvec{1 & 2 & 0 & -2\\
           1 & 0 & -2 & 0\\
           3 & 2 & -6 & -1\\
           0 & 1 & -1 & 0}\vec{x}
           =\vec{0}
\end{align}

where,
\begin{align}
   \vec{x}= \myvec{x_{1}\\x_{2}\\x_{3}\\x_{4}}
\end{align}

(\ref{eq:solutions/chem/6a matrix}) can be reduced as follows:
\begin{align}
    \myvec{1 & 2 & 0 & -2\\
           1 & 0 & -2 & 0\\
           3 & 2 & -6 & -1\\
           0 & 1 & -1 & 0}
    \xleftrightarrow[R_{3}\leftarrow \frac{R_3}{3}-R_{1}]{R_{2}\leftarrow R_2- R_1}
    \myvec{1 & 2 & 0 & -2\\
           0 & -2 & -2 & 2\\
           0 & -\frac{4}{3} & -2 & \frac{5}{3}\\
           0 & 1 & -1 & 0}\\
    \xleftrightarrow{R_2 \leftarrow -\frac{R_2}{2}}
    \myvec{1 & 2 & 0 & -2\\
          0 & 1 & 1 & -1\\
          0 & -\frac{4}{3} & -2 & \frac{5}{3}\\
          0 & 1 & -1 & 0}\\
    \xleftrightarrow[R_4 \leftarrow R_4- R_2]{R_3 \leftarrow R_3 + \frac{4}{3}R_2}
    \myvec{1 & 2 & 0 & -2\\
           0 & 1 & 1 & -1\\
           0 & 0 & -\frac{2}{3} & \frac{1}{3}\\
           0 & 0 & -2 & 1}\\
    \xleftrightarrow[R_3 \leftarrow -\frac{3}{2}R_3]{R_1 \leftarrow R_1- 2R_2}
    \myvec{1 & 0 & -2 & 0\\
           0 & 1 & 1 & -1\\
           0 & 0 & 1 & -\frac{1}{2}\\
           0 & 0 & -2 & 1}\\
    \xleftrightarrow{R_4\leftarrow R_4 + 2R_3}
    \myvec{1 & 0 & -2 & 0\\
           0 & 1 & 1 & -1\\
           0 & 0 & 1 & -\frac{1}{2}\\
           0 & 0 & 0 & 0}\\
    \xleftrightarrow[R_2\leftarrow R_2-R_3]{R_1\leftarrow R_1 + 2R_3}
    \myvec{1 & 0 & 0 & -1\\
           0 & 1 & 0 & -\frac{1}{2}\\
           0 & 0 & 1 & -\frac{1}{2}\\
           0 & 0 & 0 & 0}
\end{align}

Thus,
\begin{align}
    x_1=x_4, x_2= \frac{1}{2}x_4, x_3=\frac{1}{2}x_4\\
    \implies \quad\vec{x}= x_4\myvec{1\\ \frac{1}{2}\\ \frac{1}{2}\\1} =\myvec{2\\1\\1\\2}
\end{align} 
by substituting $x_4= 2$.

\hfill\break
%\vspace{5mm} 
Hence, (\ref{eq:solutions/chem/6abalanced}) finally becomes
\begin{align}
    2HNO_{3}+ Ca(OH)_{2}\to Ca(NO_{3})_{2}+ 2H_{2}O
\end{align}

\renewcommand{\theequation}{\theenumi}
\item Prove that the circle 
\begin{align}
\vec{x}^T\vec{x}-2a\myvec{1 & 1}\vec{x}+a^2 = 0
\end{align}
touches the coordinate axes.
\item Show that two circles can be drawn to pass through the point $\myvec{1\\2}$ and touch the coordinate axes, and find their equations.
\item Find the length of the tangent from the point $\myvec{7\\4}$ to the circle 
\begin{align}
\vec{x}^T\vec{x}-\myvec{4 & 6}\vec{x}+12 = 0
\end{align}
\numberwithin{equation}{enumi}
\item  Find the equations of the tangents to the circle
\begin{align}
\vec{x}^T\vec{x}-\myvec{4 & 3}\vec{x}+5 = 0
\end{align}
that are parallel to the line 
\begin{align}
\myvec{1 & 1}\vec{x}=0
\end{align}
\renewcommand{\theequation}{\theenumi}
\item Find the equations of the tangents to the circle
\begin{align}
\vec{x}^T\vec{x}-\myvec{7 & 5}\vec{x}+18 = 0
\end{align}
at the points $\myvec{4\\3}$ and $\myvec{3\\2}$, showing that they are parallel.
\item Find the equations of the tangent and normal to the circle
\begin{align}
\vec{x}^T\vec{x}+\myvec{-6 & 4}\vec{x}-12 = 0
\end{align}
at the point $\myvec{6\\2}$.
\numberwithin{equation}{enumi}
\item Prove that the line 
\begin{align}
\myvec{1 & 1}\vec{x} = 1
\end{align}
touches the circle
\begin{align}
\vec{x}^T\vec{x}-\myvec{8 & 6}\vec{x}+7 = 0
\end{align}
and find the equations of the parallel and perpendicular tangents.
\renewcommand{\theequation}{\theenumi}
\item Find the equation of the tangent at the origin to the circle
\begin{align}
\vec{x}^T\vec{x}+2\myvec{g & f}\vec{x} = 0
\end{align}
\numberwithin{equation}{enumi}
%
\item Prove that the line 
\begin{align}
\myvec{\cos\alpha &\sin\alpha}\vec{x} = p
\end{align}
touches the circle
\begin{align}
\norm{\vec{x}-\myvec{a\\b}}=r
\end{align}
if
\begin{align}
r = \pm \brak{p-a\cos\alpha-b\sin\alpha}
\end{align}
\renewcommand{\theequation}{\theenumi}
\item Find the points of contact of the tangents to the circle
\begin{align}
\norm{\vec{x}}=5
\end{align}
that pass through the point $\myvec{7\\1}$ and write down the equations of the tangents.
\numberwithin{equation}{enumi}

\item Prove that the tangent to the circle 
\begin{align}
\norm{\vec{x}}^2=5
\end{align}
at the point $\myvec{1\\-2}$ also touches the circle
\begin{align}
\vec{x}^T\vec{x}+\myvec{-8 & 6}\vec{x}+20 = 0
\end{align}
and find the coordinates of the point of contact.
\item Find the equations of the circles that touch the lines
\begin{align}
\myvec{0 & 1}\vec{x} &= 0
\\
\myvec{0 & 1}\vec{x} &= 4
\\
\myvec{2 & 1}\vec{x} &= 2
\end{align}
\item Find the coordinates of the middle point of the chord
\begin{align}
\myvec{1 & 7}\vec{x} &= 25
\end{align}
of the circle
\begin{align}
\norm{\vec{x}}=5
\end{align}
\renewcommand{\theequation}{\theenumi}
\item Find the equation of the chord of the circle
\begin{align}
\vec{x}^T\vec{x}-\myvec{6 & 4}\vec{x}-23 = 0
\end{align}
which has the point $\myvec{4\\1}$ as its middle point.
\item Prove that the circle
\begin{align}
\vec{x}^T\vec{x}-\myvec{6 & 4}\vec{x}+9 = 0
\end{align}
subtends an angle $\tan^{-1}\frac{12}{5}$ at the origin.
\numberwithin{equation}{enumi}
\item Find the condition that the line
\begin{align}
\myvec{l & m}\vec{x} + n &= 0
\end{align}
should touch the circle
\begin{align}
\norm{\vec{x}-\myvec{a\\b}}=r
\end{align}
\renewcommand{\theequation}{\theenumi}
\item Verify that the perpendicular bisector of the chord joining two points $\vec{x}_1, \vec{x}_2$ on the circle
\begin{align}
\vec{x}^T\vec{x}+2\myvec{g & f}\vec{x} + c = 0
\end{align}
passes through the centre.
\end{enumerate}
