\renewcommand{\theequation}{\theenumi}
\begin{enumerate}[label=\arabic*.,ref=\thesubsection.\theenumi]
\item Without drawing a figure, determine whether the points $\myvec{-1\\2}$, $\myvec{0\\0}$, $\myvec{3\\-4}$ lie outside, on the circumference, or inside the circle
\begin{align}
\vec{x}^T\vec{x}+\myvec{-5 & 2}\vec{x}-5 = 0
\end{align}
\numberwithin{equation}{enumi}
\item Find the points of intersection of the line 
\begin{align}
\myvec{3 &2}\vec{x}=12
\end{align}
and the circle 
\begin{align}
\norm{\vec{x}}^2=13
\end{align}
 and find for what values of $c$ the line 
\begin{align}
\myvec{3 &2}\vec{x}=c
\end{align}
touches the circle.
\item Prove that the line 
\begin{align}
\myvec{3 &2}\vec{x}=30
\end{align}
touches the circle
\begin{align}
\vec{x}^T\vec{x}-\myvec{10 & 2}\vec{x}+13 = 0
\end{align}
and find the coordinates of the point of contact.
\item For what values of $m$ does the line 
\begin{align}
\myvec{m &-1}\vec{x}=0
\end{align}
touch the circle
\begin{align}
\vec{x}^T\vec{x}-\myvec{6 & 2}\vec{x}+8 = 0
\end{align}
\renewcommand{\theequation}{\theenumi}
\item Prove that the circle 
\begin{align}
\vec{x}^T\vec{x}-2a\myvec{1 & 1}\vec{x}+a^2 = 0
\end{align}
touches the coordinate axes.
\item Show that two circles can be drawn to pass through the point $\myvec{1\\2}$ and touch the coordinate axes, and find their equations.
\item Find the length of the tangent from the point $\myvec{7\\4}$ to the circle 
\begin{align}
\vec{x}^T\vec{x}-\myvec{4 & 6}\vec{x}+12 = 0
\end{align}
\numberwithin{equation}{enumi}
\item  Find the equations of the tangents to the circle
\begin{align}
\vec{x}^T\vec{x}-\myvec{4 & 3}\vec{x}+5 = 0
\end{align}
that are parallel to the line 
\begin{align}
\myvec{1 & 1}\vec{x}=0
\end{align}
\renewcommand{\theequation}{\theenumi}
\item Find the equations of the tangents to the circle
\begin{align}
\vec{x}^T\vec{x}-\myvec{7 & 5}\vec{x}+18 = 0
\end{align}
at the points $\myvec{4\\3}$ and $\myvec{3\\2}$, showing that they are parallel.
\item Find the equations of the tangent and normal to the circle
\begin{align}
\vec{x}^T\vec{x}+\myvec{-6 & 4}\vec{x}-12 = 0
\end{align}
at the point $\myvec{6\\2}$.
\numberwithin{equation}{enumi}
\item Prove that the line 
\begin{align}
\myvec{1 & 1}\vec{x} = 1
\end{align}
touches the circle
\begin{align}
\vec{x}^T\vec{x}-\myvec{8 & 6}\vec{x}+7 = 0
\end{align}
and find the equations of the parallel and perpendicular tangents.
\renewcommand{\theequation}{\theenumi}
\item Find the equation of the tangent at the origin to the circle
\begin{align}
\vec{x}^T\vec{x}+2\myvec{g & f}\vec{x} = 0
\end{align}
\numberwithin{equation}{enumi}
%
\item Prove that the line 
\begin{align}
\myvec{\cos\alpha &\sin\alpha}\vec{x} = p
\end{align}
touches the circle
\begin{align}
\norm{\vec{x}-\myvec{a\\b}}=r
\end{align}
if
\begin{align}
r = \pm \brak{p-a\cos\alpha-b\sin\alpha}
\end{align}
\renewcommand{\theequation}{\theenumi}
\item Find the points of contact of the tangents to the circle
\begin{align}
\norm{\vec{x}}=5
\end{align}
that pass through the point $\myvec{7\\1}$ and write down the equations of the tangents.
\numberwithin{equation}{enumi}

\item Prove that the tangent to the circle 
\begin{align}
\norm{\vec{x}}^2=5
\end{align}
at the point $\myvec{1\\-2}$ also touches the circle
\begin{align}
\vec{x}^T\vec{x}+\myvec{-8 & 6}\vec{x}+20 = 0
\end{align}
and find the coordinates of the point of contact.
\item Find the equations of the circles that touch the lines
\begin{align}
\myvec{0 & 1}\vec{x} &= 0
\\
\myvec{0 & 1}\vec{x} &= 4
\\
\myvec{2 & 1}\vec{x} &= 2
\end{align}
\item Find the coordinates of the middle point of the chord
\begin{align}
\myvec{1 & 7}\vec{x} &= 25
\end{align}
of the circle
\begin{align}
\norm{\vec{x}}=5
\end{align}
\renewcommand{\theequation}{\theenumi}
\item Find the equation of the chord of the circle
\begin{align}
\vec{x}^T\vec{x}-\myvec{6 & 4}\vec{x}-23 = 0
\end{align}
which has the point $\myvec{4\\1}$ as its middle point.
\item Prove that the circle
\begin{align}
\vec{x}^T\vec{x}-\myvec{6 & 4}\vec{x}+9 = 0
\end{align}
subtends an angle $\tan^{-1}\frac{12}{5}$ at the origin.
\numberwithin{equation}{enumi}
\item Find the condition that the line
\begin{align}
\myvec{l & m}\vec{x} + n &= 0
\end{align}
should touch the circle
\begin{align}
\norm{\vec{x}-\myvec{a\\b}}=r
\end{align}
\renewcommand{\theequation}{\theenumi}
\item Verify that the perpendicular bisector of the chord joining two points $\vec{x}_1, \vec{x}_2$ on the circle
\begin{align}
\vec{x}^T\vec{x}+2\myvec{g & f}\vec{x} + c = 0
\end{align}
passes through the centre.
\end{enumerate}
