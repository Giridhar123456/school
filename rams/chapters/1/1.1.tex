%\begin{enumerate}[1.]
%\numberwithin{equation}{\thesubsection.enumi}

%\begin{enumerate}[label=\arabic*.,ref=\thesection]
\renewcommand{\theequation}{\theenumi}
\begin{enumerate}[label=\arabic*.,ref=\thesubsection.\theenumi]
%\begin{enumerate}[label=\arabic*.,ref=\thesubsection.\theenumi]
%\numberwithin{equation}{subsection}

%\begin{enumerate}[label=\arabic*.,ref=\thesubsection.\theenumi]
%\begin{enumerate}[label=\thesection.\arabic*,ref=\thesection.\theenumi]

%\numberwithin{equation}{enumi}
\item Find the distance between 
\begin{align}
\vec{P} = \myvec{-2\\4}, \vec{Q} = \myvec{3\\-5}
\end{align}
%Mark on a diagram the points $\myvec{-2,4}, \myvec{3,-5}$ and find the distance between them.
\solution
Let the balanced version of (\ref{eq:solutions/chem/6ato balance}) be
\begin{align}
    \label{eq:solutions/chem/6abalanced}x_{1}HNO_{3}+ x_{2}Ca(OH)_{2}\to x_{3}Ca(NO_{3})_{2}+ x_{4}H_{2}O
\end{align}

which results in the following equations:
\begin{align}
    (x_{1}+ 2x_{2}-2x_{4}) H= 0\\
    (x_{1}-2x_{3}) N= 0\\
    (3x_{1}+ 2x_{2}-6x_{3}- x_{4}) O=0\\
    (x_{2}-x_{3}) Ca= 0
\end{align}

which can be expressed as
\begin{align}
    x_{1}+ 2x_{2}+ 0.x_{3} -2x_{4} = 0\\
    x_{1}+ 0.x_{2} -2x_{3} +0.x_{4}= 0\\
    3x_{1}+ 2x_{2}-6x_{3}- x_{4} =0\\
    0.x_{1} +x_{2}-x_{3} +0.x_{4}= 0
\end{align}

resulting in the matrix equation
\begin{align}
    \label{eq:solutions/chem/6a matrix}
    \myvec{1 & 2 & 0 & -2\\
           1 & 0 & -2 & 0\\
           3 & 2 & -6 & -1\\
           0 & 1 & -1 & 0}\vec{x}
           =\vec{0}
\end{align}

where,
\begin{align}
   \vec{x}= \myvec{x_{1}\\x_{2}\\x_{3}\\x_{4}}
\end{align}

(\ref{eq:solutions/chem/6a matrix}) can be reduced as follows:
\begin{align}
    \myvec{1 & 2 & 0 & -2\\
           1 & 0 & -2 & 0\\
           3 & 2 & -6 & -1\\
           0 & 1 & -1 & 0}
    \xleftrightarrow[R_{3}\leftarrow \frac{R_3}{3}-R_{1}]{R_{2}\leftarrow R_2- R_1}
    \myvec{1 & 2 & 0 & -2\\
           0 & -2 & -2 & 2\\
           0 & -\frac{4}{3} & -2 & \frac{5}{3}\\
           0 & 1 & -1 & 0}\\
    \xleftrightarrow{R_2 \leftarrow -\frac{R_2}{2}}
    \myvec{1 & 2 & 0 & -2\\
          0 & 1 & 1 & -1\\
          0 & -\frac{4}{3} & -2 & \frac{5}{3}\\
          0 & 1 & -1 & 0}\\
    \xleftrightarrow[R_4 \leftarrow R_4- R_2]{R_3 \leftarrow R_3 + \frac{4}{3}R_2}
    \myvec{1 & 2 & 0 & -2\\
           0 & 1 & 1 & -1\\
           0 & 0 & -\frac{2}{3} & \frac{1}{3}\\
           0 & 0 & -2 & 1}\\
    \xleftrightarrow[R_3 \leftarrow -\frac{3}{2}R_3]{R_1 \leftarrow R_1- 2R_2}
    \myvec{1 & 0 & -2 & 0\\
           0 & 1 & 1 & -1\\
           0 & 0 & 1 & -\frac{1}{2}\\
           0 & 0 & -2 & 1}\\
    \xleftrightarrow{R_4\leftarrow R_4 + 2R_3}
    \myvec{1 & 0 & -2 & 0\\
           0 & 1 & 1 & -1\\
           0 & 0 & 1 & -\frac{1}{2}\\
           0 & 0 & 0 & 0}\\
    \xleftrightarrow[R_2\leftarrow R_2-R_3]{R_1\leftarrow R_1 + 2R_3}
    \myvec{1 & 0 & 0 & -1\\
           0 & 1 & 0 & -\frac{1}{2}\\
           0 & 0 & 1 & -\frac{1}{2}\\
           0 & 0 & 0 & 0}
\end{align}

Thus,
\begin{align}
    x_1=x_4, x_2= \frac{1}{2}x_4, x_3=\frac{1}{2}x_4\\
    \implies \quad\vec{x}= x_4\myvec{1\\ \frac{1}{2}\\ \frac{1}{2}\\1} =\myvec{2\\1\\1\\2}
\end{align} 
by substituting $x_4= 2$.

\hfill\break
%\vspace{5mm} 
Hence, (\ref{eq:solutions/chem/6abalanced}) finally becomes
\begin{align}
    2HNO_{3}+ Ca(OH)_{2}\to Ca(NO_{3})_{2}+ 2H_{2}O
\end{align}

%
\item
Find the length of $PQ$ for
\begin{enumerate}
\item $\vec{P}=\myvec{-1\\1}$ and $\vec{Q}=\myvec{2\\-1}$;
\item $\vec{P}=\myvec{4\\3}$ and $\vec{Q}=\myvec{-2\\2}$;
\item $\vec{P}=\myvec{a\\b}$ and $\vec{Q}=\myvec{-b\\a}$.
\end{enumerate}

\item Using direction vectors, show that  $\myvec{2\\1}, \myvec{4\\7}, \myvec{5\\4}$ and $\myvec{1\\4}$ are the vertices of a parallelogram.
\solution
Let the balanced version of (\ref{eq:solutions/chem/6ato balance}) be
\begin{align}
    \label{eq:solutions/chem/6abalanced}x_{1}HNO_{3}+ x_{2}Ca(OH)_{2}\to x_{3}Ca(NO_{3})_{2}+ x_{4}H_{2}O
\end{align}

which results in the following equations:
\begin{align}
    (x_{1}+ 2x_{2}-2x_{4}) H= 0\\
    (x_{1}-2x_{3}) N= 0\\
    (3x_{1}+ 2x_{2}-6x_{3}- x_{4}) O=0\\
    (x_{2}-x_{3}) Ca= 0
\end{align}

which can be expressed as
\begin{align}
    x_{1}+ 2x_{2}+ 0.x_{3} -2x_{4} = 0\\
    x_{1}+ 0.x_{2} -2x_{3} +0.x_{4}= 0\\
    3x_{1}+ 2x_{2}-6x_{3}- x_{4} =0\\
    0.x_{1} +x_{2}-x_{3} +0.x_{4}= 0
\end{align}

resulting in the matrix equation
\begin{align}
    \label{eq:solutions/chem/6a matrix}
    \myvec{1 & 2 & 0 & -2\\
           1 & 0 & -2 & 0\\
           3 & 2 & -6 & -1\\
           0 & 1 & -1 & 0}\vec{x}
           =\vec{0}
\end{align}

where,
\begin{align}
   \vec{x}= \myvec{x_{1}\\x_{2}\\x_{3}\\x_{4}}
\end{align}

(\ref{eq:solutions/chem/6a matrix}) can be reduced as follows:
\begin{align}
    \myvec{1 & 2 & 0 & -2\\
           1 & 0 & -2 & 0\\
           3 & 2 & -6 & -1\\
           0 & 1 & -1 & 0}
    \xleftrightarrow[R_{3}\leftarrow \frac{R_3}{3}-R_{1}]{R_{2}\leftarrow R_2- R_1}
    \myvec{1 & 2 & 0 & -2\\
           0 & -2 & -2 & 2\\
           0 & -\frac{4}{3} & -2 & \frac{5}{3}\\
           0 & 1 & -1 & 0}\\
    \xleftrightarrow{R_2 \leftarrow -\frac{R_2}{2}}
    \myvec{1 & 2 & 0 & -2\\
          0 & 1 & 1 & -1\\
          0 & -\frac{4}{3} & -2 & \frac{5}{3}\\
          0 & 1 & -1 & 0}\\
    \xleftrightarrow[R_4 \leftarrow R_4- R_2]{R_3 \leftarrow R_3 + \frac{4}{3}R_2}
    \myvec{1 & 2 & 0 & -2\\
           0 & 1 & 1 & -1\\
           0 & 0 & -\frac{2}{3} & \frac{1}{3}\\
           0 & 0 & -2 & 1}\\
    \xleftrightarrow[R_3 \leftarrow -\frac{3}{2}R_3]{R_1 \leftarrow R_1- 2R_2}
    \myvec{1 & 0 & -2 & 0\\
           0 & 1 & 1 & -1\\
           0 & 0 & 1 & -\frac{1}{2}\\
           0 & 0 & -2 & 1}\\
    \xleftrightarrow{R_4\leftarrow R_4 + 2R_3}
    \myvec{1 & 0 & -2 & 0\\
           0 & 1 & 1 & -1\\
           0 & 0 & 1 & -\frac{1}{2}\\
           0 & 0 & 0 & 0}\\
    \xleftrightarrow[R_2\leftarrow R_2-R_3]{R_1\leftarrow R_1 + 2R_3}
    \myvec{1 & 0 & 0 & -1\\
           0 & 1 & 0 & -\frac{1}{2}\\
           0 & 0 & 1 & -\frac{1}{2}\\
           0 & 0 & 0 & 0}
\end{align}

Thus,
\begin{align}
    x_1=x_4, x_2= \frac{1}{2}x_4, x_3=\frac{1}{2}x_4\\
    \implies \quad\vec{x}= x_4\myvec{1\\ \frac{1}{2}\\ \frac{1}{2}\\1} =\myvec{2\\1\\1\\2}
\end{align} 
by substituting $x_4= 2$.

\hfill\break
%\vspace{5mm} 
Hence, (\ref{eq:solutions/chem/6abalanced}) finally becomes
\begin{align}
    2HNO_{3}+ Ca(OH)_{2}\to Ca(NO_{3})_{2}+ 2H_{2}O
\end{align}

\item Using Baudhayana's theorem, show that the points $\myvec{-3\\-4}, \myvec{2\\6}$ and $\myvec{-6\\10}$  are the vertices of a right-angled
traingle.  Repeat using orthogonality.
\item Plot the points $\myvec{0\\2},\myvec{1\\1},\myvec{4\\4}\text{ and }\myvec{3\\5}$ and prove that they are the vertices of a rectangle.
\item Show that $\vec{B}=\myvec{-2\\-2},\vec{A}=\myvec{-1\\2}\text{ and }\vec{C}=\myvec{3\\1}$ are the vertices of an isosceles triangle.
\item In the last question, find the distance of the vertex $\vec{A}$ of the triangle from the middle point of the base $BC$.
\item Prove that the points $\myvec{-1\\0}$, $\myvec{0\\3}$, $\myvec{3\\2}$ and $\myvec{2\\-1}$ are the vertices of a square.
\item Prove that the points $\vec{A}=\myvec{-1\\0}$, $\vec{B}=\myvec{3\\1}$, $\vec{C}=\myvec{2\\2}$  and $\vec{D}=\myvec{-2\\1}$ are the vertices of a parallelogram.  Find $\vec{E},\vec{F},\vec{G},\vec{H}$, the mid points of $AB, BC, CD, AD$ respectively.  Show that EG and FH bisect each other.
\item Prove that the points $\myvec{21\\-2}$, $\myvec{15\\10}$, $\myvec{-5\\0}$  and $\myvec{1\\-12}$ are the vertices of a rectangle, and find the
coordinates of its centre.
\item Find the lengths of the medians of the triangle whose vertices are at the points $\myvec{1\\2}$, $\myvec{0\\3}$ and $\myvec{-1\\-2}$.
\item Find the coordinates of the points that divide the line joining the points $\myvec{-35\\-20}$ and $\myvec{5\\-10}$ into four equal parts.
\item Find the coordinates of the points of trisection of the line joining the points $\myvec{-5\\5}$ and $\myvec{25\\10}$.
\item Prove that the middle point of the line joining the points $\myvec{-5\\12}$ and $\myvec{9\\-2}$ is a point of trisection of the line
joining the points $\myvec{-8\\-5}$ and $\myvec{7\\10}$.
\item The points $\myvec{8\\5}$, $\myvec{-7\\-5}$ and $\myvec{-5\\5}$ are three of the vertices of a parallelogram.  Find the coordinates of
the remaining vertex which is to be taken as opposite to $\myvec{-7\\-5}$.
\item The point $\myvec{2\\6}$ is the intesection of the diagonals of a parallelogram two of whose vertices are at the points $\myvec{7\\16}$ and $\myvec{10\\2}$.
Find the coordinates of the remaining vertices.
\item Find the area of the triangle whose vertices are the points $\myvec{2\\3}$, $\myvec{-4\\7}$ and $\myvec{5\\-2}$.  
\item Find the coordinates of  points which divide the join of $\myvec{2\\3}$, $\myvec{-4\\5}$ externally in the ratio $2:3$, and also
externally in the ratio $3:2$.
\item Prove the centroid of $\triangle ABC$ is
\begin{equation}
\vec{O}=\frac{\vec{A}+\vec{B}+\vec{C}}{3}
\end{equation}

\end{enumerate}
%\bibliography{IEEEabrv,gvv_opt}

%\input{./chapters/chapter2} 
%%
%\newpage
%\section{$M$-ary Modulation}
%\input{./chapters/chapter3} 
%
%\newpage
%\section{BER in Rayleigh Flat Slowly Fading Channels}
	%\input{./chapters/chapter4} 




