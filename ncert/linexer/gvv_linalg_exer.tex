\documentclass[journal,12pt,twocolumn]{IEEEtran}
%
\usepackage{setspace}
\usepackage{gensymb}
%\usepackage{bm}
%\doublespacing
\singlespacing

%\usepackage{graphicx}
%\usepackage{amssymb}
%\usepackage{relsize}
\usepackage[cmex10]{amsmath}
\usepackage{siunitx}
%\usepackage{amsthm}
%\interdisplaylinepenalty=2500
%\savesymbol{iint}
%\usepackage{txfonts}
%\restoresymbol{TXF}{iint}
%\usepackage{wasysym}
\usepackage{amsthm}
\usepackage{iithtlc}
\usepackage{mathrsfs}
\usepackage{txfonts}
\usepackage{stfloats}
\usepackage{steinmetz}
%\usepackage{bm}
\usepackage{cite}
\usepackage{cases}
\usepackage{subfig}
%\usepackage{xtab}
\usepackage{longtable}
\usepackage{multirow}
%\usepackage{algorithm}
%\usepackage{algpseudocode}
\usepackage{enumitem}
\usepackage{mathtools}
\usepackage{tikz}
\usepackage{circuitikz}
\usepackage{verbatim}
\usepackage{tfrupee}
\usepackage[breaklinks=true]{hyperref}
%\usepackage{stmaryrd}
\usepackage{tkz-euclide} % loads  TikZ and tkz-base
%\usetkzobj{all}
\usetikzlibrary{calc,math}
\usetikzlibrary{fadings}
\usepackage{listings}
    \usepackage{color}                                            %%
    \usepackage{array}                                            %%
    \usepackage{longtable}                                        %%
    \usepackage{calc}                                             %%
    \usepackage{multirow}                                         %%
    \usepackage{hhline}                                           %%
    \usepackage{ifthen}                                           %%
  %optionally (for landscape tables embedded in another document): %%
    \usepackage{lscape}     
\usepackage{multicol}
\usepackage{chngcntr}
%\usepackage{enumerate}

%\usepackage{wasysym}
%\newcounter{MYtempeqncnt}
\DeclareMathOperator*{\Res}{Res}
%\renewcommand{\baselinestretch}{2}
\renewcommand\thesection{\arabic{section}}
\renewcommand\thesubsection{\thesection.\arabic{subsection}}
\renewcommand\thesubsubsection{\thesubsection.\arabic{subsubsection}}

\renewcommand\thesectiondis{\arabic{section}}
\renewcommand\thesubsectiondis{\thesectiondis.\arabic{subsection}}
\renewcommand\thesubsubsectiondis{\thesubsectiondis.\arabic{subsubsection}}

% correct bad hyphenation here
\hyphenation{op-tical net-works semi-conduc-tor}
\def\inputGnumericTable{}                                 %%

\lstset{
%language=C,
frame=single, 
breaklines=true,
columns=fullflexible
}
%\lstset{
%language=tex,
%frame=single, 
%breaklines=true
%}

\begin{document}
%


\newtheorem{theorem}{Theorem}[section]
\newtheorem{problem}{Problem}
\newtheorem{proposition}{Proposition}[section]
\newtheorem{lemma}{Lemma}[section]
\newtheorem{corollary}[theorem]{Corollary}
\newtheorem{example}{Example}[section]
\newtheorem{definition}[problem]{Definition}
%\newtheorem{thm}{Theorem}[section] 
%\newtheorem{defn}[thm]{Definition}
%\newtheorem{algorithm}{Algorithm}[section]
%\newtheorem{cor}{Corollary}
\newcommand{\BEQA}{\begin{eqnarray}}
\newcommand{\EEQA}{\end{eqnarray}}
\newcommand{\define}{\stackrel{\triangle}{=}}

\bibliographystyle{IEEEtran}
%\bibliographystyle{ieeetr}


\providecommand{\mbf}{\mathbf}
\providecommand{\pr}[1]{\ensuremath{\Pr\left(#1\right)}}
\providecommand{\qfunc}[1]{\ensuremath{Q\left(#1\right)}}
\providecommand{\sbrak}[1]{\ensuremath{{}\left[#1\right]}}
\providecommand{\lsbrak}[1]{\ensuremath{{}\left[#1\right.}}
\providecommand{\rsbrak}[1]{\ensuremath{{}\left.#1\right]}}
\providecommand{\brak}[1]{\ensuremath{\left(#1\right)}}
\providecommand{\lbrak}[1]{\ensuremath{\left(#1\right.}}
\providecommand{\rbrak}[1]{\ensuremath{\left.#1\right)}}
\providecommand{\cbrak}[1]{\ensuremath{\left\{#1\right\}}}
\providecommand{\lcbrak}[1]{\ensuremath{\left\{#1\right.}}
\providecommand{\rcbrak}[1]{\ensuremath{\left.#1\right\}}}
\theoremstyle{remark}
\newtheorem{rem}{Remark}
\newcommand{\sgn}{\mathop{\mathrm{sgn}}}
\providecommand{\abs}[1]{\left\vert#1\right\vert}
\providecommand{\res}[1]{\Res\displaylimits_{#1}} 
\providecommand{\norm}[1]{\left\lVert#1\right\rVert}
%\providecommand{\norm}[1]{\lVert#1\rVert}
\providecommand{\mtx}[1]{\mathbf{#1}}
\providecommand{\mean}[1]{E\left[ #1 \right]}
\providecommand{\fourier}{\overset{\mathcal{F}}{ \rightleftharpoons}}
%\providecommand{\hilbert}{\overset{\mathcal{H}}{ \rightleftharpoons}}
\providecommand{\system}{\overset{\mathcal{H}}{ \longleftrightarrow}}
	%\newcommand{\solution}[2]{\textbf{Solution:}{#1}}
\newcommand{\solution}{\noindent \textbf{Solution: }}
\newcommand{\cosec}{\,\text{cosec}\,}
\providecommand{\dec}[2]{\ensuremath{\overset{#1}{\underset{#2}{\gtrless}}}}
\newcommand{\myvec}[1]{\ensuremath{\begin{pmatrix}#1\end{pmatrix}}}
\newcommand{\cmyvec}[1]{\ensuremath{\begin{pmatrix*}[c]#1\end{pmatrix*}}}
\newcommand{\mydet}[1]{\ensuremath{\begin{vmatrix}#1\end{vmatrix}}}
\newcommand{\proj}[2]{\textbf{proj}_{\vec{#1}}\vec{#2}}
%\numberwithin{equation}{section}
\numberwithin{equation}{subsection}
%\numberwithin{problem}{section}
%\numberwithin{definition}{section}
\makeatletter
\@addtoreset{figure}{problem}
\makeatother

\let\StandardTheFigure\thefigure
\let\vec\mathbf
%\renewcommand{\thefigure}{\theproblem.\arabic{figure}}
\renewcommand{\thefigure}{\theproblem}
%\setlist[enumerate,1]{before=\renewcommand\theequation{\theenumi.\arabic{equation}}
%\counterwithin{equation}{enumi}


%\renewcommand{\theequation}{\arabic{subsection}.\arabic{equation}}

\def\putbox#1#2#3{\makebox[0in][l]{\makebox[#1][l]{}\raisebox{\baselineskip}[0in][0in]{\raisebox{#2}[0in][0in]{#3}}}}
     \def\rightbox#1{\makebox[0in][r]{#1}}
     \def\centbox#1{\makebox[0in]{#1}}
     \def\topbox#1{\raisebox{-\baselineskip}[0in][0in]{#1}}
     \def\midbox#1{\raisebox{-0.5\baselineskip}[0in][0in]{#1}}

\vspace{3cm}

\title{
	\logo{
Coordinate Geometry Exercises
	}
}
\author{ G V V Sharma$^{*}$% <-this % stops a space
	\thanks{*The author is with the Department
		of Electrical Engineering, Indian Institute of Technology, Hyderabad
		502285 India e-mail:  gadepall@iith.ac.in. All content in this manual is released under GNU GPL.  Free and open source.}
	
}	
%\title{
%	\logo{Matrix Analysis through Octave}{\begin{center}\includegraphics[scale=.24]{tlc}\end{center}}{}{HAMDSP}
%}


% paper title
% can use linebreaks \\ within to get better formatting as desired
%\title{Matrix Analysis through Octave}
%
%
% author names and IEEE memberships
% note positions of commas and nonbreaking spaces ( ~ ) LaTeX will not break
% a structure at a ~ so this keeps an author's name from being broken across
% two lines.
% use \thanks{} to gain access to the first footnote area
% a separate \thanks must be used for each paragraph as LaTeX2e's \thanks
% was not built to handle multiple paragraphs
%

%\author{<-this % stops a space
%\thanks{}}
%}
% note the % following the last \IEEEmembership and also \thanks - 
% these prevent an unwanted space from occurring between the last author name
% and the end of the author line. i.e., if you had this:
% 
% \author{....lastname \thanks{...} \thanks{...} }
%                     ^------------^------------^----Do not want these spaces!
%
% a space would be appended to the last name and could cause every name on that
% line to be shifted left slightly. This is one of those "LaTeX things". For
% instance, "\textbf{A} \textbf{B}" will typeset as "A B" not "AB". To get
% "AB" then you have to do: "\textbf{A}\textbf{B}"
% \thanks is no different in this regard, so shield the last } of each \thanks
% that ends a line with a % and do not let a space in before the next \thanks.
% Spaces after \IEEEmembership other than the last one are OK (and needed) as
% you are supposed to have spaces between the names. For what it is worth,
% this is a minor point as most people would not even notice if the said evil
% space somehow managed to creep in.



% The paper headers
%\markboth{Journal of \LaTeX\ Class Files,~Vol.~6, No.~1, January~2007}%
%{Shell \MakeLowercase{\textit{et al.}}: Bare Demo of IEEEtran.cls for Journals}
% The only time the second header will appear is for the odd numbered pages
% after the title page when using the twoside option.
% 
% *** Note that you probably will NOT want to include the author's ***
% *** name in the headers of peer review papers.                   ***
% You can use \ifCLASSOPTIONpeerreview for conditional compilation here if
% you desire.




% If you want to put a publisher's ID mark on the page you can do it like
% this:
%\IEEEpubid{0000--0000/00\$00.00~\copyright~2007 IEEE}
% Remember, if you use this you must call \IEEEpubidadjcol in the second
% column for its text to clear the IEEEpubid mark.



% make the title area
\maketitle

\newpage

\tableofcontents

\bigskip

\renewcommand{\thefigure}{\theenumi}
\renewcommand{\thetable}{\theenumi}
%\renewcommand{\theequation}{\theenumi}

%\begin{abstract}
%%\boldmath
%In this letter, an algorithm for evaluating the exact analytical bit error rate  (BER)  for the piecewise linear (PL) combiner for  multiple relays is presented. Previous results were available only for upto three relays. The algorithm is unique in the sense that  the actual mathematical expressions, that are prohibitively large, need not be explicitly obtained. The diversity gain due to multiple relays is shown through plots of the analytical BER, well supported by simulations. 
%
%\end{abstract}
% IEEEtran.cls defaults to using nonbold math in the Abstract.
% This preserves the distinction between vectors and scalars. However,
% if the journal you are submitting to favors bold math in the abstract,
% then you can use LaTeX's standard command \boldmath at the very start
% of the abstract to achieve this. Many IEEE journals frown on math
% in the abstract anyway.

% Note that keywords are not normally used for peerreview papers.
%\begin{IEEEkeywords}
%Cooperative diversity, decode and forward, piecewise linear
%\end{IEEEkeywords}



% For peer review papers, you can put extra information on the cover
% page as needed:
% \ifCLASSOPTIONpeerreview
% \begin{center} \bfseries EDICS Category: 3-BBND \end{center}
% \fi
%
% For peerreview papers, this IEEEtran command inserts a page break and
% creates the second title. It will be ignored for other modes.
%\IEEEpeerreviewmaketitle

\begin{abstract}
This book provides some exercises related to coordinate geometry.
The content and exercises are based on  NCERT textbooks from Class 6-12.  
\end{abstract}


\renewcommand{\theequation}{\theenumi}
%\begin{enumerate}[label=\arabic*.,ref=\theenumi]
\begin{enumerate}[label=\thesection.\arabic*.,ref=\thesection.\theenumi]
\numberwithin{equation}{enumi}

\item Find the area of the region enclosed between the two circles: $\vec{x}^T\vec{x} = 4$ and $\norm{\vec{x}-\myvec{2\\0}} = 2$.
\\
\solution 
Let the balanced version of (\ref{eq:solutions/chem/6ato balance}) be
\begin{align}
    \label{eq:solutions/chem/6abalanced}x_{1}HNO_{3}+ x_{2}Ca(OH)_{2}\to x_{3}Ca(NO_{3})_{2}+ x_{4}H_{2}O
\end{align}

which results in the following equations:
\begin{align}
    (x_{1}+ 2x_{2}-2x_{4}) H= 0\\
    (x_{1}-2x_{3}) N= 0\\
    (3x_{1}+ 2x_{2}-6x_{3}- x_{4}) O=0\\
    (x_{2}-x_{3}) Ca= 0
\end{align}

which can be expressed as
\begin{align}
    x_{1}+ 2x_{2}+ 0.x_{3} -2x_{4} = 0\\
    x_{1}+ 0.x_{2} -2x_{3} +0.x_{4}= 0\\
    3x_{1}+ 2x_{2}-6x_{3}- x_{4} =0\\
    0.x_{1} +x_{2}-x_{3} +0.x_{4}= 0
\end{align}

resulting in the matrix equation
\begin{align}
    \label{eq:solutions/chem/6a matrix}
    \myvec{1 & 2 & 0 & -2\\
           1 & 0 & -2 & 0\\
           3 & 2 & -6 & -1\\
           0 & 1 & -1 & 0}\vec{x}
           =\vec{0}
\end{align}

where,
\begin{align}
   \vec{x}= \myvec{x_{1}\\x_{2}\\x_{3}\\x_{4}}
\end{align}

(\ref{eq:solutions/chem/6a matrix}) can be reduced as follows:
\begin{align}
    \myvec{1 & 2 & 0 & -2\\
           1 & 0 & -2 & 0\\
           3 & 2 & -6 & -1\\
           0 & 1 & -1 & 0}
    \xleftrightarrow[R_{3}\leftarrow \frac{R_3}{3}-R_{1}]{R_{2}\leftarrow R_2- R_1}
    \myvec{1 & 2 & 0 & -2\\
           0 & -2 & -2 & 2\\
           0 & -\frac{4}{3} & -2 & \frac{5}{3}\\
           0 & 1 & -1 & 0}\\
    \xleftrightarrow{R_2 \leftarrow -\frac{R_2}{2}}
    \myvec{1 & 2 & 0 & -2\\
          0 & 1 & 1 & -1\\
          0 & -\frac{4}{3} & -2 & \frac{5}{3}\\
          0 & 1 & -1 & 0}\\
    \xleftrightarrow[R_4 \leftarrow R_4- R_2]{R_3 \leftarrow R_3 + \frac{4}{3}R_2}
    \myvec{1 & 2 & 0 & -2\\
           0 & 1 & 1 & -1\\
           0 & 0 & -\frac{2}{3} & \frac{1}{3}\\
           0 & 0 & -2 & 1}\\
    \xleftrightarrow[R_3 \leftarrow -\frac{3}{2}R_3]{R_1 \leftarrow R_1- 2R_2}
    \myvec{1 & 0 & -2 & 0\\
           0 & 1 & 1 & -1\\
           0 & 0 & 1 & -\frac{1}{2}\\
           0 & 0 & -2 & 1}\\
    \xleftrightarrow{R_4\leftarrow R_4 + 2R_3}
    \myvec{1 & 0 & -2 & 0\\
           0 & 1 & 1 & -1\\
           0 & 0 & 1 & -\frac{1}{2}\\
           0 & 0 & 0 & 0}\\
    \xleftrightarrow[R_2\leftarrow R_2-R_3]{R_1\leftarrow R_1 + 2R_3}
    \myvec{1 & 0 & 0 & -1\\
           0 & 1 & 0 & -\frac{1}{2}\\
           0 & 0 & 1 & -\frac{1}{2}\\
           0 & 0 & 0 & 0}
\end{align}

Thus,
\begin{align}
    x_1=x_4, x_2= \frac{1}{2}x_4, x_3=\frac{1}{2}x_4\\
    \implies \quad\vec{x}= x_4\myvec{1\\ \frac{1}{2}\\ \frac{1}{2}\\1} =\myvec{2\\1\\1\\2}
\end{align} 
by substituting $x_4= 2$.

\hfill\break
%\vspace{5mm} 
Hence, (\ref{eq:solutions/chem/6abalanced}) finally becomes
\begin{align}
    2HNO_{3}+ Ca(OH)_{2}\to Ca(NO_{3})_{2}+ 2H_{2}O
\end{align}

\item Find the equation of the circle with radius 5 whose centre lies on x-axis and passes through the point \myvec{2\\3}.
\\
\solution 
Let the balanced version of (\ref{eq:solutions/chem/6ato balance}) be
\begin{align}
    \label{eq:solutions/chem/6abalanced}x_{1}HNO_{3}+ x_{2}Ca(OH)_{2}\to x_{3}Ca(NO_{3})_{2}+ x_{4}H_{2}O
\end{align}

which results in the following equations:
\begin{align}
    (x_{1}+ 2x_{2}-2x_{4}) H= 0\\
    (x_{1}-2x_{3}) N= 0\\
    (3x_{1}+ 2x_{2}-6x_{3}- x_{4}) O=0\\
    (x_{2}-x_{3}) Ca= 0
\end{align}

which can be expressed as
\begin{align}
    x_{1}+ 2x_{2}+ 0.x_{3} -2x_{4} = 0\\
    x_{1}+ 0.x_{2} -2x_{3} +0.x_{4}= 0\\
    3x_{1}+ 2x_{2}-6x_{3}- x_{4} =0\\
    0.x_{1} +x_{2}-x_{3} +0.x_{4}= 0
\end{align}

resulting in the matrix equation
\begin{align}
    \label{eq:solutions/chem/6a matrix}
    \myvec{1 & 2 & 0 & -2\\
           1 & 0 & -2 & 0\\
           3 & 2 & -6 & -1\\
           0 & 1 & -1 & 0}\vec{x}
           =\vec{0}
\end{align}

where,
\begin{align}
   \vec{x}= \myvec{x_{1}\\x_{2}\\x_{3}\\x_{4}}
\end{align}

(\ref{eq:solutions/chem/6a matrix}) can be reduced as follows:
\begin{align}
    \myvec{1 & 2 & 0 & -2\\
           1 & 0 & -2 & 0\\
           3 & 2 & -6 & -1\\
           0 & 1 & -1 & 0}
    \xleftrightarrow[R_{3}\leftarrow \frac{R_3}{3}-R_{1}]{R_{2}\leftarrow R_2- R_1}
    \myvec{1 & 2 & 0 & -2\\
           0 & -2 & -2 & 2\\
           0 & -\frac{4}{3} & -2 & \frac{5}{3}\\
           0 & 1 & -1 & 0}\\
    \xleftrightarrow{R_2 \leftarrow -\frac{R_2}{2}}
    \myvec{1 & 2 & 0 & -2\\
          0 & 1 & 1 & -1\\
          0 & -\frac{4}{3} & -2 & \frac{5}{3}\\
          0 & 1 & -1 & 0}\\
    \xleftrightarrow[R_4 \leftarrow R_4- R_2]{R_3 \leftarrow R_3 + \frac{4}{3}R_2}
    \myvec{1 & 2 & 0 & -2\\
           0 & 1 & 1 & -1\\
           0 & 0 & -\frac{2}{3} & \frac{1}{3}\\
           0 & 0 & -2 & 1}\\
    \xleftrightarrow[R_3 \leftarrow -\frac{3}{2}R_3]{R_1 \leftarrow R_1- 2R_2}
    \myvec{1 & 0 & -2 & 0\\
           0 & 1 & 1 & -1\\
           0 & 0 & 1 & -\frac{1}{2}\\
           0 & 0 & -2 & 1}\\
    \xleftrightarrow{R_4\leftarrow R_4 + 2R_3}
    \myvec{1 & 0 & -2 & 0\\
           0 & 1 & 1 & -1\\
           0 & 0 & 1 & -\frac{1}{2}\\
           0 & 0 & 0 & 0}\\
    \xleftrightarrow[R_2\leftarrow R_2-R_3]{R_1\leftarrow R_1 + 2R_3}
    \myvec{1 & 0 & 0 & -1\\
           0 & 1 & 0 & -\frac{1}{2}\\
           0 & 0 & 1 & -\frac{1}{2}\\
           0 & 0 & 0 & 0}
\end{align}

Thus,
\begin{align}
    x_1=x_4, x_2= \frac{1}{2}x_4, x_3=\frac{1}{2}x_4\\
    \implies \quad\vec{x}= x_4\myvec{1\\ \frac{1}{2}\\ \frac{1}{2}\\1} =\myvec{2\\1\\1\\2}
\end{align} 
by substituting $x_4= 2$.

\hfill\break
%\vspace{5mm} 
Hence, (\ref{eq:solutions/chem/6abalanced}) finally becomes
\begin{align}
    2HNO_{3}+ Ca(OH)_{2}\to Ca(NO_{3})_{2}+ 2H_{2}O
\end{align}

\item Find the equation of the circle passing through \myvec{0\\0} and making intercepts a and b on the coordinate axes.
\item Find the equation of a circle with centre \myvec{2\\2} and passes through the point \myvec{4\\5}. 
\\
\solution 
Let the balanced version of (\ref{eq:solutions/chem/6ato balance}) be
\begin{align}
    \label{eq:solutions/chem/6abalanced}x_{1}HNO_{3}+ x_{2}Ca(OH)_{2}\to x_{3}Ca(NO_{3})_{2}+ x_{4}H_{2}O
\end{align}

which results in the following equations:
\begin{align}
    (x_{1}+ 2x_{2}-2x_{4}) H= 0\\
    (x_{1}-2x_{3}) N= 0\\
    (3x_{1}+ 2x_{2}-6x_{3}- x_{4}) O=0\\
    (x_{2}-x_{3}) Ca= 0
\end{align}

which can be expressed as
\begin{align}
    x_{1}+ 2x_{2}+ 0.x_{3} -2x_{4} = 0\\
    x_{1}+ 0.x_{2} -2x_{3} +0.x_{4}= 0\\
    3x_{1}+ 2x_{2}-6x_{3}- x_{4} =0\\
    0.x_{1} +x_{2}-x_{3} +0.x_{4}= 0
\end{align}

resulting in the matrix equation
\begin{align}
    \label{eq:solutions/chem/6a matrix}
    \myvec{1 & 2 & 0 & -2\\
           1 & 0 & -2 & 0\\
           3 & 2 & -6 & -1\\
           0 & 1 & -1 & 0}\vec{x}
           =\vec{0}
\end{align}

where,
\begin{align}
   \vec{x}= \myvec{x_{1}\\x_{2}\\x_{3}\\x_{4}}
\end{align}

(\ref{eq:solutions/chem/6a matrix}) can be reduced as follows:
\begin{align}
    \myvec{1 & 2 & 0 & -2\\
           1 & 0 & -2 & 0\\
           3 & 2 & -6 & -1\\
           0 & 1 & -1 & 0}
    \xleftrightarrow[R_{3}\leftarrow \frac{R_3}{3}-R_{1}]{R_{2}\leftarrow R_2- R_1}
    \myvec{1 & 2 & 0 & -2\\
           0 & -2 & -2 & 2\\
           0 & -\frac{4}{3} & -2 & \frac{5}{3}\\
           0 & 1 & -1 & 0}\\
    \xleftrightarrow{R_2 \leftarrow -\frac{R_2}{2}}
    \myvec{1 & 2 & 0 & -2\\
          0 & 1 & 1 & -1\\
          0 & -\frac{4}{3} & -2 & \frac{5}{3}\\
          0 & 1 & -1 & 0}\\
    \xleftrightarrow[R_4 \leftarrow R_4- R_2]{R_3 \leftarrow R_3 + \frac{4}{3}R_2}
    \myvec{1 & 2 & 0 & -2\\
           0 & 1 & 1 & -1\\
           0 & 0 & -\frac{2}{3} & \frac{1}{3}\\
           0 & 0 & -2 & 1}\\
    \xleftrightarrow[R_3 \leftarrow -\frac{3}{2}R_3]{R_1 \leftarrow R_1- 2R_2}
    \myvec{1 & 0 & -2 & 0\\
           0 & 1 & 1 & -1\\
           0 & 0 & 1 & -\frac{1}{2}\\
           0 & 0 & -2 & 1}\\
    \xleftrightarrow{R_4\leftarrow R_4 + 2R_3}
    \myvec{1 & 0 & -2 & 0\\
           0 & 1 & 1 & -1\\
           0 & 0 & 1 & -\frac{1}{2}\\
           0 & 0 & 0 & 0}\\
    \xleftrightarrow[R_2\leftarrow R_2-R_3]{R_1\leftarrow R_1 + 2R_3}
    \myvec{1 & 0 & 0 & -1\\
           0 & 1 & 0 & -\frac{1}{2}\\
           0 & 0 & 1 & -\frac{1}{2}\\
           0 & 0 & 0 & 0}
\end{align}

Thus,
\begin{align}
    x_1=x_4, x_2= \frac{1}{2}x_4, x_3=\frac{1}{2}x_4\\
    \implies \quad\vec{x}= x_4\myvec{1\\ \frac{1}{2}\\ \frac{1}{2}\\1} =\myvec{2\\1\\1\\2}
\end{align} 
by substituting $x_4= 2$.

\hfill\break
%\vspace{5mm} 
Hence, (\ref{eq:solutions/chem/6abalanced}) finally becomes
\begin{align}
    2HNO_{3}+ Ca(OH)_{2}\to Ca(NO_{3})_{2}+ 2H_{2}O
\end{align}

\item Find the locus of all the unit vectors in the xy-plane.
%
\item Find the points on the curve $\vec{x}^T\vec{x}-2\myvec{1 & 0}\vec{x} -3 =0$  at which the tangents are parallel to the x-axis.
%
\\
\solution
Let the balanced version of (\ref{eq:solutions/chem/6ato balance}) be
\begin{align}
    \label{eq:solutions/chem/6abalanced}x_{1}HNO_{3}+ x_{2}Ca(OH)_{2}\to x_{3}Ca(NO_{3})_{2}+ x_{4}H_{2}O
\end{align}

which results in the following equations:
\begin{align}
    (x_{1}+ 2x_{2}-2x_{4}) H= 0\\
    (x_{1}-2x_{3}) N= 0\\
    (3x_{1}+ 2x_{2}-6x_{3}- x_{4}) O=0\\
    (x_{2}-x_{3}) Ca= 0
\end{align}

which can be expressed as
\begin{align}
    x_{1}+ 2x_{2}+ 0.x_{3} -2x_{4} = 0\\
    x_{1}+ 0.x_{2} -2x_{3} +0.x_{4}= 0\\
    3x_{1}+ 2x_{2}-6x_{3}- x_{4} =0\\
    0.x_{1} +x_{2}-x_{3} +0.x_{4}= 0
\end{align}

resulting in the matrix equation
\begin{align}
    \label{eq:solutions/chem/6a matrix}
    \myvec{1 & 2 & 0 & -2\\
           1 & 0 & -2 & 0\\
           3 & 2 & -6 & -1\\
           0 & 1 & -1 & 0}\vec{x}
           =\vec{0}
\end{align}

where,
\begin{align}
   \vec{x}= \myvec{x_{1}\\x_{2}\\x_{3}\\x_{4}}
\end{align}

(\ref{eq:solutions/chem/6a matrix}) can be reduced as follows:
\begin{align}
    \myvec{1 & 2 & 0 & -2\\
           1 & 0 & -2 & 0\\
           3 & 2 & -6 & -1\\
           0 & 1 & -1 & 0}
    \xleftrightarrow[R_{3}\leftarrow \frac{R_3}{3}-R_{1}]{R_{2}\leftarrow R_2- R_1}
    \myvec{1 & 2 & 0 & -2\\
           0 & -2 & -2 & 2\\
           0 & -\frac{4}{3} & -2 & \frac{5}{3}\\
           0 & 1 & -1 & 0}\\
    \xleftrightarrow{R_2 \leftarrow -\frac{R_2}{2}}
    \myvec{1 & 2 & 0 & -2\\
          0 & 1 & 1 & -1\\
          0 & -\frac{4}{3} & -2 & \frac{5}{3}\\
          0 & 1 & -1 & 0}\\
    \xleftrightarrow[R_4 \leftarrow R_4- R_2]{R_3 \leftarrow R_3 + \frac{4}{3}R_2}
    \myvec{1 & 2 & 0 & -2\\
           0 & 1 & 1 & -1\\
           0 & 0 & -\frac{2}{3} & \frac{1}{3}\\
           0 & 0 & -2 & 1}\\
    \xleftrightarrow[R_3 \leftarrow -\frac{3}{2}R_3]{R_1 \leftarrow R_1- 2R_2}
    \myvec{1 & 0 & -2 & 0\\
           0 & 1 & 1 & -1\\
           0 & 0 & 1 & -\frac{1}{2}\\
           0 & 0 & -2 & 1}\\
    \xleftrightarrow{R_4\leftarrow R_4 + 2R_3}
    \myvec{1 & 0 & -2 & 0\\
           0 & 1 & 1 & -1\\
           0 & 0 & 1 & -\frac{1}{2}\\
           0 & 0 & 0 & 0}\\
    \xleftrightarrow[R_2\leftarrow R_2-R_3]{R_1\leftarrow R_1 + 2R_3}
    \myvec{1 & 0 & 0 & -1\\
           0 & 1 & 0 & -\frac{1}{2}\\
           0 & 0 & 1 & -\frac{1}{2}\\
           0 & 0 & 0 & 0}
\end{align}

Thus,
\begin{align}
    x_1=x_4, x_2= \frac{1}{2}x_4, x_3=\frac{1}{2}x_4\\
    \implies \quad\vec{x}= x_4\myvec{1\\ \frac{1}{2}\\ \frac{1}{2}\\1} =\myvec{2\\1\\1\\2}
\end{align} 
by substituting $x_4= 2$.

\hfill\break
%\vspace{5mm} 
Hence, (\ref{eq:solutions/chem/6abalanced}) finally becomes
\begin{align}
    2HNO_{3}+ Ca(OH)_{2}\to Ca(NO_{3})_{2}+ 2H_{2}O
\end{align}

\item  Find the area of the region in the first quadrant enclosed by x-axis, line $\myvec{1 & -\sqrt{3}}\vec{x} =0$ and the circle $\vec{x}^T\vec{x}=4$.
%
\\
\solution
Let the balanced version of (\ref{eq:solutions/chem/6ato balance}) be
\begin{align}
    \label{eq:solutions/chem/6abalanced}x_{1}HNO_{3}+ x_{2}Ca(OH)_{2}\to x_{3}Ca(NO_{3})_{2}+ x_{4}H_{2}O
\end{align}

which results in the following equations:
\begin{align}
    (x_{1}+ 2x_{2}-2x_{4}) H= 0\\
    (x_{1}-2x_{3}) N= 0\\
    (3x_{1}+ 2x_{2}-6x_{3}- x_{4}) O=0\\
    (x_{2}-x_{3}) Ca= 0
\end{align}

which can be expressed as
\begin{align}
    x_{1}+ 2x_{2}+ 0.x_{3} -2x_{4} = 0\\
    x_{1}+ 0.x_{2} -2x_{3} +0.x_{4}= 0\\
    3x_{1}+ 2x_{2}-6x_{3}- x_{4} =0\\
    0.x_{1} +x_{2}-x_{3} +0.x_{4}= 0
\end{align}

resulting in the matrix equation
\begin{align}
    \label{eq:solutions/chem/6a matrix}
    \myvec{1 & 2 & 0 & -2\\
           1 & 0 & -2 & 0\\
           3 & 2 & -6 & -1\\
           0 & 1 & -1 & 0}\vec{x}
           =\vec{0}
\end{align}

where,
\begin{align}
   \vec{x}= \myvec{x_{1}\\x_{2}\\x_{3}\\x_{4}}
\end{align}

(\ref{eq:solutions/chem/6a matrix}) can be reduced as follows:
\begin{align}
    \myvec{1 & 2 & 0 & -2\\
           1 & 0 & -2 & 0\\
           3 & 2 & -6 & -1\\
           0 & 1 & -1 & 0}
    \xleftrightarrow[R_{3}\leftarrow \frac{R_3}{3}-R_{1}]{R_{2}\leftarrow R_2- R_1}
    \myvec{1 & 2 & 0 & -2\\
           0 & -2 & -2 & 2\\
           0 & -\frac{4}{3} & -2 & \frac{5}{3}\\
           0 & 1 & -1 & 0}\\
    \xleftrightarrow{R_2 \leftarrow -\frac{R_2}{2}}
    \myvec{1 & 2 & 0 & -2\\
          0 & 1 & 1 & -1\\
          0 & -\frac{4}{3} & -2 & \frac{5}{3}\\
          0 & 1 & -1 & 0}\\
    \xleftrightarrow[R_4 \leftarrow R_4- R_2]{R_3 \leftarrow R_3 + \frac{4}{3}R_2}
    \myvec{1 & 2 & 0 & -2\\
           0 & 1 & 1 & -1\\
           0 & 0 & -\frac{2}{3} & \frac{1}{3}\\
           0 & 0 & -2 & 1}\\
    \xleftrightarrow[R_3 \leftarrow -\frac{3}{2}R_3]{R_1 \leftarrow R_1- 2R_2}
    \myvec{1 & 0 & -2 & 0\\
           0 & 1 & 1 & -1\\
           0 & 0 & 1 & -\frac{1}{2}\\
           0 & 0 & -2 & 1}\\
    \xleftrightarrow{R_4\leftarrow R_4 + 2R_3}
    \myvec{1 & 0 & -2 & 0\\
           0 & 1 & 1 & -1\\
           0 & 0 & 1 & -\frac{1}{2}\\
           0 & 0 & 0 & 0}\\
    \xleftrightarrow[R_2\leftarrow R_2-R_3]{R_1\leftarrow R_1 + 2R_3}
    \myvec{1 & 0 & 0 & -1\\
           0 & 1 & 0 & -\frac{1}{2}\\
           0 & 0 & 1 & -\frac{1}{2}\\
           0 & 0 & 0 & 0}
\end{align}

Thus,
\begin{align}
    x_1=x_4, x_2= \frac{1}{2}x_4, x_3=\frac{1}{2}x_4\\
    \implies \quad\vec{x}= x_4\myvec{1\\ \frac{1}{2}\\ \frac{1}{2}\\1} =\myvec{2\\1\\1\\2}
\end{align} 
by substituting $x_4= 2$.

\hfill\break
%\vspace{5mm} 
Hence, (\ref{eq:solutions/chem/6abalanced}) finally becomes
\begin{align}
    2HNO_{3}+ Ca(OH)_{2}\to Ca(NO_{3})_{2}+ 2H_{2}O
\end{align}

\item Find the area lying in the first quadrant and bounded by the circle $\vec{x}^T\vec{x}=4$ and the lines $x = 0$ and $x = 2$.
%
\item Find the area of the circle $4\vec{x}^T\vec{x}=9$.
\item  Find the area bounded by curves $\norm{\vec{x}-\myvec{1\\0}} = 1$ and $\norm{\vec{x}}=1$
\\
\solution
Let the balanced version of (\ref{eq:solutions/chem/6ato balance}) be
\begin{align}
    \label{eq:solutions/chem/6abalanced}x_{1}HNO_{3}+ x_{2}Ca(OH)_{2}\to x_{3}Ca(NO_{3})_{2}+ x_{4}H_{2}O
\end{align}

which results in the following equations:
\begin{align}
    (x_{1}+ 2x_{2}-2x_{4}) H= 0\\
    (x_{1}-2x_{3}) N= 0\\
    (3x_{1}+ 2x_{2}-6x_{3}- x_{4}) O=0\\
    (x_{2}-x_{3}) Ca= 0
\end{align}

which can be expressed as
\begin{align}
    x_{1}+ 2x_{2}+ 0.x_{3} -2x_{4} = 0\\
    x_{1}+ 0.x_{2} -2x_{3} +0.x_{4}= 0\\
    3x_{1}+ 2x_{2}-6x_{3}- x_{4} =0\\
    0.x_{1} +x_{2}-x_{3} +0.x_{4}= 0
\end{align}

resulting in the matrix equation
\begin{align}
    \label{eq:solutions/chem/6a matrix}
    \myvec{1 & 2 & 0 & -2\\
           1 & 0 & -2 & 0\\
           3 & 2 & -6 & -1\\
           0 & 1 & -1 & 0}\vec{x}
           =\vec{0}
\end{align}

where,
\begin{align}
   \vec{x}= \myvec{x_{1}\\x_{2}\\x_{3}\\x_{4}}
\end{align}

(\ref{eq:solutions/chem/6a matrix}) can be reduced as follows:
\begin{align}
    \myvec{1 & 2 & 0 & -2\\
           1 & 0 & -2 & 0\\
           3 & 2 & -6 & -1\\
           0 & 1 & -1 & 0}
    \xleftrightarrow[R_{3}\leftarrow \frac{R_3}{3}-R_{1}]{R_{2}\leftarrow R_2- R_1}
    \myvec{1 & 2 & 0 & -2\\
           0 & -2 & -2 & 2\\
           0 & -\frac{4}{3} & -2 & \frac{5}{3}\\
           0 & 1 & -1 & 0}\\
    \xleftrightarrow{R_2 \leftarrow -\frac{R_2}{2}}
    \myvec{1 & 2 & 0 & -2\\
          0 & 1 & 1 & -1\\
          0 & -\frac{4}{3} & -2 & \frac{5}{3}\\
          0 & 1 & -1 & 0}\\
    \xleftrightarrow[R_4 \leftarrow R_4- R_2]{R_3 \leftarrow R_3 + \frac{4}{3}R_2}
    \myvec{1 & 2 & 0 & -2\\
           0 & 1 & 1 & -1\\
           0 & 0 & -\frac{2}{3} & \frac{1}{3}\\
           0 & 0 & -2 & 1}\\
    \xleftrightarrow[R_3 \leftarrow -\frac{3}{2}R_3]{R_1 \leftarrow R_1- 2R_2}
    \myvec{1 & 0 & -2 & 0\\
           0 & 1 & 1 & -1\\
           0 & 0 & 1 & -\frac{1}{2}\\
           0 & 0 & -2 & 1}\\
    \xleftrightarrow{R_4\leftarrow R_4 + 2R_3}
    \myvec{1 & 0 & -2 & 0\\
           0 & 1 & 1 & -1\\
           0 & 0 & 1 & -\frac{1}{2}\\
           0 & 0 & 0 & 0}\\
    \xleftrightarrow[R_2\leftarrow R_2-R_3]{R_1\leftarrow R_1 + 2R_3}
    \myvec{1 & 0 & 0 & -1\\
           0 & 1 & 0 & -\frac{1}{2}\\
           0 & 0 & 1 & -\frac{1}{2}\\
           0 & 0 & 0 & 0}
\end{align}

Thus,
\begin{align}
    x_1=x_4, x_2= \frac{1}{2}x_4, x_3=\frac{1}{2}x_4\\
    \implies \quad\vec{x}= x_4\myvec{1\\ \frac{1}{2}\\ \frac{1}{2}\\1} =\myvec{2\\1\\1\\2}
\end{align} 
by substituting $x_4= 2$.

\hfill\break
%\vspace{5mm} 
Hence, (\ref{eq:solutions/chem/6abalanced}) finally becomes
\begin{align}
    2HNO_{3}+ Ca(OH)_{2}\to Ca(NO_{3})_{2}+ 2H_{2}O
\end{align}

\item Find the smaller area enclosed by the circle $\vec{x}^T\vec{x}=4$ and the line $\myvec{1 & 1}\vec{x} = 2$.


\item Find the slope of the tangent to the curve $y = \frac{x-1}{x-2}, x\ne 2$ at $x = 10$.
\\
\solution
Let the balanced version of (\ref{eq:solutions/chem/6ato balance}) be
\begin{align}
    \label{eq:solutions/chem/6abalanced}x_{1}HNO_{3}+ x_{2}Ca(OH)_{2}\to x_{3}Ca(NO_{3})_{2}+ x_{4}H_{2}O
\end{align}

which results in the following equations:
\begin{align}
    (x_{1}+ 2x_{2}-2x_{4}) H= 0\\
    (x_{1}-2x_{3}) N= 0\\
    (3x_{1}+ 2x_{2}-6x_{3}- x_{4}) O=0\\
    (x_{2}-x_{3}) Ca= 0
\end{align}

which can be expressed as
\begin{align}
    x_{1}+ 2x_{2}+ 0.x_{3} -2x_{4} = 0\\
    x_{1}+ 0.x_{2} -2x_{3} +0.x_{4}= 0\\
    3x_{1}+ 2x_{2}-6x_{3}- x_{4} =0\\
    0.x_{1} +x_{2}-x_{3} +0.x_{4}= 0
\end{align}

resulting in the matrix equation
\begin{align}
    \label{eq:solutions/chem/6a matrix}
    \myvec{1 & 2 & 0 & -2\\
           1 & 0 & -2 & 0\\
           3 & 2 & -6 & -1\\
           0 & 1 & -1 & 0}\vec{x}
           =\vec{0}
\end{align}

where,
\begin{align}
   \vec{x}= \myvec{x_{1}\\x_{2}\\x_{3}\\x_{4}}
\end{align}

(\ref{eq:solutions/chem/6a matrix}) can be reduced as follows:
\begin{align}
    \myvec{1 & 2 & 0 & -2\\
           1 & 0 & -2 & 0\\
           3 & 2 & -6 & -1\\
           0 & 1 & -1 & 0}
    \xleftrightarrow[R_{3}\leftarrow \frac{R_3}{3}-R_{1}]{R_{2}\leftarrow R_2- R_1}
    \myvec{1 & 2 & 0 & -2\\
           0 & -2 & -2 & 2\\
           0 & -\frac{4}{3} & -2 & \frac{5}{3}\\
           0 & 1 & -1 & 0}\\
    \xleftrightarrow{R_2 \leftarrow -\frac{R_2}{2}}
    \myvec{1 & 2 & 0 & -2\\
          0 & 1 & 1 & -1\\
          0 & -\frac{4}{3} & -2 & \frac{5}{3}\\
          0 & 1 & -1 & 0}\\
    \xleftrightarrow[R_4 \leftarrow R_4- R_2]{R_3 \leftarrow R_3 + \frac{4}{3}R_2}
    \myvec{1 & 2 & 0 & -2\\
           0 & 1 & 1 & -1\\
           0 & 0 & -\frac{2}{3} & \frac{1}{3}\\
           0 & 0 & -2 & 1}\\
    \xleftrightarrow[R_3 \leftarrow -\frac{3}{2}R_3]{R_1 \leftarrow R_1- 2R_2}
    \myvec{1 & 0 & -2 & 0\\
           0 & 1 & 1 & -1\\
           0 & 0 & 1 & -\frac{1}{2}\\
           0 & 0 & -2 & 1}\\
    \xleftrightarrow{R_4\leftarrow R_4 + 2R_3}
    \myvec{1 & 0 & -2 & 0\\
           0 & 1 & 1 & -1\\
           0 & 0 & 1 & -\frac{1}{2}\\
           0 & 0 & 0 & 0}\\
    \xleftrightarrow[R_2\leftarrow R_2-R_3]{R_1\leftarrow R_1 + 2R_3}
    \myvec{1 & 0 & 0 & -1\\
           0 & 1 & 0 & -\frac{1}{2}\\
           0 & 0 & 1 & -\frac{1}{2}\\
           0 & 0 & 0 & 0}
\end{align}

Thus,
\begin{align}
    x_1=x_4, x_2= \frac{1}{2}x_4, x_3=\frac{1}{2}x_4\\
    \implies \quad\vec{x}= x_4\myvec{1\\ \frac{1}{2}\\ \frac{1}{2}\\1} =\myvec{2\\1\\1\\2}
\end{align} 
by substituting $x_4= 2$.

\hfill\break
%\vspace{5mm} 
Hence, (\ref{eq:solutions/chem/6abalanced}) finally becomes
\begin{align}
    2HNO_{3}+ Ca(OH)_{2}\to Ca(NO_{3})_{2}+ 2H_{2}O
\end{align}

\item Find a point on the curve $y = (x – 2)^2$ at which the tangent is parallel to the chord joining the points \myvec{2\\ 0} and \myvec{4\\ 4}.
\\
\solution
Let the balanced version of (\ref{eq:solutions/chem/6ato balance}) be
\begin{align}
    \label{eq:solutions/chem/6abalanced}x_{1}HNO_{3}+ x_{2}Ca(OH)_{2}\to x_{3}Ca(NO_{3})_{2}+ x_{4}H_{2}O
\end{align}

which results in the following equations:
\begin{align}
    (x_{1}+ 2x_{2}-2x_{4}) H= 0\\
    (x_{1}-2x_{3}) N= 0\\
    (3x_{1}+ 2x_{2}-6x_{3}- x_{4}) O=0\\
    (x_{2}-x_{3}) Ca= 0
\end{align}

which can be expressed as
\begin{align}
    x_{1}+ 2x_{2}+ 0.x_{3} -2x_{4} = 0\\
    x_{1}+ 0.x_{2} -2x_{3} +0.x_{4}= 0\\
    3x_{1}+ 2x_{2}-6x_{3}- x_{4} =0\\
    0.x_{1} +x_{2}-x_{3} +0.x_{4}= 0
\end{align}

resulting in the matrix equation
\begin{align}
    \label{eq:solutions/chem/6a matrix}
    \myvec{1 & 2 & 0 & -2\\
           1 & 0 & -2 & 0\\
           3 & 2 & -6 & -1\\
           0 & 1 & -1 & 0}\vec{x}
           =\vec{0}
\end{align}

where,
\begin{align}
   \vec{x}= \myvec{x_{1}\\x_{2}\\x_{3}\\x_{4}}
\end{align}

(\ref{eq:solutions/chem/6a matrix}) can be reduced as follows:
\begin{align}
    \myvec{1 & 2 & 0 & -2\\
           1 & 0 & -2 & 0\\
           3 & 2 & -6 & -1\\
           0 & 1 & -1 & 0}
    \xleftrightarrow[R_{3}\leftarrow \frac{R_3}{3}-R_{1}]{R_{2}\leftarrow R_2- R_1}
    \myvec{1 & 2 & 0 & -2\\
           0 & -2 & -2 & 2\\
           0 & -\frac{4}{3} & -2 & \frac{5}{3}\\
           0 & 1 & -1 & 0}\\
    \xleftrightarrow{R_2 \leftarrow -\frac{R_2}{2}}
    \myvec{1 & 2 & 0 & -2\\
          0 & 1 & 1 & -1\\
          0 & -\frac{4}{3} & -2 & \frac{5}{3}\\
          0 & 1 & -1 & 0}\\
    \xleftrightarrow[R_4 \leftarrow R_4- R_2]{R_3 \leftarrow R_3 + \frac{4}{3}R_2}
    \myvec{1 & 2 & 0 & -2\\
           0 & 1 & 1 & -1\\
           0 & 0 & -\frac{2}{3} & \frac{1}{3}\\
           0 & 0 & -2 & 1}\\
    \xleftrightarrow[R_3 \leftarrow -\frac{3}{2}R_3]{R_1 \leftarrow R_1- 2R_2}
    \myvec{1 & 0 & -2 & 0\\
           0 & 1 & 1 & -1\\
           0 & 0 & 1 & -\frac{1}{2}\\
           0 & 0 & -2 & 1}\\
    \xleftrightarrow{R_4\leftarrow R_4 + 2R_3}
    \myvec{1 & 0 & -2 & 0\\
           0 & 1 & 1 & -1\\
           0 & 0 & 1 & -\frac{1}{2}\\
           0 & 0 & 0 & 0}\\
    \xleftrightarrow[R_2\leftarrow R_2-R_3]{R_1\leftarrow R_1 + 2R_3}
    \myvec{1 & 0 & 0 & -1\\
           0 & 1 & 0 & -\frac{1}{2}\\
           0 & 0 & 1 & -\frac{1}{2}\\
           0 & 0 & 0 & 0}
\end{align}

Thus,
\begin{align}
    x_1=x_4, x_2= \frac{1}{2}x_4, x_3=\frac{1}{2}x_4\\
    \implies \quad\vec{x}= x_4\myvec{1\\ \frac{1}{2}\\ \frac{1}{2}\\1} =\myvec{2\\1\\1\\2}
\end{align} 
by substituting $x_4= 2$.

\hfill\break
%\vspace{5mm} 
Hence, (\ref{eq:solutions/chem/6abalanced}) finally becomes
\begin{align}
    2HNO_{3}+ Ca(OH)_{2}\to Ca(NO_{3})_{2}+ 2H_{2}O
\end{align}

\item Find the equation of all lines having slope – 1 that are tangents to the curve $\frac{1}
{x -1}, x \ne 1$
\\
\solution 
Let the balanced version of (\ref{eq:solutions/chem/6ato balance}) be
\begin{align}
    \label{eq:solutions/chem/6abalanced}x_{1}HNO_{3}+ x_{2}Ca(OH)_{2}\to x_{3}Ca(NO_{3})_{2}+ x_{4}H_{2}O
\end{align}

which results in the following equations:
\begin{align}
    (x_{1}+ 2x_{2}-2x_{4}) H= 0\\
    (x_{1}-2x_{3}) N= 0\\
    (3x_{1}+ 2x_{2}-6x_{3}- x_{4}) O=0\\
    (x_{2}-x_{3}) Ca= 0
\end{align}

which can be expressed as
\begin{align}
    x_{1}+ 2x_{2}+ 0.x_{3} -2x_{4} = 0\\
    x_{1}+ 0.x_{2} -2x_{3} +0.x_{4}= 0\\
    3x_{1}+ 2x_{2}-6x_{3}- x_{4} =0\\
    0.x_{1} +x_{2}-x_{3} +0.x_{4}= 0
\end{align}

resulting in the matrix equation
\begin{align}
    \label{eq:solutions/chem/6a matrix}
    \myvec{1 & 2 & 0 & -2\\
           1 & 0 & -2 & 0\\
           3 & 2 & -6 & -1\\
           0 & 1 & -1 & 0}\vec{x}
           =\vec{0}
\end{align}

where,
\begin{align}
   \vec{x}= \myvec{x_{1}\\x_{2}\\x_{3}\\x_{4}}
\end{align}

(\ref{eq:solutions/chem/6a matrix}) can be reduced as follows:
\begin{align}
    \myvec{1 & 2 & 0 & -2\\
           1 & 0 & -2 & 0\\
           3 & 2 & -6 & -1\\
           0 & 1 & -1 & 0}
    \xleftrightarrow[R_{3}\leftarrow \frac{R_3}{3}-R_{1}]{R_{2}\leftarrow R_2- R_1}
    \myvec{1 & 2 & 0 & -2\\
           0 & -2 & -2 & 2\\
           0 & -\frac{4}{3} & -2 & \frac{5}{3}\\
           0 & 1 & -1 & 0}\\
    \xleftrightarrow{R_2 \leftarrow -\frac{R_2}{2}}
    \myvec{1 & 2 & 0 & -2\\
          0 & 1 & 1 & -1\\
          0 & -\frac{4}{3} & -2 & \frac{5}{3}\\
          0 & 1 & -1 & 0}\\
    \xleftrightarrow[R_4 \leftarrow R_4- R_2]{R_3 \leftarrow R_3 + \frac{4}{3}R_2}
    \myvec{1 & 2 & 0 & -2\\
           0 & 1 & 1 & -1\\
           0 & 0 & -\frac{2}{3} & \frac{1}{3}\\
           0 & 0 & -2 & 1}\\
    \xleftrightarrow[R_3 \leftarrow -\frac{3}{2}R_3]{R_1 \leftarrow R_1- 2R_2}
    \myvec{1 & 0 & -2 & 0\\
           0 & 1 & 1 & -1\\
           0 & 0 & 1 & -\frac{1}{2}\\
           0 & 0 & -2 & 1}\\
    \xleftrightarrow{R_4\leftarrow R_4 + 2R_3}
    \myvec{1 & 0 & -2 & 0\\
           0 & 1 & 1 & -1\\
           0 & 0 & 1 & -\frac{1}{2}\\
           0 & 0 & 0 & 0}\\
    \xleftrightarrow[R_2\leftarrow R_2-R_3]{R_1\leftarrow R_1 + 2R_3}
    \myvec{1 & 0 & 0 & -1\\
           0 & 1 & 0 & -\frac{1}{2}\\
           0 & 0 & 1 & -\frac{1}{2}\\
           0 & 0 & 0 & 0}
\end{align}

Thus,
\begin{align}
    x_1=x_4, x_2= \frac{1}{2}x_4, x_3=\frac{1}{2}x_4\\
    \implies \quad\vec{x}= x_4\myvec{1\\ \frac{1}{2}\\ \frac{1}{2}\\1} =\myvec{2\\1\\1\\2}
\end{align} 
by substituting $x_4= 2$.

\hfill\break
%\vspace{5mm} 
Hence, (\ref{eq:solutions/chem/6abalanced}) finally becomes
\begin{align}
    2HNO_{3}+ Ca(OH)_{2}\to Ca(NO_{3})_{2}+ 2H_{2}O
\end{align}

\item Find the equation of all lines having slope -2 which are tangents to the curve $\frac{1}
{x - 3} , x \ne 3$.
%
\\
\solution 
Let the balanced version of (\ref{eq:solutions/chem/6ato balance}) be
\begin{align}
    \label{eq:solutions/chem/6abalanced}x_{1}HNO_{3}+ x_{2}Ca(OH)_{2}\to x_{3}Ca(NO_{3})_{2}+ x_{4}H_{2}O
\end{align}

which results in the following equations:
\begin{align}
    (x_{1}+ 2x_{2}-2x_{4}) H= 0\\
    (x_{1}-2x_{3}) N= 0\\
    (3x_{1}+ 2x_{2}-6x_{3}- x_{4}) O=0\\
    (x_{2}-x_{3}) Ca= 0
\end{align}

which can be expressed as
\begin{align}
    x_{1}+ 2x_{2}+ 0.x_{3} -2x_{4} = 0\\
    x_{1}+ 0.x_{2} -2x_{3} +0.x_{4}= 0\\
    3x_{1}+ 2x_{2}-6x_{3}- x_{4} =0\\
    0.x_{1} +x_{2}-x_{3} +0.x_{4}= 0
\end{align}

resulting in the matrix equation
\begin{align}
    \label{eq:solutions/chem/6a matrix}
    \myvec{1 & 2 & 0 & -2\\
           1 & 0 & -2 & 0\\
           3 & 2 & -6 & -1\\
           0 & 1 & -1 & 0}\vec{x}
           =\vec{0}
\end{align}

where,
\begin{align}
   \vec{x}= \myvec{x_{1}\\x_{2}\\x_{3}\\x_{4}}
\end{align}

(\ref{eq:solutions/chem/6a matrix}) can be reduced as follows:
\begin{align}
    \myvec{1 & 2 & 0 & -2\\
           1 & 0 & -2 & 0\\
           3 & 2 & -6 & -1\\
           0 & 1 & -1 & 0}
    \xleftrightarrow[R_{3}\leftarrow \frac{R_3}{3}-R_{1}]{R_{2}\leftarrow R_2- R_1}
    \myvec{1 & 2 & 0 & -2\\
           0 & -2 & -2 & 2\\
           0 & -\frac{4}{3} & -2 & \frac{5}{3}\\
           0 & 1 & -1 & 0}\\
    \xleftrightarrow{R_2 \leftarrow -\frac{R_2}{2}}
    \myvec{1 & 2 & 0 & -2\\
          0 & 1 & 1 & -1\\
          0 & -\frac{4}{3} & -2 & \frac{5}{3}\\
          0 & 1 & -1 & 0}\\
    \xleftrightarrow[R_4 \leftarrow R_4- R_2]{R_3 \leftarrow R_3 + \frac{4}{3}R_2}
    \myvec{1 & 2 & 0 & -2\\
           0 & 1 & 1 & -1\\
           0 & 0 & -\frac{2}{3} & \frac{1}{3}\\
           0 & 0 & -2 & 1}\\
    \xleftrightarrow[R_3 \leftarrow -\frac{3}{2}R_3]{R_1 \leftarrow R_1- 2R_2}
    \myvec{1 & 0 & -2 & 0\\
           0 & 1 & 1 & -1\\
           0 & 0 & 1 & -\frac{1}{2}\\
           0 & 0 & -2 & 1}\\
    \xleftrightarrow{R_4\leftarrow R_4 + 2R_3}
    \myvec{1 & 0 & -2 & 0\\
           0 & 1 & 1 & -1\\
           0 & 0 & 1 & -\frac{1}{2}\\
           0 & 0 & 0 & 0}\\
    \xleftrightarrow[R_2\leftarrow R_2-R_3]{R_1\leftarrow R_1 + 2R_3}
    \myvec{1 & 0 & 0 & -1\\
           0 & 1 & 0 & -\frac{1}{2}\\
           0 & 0 & 1 & -\frac{1}{2}\\
           0 & 0 & 0 & 0}
\end{align}

Thus,
\begin{align}
    x_1=x_4, x_2= \frac{1}{2}x_4, x_3=\frac{1}{2}x_4\\
    \implies \quad\vec{x}= x_4\myvec{1\\ \frac{1}{2}\\ \frac{1}{2}\\1} =\myvec{2\\1\\1\\2}
\end{align} 
by substituting $x_4= 2$.

\hfill\break
%\vspace{5mm} 
Hence, (\ref{eq:solutions/chem/6abalanced}) finally becomes
\begin{align}
    2HNO_{3}+ Ca(OH)_{2}\to Ca(NO_{3})_{2}+ 2H_{2}O
\end{align}

\item Find points on the curve 
$
\vec{x}^T\myvec{\frac{1}{9} & 0 \\ 0 & \frac{1}{16}}\vec{x} = 1
$
%
at which tangents are
\begin{enumerate}
\item  parallel to x-axis
\item  parallel to y-axis.
\end{enumerate}
\solution 
Let the balanced version of (\ref{eq:solutions/chem/6ato balance}) be
\begin{align}
    \label{eq:solutions/chem/6abalanced}x_{1}HNO_{3}+ x_{2}Ca(OH)_{2}\to x_{3}Ca(NO_{3})_{2}+ x_{4}H_{2}O
\end{align}

which results in the following equations:
\begin{align}
    (x_{1}+ 2x_{2}-2x_{4}) H= 0\\
    (x_{1}-2x_{3}) N= 0\\
    (3x_{1}+ 2x_{2}-6x_{3}- x_{4}) O=0\\
    (x_{2}-x_{3}) Ca= 0
\end{align}

which can be expressed as
\begin{align}
    x_{1}+ 2x_{2}+ 0.x_{3} -2x_{4} = 0\\
    x_{1}+ 0.x_{2} -2x_{3} +0.x_{4}= 0\\
    3x_{1}+ 2x_{2}-6x_{3}- x_{4} =0\\
    0.x_{1} +x_{2}-x_{3} +0.x_{4}= 0
\end{align}

resulting in the matrix equation
\begin{align}
    \label{eq:solutions/chem/6a matrix}
    \myvec{1 & 2 & 0 & -2\\
           1 & 0 & -2 & 0\\
           3 & 2 & -6 & -1\\
           0 & 1 & -1 & 0}\vec{x}
           =\vec{0}
\end{align}

where,
\begin{align}
   \vec{x}= \myvec{x_{1}\\x_{2}\\x_{3}\\x_{4}}
\end{align}

(\ref{eq:solutions/chem/6a matrix}) can be reduced as follows:
\begin{align}
    \myvec{1 & 2 & 0 & -2\\
           1 & 0 & -2 & 0\\
           3 & 2 & -6 & -1\\
           0 & 1 & -1 & 0}
    \xleftrightarrow[R_{3}\leftarrow \frac{R_3}{3}-R_{1}]{R_{2}\leftarrow R_2- R_1}
    \myvec{1 & 2 & 0 & -2\\
           0 & -2 & -2 & 2\\
           0 & -\frac{4}{3} & -2 & \frac{5}{3}\\
           0 & 1 & -1 & 0}\\
    \xleftrightarrow{R_2 \leftarrow -\frac{R_2}{2}}
    \myvec{1 & 2 & 0 & -2\\
          0 & 1 & 1 & -1\\
          0 & -\frac{4}{3} & -2 & \frac{5}{3}\\
          0 & 1 & -1 & 0}\\
    \xleftrightarrow[R_4 \leftarrow R_4- R_2]{R_3 \leftarrow R_3 + \frac{4}{3}R_2}
    \myvec{1 & 2 & 0 & -2\\
           0 & 1 & 1 & -1\\
           0 & 0 & -\frac{2}{3} & \frac{1}{3}\\
           0 & 0 & -2 & 1}\\
    \xleftrightarrow[R_3 \leftarrow -\frac{3}{2}R_3]{R_1 \leftarrow R_1- 2R_2}
    \myvec{1 & 0 & -2 & 0\\
           0 & 1 & 1 & -1\\
           0 & 0 & 1 & -\frac{1}{2}\\
           0 & 0 & -2 & 1}\\
    \xleftrightarrow{R_4\leftarrow R_4 + 2R_3}
    \myvec{1 & 0 & -2 & 0\\
           0 & 1 & 1 & -1\\
           0 & 0 & 1 & -\frac{1}{2}\\
           0 & 0 & 0 & 0}\\
    \xleftrightarrow[R_2\leftarrow R_2-R_3]{R_1\leftarrow R_1 + 2R_3}
    \myvec{1 & 0 & 0 & -1\\
           0 & 1 & 0 & -\frac{1}{2}\\
           0 & 0 & 1 & -\frac{1}{2}\\
           0 & 0 & 0 & 0}
\end{align}

Thus,
\begin{align}
    x_1=x_4, x_2= \frac{1}{2}x_4, x_3=\frac{1}{2}x_4\\
    \implies \quad\vec{x}= x_4\myvec{1\\ \frac{1}{2}\\ \frac{1}{2}\\1} =\myvec{2\\1\\1\\2}
\end{align} 
by substituting $x_4= 2$.

\hfill\break
%\vspace{5mm} 
Hence, (\ref{eq:solutions/chem/6abalanced}) finally becomes
\begin{align}
    2HNO_{3}+ Ca(OH)_{2}\to Ca(NO_{3})_{2}+ 2H_{2}O
\end{align}

\item Find the equations of the tangent and normal to the given curves at the indicated points:
$
y = x^2
$
at \myvec{0\\0}.
\item Find the equation of the tangent line to the curve $y = x^2-2x+7$
\begin{enumerate}
%
\item  parallel to the line $\myvec{2 & -1}\vec{x}= -9$ 
\item  perpendicular to the line $\myvec{-15 & 5}\vec{x} = 13$. 
\end{enumerate}
\item  Find the equation of the tangent to the curve,
\begin{align}
y = \sqrt{3x-2}
\label{eq:solutions/1/19/Q}
\end{align}
 which is parallel to the line,
\begin{align}
\myvec{4&2}\vec{x}+5=0
\label{eq:solutions/1/19/P}
\end{align}  
\solution 
Let the balanced version of (\ref{eq:solutions/chem/6ato balance}) be
\begin{align}
    \label{eq:solutions/chem/6abalanced}x_{1}HNO_{3}+ x_{2}Ca(OH)_{2}\to x_{3}Ca(NO_{3})_{2}+ x_{4}H_{2}O
\end{align}

which results in the following equations:
\begin{align}
    (x_{1}+ 2x_{2}-2x_{4}) H= 0\\
    (x_{1}-2x_{3}) N= 0\\
    (3x_{1}+ 2x_{2}-6x_{3}- x_{4}) O=0\\
    (x_{2}-x_{3}) Ca= 0
\end{align}

which can be expressed as
\begin{align}
    x_{1}+ 2x_{2}+ 0.x_{3} -2x_{4} = 0\\
    x_{1}+ 0.x_{2} -2x_{3} +0.x_{4}= 0\\
    3x_{1}+ 2x_{2}-6x_{3}- x_{4} =0\\
    0.x_{1} +x_{2}-x_{3} +0.x_{4}= 0
\end{align}

resulting in the matrix equation
\begin{align}
    \label{eq:solutions/chem/6a matrix}
    \myvec{1 & 2 & 0 & -2\\
           1 & 0 & -2 & 0\\
           3 & 2 & -6 & -1\\
           0 & 1 & -1 & 0}\vec{x}
           =\vec{0}
\end{align}

where,
\begin{align}
   \vec{x}= \myvec{x_{1}\\x_{2}\\x_{3}\\x_{4}}
\end{align}

(\ref{eq:solutions/chem/6a matrix}) can be reduced as follows:
\begin{align}
    \myvec{1 & 2 & 0 & -2\\
           1 & 0 & -2 & 0\\
           3 & 2 & -6 & -1\\
           0 & 1 & -1 & 0}
    \xleftrightarrow[R_{3}\leftarrow \frac{R_3}{3}-R_{1}]{R_{2}\leftarrow R_2- R_1}
    \myvec{1 & 2 & 0 & -2\\
           0 & -2 & -2 & 2\\
           0 & -\frac{4}{3} & -2 & \frac{5}{3}\\
           0 & 1 & -1 & 0}\\
    \xleftrightarrow{R_2 \leftarrow -\frac{R_2}{2}}
    \myvec{1 & 2 & 0 & -2\\
          0 & 1 & 1 & -1\\
          0 & -\frac{4}{3} & -2 & \frac{5}{3}\\
          0 & 1 & -1 & 0}\\
    \xleftrightarrow[R_4 \leftarrow R_4- R_2]{R_3 \leftarrow R_3 + \frac{4}{3}R_2}
    \myvec{1 & 2 & 0 & -2\\
           0 & 1 & 1 & -1\\
           0 & 0 & -\frac{2}{3} & \frac{1}{3}\\
           0 & 0 & -2 & 1}\\
    \xleftrightarrow[R_3 \leftarrow -\frac{3}{2}R_3]{R_1 \leftarrow R_1- 2R_2}
    \myvec{1 & 0 & -2 & 0\\
           0 & 1 & 1 & -1\\
           0 & 0 & 1 & -\frac{1}{2}\\
           0 & 0 & -2 & 1}\\
    \xleftrightarrow{R_4\leftarrow R_4 + 2R_3}
    \myvec{1 & 0 & -2 & 0\\
           0 & 1 & 1 & -1\\
           0 & 0 & 1 & -\frac{1}{2}\\
           0 & 0 & 0 & 0}\\
    \xleftrightarrow[R_2\leftarrow R_2-R_3]{R_1\leftarrow R_1 + 2R_3}
    \myvec{1 & 0 & 0 & -1\\
           0 & 1 & 0 & -\frac{1}{2}\\
           0 & 0 & 1 & -\frac{1}{2}\\
           0 & 0 & 0 & 0}
\end{align}

Thus,
\begin{align}
    x_1=x_4, x_2= \frac{1}{2}x_4, x_3=\frac{1}{2}x_4\\
    \implies \quad\vec{x}= x_4\myvec{1\\ \frac{1}{2}\\ \frac{1}{2}\\1} =\myvec{2\\1\\1\\2}
\end{align} 
by substituting $x_4= 2$.

\hfill\break
%\vspace{5mm} 
Hence, (\ref{eq:solutions/chem/6abalanced}) finally becomes
\begin{align}
    2HNO_{3}+ Ca(OH)_{2}\to Ca(NO_{3})_{2}+ 2H_{2}O
\end{align}

%
\item Find the point at which the line $\myvec{-1 & 1}\vec{x} =  1$ is a tangent to the curve $y^2 = 4x$.
%
\item The line $\myvec{-m & 1}\vec{x} = 1$ is a tangent to the curve $y^2 = 4x$.  Find the value of $m$.
\item  Find the normal at the point \myvec{1\\1} on the curve $2y + x^2 = 3$ 
\item  Find the normal to the curve $x^2=4y$ passing through $\myvec{1\\2}$.
%
\item Find the area of the region bounded by the curve $y^2= x$ and the lines $x = 1, x = 4$ and the x-axis in the first quadrant.
\item  Find the area of the region bounded by $y^2=9x, x=2, x=4$ and the x-axis in the  first quadrant.
%
\item Find the area of the region bounded by $x^2 = 4y, y = 2, y = 4$ and the y-axis in the first quadrant.
\item Find the area of the region bounded by the ellipse 
$
\vec{x}^T\myvec{\frac{1}{16} & 0 \\ 0 & \frac{1}{9}}\vec{x} = 1
$

\item  Find the area of the region bounded by the ellipse 
$
\vec{x}^T\myvec{\frac{1}{4} & 0 \\ 0 & \frac{1}{9}}\vec{x} = 1
$
\item The area between $x=y^2$ and $x=4$ is divided into two equal parts by the line $x=a$, find the value of $a$.
\item  Find the area of the region bounded by the parabola $y = x^2$ and $y = \abs{x}$.
\item  Find the area bounded by the curve $x^2 = 4y$ and the line $\myvec{1 & -1}\vec{x} = -2$.
\item  Find the area of the region bounded by the curve $y^2 = 4x$ and the line $x = 3$.
%
\item Find the area of the region bounded by the curve $y^2 = x$, y-axis and the line $y = 3$.
%
\item Find the area of the region bounded by the two parabolas $y = x^2, y^2=x$.
\item Find the area lying above x-axis and included between the circle $\vec{x}^T\vec{x} -8\myvec{1 & 0}= 0$  and inside of the parabola $y^2 = 4x$.
%
\item AOBA is the part of the ellipse 
$
\vec{x}^T\myvec{9 & 0 \\ 0 & 1}\vec{x} = 36
$
in the first quadrant such that $OA = 2$ and $OB = 6$. Find the area between the arc $AB$ and the chord $AB$.
\item Find the area lying between the curves $y^2 = 4x$ and $y = 2x$.
\item  Find the area of the region bounded by the curves $y = x^2+2, y = x, x = 0$ and $ x = 3.$
%
\item Find the area under $y = x^2, x = 1, x = 2$ and x-axis.
\item Find the area between  $y = x^2$ and $y = x$.
\item Find the area of the region lying in the first quadrant and bounded by $y = 4x^2, x = 0, y = 1$ and $y = 4$.
\item Find the area enclosed by the parabola $4y = 3x^2$ and the line $\myvec{-3 & 2}\vec{x} = 12$.
%
\item Find the area of the smaller region bounded by the ellipse
$
\vec{x}^T\myvec{\frac{1}{9} & 0 \\ 0 & \frac{1}{4}}\vec{x} = 1
$
and the line 
$
\myvec{\frac{1}{a} & \frac{1}{b}}\vec{x} = 1
$
\item Find the area of the region enclosed by the parabola $x^2=y$, the line $\myvec{-1 & 1}\vec{x} = 2$ and the x-axis.
%
\item Find the area bounded by the curves
\begin{align}
\cbrak{\brak{x,y} : y > x^2, y = \abs{x}}
\end{align}
%
\item Find the area of the region
\begin{align}
\cbrak{\brak{x,y} : y^2 \le 4x, 4\vec{x}^T\vec{x} = 9}
\end{align}
%
\item Find the area of the circle $\vec{x}^T\vec{x} = 16$ exterior to the parabola $y^2 = 6$.
\end{enumerate}


\end{document}


