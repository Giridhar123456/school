\renewcommand{\theequation}{\theenumi}
\begin{enumerate}[label=\thesubsection.\arabic*.,ref=\thesubsection.\theenumi]
\numberwithin{equation}{enumi}

\item  Find the equation of a plane passing through the points $\vec{a}=\myvec{2\\5\\-3}, \vec{b}=\myvec{-2\\-3\\5}$ and $\vec{c}=\myvec{5\\3\\-3}$ 
\label{eq:plane}
\\
\solution
The equation of  plane is also  given by \eqref{eq:line_norm_eq_unit} in 3D.  Following the approach in the previous example results in the matrix equation, 
\begin{align}
\myvec{2&5&-3 \\ -2&-3&5 \\ 5&3&-3} \vec{n} &= \myvec{1\\1\\1}
\end{align}
Row reducing the augmented matrix, 
\begin{align}
\myvec{2&5&-3 & 1\\ -2&-3&5 & 1\\ 5&3&-3 & 1} 
\\
\xleftrightarrow[R_3\leftarrow 2R_3-5R_1]{R_2\leftarrow \frac{R_2+R_1}{2} }\myvec{2&5&-3 & 1\\ 0&1&1 & 1\\ 0&-19&9 & -3} 
\\
\xleftrightarrow[R_3\leftarrow \frac{R_3+19R_2}{4}]{R_1\leftarrow R_1-5R_2 }\myvec{2&0&-8 & -4\\ 0&1&1 & 1\\ 0&0&7 & 4}\\ 
\xleftrightarrow[R_3\leftarrow 7R2-R_3]{R_1\leftarrow \frac{7R_1+8R_3}{2} }\myvec{7&0&0 & 2\\ 0&7&0 & 3\\ 0&0&7 & 4} 
\\
\implies \vec{n} = \frac{1}{7}\myvec{2\\3\\4}
\end{align}
Thus, the equation of the plane passing through the given points is
%
\begin{align}
\myvec{2 & 3 & 4}\vec{x} = 7
\end{align} 
%\cite{twelve_two}.
\item  Find the angle between the two planes 
%\cite{twelve_two}
\begin{align}
\myvec{ 2 & 1 & -2}\vec{x} &=5
\\
\myvec{ 3 & -6 & -2}\vec{x} &=7
\end{align}
\\
\solution
The angle between two planes is the same as the angle between their normal vectors.  For 
\begin{align}
\vec{n}_1 = \myvec{ 2 \\ 1 \\ -2}
\vec{n}_2 = \myvec{ 3 \\ -6 \\ -2}
\end{align}
using \eqref{eq:vec_angle}, 
\begin{align}
\cos \theta = \frac{6-6+4}{\sqrt{9}\sqrt{49}} = \frac{4}{21}
\end{align}
\end{enumerate}


