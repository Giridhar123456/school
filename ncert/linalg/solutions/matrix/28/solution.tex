
Since
\begin{align}
    \vec{I}=\myvec{1 & 0\\ 0 & 1}\\
    \vec{A}=\tan\frac{\alpha}{2}\myvec{\cos{90} & -\sin{90} \\ \sin{90} & \cos{90}}\\
\Vec{I}-\Vec{A}=\myvec{1 & 0\\ 0 & 1}-\tan\frac{\alpha}{2}\myvec{\cos{90} & -\sin{90} \\ \sin{90} & \cos{90}}\\
=\frac{1}{\cos\frac{\alpha}{2}}\myvec{\cos\frac{\alpha}{2} & 0 \\ 0 & \cos\frac{\alpha}{2}}-\frac{\sin\frac{\alpha}{2}}{\cos\frac{\alpha}{2}}\myvec{\cos{90} & -\sin{90} \\ \sin{90} & \cos{90}} \\
=\frac{1}{\cos\frac{\alpha}{2}}\myvec{\cos\frac{\alpha}{2} & 0 \\ 0 & \cos\frac{\alpha}{2}}-\frac{1}{\cos\frac{\alpha}{2}}\myvec{0 & -\sin\frac{\alpha}{2} \\ \sin\frac{\alpha}{2} & 0} \\
=\frac{1}{\cos\frac{\alpha}{2}}\myvec{\cos\frac{\alpha}{2} & \sin\frac{\alpha}{2} \\ -\sin\frac{\alpha}{2} & \cos\frac{\alpha}{2}}
\end{align}
The matrix $\vec{I}-\vec{A}$ is a rotational Matrix with rotation $-\frac{\alpha}{2}$ \\\\
The Matrix $\myvec{\cos\alpha & -\sin\alpha \\ \sin\alpha & \cos\alpha}$ is also a rotational Matrix with an angle $+\alpha$.\\\\
Multiplying two rotational matrices gives the resultant rotational matrix $+\alpha-\frac{\alpha}{2}=+\frac{\alpha}{2}$\\
\begin{align}
  RHS=\myvec{I-A}\myvec{\cos\alpha & -\sin\alpha \\ \sin\alpha & cos\alpha}\\
 =\frac{1}{\cos\frac{\alpha}{2}}\myvec{\cos\frac{\alpha}{2} & \sin\frac{\alpha}{2} \\ -\sin\frac{\alpha}{2} & \cos\frac{\alpha}{2}}\myvec{\cos\alpha & -\sin\alpha \\ \sin\alpha & cos\alpha}\\
   =\frac{1}{\cos\frac{\alpha}{2}}\myvec{\cos\frac{\alpha}{2} & -\sin\frac{\alpha}{2} \\ \sin\frac{\alpha}{2} & \cos\frac{\alpha}{2}}\\
\end{align}
Solving LHS$=\vec{I}+\vec{A}$
\begin{align}
    \vec{I}+\vec{A}=\myvec{1 & \tan\frac{\alpha}{2}\\\tan\frac{\alpha}{2} & 1}\\
    =\frac{1}{\cos\frac{\alpha}{2}}\myvec{\cos\frac{\alpha}{2} & -\sin\frac{\alpha}{2} \\ \sin\frac{\alpha}{2} & \cos\frac{\alpha}{2}}
\end{align}
This term is a rotational Matrix with angle $+\frac{\alpha}{2}$.Hence both sides evaluates to be a rotational matrix with angle $+\frac{\alpha}{2}$.


