Given $A^2 = A$ $\implies$$ A^2 - A = 0 $\\
Let $\lambda $ is the eigen value then theevery eigen value satisfies its very own characteristic equation
\begin{equation}
$\implies$ $ \lambda^2 = \lambda $
\end{equation}
\begin{equation}
$  \lambda^2 - \lambda = 0 $
\end{equation}
Then $ (I+A)^3 - 7A $ can be written as $ (I+\lambda)^3 - 7\lambda $\\
\begin{equation}	
\begin{align}
$ (1+\lambda)^3 - 7\lambda  = 1^3+\lambda^3+3(1^2)\lambda+3\lambda^2 (1) - 7\lambda$
\end{align}
\end{equation}
we know that 
\begin{equation}
\begin{align}	
I^3 = I^2 = I =1
\end{align}
\end{equation}
\begin{equation}
\begin{align}
 $  = 1+\lambda^3+3 (1)\lambda+3\lambda^2 (1) -7\lambda$
\end{align}
\end{equation}
\begin{equation}
\begin{align}
 $ = 1+\lambda^3+3\lambda+3\lambda^2-7\lambda$
\end{align}
\end{equation}
\begin{equation}
\begin{align}
 $   =1+(\lambda^2\lambda)+3\lambda+3\lambda -7\lambda$
\end{align}
\end{equation}
\begin{equation}
\begin{align}
    $ = 1+(\lambda\lambda)-\lambda$
\end{align}
\end{equation}
\begin{equation}
\begin{align}
$ = 1+\lambda^2-\lambda$
\end{align}
\end{equation}
From the equation   $ \lambda^2-\lambda = 0 $
\begin{equation}
\begin{align}
    $ = 1+0$
\end{align}
\end{equation}
\begin{equation}
\begin{align}
    $ = 1$
\end{align}
\end{equation}
Option C is the valid answer.


 


  
