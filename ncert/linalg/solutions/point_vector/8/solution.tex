Let $\bm{A}$=\(\frac{1}{7}\) $\begin{pmatrix}2 \\3 \\6\end{pmatrix}$,$\bm{B}$=\(\frac{1}{7}\) $\begin{pmatrix}3 \\-6 \\2\end{pmatrix}$,$\bm{C}$=\(\frac{1}{7}\) $\begin{pmatrix}6 \\2 \\-3\end{pmatrix}$\\
\begin{align}
    \norm{\bm{A}}=\frac{1}{7}\sqrt{2^2+3^2+6^2}=1\\
    \norm{\bm{B}}=\frac{1}{7}\sqrt{3^2+-6^2+2^2}=1\\
    \norm{\bm{C}}=\frac{1}{7}\sqrt{6^2+2^2+-3^2}=1
\end{align}
When two vectors are perpendicular to each other their dot product is zero.
The dot product of $\bm{A}$,$\bm{B}$ and $\bm{C}$ with each other is
\begin{align}
    \bm{A}^T\bm{B}=\frac{1}{7} \times \frac{1}{7}(2 \times 3+3 \times -6+6 \times 2)=0\\
    \bm{B}^T\bm{C}=\frac{1}{7} \times \frac{1}{7}(2 \times 3+3 \times -6+6 \times 2)=0\\
    \bm{C}^T\bm{A}=\frac{1}{7} \times \frac{1}{7}(6 \times 2+2 \times 3+-3 \times 6)=0
\end{align}
Hence, the three unit vectors are mutually perpendicular to each other.
