%Let $X_i \in \in\{i\}\math{_{i=1}^{i=6}}$ and ${f_i}$ be the corresponding frequency. Then,
\begin{align}
    P_r (X=i)= \dfrac{{f}_{i}}{200}
\end{align}
From table \ref{eq:solutions/1/1/4/tab:givendistribution}
\begin{align}
    P_r (X=6)= \dfrac{14}{200}\\
    =0.07
\end{align}
\section{Output}
The outputs of Python program are attached below:
% Table generated by Excel2LaTeX from sheet 'sheet_random'
\begin{table}[h]
\renewcommand\thetable{2}
  \centering
  \caption{For 200 randomly generated numbers}
    \begin{tabular}{|r|r|r|}
    \toprule
    \hline
    \multicolumn{1}{|c|}{\textbf{Digit}} & \multicolumn{1}{c|}{\textbf{Frequency}} & \multicolumn{1}{c|}{\textbf{Probability}} \\
    \midrule
    \hline
    0     & 21    & 0.105 \\
    1     & 13    & 0.065 \\
    2     & 20    & 0.1 \\
    3     & 21    & 0.105 \\
    4     & 20    & 0.1 \\
    5     & 25    & 0.125 \\
    6     & 15    & 0.075 \\
    7     & 24    & 0.12 \\
    8     & 20    & 0.1 \\
    9     & 21    & 0.105 \\
    \hline
    \end{tabular}%
  \label{tab:output_200}%
\end{table}%

% Table generated by Excel2LaTeX from sheet 'sheet_random_2'
\begin{table}[htbp]
\renewcommand\thetable{3}
  \centering
  \caption{For 10000 randomly generated numbers}
    \begin{tabular}{|r|r|r|}
    \toprule
    \hline
    \multicolumn{1}{|c|}{\textbf{Digit}} & \multicolumn{1}{c|}{\textbf{Frequency}} & \multicolumn{1}{c|}{\textbf{Probability}} \\
    \midrule
    \hline
    0     & 1007  & 0.1007 \\
    1     & 988   & 0.0988 \\
    2     & 997   & 0.0997 \\
    3     & 1010  & 0.101 \\
    4     & 1005  & 0.1005 \\
    5     & 1018  & 0.1018 \\
    6     & 1000  & 0.1 \\
    7     & 984   & 0.0984 \\
    8     & 1019  & 0.1019 \\
    9     & 972   & 0.0972 \\
    \hline
    \end{tabular}%
  \label{tab:addlabel}%
\end{table}%

\section{Explanation}
The \textbf{Law Of Large Numbers} is a fundamental concept for probability and statistics. It states that  that as the number of trials increase, the experimental probability will get closer and closer to the theoretical probability.
 From the output tables 2 and 3, we can deduce that as the number of trials increase,  the ratio of the number of successful occurrences to the number of trials will tend to approach the theoretical probability of the outcome for an individual trial. 
Since all the digits are equiprobable, ideally each probability should be 1/10=0.1
 In Table 3, when number of trials are 10,000, probability of each digit is approximately 0.1 with very little deviation. eg. 0.1005.  

With 200 samples, Tables 2 and 3 are slightly different because the number of simulations is not sufficient for convergence in the probabilities.
