These equations can be written as:

\begin{equation}
\begin{aligned}
      \begin{pmatrix}
      2       & 1    & 0   & 0\\ 
      1       & -2   & 0   & 0\\
      0       & 0    & 5   & -1\\ 
      0       & 0   & 4   & 3\\
            
    \end{pmatrix} \begin{pmatrix}
      a\\ 
      b\\
      c\\
      d\\
            
    \end{pmatrix} = \begin{pmatrix}
      4\\ 
      -3\\
      11\\
      24\\
            
    \end{pmatrix} 
\end{aligned}
\end{equation}

So the coefficient matrix $A$ can be expressed as:\\

\begin{equation}
\begin{aligned}
    A=  \begin{pmatrix}
      2       & 1    & 0   & 0\\ 
      1       & -2   & 0   & 0\\
      0       & 0    & 5   & -1\\ 
      0       & 0   & 4   & 3\\
            
    \end{pmatrix} 
\end{aligned}
\end{equation}
 
And the augmented matrix $B$ can be expressed as:\\

\begin{equation}
\begin{aligned}
    B=  \begin{pmatrix}
      2       & 1    & 0   & 0  & 4\\ 
      1       & -2   & 0   & 0  & -3\\
      0       & 0    & 5   & -1  & 11\\ 
      0       & 0   & 4   & 3   & 24\\
            
    \end{pmatrix}
\end{aligned}
\end{equation}

Now, if we express the augmented matrix as Echelon form, then it will be:\\

\begin{equation}
\begin{aligned}
    B=  \begin{pmatrix}
      2       & 1    & 0   & 0  & 4\\ 
      1       & -2   & 0   & 0  & -3\\
      0       & 0    & 5   & -1  & 11\\ 
      0       & 0   & 4   & 3   & 24\\
            
    \end{pmatrix}\\
    \xleftrightarrow
    [R_4\leftarrow R_4 - R_3]{R_2\leftarrow 2R_2 - R_1} 
    \begin{pmatrix}
      2       & 1    & 0   & 0  & 4\\ 
      0       & -5   & 0   & 0  & -10\\
      0       & 0    & 5   & -1  & 11\\ 
      0       & 0   & -1   & 4   & 13\\
            
    \end{pmatrix}\\
    \xleftrightarrow
    [R_4\leftarrow 5R_4 + R_3]{R_2\leftarrow \frac{R_2}{(-5)}}
    \begin{pmatrix}
      2       & 1    & 0   & 0  & 4\\ 
      0       & 1   & 0   & 0  & 2\\
      0       & 0    & 5   & -1  & 11\\ 
      0       & 0   & 0   & 19   & 76\\
            
    \end{pmatrix}\\
    \xleftrightarrow
    []{R_4\leftarrow \frac{R_4}{(19)}}
    \begin{pmatrix}
      2       & 1    & 0   & 0  & 4\\ 
      0       & 1   & 0   & 0  & 2\\
      0       & 0    & 5   & -1  & 11\\ 
      0       & 0   & 0   & 1   & 4\\
            
    \end{pmatrix}\\
    \xleftrightarrow
    [R_1\leftarrow R_1 -R_2]{R_3\leftarrow R_3 + R_4}
    \begin{pmatrix}
      2       & 0    & 0   & 0  & 2\\ 
      0       & 1   & 0   & 0  & 2\\
      0       & 0    & 5   & 0  & 15\\ 
      0       & 0   & 0   & 1   & 4\\
            
    \end{pmatrix}\\
    \xleftrightarrow
    [R_1\leftarrow \frac{R_1}{2}]{R_3\leftarrow \frac{R_3}{5}}
    \begin{pmatrix}
      1       & 0    & 0   & 0  & 1\\ 
      0       & 1   & 0   & 0  & 2\\
      0       & 0    & 1   & 0  & 3\\ 
      0       & 0   & 0   & 1   & 4\\
            
    \end{pmatrix}
\end{aligned}
\end{equation}

Thus, 
\begin{equation}
\begin{aligned}
\begin{pmatrix}
      a\\ 
      b\\
      c\\
      d\\
            
    \end{pmatrix} = \begin{pmatrix}
      1\\ 
      2\\
      3\\
      4\\
            
    \end{pmatrix} 
\end{aligned}
\end{equation}



