
Given that 
\begin{align}\label{eq:solutions/matrix/73/eq1}
\Vec{X}\myvec{1&2&3\\4&5&6}=\myvec{-7&-8&-9\\2&4&6}
\end{align}

Equation \eqref{eq:solutions/matrix/63/eq1} can be written as 
\begin{align}\label{eq:solutions/matrix/73/eq2}
\myvec{1&4\\2&5\\3&6}\Vec{X^T}=\myvec{-7&2\\-8&4\\-9&6}
\end{align}
Equation \eqref{eq:solutions/matrix/63/eq2} can be represented as 
\begin{align}\label{eq:solutions/matrix/73/eq3}
\vec{A}\vec{x}=\vec{b}
\end{align}

where $\vec{A}=\myvec{1&4\\2&5\\3&6}$ and $\vec{b}=\myvec{-7&2\\-8&4\\-9&6}$

The set of least square solutions of $\vec{A}\vec{x}=\vec{b}$ coincides with the non empty set of solutions of equations $\vec{A}^T\vec{A}\vec{x}=\vec{A}^T\vec{b}.$ 

\begin{align}\label{eq:solutions/matrix/73/eq4}
 \hat{x}=(\vec{A}^T\vec{A})^{-1}\vec{A}^T\vec{b} 
\end{align}

\begin{align*}
\vec{A}^T\vec{A}=\myvec{1&2&3\\4&5&6}\myvec{1&4\\2&5\\3&6}
\end{align*}

\begin{align*}
\vec{A}^T\vec{A}=\myvec{14&32\\32&77}
\end{align*}

\begin{align*}
\vec{A}^T\vec{b}=\myvec{1&2&3\\4&5&6}\myvec{-7&2\\-8&4\\-9&6}   
\end{align*}

\begin{align*}
\vec{A}^T\vec{b}=\myvec{-50&28\\-122&64}
\end{align*}

\begin{align*}
(\vec{A}^T\vec{A})^{-1}=\frac{1}{54}\myvec{77&-32\\-32&64} 
\end{align*}

Using equation\eqref{eq:solutions/matrix/63/eq4} 
\begin{align*}
\hat{x}=\frac{1}{54}\myvec{77&-32\\-32&14}\myvec{-50&28\\-122&64}\\
=\frac{1}{54}\myvec{54&108\\-108&0}\\
=\myvec{1&2\\-2&0}
\end{align*}
\begin{align*}
\vec{X}=\hat{x}^T=\myvec{1&-2\\2&0}  
\end{align*}

