Given,
\begin{align}
&\triangle = \mydet{b+c & c+a & a+b\\c+a & a+b & b+c\\a+b&b+c & c+a} \label{eq:solutions/det/71ques}\\
\xleftrightarrow[]{C_1\leftarrow C_1+C_2+C_3}
&\mydet{2(a+b+c) & c+a & a+b\\2(a+b+c) & a+b & b+c\\2(a+b+c)&b+c & c+a} \label{eq:solutions/det/71eq_1}\\
&=2(a+b+c)\mydet{1 & c+a & a+b\\1 & a+b & b+c\\1&b+c & c+a} \label{eq:solutions/det/71eq_2}\\
\xleftrightarrow[]{R_1\leftarrow R_1-R_2; R_2 \leftarrow R_2-R_3}
&2(a+b+c)\mydet{0 & c-b & a-c\\0 & a-c & b-a\\1&b+c & c+a} =0 \label{eq:solutions/det/71equ_3}
\end{align}
On expanding determinant along first column from equation \eqref{eq:solutions/det/71equ_3},
\begin{multline*}
\implies 2(a+b+c)[(c-b)(b-a)-(a-c)^{2}]=0
\\
\implies 2(a+b+c)(a^{2}+b^{2}+c^{2}\\
-ab-bc-ca)=0
\\
\implies (a+b+c)(2a^{2}+2b^{2}+2c^{2}\\
-2ab-2bc-2ca)=0
\end{multline*}
\begin{multline}
\implies (a+b+c)\\
[(a-b)^{2}+(b-c)^{2}+(c-a)^{2}] = 0 \label{eq:solutions/det/71equ_4}
\end{multline}
From equation \eqref{eq:solutions/det/71equ_4} we get 2 equations,
\begin{align}
&\implies \boxed{(a+b+c) = 0} \label{eq:solutions/det/71sol_1}
\end{align}
or,
\begin{align}
&\implies (a-b)^{2}+(b-c)^{2}+(c-a)^{2} = 0 \label{eq:solutions/det/71sol_2}
\end{align}
Equation \eqref{eq:solutions/det/71sol_2} is possible only when, $a=b=c$
\begin{align}
&\implies \boxed{a=b=c} \label{eq:solutions/det/71sol_3}
\end{align}
From equation \eqref{eq:solutions/det/71sol_1} and \eqref{eq:solutions/det/71sol_3} we can say that,
$\triangle=0$ if $a+b+c=0$ or $a=b=c$.

From equation \eqref{eq:solutions/det/71sol_1} and \eqref{eq:solutions/det/71sol_3} we can say that,
$\triangle=0$ if $a+b+c=0$ or $a=b=c$.

