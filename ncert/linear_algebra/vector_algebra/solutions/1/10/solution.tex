Consider an equilateral $\triangle{ABC}$ as shown in below figure:
\begin{figure}[!ht]
    \label{Fig:solutions/1/10/1}
    \centering
    \resizebox{\columnwidth}{!}{\begin{tikzpicture}

\draw (0,0) node[anchor=north]{$A$}-- (4,6.9282) node[anchor=west]{$B$}-- (8,0) node[anchor=north]{$C$}-- cycle;

\draw (0,0) (0:0.75cm) arc (0:60:0.75cm);

\draw (30:1.15cm) node{ 60 \degree};

\begin{scope}[shift={(8,0)}]

\draw (0,0) (180:0.75cm) arc (180:120:0.75cm); \draw (150:1.15cm)node{ 60 \degree};

\end{scope}

\begin{scope}[shift={(4,6.9282)}]

\draw (0,0) (-120:0.75cm) arc (-120:-60:0.75cm);

\draw (-90:1.15cm)node{ 60 \degree};

\end{scope}

\end{tikzpicture}
}
    \caption{Equilateral $\triangle{ABC}$}
\end{figure}

In equilateral triangle all sides have equal length
\begin{align}
    \label{eq:solutions/1/101}\implies\norm{\vec{A-B}}=\norm{\vec{B-C}}=\norm{\vec{A-C}}=k
\end{align}
Let $\vec{B}=0$. Then substituting in \eqref{eq:solutions/1/101} will give
\begin{align}
    \label{eq:solutions/1/102}\norm{\vec{A}}=\norm{\vec{C}}\\
    \label{eq:solutions/1/103}\norm{\vec{A}}=\norm{\vec{A-C}}
\end{align}
Taking square on both sides in \eqref{eq:solutions/1/103}.
\begin{align}
    \implies\norm{\vec{A}}^2=\norm{\vec{A-C}}^2\\
    \implies\norm{\vec{A}}^2=\norm{\vec{A}}^2+\norm{\vec{C}}^2-2\vec{A}^T\vec{C}\\
    \implies\norm{\vec{A}}^2=\norm{\vec{A}}^2+\norm{\vec{A}}^2-2\vec{A}^T\vec{C}\\
    \implies0=\norm{\vec{A}}^2-2\vec{A}^T\vec{C}\\
    \implies2\vec{A}^T\vec{C}=\norm{\vec{A}}^2\\
    \label{eq:solutions/1/109}\implies\vec{A}^T\vec{C}=\frac{\norm{\vec{A}}^2}{2}
\end{align}

let $\theta=\angle ABC$.

Taking the inner product of sides AB and BC.
\begin{align}
    (\vec{A}-\vec{B})^T(\vec{B}-\vec{C}) =
    \norm{\vec{A}-\vec{B}}\norm{\vec{B}-\vec{C}}\cos{\theta}\\
    \label{eq:solutions/1/1011}\implies \cos{\theta}=\frac{(\vec{A}-\vec{B})^T(\vec{B}-\vec{C})}{\norm{\vec{A}-\vec{B}}\norm{\vec{B}-\vec{C}}}
\end{align}
Substitute $\vec{B}=0$ in \eqref{eq:solutions/1/1011}
\begin{align}
    \label{eq:solutions/1/1012}\implies\cos{\theta}=\frac{\vec{A}^T\vec{C}}{\norm{\vec{A}}\norm{\vec{C}}}
\end{align}
Substitute \eqref{eq:solutions/1/102},\eqref{eq:solutions/1/109} in \eqref{eq:solutions/1/1012}
\begin{align}
    \implies\cos\theta=\frac{\frac{\norm{\vec{A}}^2}{2}}{\norm{\vec{A}}^2}\\
    \implies\cos\theta=\frac{1}{2}
\end{align}
In equilateral triangle, $\angle ABC=60\degree$
\begin{align}
    \implies\cos60\degree=\frac{1}{2}
\end{align}
Hence proved.
