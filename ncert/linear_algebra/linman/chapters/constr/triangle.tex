\renewcommand{\theequation}{\theenumi}
\begin{enumerate}[label=\thesubsection.\arabic*.,ref=\thesubsection.\theenumi]
%\begin{enumerate}[label=\arabic*.,ref=\thesection.\theenumi]
\numberwithin{equation}{enumi}

%
\item Draw Fig. \ref{fig:tri_right_angle} for $a = 4, c =3$.
\label{const:tri_right_angle}
%
\begin{figure}[!ht]
\centering
\resizebox{\columnwidth}{!}{%Code by GVV Sharma
%December 6, 2019
%released under GNU GPL
%Drawing a right angled triangle

\begin{tikzpicture}[scale=2]

%Triangle sides
\def\a{4}
\def\c{3}

%Marking coordiantes
\coordinate [label=above:$A$] (A) at (0,\c);
\coordinate [label=left:$B$] (B) at (0,0);
\coordinate [label=right:$C$] (C) at (\a,0);

%Drawing triangle ABC
\draw (A) -- node[left] {$\textrm{c}$} (B) -- node[below] {$\textrm{a}$} (C) -- node[above,,xshift=2mm] {$\textrm{b}$} (A);

%Drawing and marking angles
\tkzMarkAngle[fill=orange!40,size=0.5cm,mark=](A,C,B)
\tkzMarkRightAngle[fill=blue!20,size=.3](A,B,C)
\tkzLabelAngle[pos=0.65](A,C,B){$\theta$}
\end{tikzpicture}
}
\caption{Right Angled Triangle}
\label{fig:tri_right_angle}	
\end{figure}
\\
\solution The vertices of $\triangle ABC$ are 
\begin{align}
\vec{A} = \myvec{0\\c} = \myvec{0\\3}, \vec{B} = \myvec{0\\0}, \vec{C} = \myvec{a\\0}=\myvec{4\\0}
\end{align}
%
The python code for  Fig. \ref{fig:tri_right_angle} is
\begin{lstlisting}
codes/triangle/tri_right_angle.py
\end{lstlisting}
%
and the equivalent latex-tikz code is
%
\begin{lstlisting}
figs/constr/triangle/tri_right_angle.tex
\end{lstlisting}
%
The above latex code can be compiled as a standalone document as
%
\begin{lstlisting}
figs/constr/triangle/tri_right_angle_alone.tex
\end{lstlisting}
%

\item Draw Fig. \ref{fig:tri_polar} for $a = 4, c =3$.
\label{const:tri_polar}
%
\\
\solution 
 The vertex  $\vec{A}$ can  be expressed  in {\em polar coordinate form} as
%\label{prob:tri_polar}
%
\begin{align}
\vec{A} = b\myvec{\cos \theta\\  \sin \theta} 
\end{align}
%
where
\begin{align}
b = \sqrt{a^2+c^2} = 5, \tan \theta = \frac{3}{4}
\end{align}
%The vertices of $\triangle ABC$ are 
%\begin{align}
%\vec{A} = \myvec{a\\c} = \myvec{4\\3}, \vec{B} = \myvec{a\\0}  = \myvec{4\\0}, \vec{C} = \myvec{0\\0}.
%\end{align}
%
The python code for  Fig. \ref{fig:tri_polar} is
\begin{lstlisting}
codes/triangle/tri_polar.py
\end{lstlisting}
%
and the equivalent latex-tikz code is
%
\begin{lstlisting}
figs/constr/triangle/tri_polar.tex
\end{lstlisting}
\begin{figure}[!ht]
\centering
\resizebox{\columnwidth}{!}{%Code by GVV Sharma
%December 6, 2019
%released under GNU GPL
%Drawing a right angled triangle

\begin{tikzpicture}[scale=2]

%Triangle sides
\def\a{4}
\def\c{3}

%Marking coordiantes
\coordinate [label=above:$A$] (A) at (\a,\c);
\coordinate [label=below:$B$] (B) at (\a,0);
\coordinate [label=left:$C$] (C) at (0,0);

%Drawing triangle ABC
\draw (A) -- node[left] {$\textrm{c}$} (B) -- node[below] {$\textrm{a}$} (C) -- node[above left,xshift=2mm] {$\textrm{b}$} (A);

%Drawing and marking angles
\tkzMarkAngle[fill=orange!40,size=0.5cm,mark=](B,C,A)
\tkzMarkRightAngle[fill=blue!20,size=.3](A,B,C)
\tkzLabelAngle[pos=0.65](A,C,B){$\theta$}
\end{tikzpicture}
}
\caption{Right Angled Triangle}
\label{fig:tri_polar}	
\end{figure}
%
\item Draw Fig. \ref{fig:tri_sss} with $a=6$, $b=5$  and $c=4$.  
\label{const:tri_sss}
\begin{figure}[!ht]
	\begin{center}
			\resizebox{\columnwidth}{!}{%Code by GVV Sharma
%December 7, 2019
%released under GNU GPL
%Drawing a triangle given 3 sides

\begin{tikzpicture}
[scale=2,>=stealth,point/.style={draw,circle,fill = black,inner sep=0.5pt},]

%Triangle sides
\def\a{6}
\def\b{5}
\def\c{4}
 
%Coordinates of A
%\def\p{{\a^2+\c^2-\b^2}/{(2*\a)}}
\def\p{2.25}
\def\q{{sqrt(\c^2-\p^2)}}

%Labeling points
\node (A) at (\p,\q)[point,label=above right:$A$] {};
\node (B) at (0, 0)[point,label=below left:$B$] {};
\node (C) at (\a, 0)[point,label=below right:$C$] {};

%Foot of perpendicular

\node (D) at (\p,0)[point,label=above right:$D$] {};

%Drawing triangle ABC
\draw (A) -- node[left] {$\textrm{c}$} (B) -- node[below] {$\textrm{a}$} (C) -- node[above,xshift=2mm] {$\textrm{b}$} (A);

%Drawing altitude AD
\draw (A) -- node[left] {$\textrm{h}$}(D);

%Drawing and marking angles
%\tkzMarkAngle[fill=orange!40,size=0.5cm,mark=](A,C,B)
%\tkzMarkAngle[fill=orange!40,size=0.4cm,mark=](D,B,A)
%\tkzMarkAngle[fill=green!40,size=0.5cm,mark=](B,A,C)
%\tkzMarkAngle[fill=green!40,size=0.5cm,mark=](C,B,D)
\tkzMarkRightAngle[fill=blue!20,size=.2](A,D,B)
%\tkzMarkRightAngle[fill=blue!20,size=.2](B,D,A)
%\tkzLabelAngle[pos=0.65](A,C,B){$\theta$}
%\tkzLabelAngle[pos=0.65](A,B,D){$\theta$}
%\tkzLabelAngle[pos=1](B,A,C){\rotatebox{-45}{$\alpha = 90\degree -\theta$}}
%\tkzLabelAngle[pos=0.65](C,B,D){$\alpha$}

\end{tikzpicture}
}
	\end{center}
	\caption{}
	\label{fig:tri_sss}	
\end{figure}
\\
\solution Let the vertices of  $\triangle ABC$ and $\vec{D}$ be 
\begin{align}
\label{eq:tri_basic}
\vec{A} = \myvec{p\\q}, \vec{B} = \myvec{0\\0}, \vec{C} = \myvec{a\\0}, \vec{D} = \myvec{p\\0}
\end{align}
%

Then
\begin{align}
\label{eq:c_tricoord}
AB &= \norm{\vec{A}-\vec{B}}^2 = \norm{\vec{A}}^2  = c^2 \quad \because \vec{B} = \vec{0}
\\
\label{eq:a_tricoord}
BC &= \norm{\vec{C}-\vec{B}}^2 = \norm{\vec{C}}^2  = a^2
\\
AC &= \norm{\vec{A}-\vec{C}}^2 =    b^2
\label{eq:b_tricoord}
\end{align}
%
From \eqref{eq:b_tricoord},
\begin{align}
b^2 &=\norm{\vec{A}-\vec{C}}^2 = \norm{\vec{A}-\vec{C}}^T\norm{\vec{A}-\vec{C}}  
\\
&= \vec{A}^T\vec{A}+\vec{C}^T\vec{C}-\vec{A}^T\vec{C} - \vec{C}^T\vec{A} 
\\
&= \norm{\vec{A}}^2 + \norm{\vec{C}}^2 - 2\vec{A}^T\vec{C} \quad \brak{\because \vec{A}^T\vec{C} = \vec{C}^T\vec{A} } 
\label{eq:tri_const_norm_ac}
\\
&= a^2+c^2-2ap
\end{align}
%
yielding
\begin{align}
p&= \frac{a^2+c^2-b^2}{2a}
\end{align}
%
From \eqref{eq:c_tricoord}, 
\begin{align}
\norm{\vec{A}}^2 &= c^2 = p^2+q^2
\\
\implies q&= \pm \sqrt{c^2-p^2}
\end{align}
%
The python code for  Fig. \ref{fig:tri_sss} is
\begin{lstlisting}
codes/triangle/tri_sss.py
\end{lstlisting}
%
and the equivalent latex-tikz code is
%
\begin{lstlisting}
figs/constr/triangle/tri_sss.tex
\end{lstlisting}

\end{enumerate}

