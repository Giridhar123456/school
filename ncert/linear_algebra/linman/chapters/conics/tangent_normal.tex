\renewcommand{\theequation}{\theenumi}
\begin{enumerate}[label=\thesection.\arabic*.,ref=\thesection.\theenumi]
\numberwithin{equation}{enumi}

\item 
{\em Secant: }The points of intersection of the line 
\begin{align}
L: \quad \vec{x} = \vec{q} + \mu \vec{m} \quad \mu \in \mathbb{R}
\label{eq:conic_tangent}
\end{align}
with the conic section in \eqref{eq:conic_quad_form} are given by
\begin{align}
\vec{x}_i = \vec{q} + \mu_i \vec{m}
\end{align}
%
where
\begin{multline}
\mu_i = \frac{1}
{
\vec{m}^T\vec{V}\vec{m}
}
\lbrak{-\vec{m}^T\brak{\vec{V}\vec{q}+\vec{u}}}
\\
\pm
{\small
\rbrak{\sqrt{
\sbrak{
\vec{m}^T\brak{\vec{V}\vec{q}+\vec{u}}
}^2
-
\brak
{
\vec{q}^T\vec{V}\vec{q} + 2\vec{u}^T\vec{q} +f
}
\brak{\vec{m}^T\vec{V}\vec{m}}
}
}
}
\label{eq:tangent_roots}
\end{multline}
\solution
Substituting \eqref{eq:conic_tangent}
in \eqref{eq:conic_quad_form}, 
\begin{multline}
\brak{\vec{q} + \mu \vec{m}}^T\vec{V}\brak{\vec{q} + \mu \vec{m}}  + 2 \vec{u}^T\brak{\vec{q} + \mu \vec{m}}+f = 0
\\
\implies \mu^2\vec{m}^T\vec{V}\vec{m} + 2 \mu\vec{m}^T\brak{\vec{V}\vec{q}+\vec{u}} 
\\
+ \vec{q}^T\vec{V}\vec{q} + 2\vec{u}^T\vec{q} +f = 0
\label{eq:conic_intercept}
\end{multline}
Solving the above quadratic in \eqref{eq:conic_intercept}
yields \eqref{eq:tangent_roots}.
\item 
{\em Tangent: } If $L$ in \eqref{eq:conic_tangent} touches \eqref{eq:conic_quad_form} at exactly one point $\vec{q}$, 
\begin{align}
\vec{m}^T\brak{\vec{V}\vec{q}+\vec{u}} = 0
\label{eq:conic_tangent_mq}
\end{align}
\solution  In this case, \eqref{eq:conic_intercept} has exactly one root.  Hence, 
in \eqref{eq:tangent_roots}
\begin{multline}
\sbrak{
\vec{m}^T\brak{\vec{V}\vec{q}+\vec{u}}
}^2                                                                                                                    \\
-\brak{\vec{m}^T\vec{V}\vec{m}}
\brak
{
\vec{q}^T\vec{V}\vec{q} + 2\vec{u}^T\vec{q} +f
} = 0                                                                                             
\label{eq:conic_tangent_disc}
\end{multline}                    
$\because \vec{q}$ is the point of contact, $\vec{q}$ satisfies \eqref{eq:conic_quad_form}
and 
\begin{align}
\vec{q}^T\vec{V}\vec{q} + 2\vec{u}^T\vec{q} +f = 0
\label{eq:conic_tangent_qquad}
\end{align}
Substituting \eqref{eq:conic_tangent_qquad} in \eqref{eq:conic_tangent_disc} and simplifying, we obtain \eqref{eq:conic_tangent_mq}.
\item 
The normal vector is obtained from \eqref{eq:conic_tangent_mq} and \eqref{eq:line_dir_norm}
as
%
\begin{align}
\label{eq:conic_normal_vec}
\vec{n} = \vec{V}\vec{q}+\vec{u}
\end{align}
\item 
Given the point of contact $\vec{q}$, the equation of a tangent is 
\begin{align}
\brak{\vec{V}\vec{q}+\vec{u}}^T\vec{x}+\vec{u}^T\vec{q}+f = 0
\label{eq:conic_tangent_final}
\end{align}
\solution
 From \eqref{eq:conic_normal_vec} and \eqref{eq:line_norm_eq}, the equation of the tangent is\begin{align}
\brak{\vec{V}\vec{q}+\vec{u}}^T\brak{\vec{x}-\vec{q}} &=0
\\
\implies \brak{\vec{V}\vec{q}+\vec{u}}^T\vec{x}-\vec{q}^T\vec{V}\vec{q}-\vec{u}^T\vec{q} &= 0
\end{align}
which, upon substituting from \eqref{eq:conic_tangent_qquad} and simplifying yields \eqref{eq:conic_tangent}.
\item 
If $\vec{V}^{-1}$ exists, given the normal vector $\vec{n}$, the tangent points of contact to \eqref{eq:conic_quad_form} are given by
\begin{align}
\vec{q}_i &= \vec{V}^{-1}\brak{\kappa_i \vec{n}-\vec{u}}, i = 1,2
\\
\text{where }\kappa_i &= \pm \sqrt{
\frac{
\vec{u}^T\vec{V}^{-1}\vec{u}-f
}
{
\vec{n}^T\vec{V}^{-1}\vec{n}
}
}
\label{eq:conic_tangent_qk}
\end{align}
\solution  From \eqref{eq:conic_normal_vec},
\begin{align}
\label{eq:conic_normal_vec_q}
 \vec{q} = \vec{V}^{-1}\brak{\kappa \vec{n}-\vec{u}}, \quad \kappa \in \mathbb{R}
\end{align}
Substituting \eqref{eq:conic_normal_vec_q}
in \eqref{eq:conic_tangent_qquad},
\begin{multline}
\brak{\kappa \vec{n}-\vec{u}}^T\vec{V}^{-1}\brak{\kappa \vec{n}-\vec{u}} 
\\
+ 2\vec{u}^T\vec{V}^{-1}\brak{\kappa \vec{n}-\vec{u}} +f = 0
\\
\implies 
\kappa^2 \vec{n}^T\vec{V}^{-1}\vec{n} - \vec{u}^T\vec{V}^{-1}\vec{u} + f =0
\\
\text{or, } \kappa = \pm \sqrt{\frac{\vec{u}^T\vec{V}^{-1}\vec{u}-f}{\vec{n}^T\vec{V}^{-1}\vec{n}}}\label{eq:conic_normal_k}
\end{multline}
%
Substituting \eqref{eq:conic_normal_k} in \eqref{eq:conic_normal_vec_q}
yields \eqref{eq:conic_tangent_qk}.
%
\item 
If $\vec{V}$ is not invertible,  given the normal vector $\vec{n}$, the point of contact to \eqref{eq:conic_quad_form} is given by the matrix equation
\begin{align}
\label{eq:conic_tangent_q_eigen}
\begin{pmatrix}
\vec{u+\kappa \vec{n}}^T \\ \vec{V}
\end{pmatrix}
\vec{q} &= 
\begin{pmatrix}
-f
\\
\kappa\vec{n}-\vec{u}
\end{pmatrix}
\\
\text{where }  \kappa = \frac{\vec{p}_1^T\vec{u}}{\vec{p}_1^T\vec{n}}, \quad \vec{V}\vec{p}_1 &= 0
\label{eq:conic_tangent_qk_eigen}
\end{align}
\solution If $\vec{V}$ is non-invertible, it has a zero eigenvalue.  If the corresponding eigenvector is $\vec{p}_1$, then,
\begin{align}
\vec{V}\vec{p}_1 = 0
\label{eq:conic_zero_eigen}
\end{align}
From \eqref{eq:conic_normal_vec},
\begin{align}
\label{eq:conic_zero_eigen_normal}
\kappa \vec{n} &= \vec{V} \vec{q}+\vec{u}, \quad \kappa \in \mathbb{R}
\\
\implies \kappa \vec{p}_1^T\vec{n} &= \vec{p}_1^T\vec{V} \vec{q}+\vec{p}_1^T\vec{u}
\\
\text{or, } \kappa \vec{p}_1^T\vec{n} &= \vec{p}_1^T\vec{u},  \quad \because \vec{p}_1^T \vec{V} = 0, 
\\
\text{ from } \eqref{eq:conic_zero_eigen} &
%\label{eq:conic_normal_vec_q}
\end{align}
yielding $\kappa$ in \eqref{eq:conic_tangent_qk_eigen}. From \eqref{eq:conic_zero_eigen_normal},
\begin{align}
\kappa \vec{q}^T\vec{n} &= \vec{q}^T\vec{V} \vec{q}+\vec{q}^T\vec{u}
\\
\implies \kappa \vec{q}^T\vec{n} &= -f-\vec{q}^T\vec{u} \quad \text{from } \eqref{eq:conic_tangent_qquad},
\\
\text{or, } \brak{\kappa \vec{n}+\vec{u}}\vec{q} &= -f
\label{eq:conic_zero_eigen_normal_fq}
\end{align}
\eqref{eq:conic_zero_eigen_normal} can be expressed as
\begin{align}
\label{eq:conic_zero_eigen_normal_vq}
\vec{V} \vec{q} = \kappa \vec{n} - \vec{u}.
\end{align}
\eqref{eq:conic_zero_eigen_normal_fq} and \eqref{eq:conic_zero_eigen_normal_vq} clubbed together result in \eqref{eq:conic_tangent_q_eigen}.
\item All the results related to conics are summarized in 
Table \ref{table:conics}.  
\begin{table*}[!t]
\centering
\renewcommand{\theequation}{\theenumi}
%\begin{enumerate}[label=\arabic*.,ref=\theenumi]
\begin{enumerate}[label=\thesection.\arabic*.,ref=\thesection.\theenumi]
\numberwithin{equation}{enumi}

\item Find the area of the region enclosed between the two circles: $\vec{x}^T\vec{x} = 4$ and $\norm{\vec{x}-\myvec{2\\0}} = 2$.
\\
\solution 
Let the balanced version of (\ref{eq:solutions/chem/6ato balance}) be
\begin{align}
    \label{eq:solutions/chem/6abalanced}x_{1}HNO_{3}+ x_{2}Ca(OH)_{2}\to x_{3}Ca(NO_{3})_{2}+ x_{4}H_{2}O
\end{align}

which results in the following equations:
\begin{align}
    (x_{1}+ 2x_{2}-2x_{4}) H= 0\\
    (x_{1}-2x_{3}) N= 0\\
    (3x_{1}+ 2x_{2}-6x_{3}- x_{4}) O=0\\
    (x_{2}-x_{3}) Ca= 0
\end{align}

which can be expressed as
\begin{align}
    x_{1}+ 2x_{2}+ 0.x_{3} -2x_{4} = 0\\
    x_{1}+ 0.x_{2} -2x_{3} +0.x_{4}= 0\\
    3x_{1}+ 2x_{2}-6x_{3}- x_{4} =0\\
    0.x_{1} +x_{2}-x_{3} +0.x_{4}= 0
\end{align}

resulting in the matrix equation
\begin{align}
    \label{eq:solutions/chem/6a matrix}
    \myvec{1 & 2 & 0 & -2\\
           1 & 0 & -2 & 0\\
           3 & 2 & -6 & -1\\
           0 & 1 & -1 & 0}\vec{x}
           =\vec{0}
\end{align}

where,
\begin{align}
   \vec{x}= \myvec{x_{1}\\x_{2}\\x_{3}\\x_{4}}
\end{align}

(\ref{eq:solutions/chem/6a matrix}) can be reduced as follows:
\begin{align}
    \myvec{1 & 2 & 0 & -2\\
           1 & 0 & -2 & 0\\
           3 & 2 & -6 & -1\\
           0 & 1 & -1 & 0}
    \xleftrightarrow[R_{3}\leftarrow \frac{R_3}{3}-R_{1}]{R_{2}\leftarrow R_2- R_1}
    \myvec{1 & 2 & 0 & -2\\
           0 & -2 & -2 & 2\\
           0 & -\frac{4}{3} & -2 & \frac{5}{3}\\
           0 & 1 & -1 & 0}\\
    \xleftrightarrow{R_2 \leftarrow -\frac{R_2}{2}}
    \myvec{1 & 2 & 0 & -2\\
          0 & 1 & 1 & -1\\
          0 & -\frac{4}{3} & -2 & \frac{5}{3}\\
          0 & 1 & -1 & 0}\\
    \xleftrightarrow[R_4 \leftarrow R_4- R_2]{R_3 \leftarrow R_3 + \frac{4}{3}R_2}
    \myvec{1 & 2 & 0 & -2\\
           0 & 1 & 1 & -1\\
           0 & 0 & -\frac{2}{3} & \frac{1}{3}\\
           0 & 0 & -2 & 1}\\
    \xleftrightarrow[R_3 \leftarrow -\frac{3}{2}R_3]{R_1 \leftarrow R_1- 2R_2}
    \myvec{1 & 0 & -2 & 0\\
           0 & 1 & 1 & -1\\
           0 & 0 & 1 & -\frac{1}{2}\\
           0 & 0 & -2 & 1}\\
    \xleftrightarrow{R_4\leftarrow R_4 + 2R_3}
    \myvec{1 & 0 & -2 & 0\\
           0 & 1 & 1 & -1\\
           0 & 0 & 1 & -\frac{1}{2}\\
           0 & 0 & 0 & 0}\\
    \xleftrightarrow[R_2\leftarrow R_2-R_3]{R_1\leftarrow R_1 + 2R_3}
    \myvec{1 & 0 & 0 & -1\\
           0 & 1 & 0 & -\frac{1}{2}\\
           0 & 0 & 1 & -\frac{1}{2}\\
           0 & 0 & 0 & 0}
\end{align}

Thus,
\begin{align}
    x_1=x_4, x_2= \frac{1}{2}x_4, x_3=\frac{1}{2}x_4\\
    \implies \quad\vec{x}= x_4\myvec{1\\ \frac{1}{2}\\ \frac{1}{2}\\1} =\myvec{2\\1\\1\\2}
\end{align} 
by substituting $x_4= 2$.

\hfill\break
%\vspace{5mm} 
Hence, (\ref{eq:solutions/chem/6abalanced}) finally becomes
\begin{align}
    2HNO_{3}+ Ca(OH)_{2}\to Ca(NO_{3})_{2}+ 2H_{2}O
\end{align}

\item Find the equation of the circle with radius 5 whose centre lies on x-axis and passes through the point \myvec{2\\3}.
\\
\solution 
Let the balanced version of (\ref{eq:solutions/chem/6ato balance}) be
\begin{align}
    \label{eq:solutions/chem/6abalanced}x_{1}HNO_{3}+ x_{2}Ca(OH)_{2}\to x_{3}Ca(NO_{3})_{2}+ x_{4}H_{2}O
\end{align}

which results in the following equations:
\begin{align}
    (x_{1}+ 2x_{2}-2x_{4}) H= 0\\
    (x_{1}-2x_{3}) N= 0\\
    (3x_{1}+ 2x_{2}-6x_{3}- x_{4}) O=0\\
    (x_{2}-x_{3}) Ca= 0
\end{align}

which can be expressed as
\begin{align}
    x_{1}+ 2x_{2}+ 0.x_{3} -2x_{4} = 0\\
    x_{1}+ 0.x_{2} -2x_{3} +0.x_{4}= 0\\
    3x_{1}+ 2x_{2}-6x_{3}- x_{4} =0\\
    0.x_{1} +x_{2}-x_{3} +0.x_{4}= 0
\end{align}

resulting in the matrix equation
\begin{align}
    \label{eq:solutions/chem/6a matrix}
    \myvec{1 & 2 & 0 & -2\\
           1 & 0 & -2 & 0\\
           3 & 2 & -6 & -1\\
           0 & 1 & -1 & 0}\vec{x}
           =\vec{0}
\end{align}

where,
\begin{align}
   \vec{x}= \myvec{x_{1}\\x_{2}\\x_{3}\\x_{4}}
\end{align}

(\ref{eq:solutions/chem/6a matrix}) can be reduced as follows:
\begin{align}
    \myvec{1 & 2 & 0 & -2\\
           1 & 0 & -2 & 0\\
           3 & 2 & -6 & -1\\
           0 & 1 & -1 & 0}
    \xleftrightarrow[R_{3}\leftarrow \frac{R_3}{3}-R_{1}]{R_{2}\leftarrow R_2- R_1}
    \myvec{1 & 2 & 0 & -2\\
           0 & -2 & -2 & 2\\
           0 & -\frac{4}{3} & -2 & \frac{5}{3}\\
           0 & 1 & -1 & 0}\\
    \xleftrightarrow{R_2 \leftarrow -\frac{R_2}{2}}
    \myvec{1 & 2 & 0 & -2\\
          0 & 1 & 1 & -1\\
          0 & -\frac{4}{3} & -2 & \frac{5}{3}\\
          0 & 1 & -1 & 0}\\
    \xleftrightarrow[R_4 \leftarrow R_4- R_2]{R_3 \leftarrow R_3 + \frac{4}{3}R_2}
    \myvec{1 & 2 & 0 & -2\\
           0 & 1 & 1 & -1\\
           0 & 0 & -\frac{2}{3} & \frac{1}{3}\\
           0 & 0 & -2 & 1}\\
    \xleftrightarrow[R_3 \leftarrow -\frac{3}{2}R_3]{R_1 \leftarrow R_1- 2R_2}
    \myvec{1 & 0 & -2 & 0\\
           0 & 1 & 1 & -1\\
           0 & 0 & 1 & -\frac{1}{2}\\
           0 & 0 & -2 & 1}\\
    \xleftrightarrow{R_4\leftarrow R_4 + 2R_3}
    \myvec{1 & 0 & -2 & 0\\
           0 & 1 & 1 & -1\\
           0 & 0 & 1 & -\frac{1}{2}\\
           0 & 0 & 0 & 0}\\
    \xleftrightarrow[R_2\leftarrow R_2-R_3]{R_1\leftarrow R_1 + 2R_3}
    \myvec{1 & 0 & 0 & -1\\
           0 & 1 & 0 & -\frac{1}{2}\\
           0 & 0 & 1 & -\frac{1}{2}\\
           0 & 0 & 0 & 0}
\end{align}

Thus,
\begin{align}
    x_1=x_4, x_2= \frac{1}{2}x_4, x_3=\frac{1}{2}x_4\\
    \implies \quad\vec{x}= x_4\myvec{1\\ \frac{1}{2}\\ \frac{1}{2}\\1} =\myvec{2\\1\\1\\2}
\end{align} 
by substituting $x_4= 2$.

\hfill\break
%\vspace{5mm} 
Hence, (\ref{eq:solutions/chem/6abalanced}) finally becomes
\begin{align}
    2HNO_{3}+ Ca(OH)_{2}\to Ca(NO_{3})_{2}+ 2H_{2}O
\end{align}

\item Find the equation of the circle passing through \myvec{0\\0} and making intercepts a and b on the coordinate axes.
\item Find the equation of a circle with centre \myvec{2\\2} and passes through the point \myvec{4\\5}. 
\\
\solution 
Let the balanced version of (\ref{eq:solutions/chem/6ato balance}) be
\begin{align}
    \label{eq:solutions/chem/6abalanced}x_{1}HNO_{3}+ x_{2}Ca(OH)_{2}\to x_{3}Ca(NO_{3})_{2}+ x_{4}H_{2}O
\end{align}

which results in the following equations:
\begin{align}
    (x_{1}+ 2x_{2}-2x_{4}) H= 0\\
    (x_{1}-2x_{3}) N= 0\\
    (3x_{1}+ 2x_{2}-6x_{3}- x_{4}) O=0\\
    (x_{2}-x_{3}) Ca= 0
\end{align}

which can be expressed as
\begin{align}
    x_{1}+ 2x_{2}+ 0.x_{3} -2x_{4} = 0\\
    x_{1}+ 0.x_{2} -2x_{3} +0.x_{4}= 0\\
    3x_{1}+ 2x_{2}-6x_{3}- x_{4} =0\\
    0.x_{1} +x_{2}-x_{3} +0.x_{4}= 0
\end{align}

resulting in the matrix equation
\begin{align}
    \label{eq:solutions/chem/6a matrix}
    \myvec{1 & 2 & 0 & -2\\
           1 & 0 & -2 & 0\\
           3 & 2 & -6 & -1\\
           0 & 1 & -1 & 0}\vec{x}
           =\vec{0}
\end{align}

where,
\begin{align}
   \vec{x}= \myvec{x_{1}\\x_{2}\\x_{3}\\x_{4}}
\end{align}

(\ref{eq:solutions/chem/6a matrix}) can be reduced as follows:
\begin{align}
    \myvec{1 & 2 & 0 & -2\\
           1 & 0 & -2 & 0\\
           3 & 2 & -6 & -1\\
           0 & 1 & -1 & 0}
    \xleftrightarrow[R_{3}\leftarrow \frac{R_3}{3}-R_{1}]{R_{2}\leftarrow R_2- R_1}
    \myvec{1 & 2 & 0 & -2\\
           0 & -2 & -2 & 2\\
           0 & -\frac{4}{3} & -2 & \frac{5}{3}\\
           0 & 1 & -1 & 0}\\
    \xleftrightarrow{R_2 \leftarrow -\frac{R_2}{2}}
    \myvec{1 & 2 & 0 & -2\\
          0 & 1 & 1 & -1\\
          0 & -\frac{4}{3} & -2 & \frac{5}{3}\\
          0 & 1 & -1 & 0}\\
    \xleftrightarrow[R_4 \leftarrow R_4- R_2]{R_3 \leftarrow R_3 + \frac{4}{3}R_2}
    \myvec{1 & 2 & 0 & -2\\
           0 & 1 & 1 & -1\\
           0 & 0 & -\frac{2}{3} & \frac{1}{3}\\
           0 & 0 & -2 & 1}\\
    \xleftrightarrow[R_3 \leftarrow -\frac{3}{2}R_3]{R_1 \leftarrow R_1- 2R_2}
    \myvec{1 & 0 & -2 & 0\\
           0 & 1 & 1 & -1\\
           0 & 0 & 1 & -\frac{1}{2}\\
           0 & 0 & -2 & 1}\\
    \xleftrightarrow{R_4\leftarrow R_4 + 2R_3}
    \myvec{1 & 0 & -2 & 0\\
           0 & 1 & 1 & -1\\
           0 & 0 & 1 & -\frac{1}{2}\\
           0 & 0 & 0 & 0}\\
    \xleftrightarrow[R_2\leftarrow R_2-R_3]{R_1\leftarrow R_1 + 2R_3}
    \myvec{1 & 0 & 0 & -1\\
           0 & 1 & 0 & -\frac{1}{2}\\
           0 & 0 & 1 & -\frac{1}{2}\\
           0 & 0 & 0 & 0}
\end{align}

Thus,
\begin{align}
    x_1=x_4, x_2= \frac{1}{2}x_4, x_3=\frac{1}{2}x_4\\
    \implies \quad\vec{x}= x_4\myvec{1\\ \frac{1}{2}\\ \frac{1}{2}\\1} =\myvec{2\\1\\1\\2}
\end{align} 
by substituting $x_4= 2$.

\hfill\break
%\vspace{5mm} 
Hence, (\ref{eq:solutions/chem/6abalanced}) finally becomes
\begin{align}
    2HNO_{3}+ Ca(OH)_{2}\to Ca(NO_{3})_{2}+ 2H_{2}O
\end{align}

\item Find the locus of all the unit vectors in the xy-plane.
%
\item Find the points on the curve $\vec{x}^T\vec{x}-2\myvec{1 & 0}\vec{x} -3 =0$  at which the tangents are parallel to the x-axis.
%
\\
\solution
Let the balanced version of (\ref{eq:solutions/chem/6ato balance}) be
\begin{align}
    \label{eq:solutions/chem/6abalanced}x_{1}HNO_{3}+ x_{2}Ca(OH)_{2}\to x_{3}Ca(NO_{3})_{2}+ x_{4}H_{2}O
\end{align}

which results in the following equations:
\begin{align}
    (x_{1}+ 2x_{2}-2x_{4}) H= 0\\
    (x_{1}-2x_{3}) N= 0\\
    (3x_{1}+ 2x_{2}-6x_{3}- x_{4}) O=0\\
    (x_{2}-x_{3}) Ca= 0
\end{align}

which can be expressed as
\begin{align}
    x_{1}+ 2x_{2}+ 0.x_{3} -2x_{4} = 0\\
    x_{1}+ 0.x_{2} -2x_{3} +0.x_{4}= 0\\
    3x_{1}+ 2x_{2}-6x_{3}- x_{4} =0\\
    0.x_{1} +x_{2}-x_{3} +0.x_{4}= 0
\end{align}

resulting in the matrix equation
\begin{align}
    \label{eq:solutions/chem/6a matrix}
    \myvec{1 & 2 & 0 & -2\\
           1 & 0 & -2 & 0\\
           3 & 2 & -6 & -1\\
           0 & 1 & -1 & 0}\vec{x}
           =\vec{0}
\end{align}

where,
\begin{align}
   \vec{x}= \myvec{x_{1}\\x_{2}\\x_{3}\\x_{4}}
\end{align}

(\ref{eq:solutions/chem/6a matrix}) can be reduced as follows:
\begin{align}
    \myvec{1 & 2 & 0 & -2\\
           1 & 0 & -2 & 0\\
           3 & 2 & -6 & -1\\
           0 & 1 & -1 & 0}
    \xleftrightarrow[R_{3}\leftarrow \frac{R_3}{3}-R_{1}]{R_{2}\leftarrow R_2- R_1}
    \myvec{1 & 2 & 0 & -2\\
           0 & -2 & -2 & 2\\
           0 & -\frac{4}{3} & -2 & \frac{5}{3}\\
           0 & 1 & -1 & 0}\\
    \xleftrightarrow{R_2 \leftarrow -\frac{R_2}{2}}
    \myvec{1 & 2 & 0 & -2\\
          0 & 1 & 1 & -1\\
          0 & -\frac{4}{3} & -2 & \frac{5}{3}\\
          0 & 1 & -1 & 0}\\
    \xleftrightarrow[R_4 \leftarrow R_4- R_2]{R_3 \leftarrow R_3 + \frac{4}{3}R_2}
    \myvec{1 & 2 & 0 & -2\\
           0 & 1 & 1 & -1\\
           0 & 0 & -\frac{2}{3} & \frac{1}{3}\\
           0 & 0 & -2 & 1}\\
    \xleftrightarrow[R_3 \leftarrow -\frac{3}{2}R_3]{R_1 \leftarrow R_1- 2R_2}
    \myvec{1 & 0 & -2 & 0\\
           0 & 1 & 1 & -1\\
           0 & 0 & 1 & -\frac{1}{2}\\
           0 & 0 & -2 & 1}\\
    \xleftrightarrow{R_4\leftarrow R_4 + 2R_3}
    \myvec{1 & 0 & -2 & 0\\
           0 & 1 & 1 & -1\\
           0 & 0 & 1 & -\frac{1}{2}\\
           0 & 0 & 0 & 0}\\
    \xleftrightarrow[R_2\leftarrow R_2-R_3]{R_1\leftarrow R_1 + 2R_3}
    \myvec{1 & 0 & 0 & -1\\
           0 & 1 & 0 & -\frac{1}{2}\\
           0 & 0 & 1 & -\frac{1}{2}\\
           0 & 0 & 0 & 0}
\end{align}

Thus,
\begin{align}
    x_1=x_4, x_2= \frac{1}{2}x_4, x_3=\frac{1}{2}x_4\\
    \implies \quad\vec{x}= x_4\myvec{1\\ \frac{1}{2}\\ \frac{1}{2}\\1} =\myvec{2\\1\\1\\2}
\end{align} 
by substituting $x_4= 2$.

\hfill\break
%\vspace{5mm} 
Hence, (\ref{eq:solutions/chem/6abalanced}) finally becomes
\begin{align}
    2HNO_{3}+ Ca(OH)_{2}\to Ca(NO_{3})_{2}+ 2H_{2}O
\end{align}

\item  Find the area of the region in the first quadrant enclosed by x-axis, line $\myvec{1 & -\sqrt{3}}\vec{x} =0$ and the circle $\vec{x}^T\vec{x}=4$.
%
\\
\solution
Let the balanced version of (\ref{eq:solutions/chem/6ato balance}) be
\begin{align}
    \label{eq:solutions/chem/6abalanced}x_{1}HNO_{3}+ x_{2}Ca(OH)_{2}\to x_{3}Ca(NO_{3})_{2}+ x_{4}H_{2}O
\end{align}

which results in the following equations:
\begin{align}
    (x_{1}+ 2x_{2}-2x_{4}) H= 0\\
    (x_{1}-2x_{3}) N= 0\\
    (3x_{1}+ 2x_{2}-6x_{3}- x_{4}) O=0\\
    (x_{2}-x_{3}) Ca= 0
\end{align}

which can be expressed as
\begin{align}
    x_{1}+ 2x_{2}+ 0.x_{3} -2x_{4} = 0\\
    x_{1}+ 0.x_{2} -2x_{3} +0.x_{4}= 0\\
    3x_{1}+ 2x_{2}-6x_{3}- x_{4} =0\\
    0.x_{1} +x_{2}-x_{3} +0.x_{4}= 0
\end{align}

resulting in the matrix equation
\begin{align}
    \label{eq:solutions/chem/6a matrix}
    \myvec{1 & 2 & 0 & -2\\
           1 & 0 & -2 & 0\\
           3 & 2 & -6 & -1\\
           0 & 1 & -1 & 0}\vec{x}
           =\vec{0}
\end{align}

where,
\begin{align}
   \vec{x}= \myvec{x_{1}\\x_{2}\\x_{3}\\x_{4}}
\end{align}

(\ref{eq:solutions/chem/6a matrix}) can be reduced as follows:
\begin{align}
    \myvec{1 & 2 & 0 & -2\\
           1 & 0 & -2 & 0\\
           3 & 2 & -6 & -1\\
           0 & 1 & -1 & 0}
    \xleftrightarrow[R_{3}\leftarrow \frac{R_3}{3}-R_{1}]{R_{2}\leftarrow R_2- R_1}
    \myvec{1 & 2 & 0 & -2\\
           0 & -2 & -2 & 2\\
           0 & -\frac{4}{3} & -2 & \frac{5}{3}\\
           0 & 1 & -1 & 0}\\
    \xleftrightarrow{R_2 \leftarrow -\frac{R_2}{2}}
    \myvec{1 & 2 & 0 & -2\\
          0 & 1 & 1 & -1\\
          0 & -\frac{4}{3} & -2 & \frac{5}{3}\\
          0 & 1 & -1 & 0}\\
    \xleftrightarrow[R_4 \leftarrow R_4- R_2]{R_3 \leftarrow R_3 + \frac{4}{3}R_2}
    \myvec{1 & 2 & 0 & -2\\
           0 & 1 & 1 & -1\\
           0 & 0 & -\frac{2}{3} & \frac{1}{3}\\
           0 & 0 & -2 & 1}\\
    \xleftrightarrow[R_3 \leftarrow -\frac{3}{2}R_3]{R_1 \leftarrow R_1- 2R_2}
    \myvec{1 & 0 & -2 & 0\\
           0 & 1 & 1 & -1\\
           0 & 0 & 1 & -\frac{1}{2}\\
           0 & 0 & -2 & 1}\\
    \xleftrightarrow{R_4\leftarrow R_4 + 2R_3}
    \myvec{1 & 0 & -2 & 0\\
           0 & 1 & 1 & -1\\
           0 & 0 & 1 & -\frac{1}{2}\\
           0 & 0 & 0 & 0}\\
    \xleftrightarrow[R_2\leftarrow R_2-R_3]{R_1\leftarrow R_1 + 2R_3}
    \myvec{1 & 0 & 0 & -1\\
           0 & 1 & 0 & -\frac{1}{2}\\
           0 & 0 & 1 & -\frac{1}{2}\\
           0 & 0 & 0 & 0}
\end{align}

Thus,
\begin{align}
    x_1=x_4, x_2= \frac{1}{2}x_4, x_3=\frac{1}{2}x_4\\
    \implies \quad\vec{x}= x_4\myvec{1\\ \frac{1}{2}\\ \frac{1}{2}\\1} =\myvec{2\\1\\1\\2}
\end{align} 
by substituting $x_4= 2$.

\hfill\break
%\vspace{5mm} 
Hence, (\ref{eq:solutions/chem/6abalanced}) finally becomes
\begin{align}
    2HNO_{3}+ Ca(OH)_{2}\to Ca(NO_{3})_{2}+ 2H_{2}O
\end{align}

\item Find the area lying in the first quadrant and bounded by the circle $\vec{x}^T\vec{x}=4$ and the lines $x = 0$ and $x = 2$.
%
\item Find the area of the circle $4\vec{x}^T\vec{x}=9$.
\item  Find the area bounded by curves $\norm{\vec{x}-\myvec{1\\0}} = 1$ and $\norm{\vec{x}}=1$
\\
\solution
Let the balanced version of (\ref{eq:solutions/chem/6ato balance}) be
\begin{align}
    \label{eq:solutions/chem/6abalanced}x_{1}HNO_{3}+ x_{2}Ca(OH)_{2}\to x_{3}Ca(NO_{3})_{2}+ x_{4}H_{2}O
\end{align}

which results in the following equations:
\begin{align}
    (x_{1}+ 2x_{2}-2x_{4}) H= 0\\
    (x_{1}-2x_{3}) N= 0\\
    (3x_{1}+ 2x_{2}-6x_{3}- x_{4}) O=0\\
    (x_{2}-x_{3}) Ca= 0
\end{align}

which can be expressed as
\begin{align}
    x_{1}+ 2x_{2}+ 0.x_{3} -2x_{4} = 0\\
    x_{1}+ 0.x_{2} -2x_{3} +0.x_{4}= 0\\
    3x_{1}+ 2x_{2}-6x_{3}- x_{4} =0\\
    0.x_{1} +x_{2}-x_{3} +0.x_{4}= 0
\end{align}

resulting in the matrix equation
\begin{align}
    \label{eq:solutions/chem/6a matrix}
    \myvec{1 & 2 & 0 & -2\\
           1 & 0 & -2 & 0\\
           3 & 2 & -6 & -1\\
           0 & 1 & -1 & 0}\vec{x}
           =\vec{0}
\end{align}

where,
\begin{align}
   \vec{x}= \myvec{x_{1}\\x_{2}\\x_{3}\\x_{4}}
\end{align}

(\ref{eq:solutions/chem/6a matrix}) can be reduced as follows:
\begin{align}
    \myvec{1 & 2 & 0 & -2\\
           1 & 0 & -2 & 0\\
           3 & 2 & -6 & -1\\
           0 & 1 & -1 & 0}
    \xleftrightarrow[R_{3}\leftarrow \frac{R_3}{3}-R_{1}]{R_{2}\leftarrow R_2- R_1}
    \myvec{1 & 2 & 0 & -2\\
           0 & -2 & -2 & 2\\
           0 & -\frac{4}{3} & -2 & \frac{5}{3}\\
           0 & 1 & -1 & 0}\\
    \xleftrightarrow{R_2 \leftarrow -\frac{R_2}{2}}
    \myvec{1 & 2 & 0 & -2\\
          0 & 1 & 1 & -1\\
          0 & -\frac{4}{3} & -2 & \frac{5}{3}\\
          0 & 1 & -1 & 0}\\
    \xleftrightarrow[R_4 \leftarrow R_4- R_2]{R_3 \leftarrow R_3 + \frac{4}{3}R_2}
    \myvec{1 & 2 & 0 & -2\\
           0 & 1 & 1 & -1\\
           0 & 0 & -\frac{2}{3} & \frac{1}{3}\\
           0 & 0 & -2 & 1}\\
    \xleftrightarrow[R_3 \leftarrow -\frac{3}{2}R_3]{R_1 \leftarrow R_1- 2R_2}
    \myvec{1 & 0 & -2 & 0\\
           0 & 1 & 1 & -1\\
           0 & 0 & 1 & -\frac{1}{2}\\
           0 & 0 & -2 & 1}\\
    \xleftrightarrow{R_4\leftarrow R_4 + 2R_3}
    \myvec{1 & 0 & -2 & 0\\
           0 & 1 & 1 & -1\\
           0 & 0 & 1 & -\frac{1}{2}\\
           0 & 0 & 0 & 0}\\
    \xleftrightarrow[R_2\leftarrow R_2-R_3]{R_1\leftarrow R_1 + 2R_3}
    \myvec{1 & 0 & 0 & -1\\
           0 & 1 & 0 & -\frac{1}{2}\\
           0 & 0 & 1 & -\frac{1}{2}\\
           0 & 0 & 0 & 0}
\end{align}

Thus,
\begin{align}
    x_1=x_4, x_2= \frac{1}{2}x_4, x_3=\frac{1}{2}x_4\\
    \implies \quad\vec{x}= x_4\myvec{1\\ \frac{1}{2}\\ \frac{1}{2}\\1} =\myvec{2\\1\\1\\2}
\end{align} 
by substituting $x_4= 2$.

\hfill\break
%\vspace{5mm} 
Hence, (\ref{eq:solutions/chem/6abalanced}) finally becomes
\begin{align}
    2HNO_{3}+ Ca(OH)_{2}\to Ca(NO_{3})_{2}+ 2H_{2}O
\end{align}

\item Find the smaller area enclosed by the circle $\vec{x}^T\vec{x}=4$ and the line $\myvec{1 & 1}\vec{x} = 2$.


\item Find the slope of the tangent to the curve $y = \frac{x-1}{x-2}, x\ne 2$ at $x = 10$.
\\
\solution
Let the balanced version of (\ref{eq:solutions/chem/6ato balance}) be
\begin{align}
    \label{eq:solutions/chem/6abalanced}x_{1}HNO_{3}+ x_{2}Ca(OH)_{2}\to x_{3}Ca(NO_{3})_{2}+ x_{4}H_{2}O
\end{align}

which results in the following equations:
\begin{align}
    (x_{1}+ 2x_{2}-2x_{4}) H= 0\\
    (x_{1}-2x_{3}) N= 0\\
    (3x_{1}+ 2x_{2}-6x_{3}- x_{4}) O=0\\
    (x_{2}-x_{3}) Ca= 0
\end{align}

which can be expressed as
\begin{align}
    x_{1}+ 2x_{2}+ 0.x_{3} -2x_{4} = 0\\
    x_{1}+ 0.x_{2} -2x_{3} +0.x_{4}= 0\\
    3x_{1}+ 2x_{2}-6x_{3}- x_{4} =0\\
    0.x_{1} +x_{2}-x_{3} +0.x_{4}= 0
\end{align}

resulting in the matrix equation
\begin{align}
    \label{eq:solutions/chem/6a matrix}
    \myvec{1 & 2 & 0 & -2\\
           1 & 0 & -2 & 0\\
           3 & 2 & -6 & -1\\
           0 & 1 & -1 & 0}\vec{x}
           =\vec{0}
\end{align}

where,
\begin{align}
   \vec{x}= \myvec{x_{1}\\x_{2}\\x_{3}\\x_{4}}
\end{align}

(\ref{eq:solutions/chem/6a matrix}) can be reduced as follows:
\begin{align}
    \myvec{1 & 2 & 0 & -2\\
           1 & 0 & -2 & 0\\
           3 & 2 & -6 & -1\\
           0 & 1 & -1 & 0}
    \xleftrightarrow[R_{3}\leftarrow \frac{R_3}{3}-R_{1}]{R_{2}\leftarrow R_2- R_1}
    \myvec{1 & 2 & 0 & -2\\
           0 & -2 & -2 & 2\\
           0 & -\frac{4}{3} & -2 & \frac{5}{3}\\
           0 & 1 & -1 & 0}\\
    \xleftrightarrow{R_2 \leftarrow -\frac{R_2}{2}}
    \myvec{1 & 2 & 0 & -2\\
          0 & 1 & 1 & -1\\
          0 & -\frac{4}{3} & -2 & \frac{5}{3}\\
          0 & 1 & -1 & 0}\\
    \xleftrightarrow[R_4 \leftarrow R_4- R_2]{R_3 \leftarrow R_3 + \frac{4}{3}R_2}
    \myvec{1 & 2 & 0 & -2\\
           0 & 1 & 1 & -1\\
           0 & 0 & -\frac{2}{3} & \frac{1}{3}\\
           0 & 0 & -2 & 1}\\
    \xleftrightarrow[R_3 \leftarrow -\frac{3}{2}R_3]{R_1 \leftarrow R_1- 2R_2}
    \myvec{1 & 0 & -2 & 0\\
           0 & 1 & 1 & -1\\
           0 & 0 & 1 & -\frac{1}{2}\\
           0 & 0 & -2 & 1}\\
    \xleftrightarrow{R_4\leftarrow R_4 + 2R_3}
    \myvec{1 & 0 & -2 & 0\\
           0 & 1 & 1 & -1\\
           0 & 0 & 1 & -\frac{1}{2}\\
           0 & 0 & 0 & 0}\\
    \xleftrightarrow[R_2\leftarrow R_2-R_3]{R_1\leftarrow R_1 + 2R_3}
    \myvec{1 & 0 & 0 & -1\\
           0 & 1 & 0 & -\frac{1}{2}\\
           0 & 0 & 1 & -\frac{1}{2}\\
           0 & 0 & 0 & 0}
\end{align}

Thus,
\begin{align}
    x_1=x_4, x_2= \frac{1}{2}x_4, x_3=\frac{1}{2}x_4\\
    \implies \quad\vec{x}= x_4\myvec{1\\ \frac{1}{2}\\ \frac{1}{2}\\1} =\myvec{2\\1\\1\\2}
\end{align} 
by substituting $x_4= 2$.

\hfill\break
%\vspace{5mm} 
Hence, (\ref{eq:solutions/chem/6abalanced}) finally becomes
\begin{align}
    2HNO_{3}+ Ca(OH)_{2}\to Ca(NO_{3})_{2}+ 2H_{2}O
\end{align}

\item Find a point on the curve $y = (x – 2)^2$ at which the tangent is parallel to the chord joining the points \myvec{2\\ 0} and \myvec{4\\ 4}.
\\
\solution
Let the balanced version of (\ref{eq:solutions/chem/6ato balance}) be
\begin{align}
    \label{eq:solutions/chem/6abalanced}x_{1}HNO_{3}+ x_{2}Ca(OH)_{2}\to x_{3}Ca(NO_{3})_{2}+ x_{4}H_{2}O
\end{align}

which results in the following equations:
\begin{align}
    (x_{1}+ 2x_{2}-2x_{4}) H= 0\\
    (x_{1}-2x_{3}) N= 0\\
    (3x_{1}+ 2x_{2}-6x_{3}- x_{4}) O=0\\
    (x_{2}-x_{3}) Ca= 0
\end{align}

which can be expressed as
\begin{align}
    x_{1}+ 2x_{2}+ 0.x_{3} -2x_{4} = 0\\
    x_{1}+ 0.x_{2} -2x_{3} +0.x_{4}= 0\\
    3x_{1}+ 2x_{2}-6x_{3}- x_{4} =0\\
    0.x_{1} +x_{2}-x_{3} +0.x_{4}= 0
\end{align}

resulting in the matrix equation
\begin{align}
    \label{eq:solutions/chem/6a matrix}
    \myvec{1 & 2 & 0 & -2\\
           1 & 0 & -2 & 0\\
           3 & 2 & -6 & -1\\
           0 & 1 & -1 & 0}\vec{x}
           =\vec{0}
\end{align}

where,
\begin{align}
   \vec{x}= \myvec{x_{1}\\x_{2}\\x_{3}\\x_{4}}
\end{align}

(\ref{eq:solutions/chem/6a matrix}) can be reduced as follows:
\begin{align}
    \myvec{1 & 2 & 0 & -2\\
           1 & 0 & -2 & 0\\
           3 & 2 & -6 & -1\\
           0 & 1 & -1 & 0}
    \xleftrightarrow[R_{3}\leftarrow \frac{R_3}{3}-R_{1}]{R_{2}\leftarrow R_2- R_1}
    \myvec{1 & 2 & 0 & -2\\
           0 & -2 & -2 & 2\\
           0 & -\frac{4}{3} & -2 & \frac{5}{3}\\
           0 & 1 & -1 & 0}\\
    \xleftrightarrow{R_2 \leftarrow -\frac{R_2}{2}}
    \myvec{1 & 2 & 0 & -2\\
          0 & 1 & 1 & -1\\
          0 & -\frac{4}{3} & -2 & \frac{5}{3}\\
          0 & 1 & -1 & 0}\\
    \xleftrightarrow[R_4 \leftarrow R_4- R_2]{R_3 \leftarrow R_3 + \frac{4}{3}R_2}
    \myvec{1 & 2 & 0 & -2\\
           0 & 1 & 1 & -1\\
           0 & 0 & -\frac{2}{3} & \frac{1}{3}\\
           0 & 0 & -2 & 1}\\
    \xleftrightarrow[R_3 \leftarrow -\frac{3}{2}R_3]{R_1 \leftarrow R_1- 2R_2}
    \myvec{1 & 0 & -2 & 0\\
           0 & 1 & 1 & -1\\
           0 & 0 & 1 & -\frac{1}{2}\\
           0 & 0 & -2 & 1}\\
    \xleftrightarrow{R_4\leftarrow R_4 + 2R_3}
    \myvec{1 & 0 & -2 & 0\\
           0 & 1 & 1 & -1\\
           0 & 0 & 1 & -\frac{1}{2}\\
           0 & 0 & 0 & 0}\\
    \xleftrightarrow[R_2\leftarrow R_2-R_3]{R_1\leftarrow R_1 + 2R_3}
    \myvec{1 & 0 & 0 & -1\\
           0 & 1 & 0 & -\frac{1}{2}\\
           0 & 0 & 1 & -\frac{1}{2}\\
           0 & 0 & 0 & 0}
\end{align}

Thus,
\begin{align}
    x_1=x_4, x_2= \frac{1}{2}x_4, x_3=\frac{1}{2}x_4\\
    \implies \quad\vec{x}= x_4\myvec{1\\ \frac{1}{2}\\ \frac{1}{2}\\1} =\myvec{2\\1\\1\\2}
\end{align} 
by substituting $x_4= 2$.

\hfill\break
%\vspace{5mm} 
Hence, (\ref{eq:solutions/chem/6abalanced}) finally becomes
\begin{align}
    2HNO_{3}+ Ca(OH)_{2}\to Ca(NO_{3})_{2}+ 2H_{2}O
\end{align}

\item Find the equation of all lines having slope – 1 that are tangents to the curve $\frac{1}
{x -1}, x \ne 1$
\\
\solution 
Let the balanced version of (\ref{eq:solutions/chem/6ato balance}) be
\begin{align}
    \label{eq:solutions/chem/6abalanced}x_{1}HNO_{3}+ x_{2}Ca(OH)_{2}\to x_{3}Ca(NO_{3})_{2}+ x_{4}H_{2}O
\end{align}

which results in the following equations:
\begin{align}
    (x_{1}+ 2x_{2}-2x_{4}) H= 0\\
    (x_{1}-2x_{3}) N= 0\\
    (3x_{1}+ 2x_{2}-6x_{3}- x_{4}) O=0\\
    (x_{2}-x_{3}) Ca= 0
\end{align}

which can be expressed as
\begin{align}
    x_{1}+ 2x_{2}+ 0.x_{3} -2x_{4} = 0\\
    x_{1}+ 0.x_{2} -2x_{3} +0.x_{4}= 0\\
    3x_{1}+ 2x_{2}-6x_{3}- x_{4} =0\\
    0.x_{1} +x_{2}-x_{3} +0.x_{4}= 0
\end{align}

resulting in the matrix equation
\begin{align}
    \label{eq:solutions/chem/6a matrix}
    \myvec{1 & 2 & 0 & -2\\
           1 & 0 & -2 & 0\\
           3 & 2 & -6 & -1\\
           0 & 1 & -1 & 0}\vec{x}
           =\vec{0}
\end{align}

where,
\begin{align}
   \vec{x}= \myvec{x_{1}\\x_{2}\\x_{3}\\x_{4}}
\end{align}

(\ref{eq:solutions/chem/6a matrix}) can be reduced as follows:
\begin{align}
    \myvec{1 & 2 & 0 & -2\\
           1 & 0 & -2 & 0\\
           3 & 2 & -6 & -1\\
           0 & 1 & -1 & 0}
    \xleftrightarrow[R_{3}\leftarrow \frac{R_3}{3}-R_{1}]{R_{2}\leftarrow R_2- R_1}
    \myvec{1 & 2 & 0 & -2\\
           0 & -2 & -2 & 2\\
           0 & -\frac{4}{3} & -2 & \frac{5}{3}\\
           0 & 1 & -1 & 0}\\
    \xleftrightarrow{R_2 \leftarrow -\frac{R_2}{2}}
    \myvec{1 & 2 & 0 & -2\\
          0 & 1 & 1 & -1\\
          0 & -\frac{4}{3} & -2 & \frac{5}{3}\\
          0 & 1 & -1 & 0}\\
    \xleftrightarrow[R_4 \leftarrow R_4- R_2]{R_3 \leftarrow R_3 + \frac{4}{3}R_2}
    \myvec{1 & 2 & 0 & -2\\
           0 & 1 & 1 & -1\\
           0 & 0 & -\frac{2}{3} & \frac{1}{3}\\
           0 & 0 & -2 & 1}\\
    \xleftrightarrow[R_3 \leftarrow -\frac{3}{2}R_3]{R_1 \leftarrow R_1- 2R_2}
    \myvec{1 & 0 & -2 & 0\\
           0 & 1 & 1 & -1\\
           0 & 0 & 1 & -\frac{1}{2}\\
           0 & 0 & -2 & 1}\\
    \xleftrightarrow{R_4\leftarrow R_4 + 2R_3}
    \myvec{1 & 0 & -2 & 0\\
           0 & 1 & 1 & -1\\
           0 & 0 & 1 & -\frac{1}{2}\\
           0 & 0 & 0 & 0}\\
    \xleftrightarrow[R_2\leftarrow R_2-R_3]{R_1\leftarrow R_1 + 2R_3}
    \myvec{1 & 0 & 0 & -1\\
           0 & 1 & 0 & -\frac{1}{2}\\
           0 & 0 & 1 & -\frac{1}{2}\\
           0 & 0 & 0 & 0}
\end{align}

Thus,
\begin{align}
    x_1=x_4, x_2= \frac{1}{2}x_4, x_3=\frac{1}{2}x_4\\
    \implies \quad\vec{x}= x_4\myvec{1\\ \frac{1}{2}\\ \frac{1}{2}\\1} =\myvec{2\\1\\1\\2}
\end{align} 
by substituting $x_4= 2$.

\hfill\break
%\vspace{5mm} 
Hence, (\ref{eq:solutions/chem/6abalanced}) finally becomes
\begin{align}
    2HNO_{3}+ Ca(OH)_{2}\to Ca(NO_{3})_{2}+ 2H_{2}O
\end{align}

\item Find the equation of all lines having slope -2 which are tangents to the curve $\frac{1}
{x - 3} , x \ne 3$.
%
\\
\solution 
Let the balanced version of (\ref{eq:solutions/chem/6ato balance}) be
\begin{align}
    \label{eq:solutions/chem/6abalanced}x_{1}HNO_{3}+ x_{2}Ca(OH)_{2}\to x_{3}Ca(NO_{3})_{2}+ x_{4}H_{2}O
\end{align}

which results in the following equations:
\begin{align}
    (x_{1}+ 2x_{2}-2x_{4}) H= 0\\
    (x_{1}-2x_{3}) N= 0\\
    (3x_{1}+ 2x_{2}-6x_{3}- x_{4}) O=0\\
    (x_{2}-x_{3}) Ca= 0
\end{align}

which can be expressed as
\begin{align}
    x_{1}+ 2x_{2}+ 0.x_{3} -2x_{4} = 0\\
    x_{1}+ 0.x_{2} -2x_{3} +0.x_{4}= 0\\
    3x_{1}+ 2x_{2}-6x_{3}- x_{4} =0\\
    0.x_{1} +x_{2}-x_{3} +0.x_{4}= 0
\end{align}

resulting in the matrix equation
\begin{align}
    \label{eq:solutions/chem/6a matrix}
    \myvec{1 & 2 & 0 & -2\\
           1 & 0 & -2 & 0\\
           3 & 2 & -6 & -1\\
           0 & 1 & -1 & 0}\vec{x}
           =\vec{0}
\end{align}

where,
\begin{align}
   \vec{x}= \myvec{x_{1}\\x_{2}\\x_{3}\\x_{4}}
\end{align}

(\ref{eq:solutions/chem/6a matrix}) can be reduced as follows:
\begin{align}
    \myvec{1 & 2 & 0 & -2\\
           1 & 0 & -2 & 0\\
           3 & 2 & -6 & -1\\
           0 & 1 & -1 & 0}
    \xleftrightarrow[R_{3}\leftarrow \frac{R_3}{3}-R_{1}]{R_{2}\leftarrow R_2- R_1}
    \myvec{1 & 2 & 0 & -2\\
           0 & -2 & -2 & 2\\
           0 & -\frac{4}{3} & -2 & \frac{5}{3}\\
           0 & 1 & -1 & 0}\\
    \xleftrightarrow{R_2 \leftarrow -\frac{R_2}{2}}
    \myvec{1 & 2 & 0 & -2\\
          0 & 1 & 1 & -1\\
          0 & -\frac{4}{3} & -2 & \frac{5}{3}\\
          0 & 1 & -1 & 0}\\
    \xleftrightarrow[R_4 \leftarrow R_4- R_2]{R_3 \leftarrow R_3 + \frac{4}{3}R_2}
    \myvec{1 & 2 & 0 & -2\\
           0 & 1 & 1 & -1\\
           0 & 0 & -\frac{2}{3} & \frac{1}{3}\\
           0 & 0 & -2 & 1}\\
    \xleftrightarrow[R_3 \leftarrow -\frac{3}{2}R_3]{R_1 \leftarrow R_1- 2R_2}
    \myvec{1 & 0 & -2 & 0\\
           0 & 1 & 1 & -1\\
           0 & 0 & 1 & -\frac{1}{2}\\
           0 & 0 & -2 & 1}\\
    \xleftrightarrow{R_4\leftarrow R_4 + 2R_3}
    \myvec{1 & 0 & -2 & 0\\
           0 & 1 & 1 & -1\\
           0 & 0 & 1 & -\frac{1}{2}\\
           0 & 0 & 0 & 0}\\
    \xleftrightarrow[R_2\leftarrow R_2-R_3]{R_1\leftarrow R_1 + 2R_3}
    \myvec{1 & 0 & 0 & -1\\
           0 & 1 & 0 & -\frac{1}{2}\\
           0 & 0 & 1 & -\frac{1}{2}\\
           0 & 0 & 0 & 0}
\end{align}

Thus,
\begin{align}
    x_1=x_4, x_2= \frac{1}{2}x_4, x_3=\frac{1}{2}x_4\\
    \implies \quad\vec{x}= x_4\myvec{1\\ \frac{1}{2}\\ \frac{1}{2}\\1} =\myvec{2\\1\\1\\2}
\end{align} 
by substituting $x_4= 2$.

\hfill\break
%\vspace{5mm} 
Hence, (\ref{eq:solutions/chem/6abalanced}) finally becomes
\begin{align}
    2HNO_{3}+ Ca(OH)_{2}\to Ca(NO_{3})_{2}+ 2H_{2}O
\end{align}

\item Find points on the curve 
$
\vec{x}^T\myvec{\frac{1}{9} & 0 \\ 0 & \frac{1}{16}}\vec{x} = 1
$
%
at which tangents are
\begin{enumerate}
\item  parallel to x-axis
\item  parallel to y-axis.
\end{enumerate}
\solution 
Let the balanced version of (\ref{eq:solutions/chem/6ato balance}) be
\begin{align}
    \label{eq:solutions/chem/6abalanced}x_{1}HNO_{3}+ x_{2}Ca(OH)_{2}\to x_{3}Ca(NO_{3})_{2}+ x_{4}H_{2}O
\end{align}

which results in the following equations:
\begin{align}
    (x_{1}+ 2x_{2}-2x_{4}) H= 0\\
    (x_{1}-2x_{3}) N= 0\\
    (3x_{1}+ 2x_{2}-6x_{3}- x_{4}) O=0\\
    (x_{2}-x_{3}) Ca= 0
\end{align}

which can be expressed as
\begin{align}
    x_{1}+ 2x_{2}+ 0.x_{3} -2x_{4} = 0\\
    x_{1}+ 0.x_{2} -2x_{3} +0.x_{4}= 0\\
    3x_{1}+ 2x_{2}-6x_{3}- x_{4} =0\\
    0.x_{1} +x_{2}-x_{3} +0.x_{4}= 0
\end{align}

resulting in the matrix equation
\begin{align}
    \label{eq:solutions/chem/6a matrix}
    \myvec{1 & 2 & 0 & -2\\
           1 & 0 & -2 & 0\\
           3 & 2 & -6 & -1\\
           0 & 1 & -1 & 0}\vec{x}
           =\vec{0}
\end{align}

where,
\begin{align}
   \vec{x}= \myvec{x_{1}\\x_{2}\\x_{3}\\x_{4}}
\end{align}

(\ref{eq:solutions/chem/6a matrix}) can be reduced as follows:
\begin{align}
    \myvec{1 & 2 & 0 & -2\\
           1 & 0 & -2 & 0\\
           3 & 2 & -6 & -1\\
           0 & 1 & -1 & 0}
    \xleftrightarrow[R_{3}\leftarrow \frac{R_3}{3}-R_{1}]{R_{2}\leftarrow R_2- R_1}
    \myvec{1 & 2 & 0 & -2\\
           0 & -2 & -2 & 2\\
           0 & -\frac{4}{3} & -2 & \frac{5}{3}\\
           0 & 1 & -1 & 0}\\
    \xleftrightarrow{R_2 \leftarrow -\frac{R_2}{2}}
    \myvec{1 & 2 & 0 & -2\\
          0 & 1 & 1 & -1\\
          0 & -\frac{4}{3} & -2 & \frac{5}{3}\\
          0 & 1 & -1 & 0}\\
    \xleftrightarrow[R_4 \leftarrow R_4- R_2]{R_3 \leftarrow R_3 + \frac{4}{3}R_2}
    \myvec{1 & 2 & 0 & -2\\
           0 & 1 & 1 & -1\\
           0 & 0 & -\frac{2}{3} & \frac{1}{3}\\
           0 & 0 & -2 & 1}\\
    \xleftrightarrow[R_3 \leftarrow -\frac{3}{2}R_3]{R_1 \leftarrow R_1- 2R_2}
    \myvec{1 & 0 & -2 & 0\\
           0 & 1 & 1 & -1\\
           0 & 0 & 1 & -\frac{1}{2}\\
           0 & 0 & -2 & 1}\\
    \xleftrightarrow{R_4\leftarrow R_4 + 2R_3}
    \myvec{1 & 0 & -2 & 0\\
           0 & 1 & 1 & -1\\
           0 & 0 & 1 & -\frac{1}{2}\\
           0 & 0 & 0 & 0}\\
    \xleftrightarrow[R_2\leftarrow R_2-R_3]{R_1\leftarrow R_1 + 2R_3}
    \myvec{1 & 0 & 0 & -1\\
           0 & 1 & 0 & -\frac{1}{2}\\
           0 & 0 & 1 & -\frac{1}{2}\\
           0 & 0 & 0 & 0}
\end{align}

Thus,
\begin{align}
    x_1=x_4, x_2= \frac{1}{2}x_4, x_3=\frac{1}{2}x_4\\
    \implies \quad\vec{x}= x_4\myvec{1\\ \frac{1}{2}\\ \frac{1}{2}\\1} =\myvec{2\\1\\1\\2}
\end{align} 
by substituting $x_4= 2$.

\hfill\break
%\vspace{5mm} 
Hence, (\ref{eq:solutions/chem/6abalanced}) finally becomes
\begin{align}
    2HNO_{3}+ Ca(OH)_{2}\to Ca(NO_{3})_{2}+ 2H_{2}O
\end{align}

\item Find the equations of the tangent and normal to the given curves at the indicated points:
$
y = x^2
$
at \myvec{0\\0}.
\item Find the equation of the tangent line to the curve $y = x^2-2x+7$
\begin{enumerate}
%
\item  parallel to the line $\myvec{2 & -1}\vec{x}= -9$ 
\item  perpendicular to the line $\myvec{-15 & 5}\vec{x} = 13$. 
\end{enumerate}
\item  Find the equation of the tangent to the curve,
\begin{align}
y = \sqrt{3x-2}
\label{eq:solutions/1/19/Q}
\end{align}
 which is parallel to the line,
\begin{align}
\myvec{4&2}\vec{x}+5=0
\label{eq:solutions/1/19/P}
\end{align}  
\solution 
Let the balanced version of (\ref{eq:solutions/chem/6ato balance}) be
\begin{align}
    \label{eq:solutions/chem/6abalanced}x_{1}HNO_{3}+ x_{2}Ca(OH)_{2}\to x_{3}Ca(NO_{3})_{2}+ x_{4}H_{2}O
\end{align}

which results in the following equations:
\begin{align}
    (x_{1}+ 2x_{2}-2x_{4}) H= 0\\
    (x_{1}-2x_{3}) N= 0\\
    (3x_{1}+ 2x_{2}-6x_{3}- x_{4}) O=0\\
    (x_{2}-x_{3}) Ca= 0
\end{align}

which can be expressed as
\begin{align}
    x_{1}+ 2x_{2}+ 0.x_{3} -2x_{4} = 0\\
    x_{1}+ 0.x_{2} -2x_{3} +0.x_{4}= 0\\
    3x_{1}+ 2x_{2}-6x_{3}- x_{4} =0\\
    0.x_{1} +x_{2}-x_{3} +0.x_{4}= 0
\end{align}

resulting in the matrix equation
\begin{align}
    \label{eq:solutions/chem/6a matrix}
    \myvec{1 & 2 & 0 & -2\\
           1 & 0 & -2 & 0\\
           3 & 2 & -6 & -1\\
           0 & 1 & -1 & 0}\vec{x}
           =\vec{0}
\end{align}

where,
\begin{align}
   \vec{x}= \myvec{x_{1}\\x_{2}\\x_{3}\\x_{4}}
\end{align}

(\ref{eq:solutions/chem/6a matrix}) can be reduced as follows:
\begin{align}
    \myvec{1 & 2 & 0 & -2\\
           1 & 0 & -2 & 0\\
           3 & 2 & -6 & -1\\
           0 & 1 & -1 & 0}
    \xleftrightarrow[R_{3}\leftarrow \frac{R_3}{3}-R_{1}]{R_{2}\leftarrow R_2- R_1}
    \myvec{1 & 2 & 0 & -2\\
           0 & -2 & -2 & 2\\
           0 & -\frac{4}{3} & -2 & \frac{5}{3}\\
           0 & 1 & -1 & 0}\\
    \xleftrightarrow{R_2 \leftarrow -\frac{R_2}{2}}
    \myvec{1 & 2 & 0 & -2\\
          0 & 1 & 1 & -1\\
          0 & -\frac{4}{3} & -2 & \frac{5}{3}\\
          0 & 1 & -1 & 0}\\
    \xleftrightarrow[R_4 \leftarrow R_4- R_2]{R_3 \leftarrow R_3 + \frac{4}{3}R_2}
    \myvec{1 & 2 & 0 & -2\\
           0 & 1 & 1 & -1\\
           0 & 0 & -\frac{2}{3} & \frac{1}{3}\\
           0 & 0 & -2 & 1}\\
    \xleftrightarrow[R_3 \leftarrow -\frac{3}{2}R_3]{R_1 \leftarrow R_1- 2R_2}
    \myvec{1 & 0 & -2 & 0\\
           0 & 1 & 1 & -1\\
           0 & 0 & 1 & -\frac{1}{2}\\
           0 & 0 & -2 & 1}\\
    \xleftrightarrow{R_4\leftarrow R_4 + 2R_3}
    \myvec{1 & 0 & -2 & 0\\
           0 & 1 & 1 & -1\\
           0 & 0 & 1 & -\frac{1}{2}\\
           0 & 0 & 0 & 0}\\
    \xleftrightarrow[R_2\leftarrow R_2-R_3]{R_1\leftarrow R_1 + 2R_3}
    \myvec{1 & 0 & 0 & -1\\
           0 & 1 & 0 & -\frac{1}{2}\\
           0 & 0 & 1 & -\frac{1}{2}\\
           0 & 0 & 0 & 0}
\end{align}

Thus,
\begin{align}
    x_1=x_4, x_2= \frac{1}{2}x_4, x_3=\frac{1}{2}x_4\\
    \implies \quad\vec{x}= x_4\myvec{1\\ \frac{1}{2}\\ \frac{1}{2}\\1} =\myvec{2\\1\\1\\2}
\end{align} 
by substituting $x_4= 2$.

\hfill\break
%\vspace{5mm} 
Hence, (\ref{eq:solutions/chem/6abalanced}) finally becomes
\begin{align}
    2HNO_{3}+ Ca(OH)_{2}\to Ca(NO_{3})_{2}+ 2H_{2}O
\end{align}

%
\item Find the point at which the line $\myvec{-1 & 1}\vec{x} =  1$ is a tangent to the curve $y^2 = 4x$.
%
\item The line $\myvec{-m & 1}\vec{x} = 1$ is a tangent to the curve $y^2 = 4x$.  Find the value of $m$.
\item  Find the normal at the point \myvec{1\\1} on the curve $2y + x^2 = 3$ 
\item  Find the normal to the curve $x^2=4y$ passing through $\myvec{1\\2}$.
%
\item Find the area of the region bounded by the curve $y^2= x$ and the lines $x = 1, x = 4$ and the x-axis in the first quadrant.
\item  Find the area of the region bounded by $y^2=9x, x=2, x=4$ and the x-axis in the  first quadrant.
%
\item Find the area of the region bounded by $x^2 = 4y, y = 2, y = 4$ and the y-axis in the first quadrant.
\item Find the area of the region bounded by the ellipse 
$
\vec{x}^T\myvec{\frac{1}{16} & 0 \\ 0 & \frac{1}{9}}\vec{x} = 1
$

\item  Find the area of the region bounded by the ellipse 
$
\vec{x}^T\myvec{\frac{1}{4} & 0 \\ 0 & \frac{1}{9}}\vec{x} = 1
$
\item The area between $x=y^2$ and $x=4$ is divided into two equal parts by the line $x=a$, find the value of $a$.
\item  Find the area of the region bounded by the parabola $y = x^2$ and $y = \abs{x}$.
\item  Find the area bounded by the curve $x^2 = 4y$ and the line $\myvec{1 & -1}\vec{x} = -2$.
\item  Find the area of the region bounded by the curve $y^2 = 4x$ and the line $x = 3$.
%
\item Find the area of the region bounded by the curve $y^2 = x$, y-axis and the line $y = 3$.
%
\item Find the area of the region bounded by the two parabolas $y = x^2, y^2=x$.
\item Find the area lying above x-axis and included between the circle $\vec{x}^T\vec{x} -8\myvec{1 & 0}= 0$  and inside of the parabola $y^2 = 4x$.
%
\item AOBA is the part of the ellipse 
$
\vec{x}^T\myvec{9 & 0 \\ 0 & 1}\vec{x} = 36
$
in the first quadrant such that $OA = 2$ and $OB = 6$. Find the area between the arc $AB$ and the chord $AB$.
\item Find the area lying between the curves $y^2 = 4x$ and $y = 2x$.
\item  Find the area of the region bounded by the curves $y = x^2+2, y = x, x = 0$ and $ x = 3.$
%
\item Find the area under $y = x^2, x = 1, x = 2$ and x-axis.
\item Find the area between  $y = x^2$ and $y = x$.
\item Find the area of the region lying in the first quadrant and bounded by $y = 4x^2, x = 0, y = 1$ and $y = 4$.
\item Find the area enclosed by the parabola $4y = 3x^2$ and the line $\myvec{-3 & 2}\vec{x} = 12$.
%
\item Find the area of the smaller region bounded by the ellipse
$
\vec{x}^T\myvec{\frac{1}{9} & 0 \\ 0 & \frac{1}{4}}\vec{x} = 1
$
and the line 
$
\myvec{\frac{1}{a} & \frac{1}{b}}\vec{x} = 1
$
\item Find the area of the region enclosed by the parabola $x^2=y$, the line $\myvec{-1 & 1}\vec{x} = 2$ and the x-axis.
%
\item Find the area bounded by the curves
\begin{align}
\cbrak{\brak{x,y} : y > x^2, y = \abs{x}}
\end{align}
%
\item Find the area of the region
\begin{align}
\cbrak{\brak{x,y} : y^2 \le 4x, 4\vec{x}^T\vec{x} = 9}
\end{align}
%
\item Find the area of the circle $\vec{x}^T\vec{x} = 16$ exterior to the parabola $y^2 = 6$.
\end{enumerate}

%\renewcommand{\theequation}{\theenumi}
%\begin{enumerate}[label=\arabic*.,ref=\theenumi]
\begin{enumerate}[label=\thesection.\arabic*.,ref=\thesection.\theenumi]
\numberwithin{equation}{enumi}

\item Find the area of the region enclosed between the two circles: $\vec{x}^T\vec{x} = 4$ and $\norm{\vec{x}-\myvec{2\\0}} = 2$.
\\
\solution 
Let the balanced version of (\ref{eq:solutions/chem/6ato balance}) be
\begin{align}
    \label{eq:solutions/chem/6abalanced}x_{1}HNO_{3}+ x_{2}Ca(OH)_{2}\to x_{3}Ca(NO_{3})_{2}+ x_{4}H_{2}O
\end{align}

which results in the following equations:
\begin{align}
    (x_{1}+ 2x_{2}-2x_{4}) H= 0\\
    (x_{1}-2x_{3}) N= 0\\
    (3x_{1}+ 2x_{2}-6x_{3}- x_{4}) O=0\\
    (x_{2}-x_{3}) Ca= 0
\end{align}

which can be expressed as
\begin{align}
    x_{1}+ 2x_{2}+ 0.x_{3} -2x_{4} = 0\\
    x_{1}+ 0.x_{2} -2x_{3} +0.x_{4}= 0\\
    3x_{1}+ 2x_{2}-6x_{3}- x_{4} =0\\
    0.x_{1} +x_{2}-x_{3} +0.x_{4}= 0
\end{align}

resulting in the matrix equation
\begin{align}
    \label{eq:solutions/chem/6a matrix}
    \myvec{1 & 2 & 0 & -2\\
           1 & 0 & -2 & 0\\
           3 & 2 & -6 & -1\\
           0 & 1 & -1 & 0}\vec{x}
           =\vec{0}
\end{align}

where,
\begin{align}
   \vec{x}= \myvec{x_{1}\\x_{2}\\x_{3}\\x_{4}}
\end{align}

(\ref{eq:solutions/chem/6a matrix}) can be reduced as follows:
\begin{align}
    \myvec{1 & 2 & 0 & -2\\
           1 & 0 & -2 & 0\\
           3 & 2 & -6 & -1\\
           0 & 1 & -1 & 0}
    \xleftrightarrow[R_{3}\leftarrow \frac{R_3}{3}-R_{1}]{R_{2}\leftarrow R_2- R_1}
    \myvec{1 & 2 & 0 & -2\\
           0 & -2 & -2 & 2\\
           0 & -\frac{4}{3} & -2 & \frac{5}{3}\\
           0 & 1 & -1 & 0}\\
    \xleftrightarrow{R_2 \leftarrow -\frac{R_2}{2}}
    \myvec{1 & 2 & 0 & -2\\
          0 & 1 & 1 & -1\\
          0 & -\frac{4}{3} & -2 & \frac{5}{3}\\
          0 & 1 & -1 & 0}\\
    \xleftrightarrow[R_4 \leftarrow R_4- R_2]{R_3 \leftarrow R_3 + \frac{4}{3}R_2}
    \myvec{1 & 2 & 0 & -2\\
           0 & 1 & 1 & -1\\
           0 & 0 & -\frac{2}{3} & \frac{1}{3}\\
           0 & 0 & -2 & 1}\\
    \xleftrightarrow[R_3 \leftarrow -\frac{3}{2}R_3]{R_1 \leftarrow R_1- 2R_2}
    \myvec{1 & 0 & -2 & 0\\
           0 & 1 & 1 & -1\\
           0 & 0 & 1 & -\frac{1}{2}\\
           0 & 0 & -2 & 1}\\
    \xleftrightarrow{R_4\leftarrow R_4 + 2R_3}
    \myvec{1 & 0 & -2 & 0\\
           0 & 1 & 1 & -1\\
           0 & 0 & 1 & -\frac{1}{2}\\
           0 & 0 & 0 & 0}\\
    \xleftrightarrow[R_2\leftarrow R_2-R_3]{R_1\leftarrow R_1 + 2R_3}
    \myvec{1 & 0 & 0 & -1\\
           0 & 1 & 0 & -\frac{1}{2}\\
           0 & 0 & 1 & -\frac{1}{2}\\
           0 & 0 & 0 & 0}
\end{align}

Thus,
\begin{align}
    x_1=x_4, x_2= \frac{1}{2}x_4, x_3=\frac{1}{2}x_4\\
    \implies \quad\vec{x}= x_4\myvec{1\\ \frac{1}{2}\\ \frac{1}{2}\\1} =\myvec{2\\1\\1\\2}
\end{align} 
by substituting $x_4= 2$.

\hfill\break
%\vspace{5mm} 
Hence, (\ref{eq:solutions/chem/6abalanced}) finally becomes
\begin{align}
    2HNO_{3}+ Ca(OH)_{2}\to Ca(NO_{3})_{2}+ 2H_{2}O
\end{align}

\item Find the equation of the circle with radius 5 whose centre lies on x-axis and passes through the point \myvec{2\\3}.
\\
\solution 
Let the balanced version of (\ref{eq:solutions/chem/6ato balance}) be
\begin{align}
    \label{eq:solutions/chem/6abalanced}x_{1}HNO_{3}+ x_{2}Ca(OH)_{2}\to x_{3}Ca(NO_{3})_{2}+ x_{4}H_{2}O
\end{align}

which results in the following equations:
\begin{align}
    (x_{1}+ 2x_{2}-2x_{4}) H= 0\\
    (x_{1}-2x_{3}) N= 0\\
    (3x_{1}+ 2x_{2}-6x_{3}- x_{4}) O=0\\
    (x_{2}-x_{3}) Ca= 0
\end{align}

which can be expressed as
\begin{align}
    x_{1}+ 2x_{2}+ 0.x_{3} -2x_{4} = 0\\
    x_{1}+ 0.x_{2} -2x_{3} +0.x_{4}= 0\\
    3x_{1}+ 2x_{2}-6x_{3}- x_{4} =0\\
    0.x_{1} +x_{2}-x_{3} +0.x_{4}= 0
\end{align}

resulting in the matrix equation
\begin{align}
    \label{eq:solutions/chem/6a matrix}
    \myvec{1 & 2 & 0 & -2\\
           1 & 0 & -2 & 0\\
           3 & 2 & -6 & -1\\
           0 & 1 & -1 & 0}\vec{x}
           =\vec{0}
\end{align}

where,
\begin{align}
   \vec{x}= \myvec{x_{1}\\x_{2}\\x_{3}\\x_{4}}
\end{align}

(\ref{eq:solutions/chem/6a matrix}) can be reduced as follows:
\begin{align}
    \myvec{1 & 2 & 0 & -2\\
           1 & 0 & -2 & 0\\
           3 & 2 & -6 & -1\\
           0 & 1 & -1 & 0}
    \xleftrightarrow[R_{3}\leftarrow \frac{R_3}{3}-R_{1}]{R_{2}\leftarrow R_2- R_1}
    \myvec{1 & 2 & 0 & -2\\
           0 & -2 & -2 & 2\\
           0 & -\frac{4}{3} & -2 & \frac{5}{3}\\
           0 & 1 & -1 & 0}\\
    \xleftrightarrow{R_2 \leftarrow -\frac{R_2}{2}}
    \myvec{1 & 2 & 0 & -2\\
          0 & 1 & 1 & -1\\
          0 & -\frac{4}{3} & -2 & \frac{5}{3}\\
          0 & 1 & -1 & 0}\\
    \xleftrightarrow[R_4 \leftarrow R_4- R_2]{R_3 \leftarrow R_3 + \frac{4}{3}R_2}
    \myvec{1 & 2 & 0 & -2\\
           0 & 1 & 1 & -1\\
           0 & 0 & -\frac{2}{3} & \frac{1}{3}\\
           0 & 0 & -2 & 1}\\
    \xleftrightarrow[R_3 \leftarrow -\frac{3}{2}R_3]{R_1 \leftarrow R_1- 2R_2}
    \myvec{1 & 0 & -2 & 0\\
           0 & 1 & 1 & -1\\
           0 & 0 & 1 & -\frac{1}{2}\\
           0 & 0 & -2 & 1}\\
    \xleftrightarrow{R_4\leftarrow R_4 + 2R_3}
    \myvec{1 & 0 & -2 & 0\\
           0 & 1 & 1 & -1\\
           0 & 0 & 1 & -\frac{1}{2}\\
           0 & 0 & 0 & 0}\\
    \xleftrightarrow[R_2\leftarrow R_2-R_3]{R_1\leftarrow R_1 + 2R_3}
    \myvec{1 & 0 & 0 & -1\\
           0 & 1 & 0 & -\frac{1}{2}\\
           0 & 0 & 1 & -\frac{1}{2}\\
           0 & 0 & 0 & 0}
\end{align}

Thus,
\begin{align}
    x_1=x_4, x_2= \frac{1}{2}x_4, x_3=\frac{1}{2}x_4\\
    \implies \quad\vec{x}= x_4\myvec{1\\ \frac{1}{2}\\ \frac{1}{2}\\1} =\myvec{2\\1\\1\\2}
\end{align} 
by substituting $x_4= 2$.

\hfill\break
%\vspace{5mm} 
Hence, (\ref{eq:solutions/chem/6abalanced}) finally becomes
\begin{align}
    2HNO_{3}+ Ca(OH)_{2}\to Ca(NO_{3})_{2}+ 2H_{2}O
\end{align}

\item Find the equation of the circle passing through \myvec{0\\0} and making intercepts a and b on the coordinate axes.
\item Find the equation of a circle with centre \myvec{2\\2} and passes through the point \myvec{4\\5}. 
\\
\solution 
Let the balanced version of (\ref{eq:solutions/chem/6ato balance}) be
\begin{align}
    \label{eq:solutions/chem/6abalanced}x_{1}HNO_{3}+ x_{2}Ca(OH)_{2}\to x_{3}Ca(NO_{3})_{2}+ x_{4}H_{2}O
\end{align}

which results in the following equations:
\begin{align}
    (x_{1}+ 2x_{2}-2x_{4}) H= 0\\
    (x_{1}-2x_{3}) N= 0\\
    (3x_{1}+ 2x_{2}-6x_{3}- x_{4}) O=0\\
    (x_{2}-x_{3}) Ca= 0
\end{align}

which can be expressed as
\begin{align}
    x_{1}+ 2x_{2}+ 0.x_{3} -2x_{4} = 0\\
    x_{1}+ 0.x_{2} -2x_{3} +0.x_{4}= 0\\
    3x_{1}+ 2x_{2}-6x_{3}- x_{4} =0\\
    0.x_{1} +x_{2}-x_{3} +0.x_{4}= 0
\end{align}

resulting in the matrix equation
\begin{align}
    \label{eq:solutions/chem/6a matrix}
    \myvec{1 & 2 & 0 & -2\\
           1 & 0 & -2 & 0\\
           3 & 2 & -6 & -1\\
           0 & 1 & -1 & 0}\vec{x}
           =\vec{0}
\end{align}

where,
\begin{align}
   \vec{x}= \myvec{x_{1}\\x_{2}\\x_{3}\\x_{4}}
\end{align}

(\ref{eq:solutions/chem/6a matrix}) can be reduced as follows:
\begin{align}
    \myvec{1 & 2 & 0 & -2\\
           1 & 0 & -2 & 0\\
           3 & 2 & -6 & -1\\
           0 & 1 & -1 & 0}
    \xleftrightarrow[R_{3}\leftarrow \frac{R_3}{3}-R_{1}]{R_{2}\leftarrow R_2- R_1}
    \myvec{1 & 2 & 0 & -2\\
           0 & -2 & -2 & 2\\
           0 & -\frac{4}{3} & -2 & \frac{5}{3}\\
           0 & 1 & -1 & 0}\\
    \xleftrightarrow{R_2 \leftarrow -\frac{R_2}{2}}
    \myvec{1 & 2 & 0 & -2\\
          0 & 1 & 1 & -1\\
          0 & -\frac{4}{3} & -2 & \frac{5}{3}\\
          0 & 1 & -1 & 0}\\
    \xleftrightarrow[R_4 \leftarrow R_4- R_2]{R_3 \leftarrow R_3 + \frac{4}{3}R_2}
    \myvec{1 & 2 & 0 & -2\\
           0 & 1 & 1 & -1\\
           0 & 0 & -\frac{2}{3} & \frac{1}{3}\\
           0 & 0 & -2 & 1}\\
    \xleftrightarrow[R_3 \leftarrow -\frac{3}{2}R_3]{R_1 \leftarrow R_1- 2R_2}
    \myvec{1 & 0 & -2 & 0\\
           0 & 1 & 1 & -1\\
           0 & 0 & 1 & -\frac{1}{2}\\
           0 & 0 & -2 & 1}\\
    \xleftrightarrow{R_4\leftarrow R_4 + 2R_3}
    \myvec{1 & 0 & -2 & 0\\
           0 & 1 & 1 & -1\\
           0 & 0 & 1 & -\frac{1}{2}\\
           0 & 0 & 0 & 0}\\
    \xleftrightarrow[R_2\leftarrow R_2-R_3]{R_1\leftarrow R_1 + 2R_3}
    \myvec{1 & 0 & 0 & -1\\
           0 & 1 & 0 & -\frac{1}{2}\\
           0 & 0 & 1 & -\frac{1}{2}\\
           0 & 0 & 0 & 0}
\end{align}

Thus,
\begin{align}
    x_1=x_4, x_2= \frac{1}{2}x_4, x_3=\frac{1}{2}x_4\\
    \implies \quad\vec{x}= x_4\myvec{1\\ \frac{1}{2}\\ \frac{1}{2}\\1} =\myvec{2\\1\\1\\2}
\end{align} 
by substituting $x_4= 2$.

\hfill\break
%\vspace{5mm} 
Hence, (\ref{eq:solutions/chem/6abalanced}) finally becomes
\begin{align}
    2HNO_{3}+ Ca(OH)_{2}\to Ca(NO_{3})_{2}+ 2H_{2}O
\end{align}

\item Find the locus of all the unit vectors in the xy-plane.
%
\item Find the points on the curve $\vec{x}^T\vec{x}-2\myvec{1 & 0}\vec{x} -3 =0$  at which the tangents are parallel to the x-axis.
%
\\
\solution
Let the balanced version of (\ref{eq:solutions/chem/6ato balance}) be
\begin{align}
    \label{eq:solutions/chem/6abalanced}x_{1}HNO_{3}+ x_{2}Ca(OH)_{2}\to x_{3}Ca(NO_{3})_{2}+ x_{4}H_{2}O
\end{align}

which results in the following equations:
\begin{align}
    (x_{1}+ 2x_{2}-2x_{4}) H= 0\\
    (x_{1}-2x_{3}) N= 0\\
    (3x_{1}+ 2x_{2}-6x_{3}- x_{4}) O=0\\
    (x_{2}-x_{3}) Ca= 0
\end{align}

which can be expressed as
\begin{align}
    x_{1}+ 2x_{2}+ 0.x_{3} -2x_{4} = 0\\
    x_{1}+ 0.x_{2} -2x_{3} +0.x_{4}= 0\\
    3x_{1}+ 2x_{2}-6x_{3}- x_{4} =0\\
    0.x_{1} +x_{2}-x_{3} +0.x_{4}= 0
\end{align}

resulting in the matrix equation
\begin{align}
    \label{eq:solutions/chem/6a matrix}
    \myvec{1 & 2 & 0 & -2\\
           1 & 0 & -2 & 0\\
           3 & 2 & -6 & -1\\
           0 & 1 & -1 & 0}\vec{x}
           =\vec{0}
\end{align}

where,
\begin{align}
   \vec{x}= \myvec{x_{1}\\x_{2}\\x_{3}\\x_{4}}
\end{align}

(\ref{eq:solutions/chem/6a matrix}) can be reduced as follows:
\begin{align}
    \myvec{1 & 2 & 0 & -2\\
           1 & 0 & -2 & 0\\
           3 & 2 & -6 & -1\\
           0 & 1 & -1 & 0}
    \xleftrightarrow[R_{3}\leftarrow \frac{R_3}{3}-R_{1}]{R_{2}\leftarrow R_2- R_1}
    \myvec{1 & 2 & 0 & -2\\
           0 & -2 & -2 & 2\\
           0 & -\frac{4}{3} & -2 & \frac{5}{3}\\
           0 & 1 & -1 & 0}\\
    \xleftrightarrow{R_2 \leftarrow -\frac{R_2}{2}}
    \myvec{1 & 2 & 0 & -2\\
          0 & 1 & 1 & -1\\
          0 & -\frac{4}{3} & -2 & \frac{5}{3}\\
          0 & 1 & -1 & 0}\\
    \xleftrightarrow[R_4 \leftarrow R_4- R_2]{R_3 \leftarrow R_3 + \frac{4}{3}R_2}
    \myvec{1 & 2 & 0 & -2\\
           0 & 1 & 1 & -1\\
           0 & 0 & -\frac{2}{3} & \frac{1}{3}\\
           0 & 0 & -2 & 1}\\
    \xleftrightarrow[R_3 \leftarrow -\frac{3}{2}R_3]{R_1 \leftarrow R_1- 2R_2}
    \myvec{1 & 0 & -2 & 0\\
           0 & 1 & 1 & -1\\
           0 & 0 & 1 & -\frac{1}{2}\\
           0 & 0 & -2 & 1}\\
    \xleftrightarrow{R_4\leftarrow R_4 + 2R_3}
    \myvec{1 & 0 & -2 & 0\\
           0 & 1 & 1 & -1\\
           0 & 0 & 1 & -\frac{1}{2}\\
           0 & 0 & 0 & 0}\\
    \xleftrightarrow[R_2\leftarrow R_2-R_3]{R_1\leftarrow R_1 + 2R_3}
    \myvec{1 & 0 & 0 & -1\\
           0 & 1 & 0 & -\frac{1}{2}\\
           0 & 0 & 1 & -\frac{1}{2}\\
           0 & 0 & 0 & 0}
\end{align}

Thus,
\begin{align}
    x_1=x_4, x_2= \frac{1}{2}x_4, x_3=\frac{1}{2}x_4\\
    \implies \quad\vec{x}= x_4\myvec{1\\ \frac{1}{2}\\ \frac{1}{2}\\1} =\myvec{2\\1\\1\\2}
\end{align} 
by substituting $x_4= 2$.

\hfill\break
%\vspace{5mm} 
Hence, (\ref{eq:solutions/chem/6abalanced}) finally becomes
\begin{align}
    2HNO_{3}+ Ca(OH)_{2}\to Ca(NO_{3})_{2}+ 2H_{2}O
\end{align}

\item  Find the area of the region in the first quadrant enclosed by x-axis, line $\myvec{1 & -\sqrt{3}}\vec{x} =0$ and the circle $\vec{x}^T\vec{x}=4$.
%
\\
\solution
Let the balanced version of (\ref{eq:solutions/chem/6ato balance}) be
\begin{align}
    \label{eq:solutions/chem/6abalanced}x_{1}HNO_{3}+ x_{2}Ca(OH)_{2}\to x_{3}Ca(NO_{3})_{2}+ x_{4}H_{2}O
\end{align}

which results in the following equations:
\begin{align}
    (x_{1}+ 2x_{2}-2x_{4}) H= 0\\
    (x_{1}-2x_{3}) N= 0\\
    (3x_{1}+ 2x_{2}-6x_{3}- x_{4}) O=0\\
    (x_{2}-x_{3}) Ca= 0
\end{align}

which can be expressed as
\begin{align}
    x_{1}+ 2x_{2}+ 0.x_{3} -2x_{4} = 0\\
    x_{1}+ 0.x_{2} -2x_{3} +0.x_{4}= 0\\
    3x_{1}+ 2x_{2}-6x_{3}- x_{4} =0\\
    0.x_{1} +x_{2}-x_{3} +0.x_{4}= 0
\end{align}

resulting in the matrix equation
\begin{align}
    \label{eq:solutions/chem/6a matrix}
    \myvec{1 & 2 & 0 & -2\\
           1 & 0 & -2 & 0\\
           3 & 2 & -6 & -1\\
           0 & 1 & -1 & 0}\vec{x}
           =\vec{0}
\end{align}

where,
\begin{align}
   \vec{x}= \myvec{x_{1}\\x_{2}\\x_{3}\\x_{4}}
\end{align}

(\ref{eq:solutions/chem/6a matrix}) can be reduced as follows:
\begin{align}
    \myvec{1 & 2 & 0 & -2\\
           1 & 0 & -2 & 0\\
           3 & 2 & -6 & -1\\
           0 & 1 & -1 & 0}
    \xleftrightarrow[R_{3}\leftarrow \frac{R_3}{3}-R_{1}]{R_{2}\leftarrow R_2- R_1}
    \myvec{1 & 2 & 0 & -2\\
           0 & -2 & -2 & 2\\
           0 & -\frac{4}{3} & -2 & \frac{5}{3}\\
           0 & 1 & -1 & 0}\\
    \xleftrightarrow{R_2 \leftarrow -\frac{R_2}{2}}
    \myvec{1 & 2 & 0 & -2\\
          0 & 1 & 1 & -1\\
          0 & -\frac{4}{3} & -2 & \frac{5}{3}\\
          0 & 1 & -1 & 0}\\
    \xleftrightarrow[R_4 \leftarrow R_4- R_2]{R_3 \leftarrow R_3 + \frac{4}{3}R_2}
    \myvec{1 & 2 & 0 & -2\\
           0 & 1 & 1 & -1\\
           0 & 0 & -\frac{2}{3} & \frac{1}{3}\\
           0 & 0 & -2 & 1}\\
    \xleftrightarrow[R_3 \leftarrow -\frac{3}{2}R_3]{R_1 \leftarrow R_1- 2R_2}
    \myvec{1 & 0 & -2 & 0\\
           0 & 1 & 1 & -1\\
           0 & 0 & 1 & -\frac{1}{2}\\
           0 & 0 & -2 & 1}\\
    \xleftrightarrow{R_4\leftarrow R_4 + 2R_3}
    \myvec{1 & 0 & -2 & 0\\
           0 & 1 & 1 & -1\\
           0 & 0 & 1 & -\frac{1}{2}\\
           0 & 0 & 0 & 0}\\
    \xleftrightarrow[R_2\leftarrow R_2-R_3]{R_1\leftarrow R_1 + 2R_3}
    \myvec{1 & 0 & 0 & -1\\
           0 & 1 & 0 & -\frac{1}{2}\\
           0 & 0 & 1 & -\frac{1}{2}\\
           0 & 0 & 0 & 0}
\end{align}

Thus,
\begin{align}
    x_1=x_4, x_2= \frac{1}{2}x_4, x_3=\frac{1}{2}x_4\\
    \implies \quad\vec{x}= x_4\myvec{1\\ \frac{1}{2}\\ \frac{1}{2}\\1} =\myvec{2\\1\\1\\2}
\end{align} 
by substituting $x_4= 2$.

\hfill\break
%\vspace{5mm} 
Hence, (\ref{eq:solutions/chem/6abalanced}) finally becomes
\begin{align}
    2HNO_{3}+ Ca(OH)_{2}\to Ca(NO_{3})_{2}+ 2H_{2}O
\end{align}

\item Find the area lying in the first quadrant and bounded by the circle $\vec{x}^T\vec{x}=4$ and the lines $x = 0$ and $x = 2$.
%
\item Find the area of the circle $4\vec{x}^T\vec{x}=9$.
\item  Find the area bounded by curves $\norm{\vec{x}-\myvec{1\\0}} = 1$ and $\norm{\vec{x}}=1$
\\
\solution
Let the balanced version of (\ref{eq:solutions/chem/6ato balance}) be
\begin{align}
    \label{eq:solutions/chem/6abalanced}x_{1}HNO_{3}+ x_{2}Ca(OH)_{2}\to x_{3}Ca(NO_{3})_{2}+ x_{4}H_{2}O
\end{align}

which results in the following equations:
\begin{align}
    (x_{1}+ 2x_{2}-2x_{4}) H= 0\\
    (x_{1}-2x_{3}) N= 0\\
    (3x_{1}+ 2x_{2}-6x_{3}- x_{4}) O=0\\
    (x_{2}-x_{3}) Ca= 0
\end{align}

which can be expressed as
\begin{align}
    x_{1}+ 2x_{2}+ 0.x_{3} -2x_{4} = 0\\
    x_{1}+ 0.x_{2} -2x_{3} +0.x_{4}= 0\\
    3x_{1}+ 2x_{2}-6x_{3}- x_{4} =0\\
    0.x_{1} +x_{2}-x_{3} +0.x_{4}= 0
\end{align}

resulting in the matrix equation
\begin{align}
    \label{eq:solutions/chem/6a matrix}
    \myvec{1 & 2 & 0 & -2\\
           1 & 0 & -2 & 0\\
           3 & 2 & -6 & -1\\
           0 & 1 & -1 & 0}\vec{x}
           =\vec{0}
\end{align}

where,
\begin{align}
   \vec{x}= \myvec{x_{1}\\x_{2}\\x_{3}\\x_{4}}
\end{align}

(\ref{eq:solutions/chem/6a matrix}) can be reduced as follows:
\begin{align}
    \myvec{1 & 2 & 0 & -2\\
           1 & 0 & -2 & 0\\
           3 & 2 & -6 & -1\\
           0 & 1 & -1 & 0}
    \xleftrightarrow[R_{3}\leftarrow \frac{R_3}{3}-R_{1}]{R_{2}\leftarrow R_2- R_1}
    \myvec{1 & 2 & 0 & -2\\
           0 & -2 & -2 & 2\\
           0 & -\frac{4}{3} & -2 & \frac{5}{3}\\
           0 & 1 & -1 & 0}\\
    \xleftrightarrow{R_2 \leftarrow -\frac{R_2}{2}}
    \myvec{1 & 2 & 0 & -2\\
          0 & 1 & 1 & -1\\
          0 & -\frac{4}{3} & -2 & \frac{5}{3}\\
          0 & 1 & -1 & 0}\\
    \xleftrightarrow[R_4 \leftarrow R_4- R_2]{R_3 \leftarrow R_3 + \frac{4}{3}R_2}
    \myvec{1 & 2 & 0 & -2\\
           0 & 1 & 1 & -1\\
           0 & 0 & -\frac{2}{3} & \frac{1}{3}\\
           0 & 0 & -2 & 1}\\
    \xleftrightarrow[R_3 \leftarrow -\frac{3}{2}R_3]{R_1 \leftarrow R_1- 2R_2}
    \myvec{1 & 0 & -2 & 0\\
           0 & 1 & 1 & -1\\
           0 & 0 & 1 & -\frac{1}{2}\\
           0 & 0 & -2 & 1}\\
    \xleftrightarrow{R_4\leftarrow R_4 + 2R_3}
    \myvec{1 & 0 & -2 & 0\\
           0 & 1 & 1 & -1\\
           0 & 0 & 1 & -\frac{1}{2}\\
           0 & 0 & 0 & 0}\\
    \xleftrightarrow[R_2\leftarrow R_2-R_3]{R_1\leftarrow R_1 + 2R_3}
    \myvec{1 & 0 & 0 & -1\\
           0 & 1 & 0 & -\frac{1}{2}\\
           0 & 0 & 1 & -\frac{1}{2}\\
           0 & 0 & 0 & 0}
\end{align}

Thus,
\begin{align}
    x_1=x_4, x_2= \frac{1}{2}x_4, x_3=\frac{1}{2}x_4\\
    \implies \quad\vec{x}= x_4\myvec{1\\ \frac{1}{2}\\ \frac{1}{2}\\1} =\myvec{2\\1\\1\\2}
\end{align} 
by substituting $x_4= 2$.

\hfill\break
%\vspace{5mm} 
Hence, (\ref{eq:solutions/chem/6abalanced}) finally becomes
\begin{align}
    2HNO_{3}+ Ca(OH)_{2}\to Ca(NO_{3})_{2}+ 2H_{2}O
\end{align}

\item Find the smaller area enclosed by the circle $\vec{x}^T\vec{x}=4$ and the line $\myvec{1 & 1}\vec{x} = 2$.


\item Find the slope of the tangent to the curve $y = \frac{x-1}{x-2}, x\ne 2$ at $x = 10$.
\\
\solution
Let the balanced version of (\ref{eq:solutions/chem/6ato balance}) be
\begin{align}
    \label{eq:solutions/chem/6abalanced}x_{1}HNO_{3}+ x_{2}Ca(OH)_{2}\to x_{3}Ca(NO_{3})_{2}+ x_{4}H_{2}O
\end{align}

which results in the following equations:
\begin{align}
    (x_{1}+ 2x_{2}-2x_{4}) H= 0\\
    (x_{1}-2x_{3}) N= 0\\
    (3x_{1}+ 2x_{2}-6x_{3}- x_{4}) O=0\\
    (x_{2}-x_{3}) Ca= 0
\end{align}

which can be expressed as
\begin{align}
    x_{1}+ 2x_{2}+ 0.x_{3} -2x_{4} = 0\\
    x_{1}+ 0.x_{2} -2x_{3} +0.x_{4}= 0\\
    3x_{1}+ 2x_{2}-6x_{3}- x_{4} =0\\
    0.x_{1} +x_{2}-x_{3} +0.x_{4}= 0
\end{align}

resulting in the matrix equation
\begin{align}
    \label{eq:solutions/chem/6a matrix}
    \myvec{1 & 2 & 0 & -2\\
           1 & 0 & -2 & 0\\
           3 & 2 & -6 & -1\\
           0 & 1 & -1 & 0}\vec{x}
           =\vec{0}
\end{align}

where,
\begin{align}
   \vec{x}= \myvec{x_{1}\\x_{2}\\x_{3}\\x_{4}}
\end{align}

(\ref{eq:solutions/chem/6a matrix}) can be reduced as follows:
\begin{align}
    \myvec{1 & 2 & 0 & -2\\
           1 & 0 & -2 & 0\\
           3 & 2 & -6 & -1\\
           0 & 1 & -1 & 0}
    \xleftrightarrow[R_{3}\leftarrow \frac{R_3}{3}-R_{1}]{R_{2}\leftarrow R_2- R_1}
    \myvec{1 & 2 & 0 & -2\\
           0 & -2 & -2 & 2\\
           0 & -\frac{4}{3} & -2 & \frac{5}{3}\\
           0 & 1 & -1 & 0}\\
    \xleftrightarrow{R_2 \leftarrow -\frac{R_2}{2}}
    \myvec{1 & 2 & 0 & -2\\
          0 & 1 & 1 & -1\\
          0 & -\frac{4}{3} & -2 & \frac{5}{3}\\
          0 & 1 & -1 & 0}\\
    \xleftrightarrow[R_4 \leftarrow R_4- R_2]{R_3 \leftarrow R_3 + \frac{4}{3}R_2}
    \myvec{1 & 2 & 0 & -2\\
           0 & 1 & 1 & -1\\
           0 & 0 & -\frac{2}{3} & \frac{1}{3}\\
           0 & 0 & -2 & 1}\\
    \xleftrightarrow[R_3 \leftarrow -\frac{3}{2}R_3]{R_1 \leftarrow R_1- 2R_2}
    \myvec{1 & 0 & -2 & 0\\
           0 & 1 & 1 & -1\\
           0 & 0 & 1 & -\frac{1}{2}\\
           0 & 0 & -2 & 1}\\
    \xleftrightarrow{R_4\leftarrow R_4 + 2R_3}
    \myvec{1 & 0 & -2 & 0\\
           0 & 1 & 1 & -1\\
           0 & 0 & 1 & -\frac{1}{2}\\
           0 & 0 & 0 & 0}\\
    \xleftrightarrow[R_2\leftarrow R_2-R_3]{R_1\leftarrow R_1 + 2R_3}
    \myvec{1 & 0 & 0 & -1\\
           0 & 1 & 0 & -\frac{1}{2}\\
           0 & 0 & 1 & -\frac{1}{2}\\
           0 & 0 & 0 & 0}
\end{align}

Thus,
\begin{align}
    x_1=x_4, x_2= \frac{1}{2}x_4, x_3=\frac{1}{2}x_4\\
    \implies \quad\vec{x}= x_4\myvec{1\\ \frac{1}{2}\\ \frac{1}{2}\\1} =\myvec{2\\1\\1\\2}
\end{align} 
by substituting $x_4= 2$.

\hfill\break
%\vspace{5mm} 
Hence, (\ref{eq:solutions/chem/6abalanced}) finally becomes
\begin{align}
    2HNO_{3}+ Ca(OH)_{2}\to Ca(NO_{3})_{2}+ 2H_{2}O
\end{align}

\item Find a point on the curve $y = (x – 2)^2$ at which the tangent is parallel to the chord joining the points \myvec{2\\ 0} and \myvec{4\\ 4}.
\\
\solution
Let the balanced version of (\ref{eq:solutions/chem/6ato balance}) be
\begin{align}
    \label{eq:solutions/chem/6abalanced}x_{1}HNO_{3}+ x_{2}Ca(OH)_{2}\to x_{3}Ca(NO_{3})_{2}+ x_{4}H_{2}O
\end{align}

which results in the following equations:
\begin{align}
    (x_{1}+ 2x_{2}-2x_{4}) H= 0\\
    (x_{1}-2x_{3}) N= 0\\
    (3x_{1}+ 2x_{2}-6x_{3}- x_{4}) O=0\\
    (x_{2}-x_{3}) Ca= 0
\end{align}

which can be expressed as
\begin{align}
    x_{1}+ 2x_{2}+ 0.x_{3} -2x_{4} = 0\\
    x_{1}+ 0.x_{2} -2x_{3} +0.x_{4}= 0\\
    3x_{1}+ 2x_{2}-6x_{3}- x_{4} =0\\
    0.x_{1} +x_{2}-x_{3} +0.x_{4}= 0
\end{align}

resulting in the matrix equation
\begin{align}
    \label{eq:solutions/chem/6a matrix}
    \myvec{1 & 2 & 0 & -2\\
           1 & 0 & -2 & 0\\
           3 & 2 & -6 & -1\\
           0 & 1 & -1 & 0}\vec{x}
           =\vec{0}
\end{align}

where,
\begin{align}
   \vec{x}= \myvec{x_{1}\\x_{2}\\x_{3}\\x_{4}}
\end{align}

(\ref{eq:solutions/chem/6a matrix}) can be reduced as follows:
\begin{align}
    \myvec{1 & 2 & 0 & -2\\
           1 & 0 & -2 & 0\\
           3 & 2 & -6 & -1\\
           0 & 1 & -1 & 0}
    \xleftrightarrow[R_{3}\leftarrow \frac{R_3}{3}-R_{1}]{R_{2}\leftarrow R_2- R_1}
    \myvec{1 & 2 & 0 & -2\\
           0 & -2 & -2 & 2\\
           0 & -\frac{4}{3} & -2 & \frac{5}{3}\\
           0 & 1 & -1 & 0}\\
    \xleftrightarrow{R_2 \leftarrow -\frac{R_2}{2}}
    \myvec{1 & 2 & 0 & -2\\
          0 & 1 & 1 & -1\\
          0 & -\frac{4}{3} & -2 & \frac{5}{3}\\
          0 & 1 & -1 & 0}\\
    \xleftrightarrow[R_4 \leftarrow R_4- R_2]{R_3 \leftarrow R_3 + \frac{4}{3}R_2}
    \myvec{1 & 2 & 0 & -2\\
           0 & 1 & 1 & -1\\
           0 & 0 & -\frac{2}{3} & \frac{1}{3}\\
           0 & 0 & -2 & 1}\\
    \xleftrightarrow[R_3 \leftarrow -\frac{3}{2}R_3]{R_1 \leftarrow R_1- 2R_2}
    \myvec{1 & 0 & -2 & 0\\
           0 & 1 & 1 & -1\\
           0 & 0 & 1 & -\frac{1}{2}\\
           0 & 0 & -2 & 1}\\
    \xleftrightarrow{R_4\leftarrow R_4 + 2R_3}
    \myvec{1 & 0 & -2 & 0\\
           0 & 1 & 1 & -1\\
           0 & 0 & 1 & -\frac{1}{2}\\
           0 & 0 & 0 & 0}\\
    \xleftrightarrow[R_2\leftarrow R_2-R_3]{R_1\leftarrow R_1 + 2R_3}
    \myvec{1 & 0 & 0 & -1\\
           0 & 1 & 0 & -\frac{1}{2}\\
           0 & 0 & 1 & -\frac{1}{2}\\
           0 & 0 & 0 & 0}
\end{align}

Thus,
\begin{align}
    x_1=x_4, x_2= \frac{1}{2}x_4, x_3=\frac{1}{2}x_4\\
    \implies \quad\vec{x}= x_4\myvec{1\\ \frac{1}{2}\\ \frac{1}{2}\\1} =\myvec{2\\1\\1\\2}
\end{align} 
by substituting $x_4= 2$.

\hfill\break
%\vspace{5mm} 
Hence, (\ref{eq:solutions/chem/6abalanced}) finally becomes
\begin{align}
    2HNO_{3}+ Ca(OH)_{2}\to Ca(NO_{3})_{2}+ 2H_{2}O
\end{align}

\item Find the equation of all lines having slope – 1 that are tangents to the curve $\frac{1}
{x -1}, x \ne 1$
\\
\solution 
Let the balanced version of (\ref{eq:solutions/chem/6ato balance}) be
\begin{align}
    \label{eq:solutions/chem/6abalanced}x_{1}HNO_{3}+ x_{2}Ca(OH)_{2}\to x_{3}Ca(NO_{3})_{2}+ x_{4}H_{2}O
\end{align}

which results in the following equations:
\begin{align}
    (x_{1}+ 2x_{2}-2x_{4}) H= 0\\
    (x_{1}-2x_{3}) N= 0\\
    (3x_{1}+ 2x_{2}-6x_{3}- x_{4}) O=0\\
    (x_{2}-x_{3}) Ca= 0
\end{align}

which can be expressed as
\begin{align}
    x_{1}+ 2x_{2}+ 0.x_{3} -2x_{4} = 0\\
    x_{1}+ 0.x_{2} -2x_{3} +0.x_{4}= 0\\
    3x_{1}+ 2x_{2}-6x_{3}- x_{4} =0\\
    0.x_{1} +x_{2}-x_{3} +0.x_{4}= 0
\end{align}

resulting in the matrix equation
\begin{align}
    \label{eq:solutions/chem/6a matrix}
    \myvec{1 & 2 & 0 & -2\\
           1 & 0 & -2 & 0\\
           3 & 2 & -6 & -1\\
           0 & 1 & -1 & 0}\vec{x}
           =\vec{0}
\end{align}

where,
\begin{align}
   \vec{x}= \myvec{x_{1}\\x_{2}\\x_{3}\\x_{4}}
\end{align}

(\ref{eq:solutions/chem/6a matrix}) can be reduced as follows:
\begin{align}
    \myvec{1 & 2 & 0 & -2\\
           1 & 0 & -2 & 0\\
           3 & 2 & -6 & -1\\
           0 & 1 & -1 & 0}
    \xleftrightarrow[R_{3}\leftarrow \frac{R_3}{3}-R_{1}]{R_{2}\leftarrow R_2- R_1}
    \myvec{1 & 2 & 0 & -2\\
           0 & -2 & -2 & 2\\
           0 & -\frac{4}{3} & -2 & \frac{5}{3}\\
           0 & 1 & -1 & 0}\\
    \xleftrightarrow{R_2 \leftarrow -\frac{R_2}{2}}
    \myvec{1 & 2 & 0 & -2\\
          0 & 1 & 1 & -1\\
          0 & -\frac{4}{3} & -2 & \frac{5}{3}\\
          0 & 1 & -1 & 0}\\
    \xleftrightarrow[R_4 \leftarrow R_4- R_2]{R_3 \leftarrow R_3 + \frac{4}{3}R_2}
    \myvec{1 & 2 & 0 & -2\\
           0 & 1 & 1 & -1\\
           0 & 0 & -\frac{2}{3} & \frac{1}{3}\\
           0 & 0 & -2 & 1}\\
    \xleftrightarrow[R_3 \leftarrow -\frac{3}{2}R_3]{R_1 \leftarrow R_1- 2R_2}
    \myvec{1 & 0 & -2 & 0\\
           0 & 1 & 1 & -1\\
           0 & 0 & 1 & -\frac{1}{2}\\
           0 & 0 & -2 & 1}\\
    \xleftrightarrow{R_4\leftarrow R_4 + 2R_3}
    \myvec{1 & 0 & -2 & 0\\
           0 & 1 & 1 & -1\\
           0 & 0 & 1 & -\frac{1}{2}\\
           0 & 0 & 0 & 0}\\
    \xleftrightarrow[R_2\leftarrow R_2-R_3]{R_1\leftarrow R_1 + 2R_3}
    \myvec{1 & 0 & 0 & -1\\
           0 & 1 & 0 & -\frac{1}{2}\\
           0 & 0 & 1 & -\frac{1}{2}\\
           0 & 0 & 0 & 0}
\end{align}

Thus,
\begin{align}
    x_1=x_4, x_2= \frac{1}{2}x_4, x_3=\frac{1}{2}x_4\\
    \implies \quad\vec{x}= x_4\myvec{1\\ \frac{1}{2}\\ \frac{1}{2}\\1} =\myvec{2\\1\\1\\2}
\end{align} 
by substituting $x_4= 2$.

\hfill\break
%\vspace{5mm} 
Hence, (\ref{eq:solutions/chem/6abalanced}) finally becomes
\begin{align}
    2HNO_{3}+ Ca(OH)_{2}\to Ca(NO_{3})_{2}+ 2H_{2}O
\end{align}

\item Find the equation of all lines having slope -2 which are tangents to the curve $\frac{1}
{x - 3} , x \ne 3$.
%
\\
\solution 
Let the balanced version of (\ref{eq:solutions/chem/6ato balance}) be
\begin{align}
    \label{eq:solutions/chem/6abalanced}x_{1}HNO_{3}+ x_{2}Ca(OH)_{2}\to x_{3}Ca(NO_{3})_{2}+ x_{4}H_{2}O
\end{align}

which results in the following equations:
\begin{align}
    (x_{1}+ 2x_{2}-2x_{4}) H= 0\\
    (x_{1}-2x_{3}) N= 0\\
    (3x_{1}+ 2x_{2}-6x_{3}- x_{4}) O=0\\
    (x_{2}-x_{3}) Ca= 0
\end{align}

which can be expressed as
\begin{align}
    x_{1}+ 2x_{2}+ 0.x_{3} -2x_{4} = 0\\
    x_{1}+ 0.x_{2} -2x_{3} +0.x_{4}= 0\\
    3x_{1}+ 2x_{2}-6x_{3}- x_{4} =0\\
    0.x_{1} +x_{2}-x_{3} +0.x_{4}= 0
\end{align}

resulting in the matrix equation
\begin{align}
    \label{eq:solutions/chem/6a matrix}
    \myvec{1 & 2 & 0 & -2\\
           1 & 0 & -2 & 0\\
           3 & 2 & -6 & -1\\
           0 & 1 & -1 & 0}\vec{x}
           =\vec{0}
\end{align}

where,
\begin{align}
   \vec{x}= \myvec{x_{1}\\x_{2}\\x_{3}\\x_{4}}
\end{align}

(\ref{eq:solutions/chem/6a matrix}) can be reduced as follows:
\begin{align}
    \myvec{1 & 2 & 0 & -2\\
           1 & 0 & -2 & 0\\
           3 & 2 & -6 & -1\\
           0 & 1 & -1 & 0}
    \xleftrightarrow[R_{3}\leftarrow \frac{R_3}{3}-R_{1}]{R_{2}\leftarrow R_2- R_1}
    \myvec{1 & 2 & 0 & -2\\
           0 & -2 & -2 & 2\\
           0 & -\frac{4}{3} & -2 & \frac{5}{3}\\
           0 & 1 & -1 & 0}\\
    \xleftrightarrow{R_2 \leftarrow -\frac{R_2}{2}}
    \myvec{1 & 2 & 0 & -2\\
          0 & 1 & 1 & -1\\
          0 & -\frac{4}{3} & -2 & \frac{5}{3}\\
          0 & 1 & -1 & 0}\\
    \xleftrightarrow[R_4 \leftarrow R_4- R_2]{R_3 \leftarrow R_3 + \frac{4}{3}R_2}
    \myvec{1 & 2 & 0 & -2\\
           0 & 1 & 1 & -1\\
           0 & 0 & -\frac{2}{3} & \frac{1}{3}\\
           0 & 0 & -2 & 1}\\
    \xleftrightarrow[R_3 \leftarrow -\frac{3}{2}R_3]{R_1 \leftarrow R_1- 2R_2}
    \myvec{1 & 0 & -2 & 0\\
           0 & 1 & 1 & -1\\
           0 & 0 & 1 & -\frac{1}{2}\\
           0 & 0 & -2 & 1}\\
    \xleftrightarrow{R_4\leftarrow R_4 + 2R_3}
    \myvec{1 & 0 & -2 & 0\\
           0 & 1 & 1 & -1\\
           0 & 0 & 1 & -\frac{1}{2}\\
           0 & 0 & 0 & 0}\\
    \xleftrightarrow[R_2\leftarrow R_2-R_3]{R_1\leftarrow R_1 + 2R_3}
    \myvec{1 & 0 & 0 & -1\\
           0 & 1 & 0 & -\frac{1}{2}\\
           0 & 0 & 1 & -\frac{1}{2}\\
           0 & 0 & 0 & 0}
\end{align}

Thus,
\begin{align}
    x_1=x_4, x_2= \frac{1}{2}x_4, x_3=\frac{1}{2}x_4\\
    \implies \quad\vec{x}= x_4\myvec{1\\ \frac{1}{2}\\ \frac{1}{2}\\1} =\myvec{2\\1\\1\\2}
\end{align} 
by substituting $x_4= 2$.

\hfill\break
%\vspace{5mm} 
Hence, (\ref{eq:solutions/chem/6abalanced}) finally becomes
\begin{align}
    2HNO_{3}+ Ca(OH)_{2}\to Ca(NO_{3})_{2}+ 2H_{2}O
\end{align}

\item Find points on the curve 
$
\vec{x}^T\myvec{\frac{1}{9} & 0 \\ 0 & \frac{1}{16}}\vec{x} = 1
$
%
at which tangents are
\begin{enumerate}
\item  parallel to x-axis
\item  parallel to y-axis.
\end{enumerate}
\solution 
Let the balanced version of (\ref{eq:solutions/chem/6ato balance}) be
\begin{align}
    \label{eq:solutions/chem/6abalanced}x_{1}HNO_{3}+ x_{2}Ca(OH)_{2}\to x_{3}Ca(NO_{3})_{2}+ x_{4}H_{2}O
\end{align}

which results in the following equations:
\begin{align}
    (x_{1}+ 2x_{2}-2x_{4}) H= 0\\
    (x_{1}-2x_{3}) N= 0\\
    (3x_{1}+ 2x_{2}-6x_{3}- x_{4}) O=0\\
    (x_{2}-x_{3}) Ca= 0
\end{align}

which can be expressed as
\begin{align}
    x_{1}+ 2x_{2}+ 0.x_{3} -2x_{4} = 0\\
    x_{1}+ 0.x_{2} -2x_{3} +0.x_{4}= 0\\
    3x_{1}+ 2x_{2}-6x_{3}- x_{4} =0\\
    0.x_{1} +x_{2}-x_{3} +0.x_{4}= 0
\end{align}

resulting in the matrix equation
\begin{align}
    \label{eq:solutions/chem/6a matrix}
    \myvec{1 & 2 & 0 & -2\\
           1 & 0 & -2 & 0\\
           3 & 2 & -6 & -1\\
           0 & 1 & -1 & 0}\vec{x}
           =\vec{0}
\end{align}

where,
\begin{align}
   \vec{x}= \myvec{x_{1}\\x_{2}\\x_{3}\\x_{4}}
\end{align}

(\ref{eq:solutions/chem/6a matrix}) can be reduced as follows:
\begin{align}
    \myvec{1 & 2 & 0 & -2\\
           1 & 0 & -2 & 0\\
           3 & 2 & -6 & -1\\
           0 & 1 & -1 & 0}
    \xleftrightarrow[R_{3}\leftarrow \frac{R_3}{3}-R_{1}]{R_{2}\leftarrow R_2- R_1}
    \myvec{1 & 2 & 0 & -2\\
           0 & -2 & -2 & 2\\
           0 & -\frac{4}{3} & -2 & \frac{5}{3}\\
           0 & 1 & -1 & 0}\\
    \xleftrightarrow{R_2 \leftarrow -\frac{R_2}{2}}
    \myvec{1 & 2 & 0 & -2\\
          0 & 1 & 1 & -1\\
          0 & -\frac{4}{3} & -2 & \frac{5}{3}\\
          0 & 1 & -1 & 0}\\
    \xleftrightarrow[R_4 \leftarrow R_4- R_2]{R_3 \leftarrow R_3 + \frac{4}{3}R_2}
    \myvec{1 & 2 & 0 & -2\\
           0 & 1 & 1 & -1\\
           0 & 0 & -\frac{2}{3} & \frac{1}{3}\\
           0 & 0 & -2 & 1}\\
    \xleftrightarrow[R_3 \leftarrow -\frac{3}{2}R_3]{R_1 \leftarrow R_1- 2R_2}
    \myvec{1 & 0 & -2 & 0\\
           0 & 1 & 1 & -1\\
           0 & 0 & 1 & -\frac{1}{2}\\
           0 & 0 & -2 & 1}\\
    \xleftrightarrow{R_4\leftarrow R_4 + 2R_3}
    \myvec{1 & 0 & -2 & 0\\
           0 & 1 & 1 & -1\\
           0 & 0 & 1 & -\frac{1}{2}\\
           0 & 0 & 0 & 0}\\
    \xleftrightarrow[R_2\leftarrow R_2-R_3]{R_1\leftarrow R_1 + 2R_3}
    \myvec{1 & 0 & 0 & -1\\
           0 & 1 & 0 & -\frac{1}{2}\\
           0 & 0 & 1 & -\frac{1}{2}\\
           0 & 0 & 0 & 0}
\end{align}

Thus,
\begin{align}
    x_1=x_4, x_2= \frac{1}{2}x_4, x_3=\frac{1}{2}x_4\\
    \implies \quad\vec{x}= x_4\myvec{1\\ \frac{1}{2}\\ \frac{1}{2}\\1} =\myvec{2\\1\\1\\2}
\end{align} 
by substituting $x_4= 2$.

\hfill\break
%\vspace{5mm} 
Hence, (\ref{eq:solutions/chem/6abalanced}) finally becomes
\begin{align}
    2HNO_{3}+ Ca(OH)_{2}\to Ca(NO_{3})_{2}+ 2H_{2}O
\end{align}

\item Find the equations of the tangent and normal to the given curves at the indicated points:
$
y = x^2
$
at \myvec{0\\0}.
\item Find the equation of the tangent line to the curve $y = x^2-2x+7$
\begin{enumerate}
%
\item  parallel to the line $\myvec{2 & -1}\vec{x}= -9$ 
\item  perpendicular to the line $\myvec{-15 & 5}\vec{x} = 13$. 
\end{enumerate}
\item  Find the equation of the tangent to the curve,
\begin{align}
y = \sqrt{3x-2}
\label{eq:solutions/1/19/Q}
\end{align}
 which is parallel to the line,
\begin{align}
\myvec{4&2}\vec{x}+5=0
\label{eq:solutions/1/19/P}
\end{align}  
\solution 
Let the balanced version of (\ref{eq:solutions/chem/6ato balance}) be
\begin{align}
    \label{eq:solutions/chem/6abalanced}x_{1}HNO_{3}+ x_{2}Ca(OH)_{2}\to x_{3}Ca(NO_{3})_{2}+ x_{4}H_{2}O
\end{align}

which results in the following equations:
\begin{align}
    (x_{1}+ 2x_{2}-2x_{4}) H= 0\\
    (x_{1}-2x_{3}) N= 0\\
    (3x_{1}+ 2x_{2}-6x_{3}- x_{4}) O=0\\
    (x_{2}-x_{3}) Ca= 0
\end{align}

which can be expressed as
\begin{align}
    x_{1}+ 2x_{2}+ 0.x_{3} -2x_{4} = 0\\
    x_{1}+ 0.x_{2} -2x_{3} +0.x_{4}= 0\\
    3x_{1}+ 2x_{2}-6x_{3}- x_{4} =0\\
    0.x_{1} +x_{2}-x_{3} +0.x_{4}= 0
\end{align}

resulting in the matrix equation
\begin{align}
    \label{eq:solutions/chem/6a matrix}
    \myvec{1 & 2 & 0 & -2\\
           1 & 0 & -2 & 0\\
           3 & 2 & -6 & -1\\
           0 & 1 & -1 & 0}\vec{x}
           =\vec{0}
\end{align}

where,
\begin{align}
   \vec{x}= \myvec{x_{1}\\x_{2}\\x_{3}\\x_{4}}
\end{align}

(\ref{eq:solutions/chem/6a matrix}) can be reduced as follows:
\begin{align}
    \myvec{1 & 2 & 0 & -2\\
           1 & 0 & -2 & 0\\
           3 & 2 & -6 & -1\\
           0 & 1 & -1 & 0}
    \xleftrightarrow[R_{3}\leftarrow \frac{R_3}{3}-R_{1}]{R_{2}\leftarrow R_2- R_1}
    \myvec{1 & 2 & 0 & -2\\
           0 & -2 & -2 & 2\\
           0 & -\frac{4}{3} & -2 & \frac{5}{3}\\
           0 & 1 & -1 & 0}\\
    \xleftrightarrow{R_2 \leftarrow -\frac{R_2}{2}}
    \myvec{1 & 2 & 0 & -2\\
          0 & 1 & 1 & -1\\
          0 & -\frac{4}{3} & -2 & \frac{5}{3}\\
          0 & 1 & -1 & 0}\\
    \xleftrightarrow[R_4 \leftarrow R_4- R_2]{R_3 \leftarrow R_3 + \frac{4}{3}R_2}
    \myvec{1 & 2 & 0 & -2\\
           0 & 1 & 1 & -1\\
           0 & 0 & -\frac{2}{3} & \frac{1}{3}\\
           0 & 0 & -2 & 1}\\
    \xleftrightarrow[R_3 \leftarrow -\frac{3}{2}R_3]{R_1 \leftarrow R_1- 2R_2}
    \myvec{1 & 0 & -2 & 0\\
           0 & 1 & 1 & -1\\
           0 & 0 & 1 & -\frac{1}{2}\\
           0 & 0 & -2 & 1}\\
    \xleftrightarrow{R_4\leftarrow R_4 + 2R_3}
    \myvec{1 & 0 & -2 & 0\\
           0 & 1 & 1 & -1\\
           0 & 0 & 1 & -\frac{1}{2}\\
           0 & 0 & 0 & 0}\\
    \xleftrightarrow[R_2\leftarrow R_2-R_3]{R_1\leftarrow R_1 + 2R_3}
    \myvec{1 & 0 & 0 & -1\\
           0 & 1 & 0 & -\frac{1}{2}\\
           0 & 0 & 1 & -\frac{1}{2}\\
           0 & 0 & 0 & 0}
\end{align}

Thus,
\begin{align}
    x_1=x_4, x_2= \frac{1}{2}x_4, x_3=\frac{1}{2}x_4\\
    \implies \quad\vec{x}= x_4\myvec{1\\ \frac{1}{2}\\ \frac{1}{2}\\1} =\myvec{2\\1\\1\\2}
\end{align} 
by substituting $x_4= 2$.

\hfill\break
%\vspace{5mm} 
Hence, (\ref{eq:solutions/chem/6abalanced}) finally becomes
\begin{align}
    2HNO_{3}+ Ca(OH)_{2}\to Ca(NO_{3})_{2}+ 2H_{2}O
\end{align}

%
\item Find the point at which the line $\myvec{-1 & 1}\vec{x} =  1$ is a tangent to the curve $y^2 = 4x$.
%
\item The line $\myvec{-m & 1}\vec{x} = 1$ is a tangent to the curve $y^2 = 4x$.  Find the value of $m$.
\item  Find the normal at the point \myvec{1\\1} on the curve $2y + x^2 = 3$ 
\item  Find the normal to the curve $x^2=4y$ passing through $\myvec{1\\2}$.
%
\item Find the area of the region bounded by the curve $y^2= x$ and the lines $x = 1, x = 4$ and the x-axis in the first quadrant.
\item  Find the area of the region bounded by $y^2=9x, x=2, x=4$ and the x-axis in the  first quadrant.
%
\item Find the area of the region bounded by $x^2 = 4y, y = 2, y = 4$ and the y-axis in the first quadrant.
\item Find the area of the region bounded by the ellipse 
$
\vec{x}^T\myvec{\frac{1}{16} & 0 \\ 0 & \frac{1}{9}}\vec{x} = 1
$

\item  Find the area of the region bounded by the ellipse 
$
\vec{x}^T\myvec{\frac{1}{4} & 0 \\ 0 & \frac{1}{9}}\vec{x} = 1
$
\item The area between $x=y^2$ and $x=4$ is divided into two equal parts by the line $x=a$, find the value of $a$.
\item  Find the area of the region bounded by the parabola $y = x^2$ and $y = \abs{x}$.
\item  Find the area bounded by the curve $x^2 = 4y$ and the line $\myvec{1 & -1}\vec{x} = -2$.
\item  Find the area of the region bounded by the curve $y^2 = 4x$ and the line $x = 3$.
%
\item Find the area of the region bounded by the curve $y^2 = x$, y-axis and the line $y = 3$.
%
\item Find the area of the region bounded by the two parabolas $y = x^2, y^2=x$.
\item Find the area lying above x-axis and included between the circle $\vec{x}^T\vec{x} -8\myvec{1 & 0}= 0$  and inside of the parabola $y^2 = 4x$.
%
\item AOBA is the part of the ellipse 
$
\vec{x}^T\myvec{9 & 0 \\ 0 & 1}\vec{x} = 36
$
in the first quadrant such that $OA = 2$ and $OB = 6$. Find the area between the arc $AB$ and the chord $AB$.
\item Find the area lying between the curves $y^2 = 4x$ and $y = 2x$.
\item  Find the area of the region bounded by the curves $y = x^2+2, y = x, x = 0$ and $ x = 3.$
%
\item Find the area under $y = x^2, x = 1, x = 2$ and x-axis.
\item Find the area between  $y = x^2$ and $y = x$.
\item Find the area of the region lying in the first quadrant and bounded by $y = 4x^2, x = 0, y = 1$ and $y = 4$.
\item Find the area enclosed by the parabola $4y = 3x^2$ and the line $\myvec{-3 & 2}\vec{x} = 12$.
%
\item Find the area of the smaller region bounded by the ellipse
$
\vec{x}^T\myvec{\frac{1}{9} & 0 \\ 0 & \frac{1}{4}}\vec{x} = 1
$
and the line 
$
\myvec{\frac{1}{a} & \frac{1}{b}}\vec{x} = 1
$
\item Find the area of the region enclosed by the parabola $x^2=y$, the line $\myvec{-1 & 1}\vec{x} = 2$ and the x-axis.
%
\item Find the area bounded by the curves
\begin{align}
\cbrak{\brak{x,y} : y > x^2, y = \abs{x}}
\end{align}
%
\item Find the area of the region
\begin{align}
\cbrak{\brak{x,y} : y^2 \le 4x, 4\vec{x}^T\vec{x} = 9}
\end{align}
%
\item Find the area of the circle $\vec{x}^T\vec{x} = 16$ exterior to the parabola $y^2 = 6$.
\end{enumerate}

\caption{$\vec{x}^T\vec{V}\vec{x}+2\vec{u}^T\vec{x}+f = 0$  can be expressed in the above standard form for various conics. $\vec{c}$ represents the centre/vertex of the conic. $\vec{q}$ is/are the point(s) of contact for the tangent(s). }
\label{table:conics}
\end{table*}

\end{enumerate}
