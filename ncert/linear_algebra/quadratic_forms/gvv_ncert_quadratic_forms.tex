\documentclass[journal,12pt,twocolumn]{IEEEtran}
%
\usepackage{setspace}
\usepackage{gensymb}
%\doublespacing
\singlespacing

%\usepackage{graphicx}
%\usepackage{amssymb}
%\usepackage{relsize}
\usepackage[cmex10]{amsmath}
%\usepackage{amsthm}
%\interdisplaylinepenalty=2500
%\savesymbol{iint}
%\usepackage{txfonts}
%\restoresymbol{TXF}{iint}
%\usepackage{wasysym}
\usepackage{amsthm}
\usepackage{iithtlc}
\usepackage{mathrsfs}
\usepackage{txfonts}
\usepackage{stfloats}
\usepackage{steinmetz}
\usepackage{bm}
\usepackage{cite}
\usepackage{cases}
\usepackage{subfig}
%\usepackage{xtab}
\usepackage{longtable}
\usepackage{multirow}
%\usepackage{algorithm}
%\usepackage{algpseudocode}
\usepackage{enumitem}
\usepackage{mathtools}
\usepackage{tikz}
\usepackage{circuitikz}
\usepackage{verbatim}
\usepackage{tfrupee}
\usepackage[breaklinks=true]{hyperref}
%\usepackage{stmaryrd}
\usepackage{tkz-euclide} % loads  TikZ and tkz-base
%\usetkzobj{all}
\usepackage{listings}
    \usepackage{color}                                            %%
    \usepackage{array}                                            %%
    \usepackage{longtable}                                        %%
    \usepackage{calc}                                             %%
    \usepackage{multirow}                                         %%
    \usepackage{hhline}                                           %%
    \usepackage{ifthen}                                           %%
  %optionally (for landscape tables embedded in another document): %%
    \usepackage{lscape}     
\usepackage{multicol}
\usepackage{chngcntr}
%\usepackage{enumerate}

%\usepackage{wasysym}
%\newcounter{MYtempeqncnt}
\DeclareMathOperator*{\Res}{Res}
%\renewcommand{\baselinestretch}{2}
\renewcommand\thesection{\arabic{section}}
\renewcommand\thesubsection{\thesection.\arabic{subsection}}
\renewcommand\thesubsubsection{\thesubsection.\arabic{subsubsection}}

\renewcommand\thesectiondis{\arabic{section}}
\renewcommand\thesubsectiondis{\thesectiondis.\arabic{subsection}}
\renewcommand\thesubsubsectiondis{\thesubsectiondis.\arabic{subsubsection}}

% correct bad hyphenation here
\hyphenation{op-tical net-works semi-conduc-tor}
\def\inputGnumericTable{}                                 %%

\lstset{
%language=C,
frame=single, 
breaklines=true,
columns=fullflexible
}
%\lstset{
%language=tex,
%frame=single, 
%breaklines=true
%}

\begin{document}
%


\newtheorem{theorem}{Theorem}[section]
\newtheorem{problem}{Problem}
\newtheorem{proposition}{Proposition}[section]
\newtheorem{lemma}{Lemma}[section]
\newtheorem{corollary}[theorem]{Corollary}
\newtheorem{example}{Example}[section]
\newtheorem{definition}[problem]{Definition}
%\newtheorem{thm}{Theorem}[section] 
%\newtheorem{defn}[thm]{Definition}
%\newtheorem{algorithm}{Algorithm}[section]
%\newtheorem{cor}{Corollary}
\newcommand{\BEQA}{\begin{eqnarray}}
\newcommand{\EEQA}{\end{eqnarray}}
\newcommand{\define}{\stackrel{\triangle}{=}}

\bibliographystyle{IEEEtran}
%\bibliographystyle{ieeetr}


\providecommand{\mbf}{\mathbf}
\providecommand{\pr}[1]{\ensuremath{\Pr\left(#1\right)}}
\providecommand{\qfunc}[1]{\ensuremath{Q\left(#1\right)}}
\providecommand{\sbrak}[1]{\ensuremath{{}\left[#1\right]}}
\providecommand{\lsbrak}[1]{\ensuremath{{}\left[#1\right.}}
\providecommand{\rsbrak}[1]{\ensuremath{{}\left.#1\right]}}
\providecommand{\brak}[1]{\ensuremath{\left(#1\right)}}
\providecommand{\lbrak}[1]{\ensuremath{\left(#1\right.}}
\providecommand{\rbrak}[1]{\ensuremath{\left.#1\right)}}
\providecommand{\cbrak}[1]{\ensuremath{\left\{#1\right\}}}
\providecommand{\lcbrak}[1]{\ensuremath{\left\{#1\right.}}
\providecommand{\rcbrak}[1]{\ensuremath{\left.#1\right\}}}
\theoremstyle{remark}
\newtheorem{rem}{Remark}
\newcommand{\sgn}{\mathop{\mathrm{sgn}}}
\providecommand{\abs}[1]{\left\vert#1\right\vert}
\providecommand{\res}[1]{\Res\displaylimits_{#1}} 
\providecommand{\norm}[1]{\left\lVert#1\right\rVert}
%\providecommand{\norm}[1]{\lVert#1\rVert}
\providecommand{\mtx}[1]{\mathbf{#1}}
\providecommand{\mean}[1]{E\left[ #1 \right]}
\providecommand{\fourier}{\overset{\mathcal{F}}{ \rightleftharpoons}}
%\providecommand{\hilbert}{\overset{\mathcal{H}}{ \rightleftharpoons}}
\providecommand{\system}{\overset{\mathcal{H}}{ \longleftrightarrow}}
	%\newcommand{\solution}[2]{\textbf{Solution:}{#1}}
\newcommand{\solution}{\noindent \textbf{Solution: }}
\newcommand{\cosec}{\,\text{cosec}\,}
\providecommand{\dec}[2]{\ensuremath{\overset{#1}{\underset{#2}{\gtrless}}}}
\newcommand{\myvec}[1]{\ensuremath{\begin{pmatrix}#1\end{pmatrix}}}
\newcommand{\mydet}[1]{\ensuremath{\begin{vmatrix}#1\end{vmatrix}}}
%\numberwithin{equation}{section}
\numberwithin{equation}{subsection}
%\numberwithin{problem}{section}
%\numberwithin{definition}{section}
\makeatletter
\@addtoreset{figure}{problem}
\makeatother

\let\StandardTheFigure\thefigure
\let\vec\mathbf
%\renewcommand{\thefigure}{\theproblem.\arabic{figure}}
\renewcommand{\thefigure}{\theproblem}
%\setlist[enumerate,1]{before=\renewcommand\theequation{\theenumi.\arabic{equation}}
%\counterwithin{equation}{enumi}


%\renewcommand{\theequation}{\arabic{subsection}.\arabic{equation}}

\def\putbox#1#2#3{\makebox[0in][l]{\makebox[#1][l]{}\raisebox{\baselineskip}[0in][0in]{\raisebox{#2}[0in][0in]{#3}}}}
     \def\rightbox#1{\makebox[0in][r]{#1}}
     \def\centbox#1{\makebox[0in]{#1}}
     \def\topbox#1{\raisebox{-\baselineskip}[0in][0in]{#1}}
     \def\midbox#1{\raisebox{-0.5\baselineskip}[0in][0in]{#1}}

\vspace{3cm}

\title{
	\logo{
Quadratic Forms
	}
}
\author{ G V V Sharma$^{*}$% <-this % stops a space
	\thanks{*The author is with the Department
		of Electrical Engineering, Indian Institute of Technology, Hyderabad
		502285 India e-mail:  gadepall@iith.ac.in. All content in this manual is released under GNU GPL.  Free and open source.}
	
}	
%\title{
%	\logo{Matrix Analysis through Octave}{\begin{center}\includegraphics[scale=.24]{tlc}\end{center}}{}{HAMDSP}
%}


% paper title
% can use linebreaks \\ within to get better formatting as desired
%\title{Matrix Analysis through Octave}
%
%
% author names and IEEE memberships
% note positions of commas and nonbreaking spaces ( ~ ) LaTeX will not break
% a structure at a ~ so this keeps an author's name from being broken across
% two lines.
% use \thanks{} to gain access to the first footnote area
% a separate \thanks must be used for each paragraph as LaTeX2e's \thanks
% was not built to handle multiple paragraphs
%

%\author{<-this % stops a space
%\thanks{}}
%}
% note the % following the last \IEEEmembership and also \thanks - 
% these prevent an unwanted space from occurring between the last author name
% and the end of the author line. i.e., if you had this:
% 
% \author{....lastname \thanks{...} \thanks{...} }
%                     ^------------^------------^----Do not want these spaces!
%
% a space would be appended to the last name and could cause every name on that
% line to be shifted left slightly. This is one of those "LaTeX things". For
% instance, "\textbf{A} \textbf{B}" will typeset as "A B" not "AB". To get
% "AB" then you have to do: "\textbf{A}\textbf{B}"
% \thanks is no different in this regard, so shield the last } of each \thanks
% that ends a line with a % and do not let a space in before the next \thanks.
% Spaces after \IEEEmembership other than the last one are OK (and needed) as
% you are supposed to have spaces between the names. For what it is worth,
% this is a minor point as most people would not even notice if the said evil
% space somehow managed to creep in.



% The paper headers
%\markboth{Journal of \LaTeX\ Class Files,~Vol.~6, No.~1, January~2007}%
%{Shell \MakeLowercase{\textit{et al.}}: Bare Demo of IEEEtran.cls for Journals}
% The only time the second header will appear is for the odd numbered pages
% after the title page when using the twoside option.
% 
% *** Note that you probably will NOT want to include the author's ***
% *** name in the headers of peer review papers.                   ***
% You can use \ifCLASSOPTIONpeerreview for conditional compilation here if
% you desire.




% If you want to put a publisher's ID mark on the page you can do it like
% this:
%\IEEEpubid{0000--0000/00\$00.00~\copyright~2007 IEEE}
% Remember, if you use this you must call \IEEEpubidadjcol in the second
% column for its text to clear the IEEEpubid mark.



% make the title area
\maketitle

\newpage

\tableofcontents

\bigskip

\renewcommand{\thefigure}{\theenumi}
\renewcommand{\thetable}{\theenumi}
%\renewcommand{\theequation}{\theenumi}

%\begin{abstract}
%%\boldmath
%In this letter, an algorithm for evaluating the exact analytical bit error rate  (BER)  for the piecewise linear (PL) combiner for  multiple relays is presented. Previous results were available only for upto three relays. The algorithm is unique in the sense that  the actual mathematical expressions, that are prohibitively large, need not be explicitly obtained. The diversity gain due to multiple relays is shown through plots of the analytical BER, well supported by simulations. 
%
%\end{abstract}
% IEEEtran.cls defaults to using nonbold math in the Abstract.
% This preserves the distinction between vectors and scalars. However,
% if the journal you are submitting to favors bold math in the abstract,
% then you can use LaTeX's standard command \boldmath at the very start
% of the abstract to achieve this. Many IEEE journals frown on math
% in the abstract anyway.

% Note that keywords are not normally used for peerreview papers.
%\begin{IEEEkeywords}
%Cooperative diversity, decode and forward, piecewise linear
%\end{IEEEkeywords}



% For peer review papers, you can put extra information on the cover
% page as needed:
% \ifCLASSOPTIONpeerreview
% \begin{center} \bfseries EDICS Category: 3-BBND \end{center}
% \fi
%
% For peerreview papers, this IEEEtran command inserts a page break and
% creates the second title. It will be ignored for other modes.
%\IEEEpeerreviewmaketitle

\begin{abstract}
This book provides a computational approach to school geometry based on the NCERT textbooks from Class 6-12.  Links to sample Python codes are available in the text.  
\end{abstract}
Download python codes using 
\begin{lstlisting}
svn co https://github.com/gadepall/school/trunk/ncert/computation/codes
\end{lstlisting}

\section{Examples}
\renewcommand{\theequation}{\theenumi}
\begin{enumerate}[label=\thesection.\arabic*.,ref=\thesection.\theenumi]
\numberwithin{equation}{enumi}

\renewcommand{\theequation}{\theenumi}
\begin{enumerate}[label=\arabic*.,ref=\thesubsection.\theenumi]
\numberwithin{equation}{enumi}
\item 
\end{enumerate}
 
\renewcommand{\theequation}{\theenumi}
\begin{enumerate}[label=\arabic*.,ref=\thesubsection.\theenumi]
\numberwithin{equation}{enumi}
%
%\item Draw the circumcircle of $\triangle ABC$, where 
%
\item Find the value of  the following polynomial at the indicated value of variables 
\begin{align}
p(x) = 5x^2– 3x + 7 \text{  at } x = 1.
\end{align}
\item Verify whether 2 and 0 are zeroes of the polynomial $x^2-2x$.
\item Find $p(0)$, $p(1)$ and $p(2)$ for each of the following polynomials: 
\begin{enumerate}
\item $p(y) = y^2$. 
\item $p(x) = (x – 1) (x + 1)$.
\end{enumerate}
\item Find the roots of the equation  $2x^2– 5x + 3 = 0$ .
\item Find the roots of the quadratic equation $6x^2– x – 2 = 0.$
\item Find the roots of the quadratic equation $3x^2 -2 \sqrt{6}x+ 2 = 0$.
%\item Verify whether the following are zeroes of the polynomial, indicated against them. (i) p(x) = 3x + 1, x =
%\begin{enumerate}
%
%\item $ p(x) = x^2-1, x = 1, -1$
%\item $ p(x) = 5x -\pi, x = \frac{4}{5}$
%\item $ p(x) = \brak{x+1} \brak{x-2}, x = -1,2$
%\item $ p(x) = x^2, x = 0$.
%\item $ p(x) = 3x^2-1, x = -\frac{1}{\sqrt{3}}, \frac{2}{\sqrt{3}}$.
%
\item Factorise $6x^2+ 17x + 5$.
\item Factorise $y^2 – 5y + 6$.
\item Find the zeroes of the quadratic polynomial $x^2+7x+10$ and verify the relationship between the zeroes and the coefficients.
\item Find the zeroes of the polynomial $x^2-3$ and verify the relationship between the zeroes and the coefficients.
\item Find a quadratic polynomial, the sum and product of whose zeroes are – 3 and 2, respectively.
%
\item Find the roots of the equation $5x^2  – 6x – 2 = 0 $.
\item Find the roots of $4x^2 + 3x + 5 = 0 $.
\item Find the roots of the following quadratic equations, if they exist.
\begin{enumerate}
\item 	$3x^2-5x+2 = 0$
\item 	$x^2+4x+5 = 0$
\item 	$2x^2-2\sqrt{2}x+1 = 0$
\end{enumerate}
%
\item Find the discriminant of the quadratic equation $2x^2-4x+3 = 0$
hence find the nature of its roots.
\item Find the discriminant of the quadratic equation $3x^2-2x+\frac{1}{3} = 0$
hence find the nature of its roots.
\item Solve $x^2+ 2 = 0 $.
\item Solve $x^2+ x+1 = 0 $.
\item Solve $\sqrt{5}x^2+ x+\sqrt{5} = 0 $.
%
\item Find the coordinates of the focus, axis, the equation of the directrix and latus rectum of the parabola $y^2 = 8x$.
%
\item Find the equation of the parabola with focus \myvec{2\\0} and directrix $\myvec{1 & 0}\vec{x} = -2$.\item Find the equation of the parabola with vertex at \myvec{0\\ 0} and focus at \myvec{0\\ 2}.
\item Find the equation of the parabola which is symmetric about the y-axis, and passes through the point \myvec{2\\–3}.
\item Find the coordinates of the foci, the vertices, the length of major axis, the minor axis, the eccentricity and the latus rectum of the ellipse 
%
\begin{align}
\vec{x}^T\myvec{\frac{1}{25} & 0 \\ 0 & \frac{1}{9}}\vec{x} = 1
\end{align}
%
\item Find the coordinates of the foci, the vertices, the lengths of major and minor axes and the eccentricity of the ellipse 
%
\begin{align}
\vec{x}^T\myvec{9 & 0 \\ 0 & 4}\vec{x} = 36
\end{align}
%
\item Find the equation of the ellipse whose vertices are $\myvec{\pm 13\\ 0}$ and foci are $\myvec{\pm 5\\ 0}$.
%
\item Find the equation of the ellipse, whose length of the major axis is 20 and foci are $\myvec{0\\ \pm 5}$
%
\item Find the equation of the ellipse, with major axis along the x-axis and passing through the points \myvec{4\\ 3} and \myvec{– 1\\4}.
%
\item Find the coordinates of the foci and the vertices, the eccentricity,the length of the latus rectum of the hyperbolas
\begin{enumerate}
\item 
$
\vec{x}^T\myvec{\frac{1}{9} & 0 \\ 0 & -\frac{1}{16}}\vec{x} = 1
$
\item 
$
\vec{x}^T\myvec{1 & 0 \\ 0 & -16}\vec{x} = 16
$
\end{enumerate}
\item Find the equation of the hyperbola with  vertices $\myvec{0 \\ \pm \frac{\sqrt{11}}{2}}$, foci $\myvec{ 0\\ \pm 3}$
\item Find the equation of the hyperbola with   foci $\myvec{ 0\\ \pm 12}$ and length of latus rectum 36.
%
\item Find the equation of all lines having slope 2 and being tangent to the curve
\begin{align}
y + \frac{2}{x-3} = 0
\end{align}
%
\item Find the point at which the tangent to the curve $y = \sqrt{4x-3}-1$ has its 
slope $\frac{2}{3}$.
%
\item Find the roots of the following equations:
\begin{enumerate}
\item  $x + \frac{1}{x} = 3, x \ne =0 $
\item  $ \frac{1}{x} + \frac{1}{x-2}=3, x\ne =0, 2 $
\end{enumerate}
%
\item Find points on the curve 
$
\vec{x}^T\myvec{\frac{1}{4} & 0 \\ 0 & \frac{1}{25}}\vec{x} = 1
$
at which the tangents are 
\begin{enumerate}
\item parallel to x-axis
\item parallel to y-axis
\end{enumerate}
 \item Find the equation of the normal to the curve $x^2= 4y$
which passes through the point \myvec{1\\ 2}.
%
\item Find the area enclosed by the ellipse
$
\vec{x}^T\myvec{\frac{1}{a^2} & 0 \\ 0 & \frac{1}{b^2}}\vec{x} = 1
$
%
\item Find the area of the region bounded by the curve $y = x^2$
and the line $y = 4$.
%
\item Find the area bounded by the ellipse
$
\vec{x}^T\myvec{\frac{1}{a^2} & 0 \\ 0 & \frac{1}{b^2}}\vec{x} = 1
$
and $x = ae$, where, $b^2 = a^2 (1 – e^2 )$ and $e < 1$.
%
\item Prove that the curves $y^2 = 4x$ and $x^2 = 4y$ divide the area of the square bounded by $x = 0, x = 4, y =4$ and $y = 0$ into three equal parts.
%
\item Find the area of the region
\begin{align}
\cbrak{\brak{x,y} = 0\le y\le x^2+1, 0\le y \le x+1, 0 \le x \le 2}
\end{align}
\item Find the interals in which the function 
\begin{align}
f(x)  = x^2-4x+6
\end{align}
%
is 
\begin{enumerate}
\item increasing
\item decreasing.
\end{enumerate}
%
\item Find the shortest distance of the point $\myvec{0\\c}$ from the parabola $y = x^2$, where $\frac{1}{2} \le c \le 5$.
%
\item An apache helicopter of enemy is flying along the curve given by $y = x^2+7$.  A soldier, placed at \myvec{3\\7}, wants to shoot down the helicopter when it is nearest to him.  Find  the nearest distance.
%
\item Examine whether the function $f$ given by $f(x) = x^2$ is continuous at $x = 0$.
%
\item Discuss the continuity of the function $f$ defined by 
%
\begin{align}
f(x)  = 
\begin{cases}
x & x \ge 0
\\
x^2 & x < 0
\end{cases}
\end{align}
%
\item Verify Rolle's theorem for the function $y = x^2+2, a = -2$ and $b = 2$.
\item Verify Mean Value Theorem for the function $f(x) = x^2$ in the interval $\sbrak{2,-4}$.
%\end{align}
\item Find the derivative of $f(x) = x^2$.
\item Find the derivative of $ x^2 - 2$ at $x = 10$.
\item Find the derivative of $ \brak{x-1} \brak{x-2}$.
%
%
\item Find 
\begin{align}
\int_{0}^{2} \brak{x^2+1}\,dx
\end{align}
%
as a limit of a sum.
\item Evaluate the following integral:
%
\begin{align}
\int_{2}^{3}x^2 \,dx
\end{align}
%
\item Form the differntial equation representing the family of ellipses having foci on x-axis and cenre at the origin.
%
\item Form the differntial equation representing the family of parabolas having vertex at origin and axis along positive direction of x-axis.
\item Form a differntial equation representing the following family of curves
%
\begin{align}
y^2 = a\brak{b^2-x^2}
\end{align}
%
\end{enumerate}
 
\renewcommand{\theequation}{\theenumi}
%\begin{enumerate}[label=\arabic*.,ref=\theenumi]
\begin{enumerate}[label=\thesection.\arabic*.,ref=\thesection.\theenumi]
\numberwithin{equation}{enumi}

\item Find the area of the region enclosed between the two circles: $\vec{x}^T\vec{x} = 4$ and $\norm{\vec{x}-\myvec{2\\0}} = 2$.
\\
\solution 
Let the balanced version of (\ref{eq:solutions/chem/6ato balance}) be
\begin{align}
    \label{eq:solutions/chem/6abalanced}x_{1}HNO_{3}+ x_{2}Ca(OH)_{2}\to x_{3}Ca(NO_{3})_{2}+ x_{4}H_{2}O
\end{align}

which results in the following equations:
\begin{align}
    (x_{1}+ 2x_{2}-2x_{4}) H= 0\\
    (x_{1}-2x_{3}) N= 0\\
    (3x_{1}+ 2x_{2}-6x_{3}- x_{4}) O=0\\
    (x_{2}-x_{3}) Ca= 0
\end{align}

which can be expressed as
\begin{align}
    x_{1}+ 2x_{2}+ 0.x_{3} -2x_{4} = 0\\
    x_{1}+ 0.x_{2} -2x_{3} +0.x_{4}= 0\\
    3x_{1}+ 2x_{2}-6x_{3}- x_{4} =0\\
    0.x_{1} +x_{2}-x_{3} +0.x_{4}= 0
\end{align}

resulting in the matrix equation
\begin{align}
    \label{eq:solutions/chem/6a matrix}
    \myvec{1 & 2 & 0 & -2\\
           1 & 0 & -2 & 0\\
           3 & 2 & -6 & -1\\
           0 & 1 & -1 & 0}\vec{x}
           =\vec{0}
\end{align}

where,
\begin{align}
   \vec{x}= \myvec{x_{1}\\x_{2}\\x_{3}\\x_{4}}
\end{align}

(\ref{eq:solutions/chem/6a matrix}) can be reduced as follows:
\begin{align}
    \myvec{1 & 2 & 0 & -2\\
           1 & 0 & -2 & 0\\
           3 & 2 & -6 & -1\\
           0 & 1 & -1 & 0}
    \xleftrightarrow[R_{3}\leftarrow \frac{R_3}{3}-R_{1}]{R_{2}\leftarrow R_2- R_1}
    \myvec{1 & 2 & 0 & -2\\
           0 & -2 & -2 & 2\\
           0 & -\frac{4}{3} & -2 & \frac{5}{3}\\
           0 & 1 & -1 & 0}\\
    \xleftrightarrow{R_2 \leftarrow -\frac{R_2}{2}}
    \myvec{1 & 2 & 0 & -2\\
          0 & 1 & 1 & -1\\
          0 & -\frac{4}{3} & -2 & \frac{5}{3}\\
          0 & 1 & -1 & 0}\\
    \xleftrightarrow[R_4 \leftarrow R_4- R_2]{R_3 \leftarrow R_3 + \frac{4}{3}R_2}
    \myvec{1 & 2 & 0 & -2\\
           0 & 1 & 1 & -1\\
           0 & 0 & -\frac{2}{3} & \frac{1}{3}\\
           0 & 0 & -2 & 1}\\
    \xleftrightarrow[R_3 \leftarrow -\frac{3}{2}R_3]{R_1 \leftarrow R_1- 2R_2}
    \myvec{1 & 0 & -2 & 0\\
           0 & 1 & 1 & -1\\
           0 & 0 & 1 & -\frac{1}{2}\\
           0 & 0 & -2 & 1}\\
    \xleftrightarrow{R_4\leftarrow R_4 + 2R_3}
    \myvec{1 & 0 & -2 & 0\\
           0 & 1 & 1 & -1\\
           0 & 0 & 1 & -\frac{1}{2}\\
           0 & 0 & 0 & 0}\\
    \xleftrightarrow[R_2\leftarrow R_2-R_3]{R_1\leftarrow R_1 + 2R_3}
    \myvec{1 & 0 & 0 & -1\\
           0 & 1 & 0 & -\frac{1}{2}\\
           0 & 0 & 1 & -\frac{1}{2}\\
           0 & 0 & 0 & 0}
\end{align}

Thus,
\begin{align}
    x_1=x_4, x_2= \frac{1}{2}x_4, x_3=\frac{1}{2}x_4\\
    \implies \quad\vec{x}= x_4\myvec{1\\ \frac{1}{2}\\ \frac{1}{2}\\1} =\myvec{2\\1\\1\\2}
\end{align} 
by substituting $x_4= 2$.

\hfill\break
%\vspace{5mm} 
Hence, (\ref{eq:solutions/chem/6abalanced}) finally becomes
\begin{align}
    2HNO_{3}+ Ca(OH)_{2}\to Ca(NO_{3})_{2}+ 2H_{2}O
\end{align}

\item Find the equation of the circle with radius 5 whose centre lies on x-axis and passes through the point \myvec{2\\3}.
\\
\solution 
Let the balanced version of (\ref{eq:solutions/chem/6ato balance}) be
\begin{align}
    \label{eq:solutions/chem/6abalanced}x_{1}HNO_{3}+ x_{2}Ca(OH)_{2}\to x_{3}Ca(NO_{3})_{2}+ x_{4}H_{2}O
\end{align}

which results in the following equations:
\begin{align}
    (x_{1}+ 2x_{2}-2x_{4}) H= 0\\
    (x_{1}-2x_{3}) N= 0\\
    (3x_{1}+ 2x_{2}-6x_{3}- x_{4}) O=0\\
    (x_{2}-x_{3}) Ca= 0
\end{align}

which can be expressed as
\begin{align}
    x_{1}+ 2x_{2}+ 0.x_{3} -2x_{4} = 0\\
    x_{1}+ 0.x_{2} -2x_{3} +0.x_{4}= 0\\
    3x_{1}+ 2x_{2}-6x_{3}- x_{4} =0\\
    0.x_{1} +x_{2}-x_{3} +0.x_{4}= 0
\end{align}

resulting in the matrix equation
\begin{align}
    \label{eq:solutions/chem/6a matrix}
    \myvec{1 & 2 & 0 & -2\\
           1 & 0 & -2 & 0\\
           3 & 2 & -6 & -1\\
           0 & 1 & -1 & 0}\vec{x}
           =\vec{0}
\end{align}

where,
\begin{align}
   \vec{x}= \myvec{x_{1}\\x_{2}\\x_{3}\\x_{4}}
\end{align}

(\ref{eq:solutions/chem/6a matrix}) can be reduced as follows:
\begin{align}
    \myvec{1 & 2 & 0 & -2\\
           1 & 0 & -2 & 0\\
           3 & 2 & -6 & -1\\
           0 & 1 & -1 & 0}
    \xleftrightarrow[R_{3}\leftarrow \frac{R_3}{3}-R_{1}]{R_{2}\leftarrow R_2- R_1}
    \myvec{1 & 2 & 0 & -2\\
           0 & -2 & -2 & 2\\
           0 & -\frac{4}{3} & -2 & \frac{5}{3}\\
           0 & 1 & -1 & 0}\\
    \xleftrightarrow{R_2 \leftarrow -\frac{R_2}{2}}
    \myvec{1 & 2 & 0 & -2\\
          0 & 1 & 1 & -1\\
          0 & -\frac{4}{3} & -2 & \frac{5}{3}\\
          0 & 1 & -1 & 0}\\
    \xleftrightarrow[R_4 \leftarrow R_4- R_2]{R_3 \leftarrow R_3 + \frac{4}{3}R_2}
    \myvec{1 & 2 & 0 & -2\\
           0 & 1 & 1 & -1\\
           0 & 0 & -\frac{2}{3} & \frac{1}{3}\\
           0 & 0 & -2 & 1}\\
    \xleftrightarrow[R_3 \leftarrow -\frac{3}{2}R_3]{R_1 \leftarrow R_1- 2R_2}
    \myvec{1 & 0 & -2 & 0\\
           0 & 1 & 1 & -1\\
           0 & 0 & 1 & -\frac{1}{2}\\
           0 & 0 & -2 & 1}\\
    \xleftrightarrow{R_4\leftarrow R_4 + 2R_3}
    \myvec{1 & 0 & -2 & 0\\
           0 & 1 & 1 & -1\\
           0 & 0 & 1 & -\frac{1}{2}\\
           0 & 0 & 0 & 0}\\
    \xleftrightarrow[R_2\leftarrow R_2-R_3]{R_1\leftarrow R_1 + 2R_3}
    \myvec{1 & 0 & 0 & -1\\
           0 & 1 & 0 & -\frac{1}{2}\\
           0 & 0 & 1 & -\frac{1}{2}\\
           0 & 0 & 0 & 0}
\end{align}

Thus,
\begin{align}
    x_1=x_4, x_2= \frac{1}{2}x_4, x_3=\frac{1}{2}x_4\\
    \implies \quad\vec{x}= x_4\myvec{1\\ \frac{1}{2}\\ \frac{1}{2}\\1} =\myvec{2\\1\\1\\2}
\end{align} 
by substituting $x_4= 2$.

\hfill\break
%\vspace{5mm} 
Hence, (\ref{eq:solutions/chem/6abalanced}) finally becomes
\begin{align}
    2HNO_{3}+ Ca(OH)_{2}\to Ca(NO_{3})_{2}+ 2H_{2}O
\end{align}

\item Find the equation of the circle passing through \myvec{0\\0} and making intercepts a and b on the coordinate axes.
\item Find the equation of a circle with centre \myvec{2\\2} and passes through the point \myvec{4\\5}. 
\\
\solution 
Let the balanced version of (\ref{eq:solutions/chem/6ato balance}) be
\begin{align}
    \label{eq:solutions/chem/6abalanced}x_{1}HNO_{3}+ x_{2}Ca(OH)_{2}\to x_{3}Ca(NO_{3})_{2}+ x_{4}H_{2}O
\end{align}

which results in the following equations:
\begin{align}
    (x_{1}+ 2x_{2}-2x_{4}) H= 0\\
    (x_{1}-2x_{3}) N= 0\\
    (3x_{1}+ 2x_{2}-6x_{3}- x_{4}) O=0\\
    (x_{2}-x_{3}) Ca= 0
\end{align}

which can be expressed as
\begin{align}
    x_{1}+ 2x_{2}+ 0.x_{3} -2x_{4} = 0\\
    x_{1}+ 0.x_{2} -2x_{3} +0.x_{4}= 0\\
    3x_{1}+ 2x_{2}-6x_{3}- x_{4} =0\\
    0.x_{1} +x_{2}-x_{3} +0.x_{4}= 0
\end{align}

resulting in the matrix equation
\begin{align}
    \label{eq:solutions/chem/6a matrix}
    \myvec{1 & 2 & 0 & -2\\
           1 & 0 & -2 & 0\\
           3 & 2 & -6 & -1\\
           0 & 1 & -1 & 0}\vec{x}
           =\vec{0}
\end{align}

where,
\begin{align}
   \vec{x}= \myvec{x_{1}\\x_{2}\\x_{3}\\x_{4}}
\end{align}

(\ref{eq:solutions/chem/6a matrix}) can be reduced as follows:
\begin{align}
    \myvec{1 & 2 & 0 & -2\\
           1 & 0 & -2 & 0\\
           3 & 2 & -6 & -1\\
           0 & 1 & -1 & 0}
    \xleftrightarrow[R_{3}\leftarrow \frac{R_3}{3}-R_{1}]{R_{2}\leftarrow R_2- R_1}
    \myvec{1 & 2 & 0 & -2\\
           0 & -2 & -2 & 2\\
           0 & -\frac{4}{3} & -2 & \frac{5}{3}\\
           0 & 1 & -1 & 0}\\
    \xleftrightarrow{R_2 \leftarrow -\frac{R_2}{2}}
    \myvec{1 & 2 & 0 & -2\\
          0 & 1 & 1 & -1\\
          0 & -\frac{4}{3} & -2 & \frac{5}{3}\\
          0 & 1 & -1 & 0}\\
    \xleftrightarrow[R_4 \leftarrow R_4- R_2]{R_3 \leftarrow R_3 + \frac{4}{3}R_2}
    \myvec{1 & 2 & 0 & -2\\
           0 & 1 & 1 & -1\\
           0 & 0 & -\frac{2}{3} & \frac{1}{3}\\
           0 & 0 & -2 & 1}\\
    \xleftrightarrow[R_3 \leftarrow -\frac{3}{2}R_3]{R_1 \leftarrow R_1- 2R_2}
    \myvec{1 & 0 & -2 & 0\\
           0 & 1 & 1 & -1\\
           0 & 0 & 1 & -\frac{1}{2}\\
           0 & 0 & -2 & 1}\\
    \xleftrightarrow{R_4\leftarrow R_4 + 2R_3}
    \myvec{1 & 0 & -2 & 0\\
           0 & 1 & 1 & -1\\
           0 & 0 & 1 & -\frac{1}{2}\\
           0 & 0 & 0 & 0}\\
    \xleftrightarrow[R_2\leftarrow R_2-R_3]{R_1\leftarrow R_1 + 2R_3}
    \myvec{1 & 0 & 0 & -1\\
           0 & 1 & 0 & -\frac{1}{2}\\
           0 & 0 & 1 & -\frac{1}{2}\\
           0 & 0 & 0 & 0}
\end{align}

Thus,
\begin{align}
    x_1=x_4, x_2= \frac{1}{2}x_4, x_3=\frac{1}{2}x_4\\
    \implies \quad\vec{x}= x_4\myvec{1\\ \frac{1}{2}\\ \frac{1}{2}\\1} =\myvec{2\\1\\1\\2}
\end{align} 
by substituting $x_4= 2$.

\hfill\break
%\vspace{5mm} 
Hence, (\ref{eq:solutions/chem/6abalanced}) finally becomes
\begin{align}
    2HNO_{3}+ Ca(OH)_{2}\to Ca(NO_{3})_{2}+ 2H_{2}O
\end{align}

\item Find the locus of all the unit vectors in the xy-plane.
%
\item Find the points on the curve $\vec{x}^T\vec{x}-2\myvec{1 & 0}\vec{x} -3 =0$  at which the tangents are parallel to the x-axis.
%
\\
\solution
Let the balanced version of (\ref{eq:solutions/chem/6ato balance}) be
\begin{align}
    \label{eq:solutions/chem/6abalanced}x_{1}HNO_{3}+ x_{2}Ca(OH)_{2}\to x_{3}Ca(NO_{3})_{2}+ x_{4}H_{2}O
\end{align}

which results in the following equations:
\begin{align}
    (x_{1}+ 2x_{2}-2x_{4}) H= 0\\
    (x_{1}-2x_{3}) N= 0\\
    (3x_{1}+ 2x_{2}-6x_{3}- x_{4}) O=0\\
    (x_{2}-x_{3}) Ca= 0
\end{align}

which can be expressed as
\begin{align}
    x_{1}+ 2x_{2}+ 0.x_{3} -2x_{4} = 0\\
    x_{1}+ 0.x_{2} -2x_{3} +0.x_{4}= 0\\
    3x_{1}+ 2x_{2}-6x_{3}- x_{4} =0\\
    0.x_{1} +x_{2}-x_{3} +0.x_{4}= 0
\end{align}

resulting in the matrix equation
\begin{align}
    \label{eq:solutions/chem/6a matrix}
    \myvec{1 & 2 & 0 & -2\\
           1 & 0 & -2 & 0\\
           3 & 2 & -6 & -1\\
           0 & 1 & -1 & 0}\vec{x}
           =\vec{0}
\end{align}

where,
\begin{align}
   \vec{x}= \myvec{x_{1}\\x_{2}\\x_{3}\\x_{4}}
\end{align}

(\ref{eq:solutions/chem/6a matrix}) can be reduced as follows:
\begin{align}
    \myvec{1 & 2 & 0 & -2\\
           1 & 0 & -2 & 0\\
           3 & 2 & -6 & -1\\
           0 & 1 & -1 & 0}
    \xleftrightarrow[R_{3}\leftarrow \frac{R_3}{3}-R_{1}]{R_{2}\leftarrow R_2- R_1}
    \myvec{1 & 2 & 0 & -2\\
           0 & -2 & -2 & 2\\
           0 & -\frac{4}{3} & -2 & \frac{5}{3}\\
           0 & 1 & -1 & 0}\\
    \xleftrightarrow{R_2 \leftarrow -\frac{R_2}{2}}
    \myvec{1 & 2 & 0 & -2\\
          0 & 1 & 1 & -1\\
          0 & -\frac{4}{3} & -2 & \frac{5}{3}\\
          0 & 1 & -1 & 0}\\
    \xleftrightarrow[R_4 \leftarrow R_4- R_2]{R_3 \leftarrow R_3 + \frac{4}{3}R_2}
    \myvec{1 & 2 & 0 & -2\\
           0 & 1 & 1 & -1\\
           0 & 0 & -\frac{2}{3} & \frac{1}{3}\\
           0 & 0 & -2 & 1}\\
    \xleftrightarrow[R_3 \leftarrow -\frac{3}{2}R_3]{R_1 \leftarrow R_1- 2R_2}
    \myvec{1 & 0 & -2 & 0\\
           0 & 1 & 1 & -1\\
           0 & 0 & 1 & -\frac{1}{2}\\
           0 & 0 & -2 & 1}\\
    \xleftrightarrow{R_4\leftarrow R_4 + 2R_3}
    \myvec{1 & 0 & -2 & 0\\
           0 & 1 & 1 & -1\\
           0 & 0 & 1 & -\frac{1}{2}\\
           0 & 0 & 0 & 0}\\
    \xleftrightarrow[R_2\leftarrow R_2-R_3]{R_1\leftarrow R_1 + 2R_3}
    \myvec{1 & 0 & 0 & -1\\
           0 & 1 & 0 & -\frac{1}{2}\\
           0 & 0 & 1 & -\frac{1}{2}\\
           0 & 0 & 0 & 0}
\end{align}

Thus,
\begin{align}
    x_1=x_4, x_2= \frac{1}{2}x_4, x_3=\frac{1}{2}x_4\\
    \implies \quad\vec{x}= x_4\myvec{1\\ \frac{1}{2}\\ \frac{1}{2}\\1} =\myvec{2\\1\\1\\2}
\end{align} 
by substituting $x_4= 2$.

\hfill\break
%\vspace{5mm} 
Hence, (\ref{eq:solutions/chem/6abalanced}) finally becomes
\begin{align}
    2HNO_{3}+ Ca(OH)_{2}\to Ca(NO_{3})_{2}+ 2H_{2}O
\end{align}

\item  Find the area of the region in the first quadrant enclosed by x-axis, line $\myvec{1 & -\sqrt{3}}\vec{x} =0$ and the circle $\vec{x}^T\vec{x}=4$.
%
\\
\solution
Let the balanced version of (\ref{eq:solutions/chem/6ato balance}) be
\begin{align}
    \label{eq:solutions/chem/6abalanced}x_{1}HNO_{3}+ x_{2}Ca(OH)_{2}\to x_{3}Ca(NO_{3})_{2}+ x_{4}H_{2}O
\end{align}

which results in the following equations:
\begin{align}
    (x_{1}+ 2x_{2}-2x_{4}) H= 0\\
    (x_{1}-2x_{3}) N= 0\\
    (3x_{1}+ 2x_{2}-6x_{3}- x_{4}) O=0\\
    (x_{2}-x_{3}) Ca= 0
\end{align}

which can be expressed as
\begin{align}
    x_{1}+ 2x_{2}+ 0.x_{3} -2x_{4} = 0\\
    x_{1}+ 0.x_{2} -2x_{3} +0.x_{4}= 0\\
    3x_{1}+ 2x_{2}-6x_{3}- x_{4} =0\\
    0.x_{1} +x_{2}-x_{3} +0.x_{4}= 0
\end{align}

resulting in the matrix equation
\begin{align}
    \label{eq:solutions/chem/6a matrix}
    \myvec{1 & 2 & 0 & -2\\
           1 & 0 & -2 & 0\\
           3 & 2 & -6 & -1\\
           0 & 1 & -1 & 0}\vec{x}
           =\vec{0}
\end{align}

where,
\begin{align}
   \vec{x}= \myvec{x_{1}\\x_{2}\\x_{3}\\x_{4}}
\end{align}

(\ref{eq:solutions/chem/6a matrix}) can be reduced as follows:
\begin{align}
    \myvec{1 & 2 & 0 & -2\\
           1 & 0 & -2 & 0\\
           3 & 2 & -6 & -1\\
           0 & 1 & -1 & 0}
    \xleftrightarrow[R_{3}\leftarrow \frac{R_3}{3}-R_{1}]{R_{2}\leftarrow R_2- R_1}
    \myvec{1 & 2 & 0 & -2\\
           0 & -2 & -2 & 2\\
           0 & -\frac{4}{3} & -2 & \frac{5}{3}\\
           0 & 1 & -1 & 0}\\
    \xleftrightarrow{R_2 \leftarrow -\frac{R_2}{2}}
    \myvec{1 & 2 & 0 & -2\\
          0 & 1 & 1 & -1\\
          0 & -\frac{4}{3} & -2 & \frac{5}{3}\\
          0 & 1 & -1 & 0}\\
    \xleftrightarrow[R_4 \leftarrow R_4- R_2]{R_3 \leftarrow R_3 + \frac{4}{3}R_2}
    \myvec{1 & 2 & 0 & -2\\
           0 & 1 & 1 & -1\\
           0 & 0 & -\frac{2}{3} & \frac{1}{3}\\
           0 & 0 & -2 & 1}\\
    \xleftrightarrow[R_3 \leftarrow -\frac{3}{2}R_3]{R_1 \leftarrow R_1- 2R_2}
    \myvec{1 & 0 & -2 & 0\\
           0 & 1 & 1 & -1\\
           0 & 0 & 1 & -\frac{1}{2}\\
           0 & 0 & -2 & 1}\\
    \xleftrightarrow{R_4\leftarrow R_4 + 2R_3}
    \myvec{1 & 0 & -2 & 0\\
           0 & 1 & 1 & -1\\
           0 & 0 & 1 & -\frac{1}{2}\\
           0 & 0 & 0 & 0}\\
    \xleftrightarrow[R_2\leftarrow R_2-R_3]{R_1\leftarrow R_1 + 2R_3}
    \myvec{1 & 0 & 0 & -1\\
           0 & 1 & 0 & -\frac{1}{2}\\
           0 & 0 & 1 & -\frac{1}{2}\\
           0 & 0 & 0 & 0}
\end{align}

Thus,
\begin{align}
    x_1=x_4, x_2= \frac{1}{2}x_4, x_3=\frac{1}{2}x_4\\
    \implies \quad\vec{x}= x_4\myvec{1\\ \frac{1}{2}\\ \frac{1}{2}\\1} =\myvec{2\\1\\1\\2}
\end{align} 
by substituting $x_4= 2$.

\hfill\break
%\vspace{5mm} 
Hence, (\ref{eq:solutions/chem/6abalanced}) finally becomes
\begin{align}
    2HNO_{3}+ Ca(OH)_{2}\to Ca(NO_{3})_{2}+ 2H_{2}O
\end{align}

\item Find the area lying in the first quadrant and bounded by the circle $\vec{x}^T\vec{x}=4$ and the lines $x = 0$ and $x = 2$.
%
\item Find the area of the circle $4\vec{x}^T\vec{x}=9$.
\item  Find the area bounded by curves $\norm{\vec{x}-\myvec{1\\0}} = 1$ and $\norm{\vec{x}}=1$
\\
\solution
Let the balanced version of (\ref{eq:solutions/chem/6ato balance}) be
\begin{align}
    \label{eq:solutions/chem/6abalanced}x_{1}HNO_{3}+ x_{2}Ca(OH)_{2}\to x_{3}Ca(NO_{3})_{2}+ x_{4}H_{2}O
\end{align}

which results in the following equations:
\begin{align}
    (x_{1}+ 2x_{2}-2x_{4}) H= 0\\
    (x_{1}-2x_{3}) N= 0\\
    (3x_{1}+ 2x_{2}-6x_{3}- x_{4}) O=0\\
    (x_{2}-x_{3}) Ca= 0
\end{align}

which can be expressed as
\begin{align}
    x_{1}+ 2x_{2}+ 0.x_{3} -2x_{4} = 0\\
    x_{1}+ 0.x_{2} -2x_{3} +0.x_{4}= 0\\
    3x_{1}+ 2x_{2}-6x_{3}- x_{4} =0\\
    0.x_{1} +x_{2}-x_{3} +0.x_{4}= 0
\end{align}

resulting in the matrix equation
\begin{align}
    \label{eq:solutions/chem/6a matrix}
    \myvec{1 & 2 & 0 & -2\\
           1 & 0 & -2 & 0\\
           3 & 2 & -6 & -1\\
           0 & 1 & -1 & 0}\vec{x}
           =\vec{0}
\end{align}

where,
\begin{align}
   \vec{x}= \myvec{x_{1}\\x_{2}\\x_{3}\\x_{4}}
\end{align}

(\ref{eq:solutions/chem/6a matrix}) can be reduced as follows:
\begin{align}
    \myvec{1 & 2 & 0 & -2\\
           1 & 0 & -2 & 0\\
           3 & 2 & -6 & -1\\
           0 & 1 & -1 & 0}
    \xleftrightarrow[R_{3}\leftarrow \frac{R_3}{3}-R_{1}]{R_{2}\leftarrow R_2- R_1}
    \myvec{1 & 2 & 0 & -2\\
           0 & -2 & -2 & 2\\
           0 & -\frac{4}{3} & -2 & \frac{5}{3}\\
           0 & 1 & -1 & 0}\\
    \xleftrightarrow{R_2 \leftarrow -\frac{R_2}{2}}
    \myvec{1 & 2 & 0 & -2\\
          0 & 1 & 1 & -1\\
          0 & -\frac{4}{3} & -2 & \frac{5}{3}\\
          0 & 1 & -1 & 0}\\
    \xleftrightarrow[R_4 \leftarrow R_4- R_2]{R_3 \leftarrow R_3 + \frac{4}{3}R_2}
    \myvec{1 & 2 & 0 & -2\\
           0 & 1 & 1 & -1\\
           0 & 0 & -\frac{2}{3} & \frac{1}{3}\\
           0 & 0 & -2 & 1}\\
    \xleftrightarrow[R_3 \leftarrow -\frac{3}{2}R_3]{R_1 \leftarrow R_1- 2R_2}
    \myvec{1 & 0 & -2 & 0\\
           0 & 1 & 1 & -1\\
           0 & 0 & 1 & -\frac{1}{2}\\
           0 & 0 & -2 & 1}\\
    \xleftrightarrow{R_4\leftarrow R_4 + 2R_3}
    \myvec{1 & 0 & -2 & 0\\
           0 & 1 & 1 & -1\\
           0 & 0 & 1 & -\frac{1}{2}\\
           0 & 0 & 0 & 0}\\
    \xleftrightarrow[R_2\leftarrow R_2-R_3]{R_1\leftarrow R_1 + 2R_3}
    \myvec{1 & 0 & 0 & -1\\
           0 & 1 & 0 & -\frac{1}{2}\\
           0 & 0 & 1 & -\frac{1}{2}\\
           0 & 0 & 0 & 0}
\end{align}

Thus,
\begin{align}
    x_1=x_4, x_2= \frac{1}{2}x_4, x_3=\frac{1}{2}x_4\\
    \implies \quad\vec{x}= x_4\myvec{1\\ \frac{1}{2}\\ \frac{1}{2}\\1} =\myvec{2\\1\\1\\2}
\end{align} 
by substituting $x_4= 2$.

\hfill\break
%\vspace{5mm} 
Hence, (\ref{eq:solutions/chem/6abalanced}) finally becomes
\begin{align}
    2HNO_{3}+ Ca(OH)_{2}\to Ca(NO_{3})_{2}+ 2H_{2}O
\end{align}

\item Find the smaller area enclosed by the circle $\vec{x}^T\vec{x}=4$ and the line $\myvec{1 & 1}\vec{x} = 2$.


\item Find the slope of the tangent to the curve $y = \frac{x-1}{x-2}, x\ne 2$ at $x = 10$.
\\
\solution
Let the balanced version of (\ref{eq:solutions/chem/6ato balance}) be
\begin{align}
    \label{eq:solutions/chem/6abalanced}x_{1}HNO_{3}+ x_{2}Ca(OH)_{2}\to x_{3}Ca(NO_{3})_{2}+ x_{4}H_{2}O
\end{align}

which results in the following equations:
\begin{align}
    (x_{1}+ 2x_{2}-2x_{4}) H= 0\\
    (x_{1}-2x_{3}) N= 0\\
    (3x_{1}+ 2x_{2}-6x_{3}- x_{4}) O=0\\
    (x_{2}-x_{3}) Ca= 0
\end{align}

which can be expressed as
\begin{align}
    x_{1}+ 2x_{2}+ 0.x_{3} -2x_{4} = 0\\
    x_{1}+ 0.x_{2} -2x_{3} +0.x_{4}= 0\\
    3x_{1}+ 2x_{2}-6x_{3}- x_{4} =0\\
    0.x_{1} +x_{2}-x_{3} +0.x_{4}= 0
\end{align}

resulting in the matrix equation
\begin{align}
    \label{eq:solutions/chem/6a matrix}
    \myvec{1 & 2 & 0 & -2\\
           1 & 0 & -2 & 0\\
           3 & 2 & -6 & -1\\
           0 & 1 & -1 & 0}\vec{x}
           =\vec{0}
\end{align}

where,
\begin{align}
   \vec{x}= \myvec{x_{1}\\x_{2}\\x_{3}\\x_{4}}
\end{align}

(\ref{eq:solutions/chem/6a matrix}) can be reduced as follows:
\begin{align}
    \myvec{1 & 2 & 0 & -2\\
           1 & 0 & -2 & 0\\
           3 & 2 & -6 & -1\\
           0 & 1 & -1 & 0}
    \xleftrightarrow[R_{3}\leftarrow \frac{R_3}{3}-R_{1}]{R_{2}\leftarrow R_2- R_1}
    \myvec{1 & 2 & 0 & -2\\
           0 & -2 & -2 & 2\\
           0 & -\frac{4}{3} & -2 & \frac{5}{3}\\
           0 & 1 & -1 & 0}\\
    \xleftrightarrow{R_2 \leftarrow -\frac{R_2}{2}}
    \myvec{1 & 2 & 0 & -2\\
          0 & 1 & 1 & -1\\
          0 & -\frac{4}{3} & -2 & \frac{5}{3}\\
          0 & 1 & -1 & 0}\\
    \xleftrightarrow[R_4 \leftarrow R_4- R_2]{R_3 \leftarrow R_3 + \frac{4}{3}R_2}
    \myvec{1 & 2 & 0 & -2\\
           0 & 1 & 1 & -1\\
           0 & 0 & -\frac{2}{3} & \frac{1}{3}\\
           0 & 0 & -2 & 1}\\
    \xleftrightarrow[R_3 \leftarrow -\frac{3}{2}R_3]{R_1 \leftarrow R_1- 2R_2}
    \myvec{1 & 0 & -2 & 0\\
           0 & 1 & 1 & -1\\
           0 & 0 & 1 & -\frac{1}{2}\\
           0 & 0 & -2 & 1}\\
    \xleftrightarrow{R_4\leftarrow R_4 + 2R_3}
    \myvec{1 & 0 & -2 & 0\\
           0 & 1 & 1 & -1\\
           0 & 0 & 1 & -\frac{1}{2}\\
           0 & 0 & 0 & 0}\\
    \xleftrightarrow[R_2\leftarrow R_2-R_3]{R_1\leftarrow R_1 + 2R_3}
    \myvec{1 & 0 & 0 & -1\\
           0 & 1 & 0 & -\frac{1}{2}\\
           0 & 0 & 1 & -\frac{1}{2}\\
           0 & 0 & 0 & 0}
\end{align}

Thus,
\begin{align}
    x_1=x_4, x_2= \frac{1}{2}x_4, x_3=\frac{1}{2}x_4\\
    \implies \quad\vec{x}= x_4\myvec{1\\ \frac{1}{2}\\ \frac{1}{2}\\1} =\myvec{2\\1\\1\\2}
\end{align} 
by substituting $x_4= 2$.

\hfill\break
%\vspace{5mm} 
Hence, (\ref{eq:solutions/chem/6abalanced}) finally becomes
\begin{align}
    2HNO_{3}+ Ca(OH)_{2}\to Ca(NO_{3})_{2}+ 2H_{2}O
\end{align}

\item Find a point on the curve $y = (x – 2)^2$ at which the tangent is parallel to the chord joining the points \myvec{2\\ 0} and \myvec{4\\ 4}.
\\
\solution
Let the balanced version of (\ref{eq:solutions/chem/6ato balance}) be
\begin{align}
    \label{eq:solutions/chem/6abalanced}x_{1}HNO_{3}+ x_{2}Ca(OH)_{2}\to x_{3}Ca(NO_{3})_{2}+ x_{4}H_{2}O
\end{align}

which results in the following equations:
\begin{align}
    (x_{1}+ 2x_{2}-2x_{4}) H= 0\\
    (x_{1}-2x_{3}) N= 0\\
    (3x_{1}+ 2x_{2}-6x_{3}- x_{4}) O=0\\
    (x_{2}-x_{3}) Ca= 0
\end{align}

which can be expressed as
\begin{align}
    x_{1}+ 2x_{2}+ 0.x_{3} -2x_{4} = 0\\
    x_{1}+ 0.x_{2} -2x_{3} +0.x_{4}= 0\\
    3x_{1}+ 2x_{2}-6x_{3}- x_{4} =0\\
    0.x_{1} +x_{2}-x_{3} +0.x_{4}= 0
\end{align}

resulting in the matrix equation
\begin{align}
    \label{eq:solutions/chem/6a matrix}
    \myvec{1 & 2 & 0 & -2\\
           1 & 0 & -2 & 0\\
           3 & 2 & -6 & -1\\
           0 & 1 & -1 & 0}\vec{x}
           =\vec{0}
\end{align}

where,
\begin{align}
   \vec{x}= \myvec{x_{1}\\x_{2}\\x_{3}\\x_{4}}
\end{align}

(\ref{eq:solutions/chem/6a matrix}) can be reduced as follows:
\begin{align}
    \myvec{1 & 2 & 0 & -2\\
           1 & 0 & -2 & 0\\
           3 & 2 & -6 & -1\\
           0 & 1 & -1 & 0}
    \xleftrightarrow[R_{3}\leftarrow \frac{R_3}{3}-R_{1}]{R_{2}\leftarrow R_2- R_1}
    \myvec{1 & 2 & 0 & -2\\
           0 & -2 & -2 & 2\\
           0 & -\frac{4}{3} & -2 & \frac{5}{3}\\
           0 & 1 & -1 & 0}\\
    \xleftrightarrow{R_2 \leftarrow -\frac{R_2}{2}}
    \myvec{1 & 2 & 0 & -2\\
          0 & 1 & 1 & -1\\
          0 & -\frac{4}{3} & -2 & \frac{5}{3}\\
          0 & 1 & -1 & 0}\\
    \xleftrightarrow[R_4 \leftarrow R_4- R_2]{R_3 \leftarrow R_3 + \frac{4}{3}R_2}
    \myvec{1 & 2 & 0 & -2\\
           0 & 1 & 1 & -1\\
           0 & 0 & -\frac{2}{3} & \frac{1}{3}\\
           0 & 0 & -2 & 1}\\
    \xleftrightarrow[R_3 \leftarrow -\frac{3}{2}R_3]{R_1 \leftarrow R_1- 2R_2}
    \myvec{1 & 0 & -2 & 0\\
           0 & 1 & 1 & -1\\
           0 & 0 & 1 & -\frac{1}{2}\\
           0 & 0 & -2 & 1}\\
    \xleftrightarrow{R_4\leftarrow R_4 + 2R_3}
    \myvec{1 & 0 & -2 & 0\\
           0 & 1 & 1 & -1\\
           0 & 0 & 1 & -\frac{1}{2}\\
           0 & 0 & 0 & 0}\\
    \xleftrightarrow[R_2\leftarrow R_2-R_3]{R_1\leftarrow R_1 + 2R_3}
    \myvec{1 & 0 & 0 & -1\\
           0 & 1 & 0 & -\frac{1}{2}\\
           0 & 0 & 1 & -\frac{1}{2}\\
           0 & 0 & 0 & 0}
\end{align}

Thus,
\begin{align}
    x_1=x_4, x_2= \frac{1}{2}x_4, x_3=\frac{1}{2}x_4\\
    \implies \quad\vec{x}= x_4\myvec{1\\ \frac{1}{2}\\ \frac{1}{2}\\1} =\myvec{2\\1\\1\\2}
\end{align} 
by substituting $x_4= 2$.

\hfill\break
%\vspace{5mm} 
Hence, (\ref{eq:solutions/chem/6abalanced}) finally becomes
\begin{align}
    2HNO_{3}+ Ca(OH)_{2}\to Ca(NO_{3})_{2}+ 2H_{2}O
\end{align}

\item Find the equation of all lines having slope – 1 that are tangents to the curve $\frac{1}
{x -1}, x \ne 1$
\\
\solution 
Let the balanced version of (\ref{eq:solutions/chem/6ato balance}) be
\begin{align}
    \label{eq:solutions/chem/6abalanced}x_{1}HNO_{3}+ x_{2}Ca(OH)_{2}\to x_{3}Ca(NO_{3})_{2}+ x_{4}H_{2}O
\end{align}

which results in the following equations:
\begin{align}
    (x_{1}+ 2x_{2}-2x_{4}) H= 0\\
    (x_{1}-2x_{3}) N= 0\\
    (3x_{1}+ 2x_{2}-6x_{3}- x_{4}) O=0\\
    (x_{2}-x_{3}) Ca= 0
\end{align}

which can be expressed as
\begin{align}
    x_{1}+ 2x_{2}+ 0.x_{3} -2x_{4} = 0\\
    x_{1}+ 0.x_{2} -2x_{3} +0.x_{4}= 0\\
    3x_{1}+ 2x_{2}-6x_{3}- x_{4} =0\\
    0.x_{1} +x_{2}-x_{3} +0.x_{4}= 0
\end{align}

resulting in the matrix equation
\begin{align}
    \label{eq:solutions/chem/6a matrix}
    \myvec{1 & 2 & 0 & -2\\
           1 & 0 & -2 & 0\\
           3 & 2 & -6 & -1\\
           0 & 1 & -1 & 0}\vec{x}
           =\vec{0}
\end{align}

where,
\begin{align}
   \vec{x}= \myvec{x_{1}\\x_{2}\\x_{3}\\x_{4}}
\end{align}

(\ref{eq:solutions/chem/6a matrix}) can be reduced as follows:
\begin{align}
    \myvec{1 & 2 & 0 & -2\\
           1 & 0 & -2 & 0\\
           3 & 2 & -6 & -1\\
           0 & 1 & -1 & 0}
    \xleftrightarrow[R_{3}\leftarrow \frac{R_3}{3}-R_{1}]{R_{2}\leftarrow R_2- R_1}
    \myvec{1 & 2 & 0 & -2\\
           0 & -2 & -2 & 2\\
           0 & -\frac{4}{3} & -2 & \frac{5}{3}\\
           0 & 1 & -1 & 0}\\
    \xleftrightarrow{R_2 \leftarrow -\frac{R_2}{2}}
    \myvec{1 & 2 & 0 & -2\\
          0 & 1 & 1 & -1\\
          0 & -\frac{4}{3} & -2 & \frac{5}{3}\\
          0 & 1 & -1 & 0}\\
    \xleftrightarrow[R_4 \leftarrow R_4- R_2]{R_3 \leftarrow R_3 + \frac{4}{3}R_2}
    \myvec{1 & 2 & 0 & -2\\
           0 & 1 & 1 & -1\\
           0 & 0 & -\frac{2}{3} & \frac{1}{3}\\
           0 & 0 & -2 & 1}\\
    \xleftrightarrow[R_3 \leftarrow -\frac{3}{2}R_3]{R_1 \leftarrow R_1- 2R_2}
    \myvec{1 & 0 & -2 & 0\\
           0 & 1 & 1 & -1\\
           0 & 0 & 1 & -\frac{1}{2}\\
           0 & 0 & -2 & 1}\\
    \xleftrightarrow{R_4\leftarrow R_4 + 2R_3}
    \myvec{1 & 0 & -2 & 0\\
           0 & 1 & 1 & -1\\
           0 & 0 & 1 & -\frac{1}{2}\\
           0 & 0 & 0 & 0}\\
    \xleftrightarrow[R_2\leftarrow R_2-R_3]{R_1\leftarrow R_1 + 2R_3}
    \myvec{1 & 0 & 0 & -1\\
           0 & 1 & 0 & -\frac{1}{2}\\
           0 & 0 & 1 & -\frac{1}{2}\\
           0 & 0 & 0 & 0}
\end{align}

Thus,
\begin{align}
    x_1=x_4, x_2= \frac{1}{2}x_4, x_3=\frac{1}{2}x_4\\
    \implies \quad\vec{x}= x_4\myvec{1\\ \frac{1}{2}\\ \frac{1}{2}\\1} =\myvec{2\\1\\1\\2}
\end{align} 
by substituting $x_4= 2$.

\hfill\break
%\vspace{5mm} 
Hence, (\ref{eq:solutions/chem/6abalanced}) finally becomes
\begin{align}
    2HNO_{3}+ Ca(OH)_{2}\to Ca(NO_{3})_{2}+ 2H_{2}O
\end{align}

\item Find the equation of all lines having slope -2 which are tangents to the curve $\frac{1}
{x - 3} , x \ne 3$.
%
\\
\solution 
Let the balanced version of (\ref{eq:solutions/chem/6ato balance}) be
\begin{align}
    \label{eq:solutions/chem/6abalanced}x_{1}HNO_{3}+ x_{2}Ca(OH)_{2}\to x_{3}Ca(NO_{3})_{2}+ x_{4}H_{2}O
\end{align}

which results in the following equations:
\begin{align}
    (x_{1}+ 2x_{2}-2x_{4}) H= 0\\
    (x_{1}-2x_{3}) N= 0\\
    (3x_{1}+ 2x_{2}-6x_{3}- x_{4}) O=0\\
    (x_{2}-x_{3}) Ca= 0
\end{align}

which can be expressed as
\begin{align}
    x_{1}+ 2x_{2}+ 0.x_{3} -2x_{4} = 0\\
    x_{1}+ 0.x_{2} -2x_{3} +0.x_{4}= 0\\
    3x_{1}+ 2x_{2}-6x_{3}- x_{4} =0\\
    0.x_{1} +x_{2}-x_{3} +0.x_{4}= 0
\end{align}

resulting in the matrix equation
\begin{align}
    \label{eq:solutions/chem/6a matrix}
    \myvec{1 & 2 & 0 & -2\\
           1 & 0 & -2 & 0\\
           3 & 2 & -6 & -1\\
           0 & 1 & -1 & 0}\vec{x}
           =\vec{0}
\end{align}

where,
\begin{align}
   \vec{x}= \myvec{x_{1}\\x_{2}\\x_{3}\\x_{4}}
\end{align}

(\ref{eq:solutions/chem/6a matrix}) can be reduced as follows:
\begin{align}
    \myvec{1 & 2 & 0 & -2\\
           1 & 0 & -2 & 0\\
           3 & 2 & -6 & -1\\
           0 & 1 & -1 & 0}
    \xleftrightarrow[R_{3}\leftarrow \frac{R_3}{3}-R_{1}]{R_{2}\leftarrow R_2- R_1}
    \myvec{1 & 2 & 0 & -2\\
           0 & -2 & -2 & 2\\
           0 & -\frac{4}{3} & -2 & \frac{5}{3}\\
           0 & 1 & -1 & 0}\\
    \xleftrightarrow{R_2 \leftarrow -\frac{R_2}{2}}
    \myvec{1 & 2 & 0 & -2\\
          0 & 1 & 1 & -1\\
          0 & -\frac{4}{3} & -2 & \frac{5}{3}\\
          0 & 1 & -1 & 0}\\
    \xleftrightarrow[R_4 \leftarrow R_4- R_2]{R_3 \leftarrow R_3 + \frac{4}{3}R_2}
    \myvec{1 & 2 & 0 & -2\\
           0 & 1 & 1 & -1\\
           0 & 0 & -\frac{2}{3} & \frac{1}{3}\\
           0 & 0 & -2 & 1}\\
    \xleftrightarrow[R_3 \leftarrow -\frac{3}{2}R_3]{R_1 \leftarrow R_1- 2R_2}
    \myvec{1 & 0 & -2 & 0\\
           0 & 1 & 1 & -1\\
           0 & 0 & 1 & -\frac{1}{2}\\
           0 & 0 & -2 & 1}\\
    \xleftrightarrow{R_4\leftarrow R_4 + 2R_3}
    \myvec{1 & 0 & -2 & 0\\
           0 & 1 & 1 & -1\\
           0 & 0 & 1 & -\frac{1}{2}\\
           0 & 0 & 0 & 0}\\
    \xleftrightarrow[R_2\leftarrow R_2-R_3]{R_1\leftarrow R_1 + 2R_3}
    \myvec{1 & 0 & 0 & -1\\
           0 & 1 & 0 & -\frac{1}{2}\\
           0 & 0 & 1 & -\frac{1}{2}\\
           0 & 0 & 0 & 0}
\end{align}

Thus,
\begin{align}
    x_1=x_4, x_2= \frac{1}{2}x_4, x_3=\frac{1}{2}x_4\\
    \implies \quad\vec{x}= x_4\myvec{1\\ \frac{1}{2}\\ \frac{1}{2}\\1} =\myvec{2\\1\\1\\2}
\end{align} 
by substituting $x_4= 2$.

\hfill\break
%\vspace{5mm} 
Hence, (\ref{eq:solutions/chem/6abalanced}) finally becomes
\begin{align}
    2HNO_{3}+ Ca(OH)_{2}\to Ca(NO_{3})_{2}+ 2H_{2}O
\end{align}

\item Find points on the curve 
$
\vec{x}^T\myvec{\frac{1}{9} & 0 \\ 0 & \frac{1}{16}}\vec{x} = 1
$
%
at which tangents are
\begin{enumerate}
\item  parallel to x-axis
\item  parallel to y-axis.
\end{enumerate}
\solution 
Let the balanced version of (\ref{eq:solutions/chem/6ato balance}) be
\begin{align}
    \label{eq:solutions/chem/6abalanced}x_{1}HNO_{3}+ x_{2}Ca(OH)_{2}\to x_{3}Ca(NO_{3})_{2}+ x_{4}H_{2}O
\end{align}

which results in the following equations:
\begin{align}
    (x_{1}+ 2x_{2}-2x_{4}) H= 0\\
    (x_{1}-2x_{3}) N= 0\\
    (3x_{1}+ 2x_{2}-6x_{3}- x_{4}) O=0\\
    (x_{2}-x_{3}) Ca= 0
\end{align}

which can be expressed as
\begin{align}
    x_{1}+ 2x_{2}+ 0.x_{3} -2x_{4} = 0\\
    x_{1}+ 0.x_{2} -2x_{3} +0.x_{4}= 0\\
    3x_{1}+ 2x_{2}-6x_{3}- x_{4} =0\\
    0.x_{1} +x_{2}-x_{3} +0.x_{4}= 0
\end{align}

resulting in the matrix equation
\begin{align}
    \label{eq:solutions/chem/6a matrix}
    \myvec{1 & 2 & 0 & -2\\
           1 & 0 & -2 & 0\\
           3 & 2 & -6 & -1\\
           0 & 1 & -1 & 0}\vec{x}
           =\vec{0}
\end{align}

where,
\begin{align}
   \vec{x}= \myvec{x_{1}\\x_{2}\\x_{3}\\x_{4}}
\end{align}

(\ref{eq:solutions/chem/6a matrix}) can be reduced as follows:
\begin{align}
    \myvec{1 & 2 & 0 & -2\\
           1 & 0 & -2 & 0\\
           3 & 2 & -6 & -1\\
           0 & 1 & -1 & 0}
    \xleftrightarrow[R_{3}\leftarrow \frac{R_3}{3}-R_{1}]{R_{2}\leftarrow R_2- R_1}
    \myvec{1 & 2 & 0 & -2\\
           0 & -2 & -2 & 2\\
           0 & -\frac{4}{3} & -2 & \frac{5}{3}\\
           0 & 1 & -1 & 0}\\
    \xleftrightarrow{R_2 \leftarrow -\frac{R_2}{2}}
    \myvec{1 & 2 & 0 & -2\\
          0 & 1 & 1 & -1\\
          0 & -\frac{4}{3} & -2 & \frac{5}{3}\\
          0 & 1 & -1 & 0}\\
    \xleftrightarrow[R_4 \leftarrow R_4- R_2]{R_3 \leftarrow R_3 + \frac{4}{3}R_2}
    \myvec{1 & 2 & 0 & -2\\
           0 & 1 & 1 & -1\\
           0 & 0 & -\frac{2}{3} & \frac{1}{3}\\
           0 & 0 & -2 & 1}\\
    \xleftrightarrow[R_3 \leftarrow -\frac{3}{2}R_3]{R_1 \leftarrow R_1- 2R_2}
    \myvec{1 & 0 & -2 & 0\\
           0 & 1 & 1 & -1\\
           0 & 0 & 1 & -\frac{1}{2}\\
           0 & 0 & -2 & 1}\\
    \xleftrightarrow{R_4\leftarrow R_4 + 2R_3}
    \myvec{1 & 0 & -2 & 0\\
           0 & 1 & 1 & -1\\
           0 & 0 & 1 & -\frac{1}{2}\\
           0 & 0 & 0 & 0}\\
    \xleftrightarrow[R_2\leftarrow R_2-R_3]{R_1\leftarrow R_1 + 2R_3}
    \myvec{1 & 0 & 0 & -1\\
           0 & 1 & 0 & -\frac{1}{2}\\
           0 & 0 & 1 & -\frac{1}{2}\\
           0 & 0 & 0 & 0}
\end{align}

Thus,
\begin{align}
    x_1=x_4, x_2= \frac{1}{2}x_4, x_3=\frac{1}{2}x_4\\
    \implies \quad\vec{x}= x_4\myvec{1\\ \frac{1}{2}\\ \frac{1}{2}\\1} =\myvec{2\\1\\1\\2}
\end{align} 
by substituting $x_4= 2$.

\hfill\break
%\vspace{5mm} 
Hence, (\ref{eq:solutions/chem/6abalanced}) finally becomes
\begin{align}
    2HNO_{3}+ Ca(OH)_{2}\to Ca(NO_{3})_{2}+ 2H_{2}O
\end{align}

\item Find the equations of the tangent and normal to the given curves at the indicated points:
$
y = x^2
$
at \myvec{0\\0}.
\item Find the equation of the tangent line to the curve $y = x^2-2x+7$
\begin{enumerate}
%
\item  parallel to the line $\myvec{2 & -1}\vec{x}= -9$ 
\item  perpendicular to the line $\myvec{-15 & 5}\vec{x} = 13$. 
\end{enumerate}
\item  Find the equation of the tangent to the curve,
\begin{align}
y = \sqrt{3x-2}
\label{eq:solutions/1/19/Q}
\end{align}
 which is parallel to the line,
\begin{align}
\myvec{4&2}\vec{x}+5=0
\label{eq:solutions/1/19/P}
\end{align}  
\solution 
Let the balanced version of (\ref{eq:solutions/chem/6ato balance}) be
\begin{align}
    \label{eq:solutions/chem/6abalanced}x_{1}HNO_{3}+ x_{2}Ca(OH)_{2}\to x_{3}Ca(NO_{3})_{2}+ x_{4}H_{2}O
\end{align}

which results in the following equations:
\begin{align}
    (x_{1}+ 2x_{2}-2x_{4}) H= 0\\
    (x_{1}-2x_{3}) N= 0\\
    (3x_{1}+ 2x_{2}-6x_{3}- x_{4}) O=0\\
    (x_{2}-x_{3}) Ca= 0
\end{align}

which can be expressed as
\begin{align}
    x_{1}+ 2x_{2}+ 0.x_{3} -2x_{4} = 0\\
    x_{1}+ 0.x_{2} -2x_{3} +0.x_{4}= 0\\
    3x_{1}+ 2x_{2}-6x_{3}- x_{4} =0\\
    0.x_{1} +x_{2}-x_{3} +0.x_{4}= 0
\end{align}

resulting in the matrix equation
\begin{align}
    \label{eq:solutions/chem/6a matrix}
    \myvec{1 & 2 & 0 & -2\\
           1 & 0 & -2 & 0\\
           3 & 2 & -6 & -1\\
           0 & 1 & -1 & 0}\vec{x}
           =\vec{0}
\end{align}

where,
\begin{align}
   \vec{x}= \myvec{x_{1}\\x_{2}\\x_{3}\\x_{4}}
\end{align}

(\ref{eq:solutions/chem/6a matrix}) can be reduced as follows:
\begin{align}
    \myvec{1 & 2 & 0 & -2\\
           1 & 0 & -2 & 0\\
           3 & 2 & -6 & -1\\
           0 & 1 & -1 & 0}
    \xleftrightarrow[R_{3}\leftarrow \frac{R_3}{3}-R_{1}]{R_{2}\leftarrow R_2- R_1}
    \myvec{1 & 2 & 0 & -2\\
           0 & -2 & -2 & 2\\
           0 & -\frac{4}{3} & -2 & \frac{5}{3}\\
           0 & 1 & -1 & 0}\\
    \xleftrightarrow{R_2 \leftarrow -\frac{R_2}{2}}
    \myvec{1 & 2 & 0 & -2\\
          0 & 1 & 1 & -1\\
          0 & -\frac{4}{3} & -2 & \frac{5}{3}\\
          0 & 1 & -1 & 0}\\
    \xleftrightarrow[R_4 \leftarrow R_4- R_2]{R_3 \leftarrow R_3 + \frac{4}{3}R_2}
    \myvec{1 & 2 & 0 & -2\\
           0 & 1 & 1 & -1\\
           0 & 0 & -\frac{2}{3} & \frac{1}{3}\\
           0 & 0 & -2 & 1}\\
    \xleftrightarrow[R_3 \leftarrow -\frac{3}{2}R_3]{R_1 \leftarrow R_1- 2R_2}
    \myvec{1 & 0 & -2 & 0\\
           0 & 1 & 1 & -1\\
           0 & 0 & 1 & -\frac{1}{2}\\
           0 & 0 & -2 & 1}\\
    \xleftrightarrow{R_4\leftarrow R_4 + 2R_3}
    \myvec{1 & 0 & -2 & 0\\
           0 & 1 & 1 & -1\\
           0 & 0 & 1 & -\frac{1}{2}\\
           0 & 0 & 0 & 0}\\
    \xleftrightarrow[R_2\leftarrow R_2-R_3]{R_1\leftarrow R_1 + 2R_3}
    \myvec{1 & 0 & 0 & -1\\
           0 & 1 & 0 & -\frac{1}{2}\\
           0 & 0 & 1 & -\frac{1}{2}\\
           0 & 0 & 0 & 0}
\end{align}

Thus,
\begin{align}
    x_1=x_4, x_2= \frac{1}{2}x_4, x_3=\frac{1}{2}x_4\\
    \implies \quad\vec{x}= x_4\myvec{1\\ \frac{1}{2}\\ \frac{1}{2}\\1} =\myvec{2\\1\\1\\2}
\end{align} 
by substituting $x_4= 2$.

\hfill\break
%\vspace{5mm} 
Hence, (\ref{eq:solutions/chem/6abalanced}) finally becomes
\begin{align}
    2HNO_{3}+ Ca(OH)_{2}\to Ca(NO_{3})_{2}+ 2H_{2}O
\end{align}

%
\item Find the point at which the line $\myvec{-1 & 1}\vec{x} =  1$ is a tangent to the curve $y^2 = 4x$.
%
\item The line $\myvec{-m & 1}\vec{x} = 1$ is a tangent to the curve $y^2 = 4x$.  Find the value of $m$.
\item  Find the normal at the point \myvec{1\\1} on the curve $2y + x^2 = 3$ 
\item  Find the normal to the curve $x^2=4y$ passing through $\myvec{1\\2}$.
%
\item Find the area of the region bounded by the curve $y^2= x$ and the lines $x = 1, x = 4$ and the x-axis in the first quadrant.
\item  Find the area of the region bounded by $y^2=9x, x=2, x=4$ and the x-axis in the  first quadrant.
%
\item Find the area of the region bounded by $x^2 = 4y, y = 2, y = 4$ and the y-axis in the first quadrant.
\item Find the area of the region bounded by the ellipse 
$
\vec{x}^T\myvec{\frac{1}{16} & 0 \\ 0 & \frac{1}{9}}\vec{x} = 1
$

\item  Find the area of the region bounded by the ellipse 
$
\vec{x}^T\myvec{\frac{1}{4} & 0 \\ 0 & \frac{1}{9}}\vec{x} = 1
$
\item The area between $x=y^2$ and $x=4$ is divided into two equal parts by the line $x=a$, find the value of $a$.
\item  Find the area of the region bounded by the parabola $y = x^2$ and $y = \abs{x}$.
\item  Find the area bounded by the curve $x^2 = 4y$ and the line $\myvec{1 & -1}\vec{x} = -2$.
\item  Find the area of the region bounded by the curve $y^2 = 4x$ and the line $x = 3$.
%
\item Find the area of the region bounded by the curve $y^2 = x$, y-axis and the line $y = 3$.
%
\item Find the area of the region bounded by the two parabolas $y = x^2, y^2=x$.
\item Find the area lying above x-axis and included between the circle $\vec{x}^T\vec{x} -8\myvec{1 & 0}= 0$  and inside of the parabola $y^2 = 4x$.
%
\item AOBA is the part of the ellipse 
$
\vec{x}^T\myvec{9 & 0 \\ 0 & 1}\vec{x} = 36
$
in the first quadrant such that $OA = 2$ and $OB = 6$. Find the area between the arc $AB$ and the chord $AB$.
\item Find the area lying between the curves $y^2 = 4x$ and $y = 2x$.
\item  Find the area of the region bounded by the curves $y = x^2+2, y = x, x = 0$ and $ x = 3.$
%
\item Find the area under $y = x^2, x = 1, x = 2$ and x-axis.
\item Find the area between  $y = x^2$ and $y = x$.
\item Find the area of the region lying in the first quadrant and bounded by $y = 4x^2, x = 0, y = 1$ and $y = 4$.
\item Find the area enclosed by the parabola $4y = 3x^2$ and the line $\myvec{-3 & 2}\vec{x} = 12$.
%
\item Find the area of the smaller region bounded by the ellipse
$
\vec{x}^T\myvec{\frac{1}{9} & 0 \\ 0 & \frac{1}{4}}\vec{x} = 1
$
and the line 
$
\myvec{\frac{1}{a} & \frac{1}{b}}\vec{x} = 1
$
\item Find the area of the region enclosed by the parabola $x^2=y$, the line $\myvec{-1 & 1}\vec{x} = 2$ and the x-axis.
%
\item Find the area bounded by the curves
\begin{align}
\cbrak{\brak{x,y} : y > x^2, y = \abs{x}}
\end{align}
%
\item Find the area of the region
\begin{align}
\cbrak{\brak{x,y} : y^2 \le 4x, 4\vec{x}^T\vec{x} = 9}
\end{align}
%
\item Find the area of the circle $\vec{x}^T\vec{x} = 16$ exterior to the parabola $y^2 = 6$.
\end{enumerate}
 

\end{enumerate}

\section{Exercises}
\renewcommand{\theequation}{\theenumi}
\begin{enumerate}[label=\thesection.\arabic*.,ref=\thesection.\theenumi]
\numberwithin{equation}{enumi}
%\renewcommand{\theequation}{\theenumi}
%\begin{enumerate}[label=\arabic*.,ref=\thesubsection.\theenumi]
%\numberwithin{equation}{enumi}

\item Find the equation of the circle with radius 5 whose centre lies on x-axis and passes through the point \myvec{2\\3}.
\item Find the equation of the circle passing through \myvec{0\\0} and making intercepts a and b on the coordinate axes.
\item Find the equation of a circle with centre \myvec{2\\2} and passes through the point \myvec{4\\5}. 
\item Find the locus of all the unit vectors in the xy-plane.
%
\item Find the points on the curve $\vec{x}^T\vec{x}-2\myvec{1 & 0}\vec{x} -3 =0$  at which the tangents are parallel to the x-axis.
%
\item  Find the area of the region in the first quadrant enclosed by x-axis, line $\myvec{1 & -\sqrt{3}}\vec{x} =0$ and the circle $\vec{x}^T\vec{x}=4$.
%
\item Find the area lying in the first quadrant and bounded by the circle $\vec{x}^T\vec{x}=4$ and the lines $x = 0$ and $x = 2$.
%
\item Find the area of the circle $4\vec{x}^T\vec{x}=9$.
\item  Find the area bounded by curves $\norm{\vec{x}-\myvec{1\\0}} = 1$ and $\norm{\vec{x}}=1$
\item Find the smaller area enclosed by the circle $\vec{x}^T\vec{x}=4$ and the line $\myvec{1 & 1}\vec{x} = 2$.
%\item The sum of the perimeter of a circle and square is $k$, where $k$ is some constant. Prove that the sum of their areas is least when the side of square is double the radius of the circle.
%\item A window is in the form of a rectangle surmounted by a semicircular opening. The total perimeter of the window is 10 m. Find the dimensions of the window to admit maximum light through the whole opening.
%
\item If 
$
\brak{x-a}^2+\brak{y-b}^2 = c^2,
$
for some $c > 0$, prove that 
\begin{align}
\frac{\brak{1+y_2}^\frac{3}{2}}{y_2}
\end{align}
%
is a constant independent of $a$ and $b$.
%
\item Form the differential equation of the family of circles touching the y-axis at origin.
\item Form the differential equation of the family of circles having centre on y-axis and radius 3 units.
\item Form the differntial equation of the fmaily of circles touching the x-axis at the origin.
%
\item Form the differential equation of the family of circles in the second quadrant and touching the coordinate axes.
\item Factorise $6x^2+ 17x + 5$.
\item Factorise $y^2 – 5y + 6$.
\item Find the zeroes of the quadratic polynomial $x^2+7x+10$ and verify the relationship between the zeroes and the coefficients.
\item Find the zeroes of the polynomial $x^2-3$ and verify the relationship between the zeroes and the coefficients.
\item Find a quadratic polynomial, the sum and product of whose zeroes are – 3 and 2, respectively.
%
\item Find the roots of the equation $5x^2  – 6x – 2 = 0 $.
\item Find the roots of $4x^2 + 3x + 5 = 0 $.
\item Find the roots of the following quadratic equations, if they exist.
\begin{enumerate}
\item 	$3x^2-5x+2 = 0$
\item 	$x^2+4x+5 = 0$
\item 	$2x^2-2\sqrt{2}x+1 = 0$
\end{enumerate}
%
\item Find the discriminant of the quadratic equation $2x^2-4x+3 = 0$
hence find the nature of its roots.
\item Find the discriminant of the quadratic equation $3x^2-2x+\frac{1}{3} = 0$
hence find the nature of its roots.
\item Solve $x^2+ 2 = 0 $.
\item Solve $x^2+ x+1 = 0 $.
\item Solve $\sqrt{5}x^2+ x+\sqrt{5} = 0 $.
%
\item Find the coordinates of the focus, axis, the equation of the directrix and latus rectum of the parabola $y^2 = 8x$.
%
\item Find the equation of the parabola with focus \myvec{2\\0} and directrix $\myvec{1 & 0}\vec{x} = -2$.\item Find the equation of the parabola with vertex at \myvec{0\\ 0} and focus at \myvec{0\\ 2}.
\item Find the equation of the parabola which is symmetric about the y-axis, and passes through the point \myvec{2\\–3}.
\item Find the coordinates of the foci, the vertices, the length of major axis, the minor axis, the eccentricity and the latus rectum of the ellipse 
%
\begin{align}
\vec{x}^T\myvec{\frac{1}{25} & 0 \\ 0 & \frac{1}{9}}\vec{x} = 1
\end{align}
%
\item Find the coordinates of the foci, the vertices, the lengths of major and minor axes and the eccentricity of the ellipse 
%
\begin{align}
\vec{x}^T\myvec{9 & 0 \\ 0 & 4}\vec{x} = 36
\end{align}
%
\item Find the equation of the ellipse whose vertices are $\myvec{\pm 13\\ 0}$ and foci are $\myvec{\pm 5\\ 0}$.
%
\item Find the equation of the ellipse, whose length of the major axis is 20 and foci are $\myvec{0\\ \pm 5}$
%
\item Find the equation of the ellipse, with major axis along the x-axis and passing through the points \myvec{4\\ 3} and \myvec{– 1\\4}.
%
\item Find the coordinates of the foci and the vertices, the eccentricity,the length of the latus rectum of the hyperbolas
\begin{enumerate}
\item 
$
\vec{x}^T\myvec{\frac{1}{9} & 0 \\ 0 & -\frac{1}{16}}\vec{x} = 1
$
\item 
$
\vec{x}^T\myvec{1 & 0 \\ 0 & -16}\vec{x} = 16
$
\end{enumerate}
\item Find the equation of the hyperbola with  vertices $\myvec{0 \\ \pm \frac{\sqrt{11}}{2}}$, foci $\myvec{ 0\\ \pm 3}$
\item Find the equation of the hyperbola with   foci $\myvec{ 0\\ \pm 12}$ and length of latus rectum 36.
%
\item Find the equation of all lines having slope 2 and being tangent to the curve
\begin{align}
y + \frac{2}{x-3} = 0
\end{align}
%
\item Find the point at which the tangent to the curve $y = \sqrt{4x-3}-1$ has its 
slope $\frac{2}{3}$.
%
\item Find the roots of the following equations:
\begin{enumerate}
\item  $x + \frac{1}{x} = 3, x \ne =0 $
\item  $ \frac{1}{x} + \frac{1}{x-2}=3, x\ne =0, 2 $
\end{enumerate}
%
\item Find points on the curve 
$
\vec{x}^T\myvec{\frac{1}{4} & 0 \\ 0 & \frac{1}{25}}\vec{x} = 1
$
at which the tangents are 
\begin{enumerate}
\item parallel to x-axis
\item parallel to y-axis
\end{enumerate}
 \item Find the equation of the normal to the curve $x^2= 4y$
which passes through the point \myvec{1\\ 2}.
%
\item Find the area enclosed by the ellipse
$
\vec{x}^T\myvec{\frac{1}{a^2} & 0 \\ 0 & \frac{1}{b^2}}\vec{x} = 1
$
%
\item Find the area of the region bounded by the curve $y = x^2$
and the line $y = 4$.
%
\item Find the area bounded by the ellipse
$
\vec{x}^T\myvec{\frac{1}{a^2} & 0 \\ 0 & \frac{1}{b^2}}\vec{x} = 1
$
and $x = ae$, where, $b^2 = a^2 (1 – e^2 )$ and $e < 1$.
%
\item Prove that the curves $y^2 = 4x$ and $x^2 = 4y$ divide the area of the square bounded by $x = 0, x = 4, y =4$ and $y = 0$ into three equal parts.
%
\item Find the area of the region
\begin{align}
\cbrak{\brak{x,y} = 0\le y\le x^2+1, 0\le y \le x+1, 0 \le x \le 2}
\end{align}
\item Find the intervals in which the function 
\begin{align}
f(x)  = x^2-4x+6
\end{align}
%
is 
\begin{enumerate}
\item increasing
\item decreasing.
\end{enumerate}
%
%\item Find the shortest distance of the point $\myvec{0\\c}$ from the parabola $y = x^2$, where $\frac{1}{2} \le c \le 5$.
%
%\item An apache helicopter of enemy is flying along the curve given by $y = x^2+7$.  A soldier, placed at \myvec{3\\7}, wants to shoot down the helicopter when it is nearest to him.  Find  the nearest distance.
%
\item Examine whether the function $f$ given by $f(x) = x^2$ is continuous at $x = 0$.
%
\item Discuss the continuity of the function $f$ defined by 
%
\begin{align}
f(x)  = 
\begin{cases}
x & x \ge 0
\\
x^2 & x < 0
\end{cases}
\end{align}
%
\item Verify Rolle's theorem for the function $y = x^2+2, a = -2$ and $b = 2$.
\item Verify Mean Value Theorem for the function $f(x) = x^2$ in the interval $\sbrak{2,-4}$.
%\end{align}
\item Find the derivative of $f(x) = x^2$.
\item Find the derivative of $ x^2 - 2$ at $x = 10$.
\item Find the derivative of $ \brak{x-1} \brak{x-2}$.
%
%
\item Find 
\begin{align}
\int_{0}^{2} \brak{x^2+1}\,dx
\end{align}
%
as a limit of a sum.
\item Evaluate the following integral:
%
\begin{align}
\int_{2}^{3}x^2 \,dx
\end{align}
%
\item Form the differntial equation representing the family of ellipses having foci on x-axis and cenre at the origin.
%
\item Form the differntial equation representing the family of parabolas having vertex at origin and axis along positive direction of x-axis.
\item Form a differntial equation representing the following family of curves
%
\begin{align}
y^2 = a\brak{b^2-x^2}
\end{align}
%
\item  A cricket ball is thrown at a speed of 28 $m s^{-1}$
in a direction 30$\degree$ above
the horizontal. Calculate 
\begin{enumerate}
\item  the maximum height, 
\item  the time taken by the ball to return to the same level, and 
\item  the distance from the thrower to the point where the ball returns to the same level.
\end{enumerate}
\item Find the roots of the equation  $2x^2– 5x + 3 = 0$ .
\item Find the value of  the following polynomial at the indicated value of variables 
\begin{align}
p(x) = 5x^2– 3x + 7 \text{  at } x = 1.
\end{align}
%\end{enumerate}
 
%\renewcommand{\theequation}{\theenumi}
%\begin{enumerate}[label=\arabic*.,ref=\thesubsection.\theenumi]
%\numberwithin{equation}{enumi}

\item Find the nature of the roots of the following quadratic equations. If the real roots exist, find them:
\begin{enumerate}
\item 	$2x^2-3x+5 = 0$
\item 	$2x^2-6x+3 = 0$
\item 	$3x^2-4\sqrt{3}x+4 = 0$
\end{enumerate}
\item Solve each of the following equations
%
\begin{enumerate}
\item 	$x^2+3 = 0$
\item 	$2x^2+x+1 = 0$
\item 	$x^2+3x+9 = 0$
\item 	$-x^2+x-2 = 0$
\item 	$x^2+3x+5 = 0$
\item 	$x^2-3x+2 = 0$
\item 	$\sqrt{2}x^2+x+\sqrt{2} = 0$
\item 	$\sqrt{3}x^2-\sqrt{2}x+3\sqrt{3} = 0$
\item 	$x^2+x+\frac{1}{\sqrt{2}} = 0$
\item 	$x^2+\frac{x}{\sqrt{2}}+1 = 0$
\end{enumerate}
%
\item In each of the following exercises, find the coordinates of the focus, axis of the parabola, the equation of the directrix and the length of the latus rectum
\begin{enumerate}
\item $y^2 = 12x$
\item $x^2 = 6y$
\item $y^2 = -8x$
\item $x^2 = -16y$
\item $y^2 = 10x$
\item $x^2 = -9y$
\end{enumerate}
%
\item In each of the following exercises, find the equation of the parabola that satisfies the following conditions:
\begin{enumerate}
\item Focus \myvec{6\\0}, directrix $\myvec{1 & 0} = -6$.
\item Focus \myvec{0\\-3}, directrix $\myvec{0 & 1} = 3$.
\item Focus \myvec{3\\0}, vertex \myvec{0 & 0}.
\item Focus \myvec{-2\\0}, vertex \myvec{0 & 0}.
\item vertex \myvec{0 & 0} passing through \myvec{2\\2} and axis is along the x-axis
\item vertex \myvec{0 & 0} passing through \myvec{5\\2} and symmetric with respect to the y-axis.
\end{enumerate}
%
\item In each of the exercises, find the coordinates of the foci, the vertices, the length of major axis, the minor axis, the eccentricity and the length of the latus rectum of the ellipse.
%
\begin{enumerate}
\item 
$
\vec{x}^T\myvec{\frac{1}{36} & 0 \\ 0 & \frac{1}{16}}\vec{x} = 1
$
\item 
$
\vec{x}^T\myvec{\frac{1}{4} & 0 \\ 0 & \frac{1}{25}}\vec{x} = 1
$
\item 
$
\vec{x}^T\myvec{\frac{1}{16} & 0 \\ 0 & \frac{1}{9}}\vec{x} = 1
$
\item 
$
\vec{x}^T\myvec{\frac{1}{25} & 0 \\ 0 & \frac{1}{100}}\vec{x} = 1
$
\item 
$
\vec{x}^T\myvec{\frac{1}{49} & 0 \\ 0 & \frac{1}{36}}\vec{x} = 1
$
\item 
$
\vec{x}^T\myvec{\frac{1}{100} & 0 \\ 0 & \frac{1}{16}}\vec{x} = 1
$
%
\item 
$
\vec{x}^T\myvec{36 & 0 \\ 0 & 4}\vec{x} = 144
$
%
\item 
$
\vec{x}^T\myvec{16 & 0 \\ 0 & 1}\vec{x} = 16
$
%
\item 
$
\vec{x}^T\myvec{4 & 0 \\ 0 & 9}\vec{x} = 36
$
%
\end{enumerate}
%
\item In each of the following, find the equation for the ellipse that satisfies the given conditions:
%
\begin{enumerate}
\item Vertices $\myvec{\pm 5\\ 0}$, foci $\myvec{\pm 4\\ 0}$ \item  Vertices $\myvec{0\\ \pm 13}$, foci $\myvec{0\\ \pm 5}$ \item  Vertices $\myvec{\pm 6\\ 0}$, foci $\myvec{\pm 4\\ 0}$ \item  Ends of major axis $\myvec{\pm 3\\ 0}$, ends of minor axis $\myvec{0\\ \pm 2}$
\item  Ends of major axis $\myvec{0\\ \pm 5 }$, ends of minor axis $\myvec{\pm 1\\ 0}$ \item  Length of major axis 26, foci $\myvec{\pm 5\\ 0}$ \item  Length of minor axis 16, foci $\myvec{0\\ \pm 6}$. \item  Foci $\myvec{\pm 3\\ 0}$, a = 4 \item  b = 3, c = 4, centre at the origin; foci on the x axis. \item  Centre at $\myvec{0\\0}$, major axis on the y-axis and passes through the points $\myvec{3\\ 2}$ and $\myvec{1\\6}$.
\item  Major axis on the x-axis and passes through the points $\myvec{4\\3}$ and $\myvec{6\\2}$.
\end{enumerate}
%
%
\item In each of the exercises, find the coordinates of the foci, the vertices, the length of major axis, the minor axis, the eccentricity and the length of the latus rectum of the ellipse.
%
\begin{enumerate}
\item 
$
\vec{x}^T\myvec{\frac{1}{16} & 0 \\ 0 & -\frac{1}{9}}\vec{x} = 1
$
\item 
$
\vec{x}^T\myvec{\frac{1}{9} & 0 \\ 0 & -\frac{1}{27}}\vec{x} = 1
$
%
\item 
$
\vec{x}^T\myvec{9 & 0 \\ 0 & -4}\vec{x} = 36
$
%
\item 
$
\vec{x}^T\myvec{16 & 0 \\ 0 & -9}\vec{x} = 576
$
%
\item 
$
\vec{x}^T\myvec{5 & 0 \\ 0 & -9}\vec{x} = 36
$
\item 
$
\vec{x}^T\myvec{49 & 0 \\ 0 & -16}\vec{x} = 784
$
%
%
\end{enumerate}
\item In each of the following, find the equation for the ellipse that satisfies the given conditions:
%
\begin{enumerate}
\item Vertices $\myvec{\pm 2\\ 0}$, foci $\myvec{\pm 3\\ 0}$ 
\item  Vertices $\myvec{0\\ \pm 5}$, foci $\myvec{0\\ \pm 8}$ 
\item  Vertices $\myvec{ 0\\ \pm 3}$, foci $\myvec{0 \\\pm 5}$ 
\item  Transverse axis length 8, foci $\myvec{\pm 5\\ 0}$.
\item  Conjugate axis length 24, foci $\myvec{0 \\ \pm 13}$.
\item  Latus rectum  length 8, foci $\myvec{ \pm 3\sqrt{5} \\ 0}$.
\item  Latus rectum  lenght 12, foci $\myvec{ \pm 4 \\ 0}$.
\item  Ends of major axis $\myvec{0\\ \pm 5 }$, ends of minor axis $\myvec{\pm 1\\ 0}$ 
\item  Vertices $\myvec{  \pm 7 \\ 0}$, $e = \frac{4}{3}$
\item  Foci $\myvec{ 0\\  \pm \sqrt{10}}$, passing through $\myvec{2\\3}$.

\end{enumerate}
%
\item Find the slope of the tangent to the curve $y = \frac{x-1}{x-2}, x\ne 2$ at $x = 10$.
\item Find a point on the curve $y = (x – 2)^2$ at which the tangent is parallel to the chord joining the points \myvec{2\\ 0} and \myvec{4\\ 4}.
\item Find the equation of all lines having slope – 1 that are tangents to the curve $\frac{1}
{x -1}, x \ne 1$
\item Find the equation of all lines having slope 2 which are tangents to the curve $\frac{1}
{x - 3} , x \ne 3$.
%
\item Find points on the curve 
$
\vec{x}^T\myvec{\frac{1}{9} & 0 \\ 0 & \frac{1}{16}}\vec{x} = 1
$
%
at which tangents are
\begin{enumerate}
\item  parallel to x-axis
\item  parallel to y-axis.
\end{enumerate}
\item Find the equations of the tangent and normal to the given curves at the indicated points:
$
y = x^2
$
at \myvec{0\\0}.
\item Find the equation of the tangent line to the curve $y = x^2-2x+7$
\begin{enumerate}
%
\item  parallel to the line $\myvec{2 & -1}\vec{x}= -9$ 
\item  perpendicular to the line $\myvec{-15 & 5}\vec{x} = 13$. 
\end{enumerate}
\item Find the equation of the tangent to the curve $y = \sqrt{3x - 2}$ which is parallel to the line $\myvec{4 & 2}\vec{x}+ 5 =0$ .
\item Find the point at which the line $\myvec{-1 & 1}\vec{x} =  1$ is a tangent to the curve $y^2 = 4x$.
%
\item The line $\myvec{-m & 1}\vec{x} = 1$ is a tangent to the curve $y^2 = 4x$.  Find the value of $m$.
\item  Find the normal at the point \myvec{1\\1} on the curve $2y + x^2 = 3$ 
\item  Find the normal to the curve $x^2=4y$ passing through $\myvec{1\\2}$.
%
\item Find the area of the region bounded by the curve $y^2= x$ and the lines $x = 1, x = 4$ and the x-axis in the first quadrant.
\item  Find the area of the region bounded by $y^2=9x, x=2, x=4$ and the x-axis in the  first quadrant.
%
\item Find the area of the region bounded by $x^2 = 4y, y = 2, y = 4$ and the y-axis in the first quadrant.
\item Find the area of the region bounded by the ellipse 
$
\vec{x}^T\myvec{\frac{1}{16} & 0 \\ 0 & \frac{1}{9}}\vec{x} = 1
$

\item  Find the area of the region bounded by the ellipse 
$
\vec{x}^T\myvec{\frac{1}{4} & 0 \\ 0 & \frac{1}{9}}\vec{x} = 1
$
\item The area between $x=y^2$ and $x=4$ is divided into two equal parts by the line $x=a$, find the value of $a$.
\item  Find the area of the region bounded by the parabola $y = x^2$ and $y = \abs{x}$.
\item  Find the area bounded by the curve $x^2 = 4y$ and the line $\myvec{1 & -1}\vec{x} = -2$.
\item  Find the area of the region bounded by the curve $y^2 = 4x$ and the line $x = 3$.
%
\item Find the area of the region bounded by the curve $y^2 = x$, y-axis and the line $y = 3$.
%
\item Find the area of the region bounded by the two parabolas $y = x^2, y^2=x$.
\item Find the area lying above x-axis and included between the circle $\vec{x}^T\vec{x} -8\myvec{1 & 0}= 0$  and inside of the parabola $y^2 = 4x$.
%
\item AOBA is the part of the ellipse 
$
\vec{x}^T\myvec{9 & 0 \\ 0 & 1}\vec{x} = 36
$
in the first quadrant such that $OA = 2$ and $OB = 6$. Find the area between the arc $AB$ and the chord $AB$.
\item Find the area lying between the curves $y^2 = 4x$ and $y = 2x$.
\item  Find the area of the region bounded by the curves $y = x^2+2, y = x, x = 0$ and $ x = 3.$
%
\item Find the area under $y = x^2, x = 1, x = 2$ and x-axis.
\item Find the area between  $y = x^2$ and $y = x$.
\item Find the area of the region lying in the first quadrant and bounded by $y = 4x^2, x = 0, y = 1$ and $y = 4$.
\item Find the area enclosed by the parabola $4y = 3x^2$ and the line $\myvec{-3 & 2}\vec{x} = 12$.
%
\item Find the area of the smaller region bounded by the ellipse
$
\vec{x}^T\myvec{\frac{1}{9} & 0 \\ 0 & \frac{1}{4}}\vec{x} = 1
$
and the line 
$
\myvec{\frac{1}{a} & \frac{1}{b}}\vec{x} = 1
$
\item Find the area of the region enclosed by the parabola $x^2=y$, the line $\myvec{-1 & 1}\vec{x} = 2$ and the x-axis.
%
\item Find the area bounded by the curves
\begin{align}
\cbrak{\brak{x,y} : y > x^2, y = \abs{x}}
\end{align}
%
\item Find the area of the region
\begin{align}
\cbrak{\brak{x,y} : y^2 \le 4x, 4\vec{x}^T\vec{x} = 9}
\end{align}
%
\item Find the area of the circle $\vec{x}^T\vec{x} = 16$ exterior to the parabola $y^2 = 6$.
%
\item Find the intervals in which the function given by 
\begin{align}
f(x)  = 2x^2-3x
\end{align}
%
is 
\begin{enumerate}
\item increasing
\item decreasing.
\end{enumerate}
%
\item Find the intervals in which the following functions are strictly increasing or decreasing
%
\begin{enumerate}
\item $x^2+2x-5$
\item $10-6x-2x^2$
\item $6-9x-x^2$
\end{enumerate}
%
\item Prove that the function $f$ given by $f(x) = x^2-x+1$ is neither strictly increasing nor decreasing on $\brak{1,-1}$.
%%
%\item Find the maximum and minimum values, if any, of the following functions given by 
%%
%\begin{enumerate}
%\item $f(x) = \brak{2x-1}^2+3$
%\item $f(x) = 9x^2+12x+2$
%\item $f(x) = -\brak{x-1}^2+10$
%\item $f(x) = x^2$.
%\end{enumerate}
%\item Find the absolute maximum and absoute minimum value of the following functions in the given intervals
%%
%\begin{enumerate}
%\item $f(x) = 4x - \frac{1}{2}x^2, x \in \brak{-2,\frac{9}{2}}$
%\item $f(x) = \brak{x-1}^2 + 3,  x \in \brak{-3,1}$
%\end{enumerate}
%%
%\item Find the maximum profit that a company can make, if the profit function is given by
%\begin{align}
%p(x) = 41-72x - 18x^2
%\end{align}
%%
%\item Find the point on the curve $x^2=2y$ which is nearest to the point $\myvec{0\\5}$.
%\item Find the maximum area of an isosceles triangle inscribed in the ellipse 
%%
%\begin{align}
%\vec{x}^T\myvec{a^2 & 0 \\ 0 & b^2}\vec{x} = a^2b^2
%\end{align}
%%
%with its vertex at one end of the major axis.
%
\item Examine the continuity of the function $f(x) = 2x^2– 1$ at $x = 3$.
\item Find all points of discontinuity of $f$, where $f$is defined by
\begin{align}
f(x)=
\begin{cases}
x+1, & x \ge 1,
\\
x^2+1, & x < 1,
\end{cases}
\end{align}
%
\item For what value of $\lambda $is the function defined by 	
%
\begin{align}
f(x)=
\begin{cases}
\lambda\brak{x^2-2x}, & x \le 0,
\\
4x+1, & x > 0
\end{cases}
\end{align}
%
continuous at $x = 0$? What about continuity at $x = 1$?\item For what value of $k$ is the following function 
%
continuous at the given point.
\begin{align}
f(x)=
\begin{cases}
kx^2, & x \le 2,
\\
3, & x > 2,
\end{cases}
\quad x =2
\end{align}
%
\item Find $\frac{dy}{dx}$ in the following
\begin{align}
x^2 +xy + y^2 = 100
\end{align}
%
\item Verify Rolle's theorem for the function 
\label{prob:conics_ex_eq_rolle}.
$
f(x) = x^2+2x-8, x \in \sbrak{-4,2}
$
\item Examine if Rolle's theorem is applicable to the following function
$
f(x) = x^2-1, x \in \sbrak{1,2}.
$
Can you say some thing about the converse of Rolle's theorem from this example?
\item  Examine the applicability of the mean value theorem for the  function in Problem \ref{prob:conics_ex_eq_rolle}.
%
\item Find $\lim_{x\to 1} \pi r^2$.
%
\item Find $\lim_{x\to 0} f(x)$ where
\begin{align}
f(x) = 
\begin{cases}
x^2-1 & x \le 1
\\
-x^2-1, & x > 1
\end{cases}
\end{align}
%
\item For some constans $a$ and $b$, find the derivative of
%
\begin{align}
\brak{x-a}\brak{x-b}
\end{align}
%
%
\item Integrate the following as limit of sums:
\begin{enumerate}[label = (\roman*)]
\item $\int_{2}^{3}x^2\, dx$
\item $\int_{1}^{4}\brak{x^2-x}\, dx$
\end{enumerate}
\item Form the differential equation of the family of parabolas having vertex at origin and axis along positive y-axis.
\item  Form the differential equation of the family of ellipses having foci on y-axis and centre at origin.
\item  Form the differential equation of the family of hyperbolas having foci on x-axis and centre at origin.
\item The ceiling of a long hall is 25 m high. What is the maximum horizontal distance that a ball thrown with a speed of 40 $m s^{-1}$
can go without hitting the ceiling of the hall ?
\item  A cricketer can throw a ball to a maximum horizontal distance of 100 m. How much high above the ground can the cricketer throw the same ball ?
\item  Find the normal to the curve $x^2=4y$ passing through $\myvec{1\\2}$.
%
\item Find the area of the region bounded by the curve $y^2= x$ and the lines $x = 1, x = 4$ and the x-axis in the first quadrant.
\item  Find the area of the region bounded by $y^2=9x, x=2, x=4$ and the x-axis in the  first quadrant.
%
\item Find the area of the region bounded by $x^2 = 4y, y = 2, y = 4$ and the y-axis in the first quadrant.
\item Find the area of the region bounded by the ellipse 
$
\vec{x}^T\myvec{\frac{1}{16} & 0 \\ 0 & \frac{1}{9}}\vec{x} = 1
$

\item  Find the area of the region bounded by the ellipse 
$
\vec{x}^T\myvec{\frac{1}{4} & 0 \\ 0 & \frac{1}{9}}\vec{x} = 1
$
\item The area between $x=y^2$ and $x=4$ is divided into two equal parts by the line $x=a$, find the value of $a$.
\item  Find the area of the region bounded by the parabola $y = x^2$ and $y = \abs{x}$.
\item  Find the area bounded by the curve $x^2 = 4y$ and the line $\myvec{1 & -1}\vec{x} = -2$.
\item  Find the area of the region bounded by the curve $y^2 = 4x$ and the line $x = 3$.
%
\item Find the area of the region bounded by the curve $y^2 = x$, y-axis and the line $y = 3$.
%
\item Find the area of the region bounded by the two parabolas $y = x^2, y^2=x$.
\item Find the area lying above x-axis and included between the circle $\vec{x}^T\vec{x} -8\myvec{1 & 0}= 0$  and inside of the parabola $y^2 = 4x$.
%
\item AOBA is the part of the ellipse 
$
\vec{x}^T\myvec{9 & 0 \\ 0 & 1}\vec{x} = 36
$
in the first quadrant such that $OA = 2$ and $OB = 6$. Find the area between the arc $AB$ and the chord $AB$.
\item Find the area lying between the curves $y^2 = 4x$ and $y = 2x$.
\item  Find the area of the region bounded by the curves $y = x^2+2, y = x, x = 0$ and $ x = 3.$
%
\item Find the area under $y = x^2, x = 1, x = 2$ and x-axis.
\item Find the area between  $y = x^2$ and $y = x$.
\item Find the area of the region lying in the first quadrant and bounded by $y = 4x^2, x = 0, y = 1$ and $y = 4$.
\item Find the area enclosed by the parabola $4y = 3x^2$ and the line $\myvec{-3 & 2}\vec{x} = 12$.
%
\item Find the area of the smaller region bounded by the ellipse
$
\vec{x}^T\myvec{\frac{1}{9} & 0 \\ 0 & \frac{1}{4}}\vec{x} = 1
$
and the line 
$
\myvec{\frac{1}{a} & \frac{1}{b}}\vec{x} = 1
$
\item Find the area of the region enclosed by the parabola $x^2=y$, the line $\myvec{-1 & 1}\vec{x} = 2$ and the x-axis.
%
\item Find the area bounded by the curves
\begin{align}
\cbrak{\brak{x,y} : y > x^2, y = \abs{x}}
\end{align}
%
\item Find the area of the region
\begin{align}
\cbrak{\brak{x,y} : y^2 \le 4x, 4\vec{x}^T\vec{x} = 9}
\end{align}
%
\item Find the area of the circle $\vec{x}^T\vec{x} = 16$ exterior to the parabola $y^2 = 6$.
\item Find the equation of the circle passing through \myvec{0\\0} and making intercepts a and b on the coordinate axes.
\item Find the locus of all the unit vectors in the xy-plane.
\item Find the area lying in the first quadrant and bounded by the circle $\vec{x}^T\vec{x}=4$ and the lines $x = 0$ and $x = 2$.
%
\item Find the area of the circle $4\vec{x}^T\vec{x}=9$.
\item  Find the equation of the tangent to the curve,
\begin{align}
y = \sqrt{3x-2}
\label{eq:solutions/conics/1/19/Q}
\end{align}
 which is parallel to the line,
\begin{align}
\myvec{4&2}\vec{x}+5=0
\label{eq:solutions/conics/1/19/P}
\end{align}  
\item Find the equations of the tangent and normal to the given curves at the indicated points:
$
y = x^2
$
at \myvec{0\\0}.
\item The line 
\begin{align}
\myvec{-m & 1}\vec{x} = 1 \label{eq:solutions/conics/3/2/21/eq: 1}
\end{align}
is a tangent to the curve $y^2 = 4x$. Find the value of m.
%\end{enumerate}
 

\end{enumerate}

%
%
\end{document}


