
We have to prove that $P$ is equidistant from $A$ and $B$ i.e. length of lines $AP$ and $BP$ are equal. 

Let $DP$ be the perpendicular bisector of line $AB$. So,
\begin{align}
    \vec{A}-\vec{D} = \vec{D} - \vec{B} \\
    \norm{\vec{A}-\vec{D}} = \norm{\vec{D} - \vec{B}} = k \\
    \norm{\vec{D}-\vec{P}} = l
\end{align}

%\renewcommand{\thefigure}{1}
\begin{figure}[!ht]
\centering
\resizebox{\columnwidth}{!}{
  	\begin{tikzpicture}
  	\coordinate (A) at (0,6);
  	\coordinate (B) at (-3,0);
  	\coordinate (C) at (3,0);
  	\coordinate (E) at (1.5,3);
  	\coordinate (F) at (-1.5,3);
  	\draw (A)node[above]{$A$}--(B)node[below]{$B$}--(C)node[below]{$C$}--cycle;
  	\draw (B)node[below]{$B$}--(E)node[above]{$E$}--(A)node[above]{$A$}--cycle;
  	\draw(C)node[below]{$C$}--(F)node[above]{$F$}--(A)node[above]{$A$}--cycle;
  	\tkzMarkRightAngle(B,E,A)
  	\tkzMarkRightAngle(C,F,A)
  	\end{tikzpicture}
}
\caption{$PD \perp AB$ by Latex-Tikz}
\label{fig:solutions/1/20/perp_bisector}
\end{figure}


Finding the length of line $AP$, 
\begin{multline}
    (\vec{A}-\vec{P})^T(\vec{A}-\vec{P}) = (\vec{A}- \vec{D}+\vec{D}-\vec{P})^T(\vec{A}-\vec{D}+\vec{D}-\vec{P}) \\
    =[(\vec{A}-\vec{D})^T+(\vec{D}-\vec{P})^T][(\vec{A}-\vec{D})+(\vec{D}-\vec{P})] \\
    =(\vec{A}-\vec{D})^T(\vec{A}-\vec{D}) + (\vec{A}-\vec{D})^T(\vec{D}-\vec{P}) + \\ (\vec{D}-\vec{P})^T(\vec{A}-\vec{D}) + (\vec{D}-\vec{P})^T(\vec{D}-\vec{P})
\end{multline}

Since, line $AB$ is perpendicular to line $DP$ the inner product is zero. 
\begin{align}
    (\vec{A}-\vec{D})^T(\vec{D}-\vec{P}) = (\vec{D}-\vec{P})^T(\vec{A}-\vec{D}) = 0
\end{align}
Thus, 
\begin{multline} \label{eq:solutions/1/20/len_ap}
    (\vec{A}-\vec{P})^T(\vec{A}-\vec{P}) = (\vec{A}-\vec{D})^T(\vec{A}-\vec{D}) + (\vec{D}-\vec{P})^T(\vec{D}-\vec{P}) \\
  \implies\norm{\vec{A}-\vec{P}}^2 = \norm{\vec{A}-\vec{D}}^2 + \norm{\vec{D}-\vec{P}}^2 \\
  \implies\norm{\vec{A}-\vec{P}} = \sqrt{k^2+l^2}
\end{multline}

Next finding the length of line $BP$,
\begin{multline}
    (\vec{B}-\vec{P})^T(\vec{B}-\vec{P}) = (\vec{B}- \vec{D}+\vec{D}-\vec{P})^T(\vec{B}-\vec{D}+\vec{D}-\vec{P}) \\
    =[(\vec{B}-\vec{D})^T+(\vec{D}-\vec{P})^T][(\vec{B}-\vec{D})+(\vec{D}-\vec{P})] \\
    =(\vec{B}-\vec{D})^T(\vec{B}-\vec{D}) + (\vec{B}-\vec{D})^T(\vec{D}-\vec{P}) + \\ (\vec{D}-\vec{P})^T(\vec{B}-\vec{D}) + (\vec{D}-\vec{P})^T(\vec{D}-\vec{P})
\end{multline}
Again since the inner product of lines $AB$ and $DP$ is zero, 
\begin{multline} \label{eq:solutions/1/20/len_bp}
  \implies(\vec{B}-\vec{P})^T(\vec{B}-\vec{P}) = (\vec{B}-\vec{D})^T(\vec{B}-\vec{D}) + (\vec{D}-\vec{P})^T(\vec{D}-\vec{P}) \\
  \implies\norm{\vec{B}-\vec{P}}^2 = \norm{\vec{B}-\vec{D}}^2 + \norm{\vec{D}-\vec{P}}^2 \\
  \implies\norm{\vec{B}-\vec{P}} = \sqrt{k^2+l^2}
\end{multline}

From equations \eqref{eq:solutions/1/20/len_ap} and \eqref{eq:solutions/1/20/len_bp} we get, 
\begin{align}
    \norm{\vec{A}-\vec{P}} = \norm{\vec{B}-\vec{P}}
\end{align}
Lengths of line $AP$ and $BP$ are equal. Hence, $P$ is equidistant from $A$ and $B$. 

