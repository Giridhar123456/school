Consider an equilateral $\triangle{ABC}$ as shown in figure:     \ref{eq:solutions/1/12/fig:es1}. Let the angle bisector of $\angle$A intersect side $\vec{BC}$ at a point D between B and C.
\begin{figure}[!ht]
    \centering
    \resizebox{\columnwidth}{!}{
  	\begin{tikzpicture}
  	\coordinate (A) at (0,6);
  	\coordinate (B) at (-3,0);
  	\coordinate (C) at (3,0);
  	\coordinate (E) at (1.5,3);
  	\coordinate (F) at (-1.5,3);
  	\draw (A)node[above]{$A$}--(B)node[below]{$B$}--(C)node[below]{$C$}--cycle;
  	\draw (B)node[below]{$B$}--(E)node[above]{$E$}--(A)node[above]{$A$}--cycle;
  	\draw(C)node[below]{$C$}--(F)node[above]{$F$}--(A)node[above]{$A$}--cycle;
  	\tkzMarkRightAngle(B,E,A)
  	\tkzMarkRightAngle(C,F,A)
  	\end{tikzpicture}
}
    \caption{Equilateral $\triangle{ABC}$}
    \label{eq:solutions/1/12/fig:es1}
\end{figure}
\begin{align}
\label{eq:solutions/1/12/eq:1}\norm{\vec{A-B}}=\norm{\vec{B-C}}=\norm{\vec{A-C}}\\
\angle ABD = \angle BAC = \angle ACD = 60\degree\\
\angle BAD = \angle CAD = 30\degree
\end{align}
Using angle bisector theorem ie. triangle will divide the opposite side into two segments that are proportional to the other two sides of the triangle.\\
\begin{align}
    \frac{\norm{\vec{A-B}}}{\norm{\vec{B-D}}}= \frac{\norm{\vec{A-C}}}{\norm{\vec{B-C}}-\norm{\vec{B-D}}}
\end{align}
Using \eqref{eq:solutions/1/12/eq:1}
\begin{align}
     \frac{\norm{\vec{A-B}}}{\norm{\vec{B-D}}}= \frac{\norm{\vec{A-B}}}{\norm{\vec{A-B}}-\norm{\vec{B-D}}}\\
     \implies\norm{\vec{A-B}} = 2\norm{\vec{B-D}}\label{eq:solutions/1/12/eq:es22}
\end{align}
Taking the inner product of sides BA and AD.
\begin{align}
    \brak{\vec{B}-\vec{A}}^T\brak{\vec{A}-\vec{D}} =
    \norm{\vec{B}-\vec{A}}\norm{\vec{A}-\vec{D}}\cos{\theta}\\
    \cos{\angle BAD}=\frac{\brak{\vec{B}-\vec{A}}^T\brak{\vec{A}-\vec{D}}}{\norm{\vec{B}-\vec{A}}\norm{\vec{A}-\vec{D}}}\label{eq:solutions/1/12/eq:13}
\end{align}
%
%
%
Now To Find AD.
\begin{multline}
 \brak{\vec{B-A}}^{T}\brak{\vec{B-A}}\\\ =\brak{\vec{B-D+D-A}}^{T}\brak{\vec{B-D+D-A}}\\
 = [\brak{\vec{B-D}}^{T}+\brak{\vec{D-A}}^{T}][\brak{\vec{B-D}}+\brak{\vec{D-A}}]\\
 =\brak{\vec{B-D}}^{T}\brak{\vec{B-D}} + \brak{\vec{B-D}}^{T}\brak{\vec{D-A}} +\\ \brak{\vec{D-A}}^{T}\brak{\vec{B-D}} + \brak{\vec{D-A}}^{T}\brak{\vec{D-A}}
\end{multline}
\begin{align}
\because \triangle ABD = \angle BAD + \angle ADB + \angle DBA\\
180\degree = 30\degree + \angle ADB + 60\degree.\\
\implies \angle ADB = 180\degree -\brak{60\degree + 30\degree} = 90\degree.\\
  \implies\brak{\vec{B-D}}^{T}\brak{\vec{D-A}}=0\\ \implies\brak{\vec{D-A}}^{T}\brak{\vec{B-D}} =0
\end{align}
which gives
\begin{multline}
    \brak{\vec{B-A}}^{T}\brak{\vec{B-A}} =\\\ \brak{\vec{B-D}}^{T}\brak{\vec{B-D}} + \brak{\vec{D-A}}^{T}\brak{\vec{D-A}}\\
    \norm{\vec{B-A}}^2=\norm{\vec{B-D}}^2 + \norm{\vec{D-A}}^2\\
    \implies\norm{\vec{D-A}}^2=\norm{\vec{B-A}}^2-\norm{\vec{B-D}}^2\label{eq:solutions/1/12/eq:5}
\end{multline}
Using Eq \eqref{eq:solutions/1/12/eq:es22} we get,
\begin{align}
\norm{\vec{D-A}}^2=\norm{\vec{B-A}}^2-\frac{1}{4}\norm{\vec{B-A}}^2\label{eq:solutions/1/12/eq:24}\\
 \norm{\vec{D-A}}=\frac{\sqrt{3}}{2}\norm{\vec{B-A}}\\
  \implies\norm{\vec{B-A}} = \frac{2}{\sqrt{3}}\norm{\vec{D-A}}\label{eq:solutions/1/12/eq:6}
\end{align}
Let $\vec{A}=0$. Substituting in \eqref{eq:solutions/1/12/eq:es22} and \eqref{eq:solutions/1/12/eq:6}
\begin{align}
    \label{eq:solutions/1/12/eq:7}\norm{\vec{B}}=2\norm{\vec{B-D}}\\
    \label{eq:solutions/1/12/eq:8}\norm{\vec{B}}=\frac{2}{\sqrt{3}}\norm{\vec{D}}
\end{align}
Square on both sides in \eqref{eq:solutions/1/12/eq:7} and \eqref{eq:solutions/1/12/eq:8}, we get,
\begin{align}
    \norm{\vec{B}}^2=4\norm{\vec{B-D}}^2\\
    \frac{1}{4}\norm{\vec{B}}^2=\norm{\vec{B}}^2+\norm{\vec{D}}^2-2\vec{B}^T\vec{D}\label{eq:solutions/1/12/eq:9}\\
\norm{\vec{B}}^2 = \frac{4}{3}\norm{\vec{D}}^2\label{eq:solutions/1/12/eq:10}
\end{align}
Solving \eqref{eq:solutions/1/12/eq:9} and \eqref{eq:solutions/1/12/eq:10} we get,
\begin{align}
    \frac{1}{3}\norm{\vec{D}}^2=\frac{4}{3}\norm{\vec{D}}^2+\norm{\vec{D}}^2-2\vec{D}^T\vec{D}\\
    0=\norm{\vec{D}}^2-2\vec{B}^T\vec{D}\\
     \implies\vec{B}^T\vec{D}=\norm{\vec{D}}^2\label{eq:solutions/1/12/eq:15}
\end{align}
Substitute $\vec{A}=0$ in \eqref{eq:solutions/1/12/eq:13} we get,
\begin{align}
    \cos{\angle{BAD}}=\frac{\vec{B}^T\vec{D}}{\norm{\vec{B}}\norm{\vec{D}}}\\
 \because\norm{\vec{B}}=\frac{2}{\sqrt{3}}\norm{\vec{D}}\\
     \label{eq:solutions/1/12/eq:12}\implies\cos{\angle{BAD}}=\frac{\vec{B}^T\vec{D}}{\frac{2}{\sqrt{3}}\norm{\vec{D}}\norm{\vec{D}}} = \frac{\vec{B}^T\vec{D}}{\frac{2}{\sqrt{3}}\norm{\vec{D}}^2}
\end{align}
Substitute \eqref{eq:solutions/1/12/eq:15} in \eqref{eq:solutions/1/12/eq:12}
\begin{align}
\cos\angle{BAD} = \frac{{\frac{\sqrt{3}}{2}\norm{\vec{D}}^2}}{\norm{\vec{D}}^2}= \frac{\sqrt{3}}{2}\\
\because \angle{BAD}= 30\degree\\
\implies \cos{30\degree}= \frac{\sqrt{3}}{2}\\
    \because \cos^2\theta +\sin^2\theta = 1\\
     \sin30\degree = \sqrt{1-\cos^230\degree}\\
    \implies\sin30\degree = \frac{1}{2}.
\end{align} 
