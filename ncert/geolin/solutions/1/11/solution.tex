Consider an equilateral triangle \textbf{ABC}. Since, $\triangle$\textbf{ABC} is an equilateral, all of its angles are $60\degree$. Now, The direction vector of all the sides are given as,
\begin{align}
\vec{AB}=\norm{\vec{A-B}}
\end{align}
\begin{align}
\vec{BC}=\norm{\vec{B-C}}
\end{align}  
\begin{align}
\vec{AC}=\norm{\vec{A-C}}
\end{align} 
Now for an equilateral triangle,
\begin{align}
\norm{\vec{A-B}} = \norm{\vec{B-C}} = \norm{\vec{A-C}}
\label{eq:solutions/1/112}
\end{align}
Let, $\vec{B}$ be the origin.Hence, $\vec{B}$ = 0. Hence substituting in the equation \eqref{eq:solutions/1/112} we get,
\begin{align}
\norm{\vec{A}} = \norm{\vec{C}} =\norm{\vec{A}-\vec{C}}
\label{e1}
\end{align}
Squaring $\norm{\vec{A}-\vec{C}}$ we get,
\begin{align}
\norm{\vec{A}-\vec{C}}^2 = \norm{\vec{A}}^2+\norm{\vec{C}}^2-2\vec{A}^T\vec{C}
\end{align} 
Substituting from equation \eqref{e1} in above equation, 
\begin{align}
\implies \norm{\vec{A}}^2 = 2\norm{\vec{A}}^2-2\vec{A}^T\vec{C}
\end{align}
\begin{align}
\implies \norm{\vec{A}}^2 = 2\vec{A}^T\vec{C}
\label{eq:solutions/1/114}
\end{align}
\begin{figure}[!ht]
\centering
\resizebox{\columnwidth}{!}{\begin{tikzpicture}
\draw (0,0) node[anchor=north]{$B$}-- (4,6.9282) node[anchor=west]{$A$}-- (8,0) node[anchor=north]{$C$}-- cycle;
\draw (0,0) (0:0.75cm) arc (0:60:0.75cm);
\draw (30:1.15cm) node{$\theta$};
\end{tikzpicture}}
\caption{Equilateral Triangle}
\label{fig:solutions/1/111}
\end{figure}
In figure \ref{fig:solutions/1/111}, taking inner products of side $\vec{AB}$ and $\vec{BC}$ we get,
\begin{align}
(\vec{AB})^T\vec{BC} = \norm{\vec{AB}}\norm{\vec{BC}}\cos\theta
\label{eq:solutions/1/113}
\end{align}
Substituting these results in \eqref{eq:solutions/1/113} and solving for $\cos\theta$ we get,
\begin{align}
\cos\theta = \frac{(\vec{A}-\vec{B})^T(\vec{B}-\vec{C})}{\norm{\vec{A}-\vec{B}}\norm{\vec{B}-\vec{C}}}
\end{align}
\begin{align}
\implies \cos\theta = \frac{\vec{A}^T\vec{C}}{\norm{\vec{A}}\norm{\vec{C}}}
\label{e2}
\end{align}
Imposing the results of \eqref{eq:solutions/1/114} in \eqref{e2} we get,
\begin{align}
\implies \cos\theta = \frac{\vec{A}^T\vec{C}}{2\vec{A}^T\vec{C}}
\end{align}
\begin{align}
\implies \cos\theta = \frac{1}{2}
\end{align}
$\therefore \theta = 60\degree$
\begin{align}
\therefore \cos60\degree = \frac{1}{2}
\end{align}
Now using the property,
\begin{align}
\cos^2\theta +\sin^2\theta = 1
\end{align} 
$\therefore$ at $\theta = 60\degree, $
\begin{align}
\implies \sin60\degree = \sqrt{1-\cos^260\degree}
\end{align}
\begin{align}
\implies\sin60\degree = \frac{\sqrt{3}}{2}.
\end{align}
