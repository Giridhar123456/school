\renewcommand{\theequation}{\theenumi}
\begin{enumerate}[label=\arabic*.,ref=\thesubsection.\theenumi]
\numberwithin{equation}{enumi}
\item The angles of quadrilateral are in the ratio 3 : 5 : 9 : 13. Find all the angles of the quadrilateral.

\item $ABCD$ is a cyclic quadrilateral with 
\begin{align}
\angle A &= 4y+20
\\
\angle B &= 3y-5
\\
\angle C &= -4x
\\
\angle D &= -7x+5
\end{align}
%
Find its angles.
\item Draw a quadrilateral in the Cartesian plane, whose vertices are \myvec{– 4\\ 5}, \myvec{0\\ 7}, \myvec{5\\ – 5} and \myvec{– 4\\ –2}. Also, find its area.
\item Find the area of a rhombus if its vertices are \myvec{3\\0}, \myvec{4\\5}, \myvec{-1\\4} and \myvec{-2\\-1} taken in order.
\item Without using distance formula, show that points \myvec{– 2\\ – 1}, \myvec{4\\ 0}, \myvec{3\\ 3} and \myvec{–3\\ 2} are the vertices of a parallelogram.
\item  Find the area of the quadrilateral whose vertices, taken in order, are 
 \myvec{-4\\2},  \myvec{-3\\-5},  \myvec{3\\-2},  \myvec{2\\3}. 
\item The two opposite vertices of a square are \myvec{-1\\2},  \myvec{3\\2}. Find the coordinates of the other two vertices.
\item $ABCD$ is a rectangle formed by the points $\vec{A} = \myvec{-1\\-1}, \vec{B} = \myvec{-1\\4}, \vec{C} = \myvec{5\\4}, \vec{D} = \myvec{5\\-1}$. $ \vec{P}, \vec{Q}, \vec{R}, \vec{S}$ are the mid points of $AB, BC, CD, DA$ respectively.  Is the quadrilateral $PQRS$ a 
\begin{enumerate}
\item square?
\item rectangle?
\item rhombus?
\end{enumerate}
\item Find the area of a parallelogram whose adjacent sides are given by the vectors \myvec{3\\1\\4} and \myvec{1\\-1\\1}.
\item Find the area of a parallelogram whose adjacent sides are determined by the vectors $\vec{a} = \myvec{1\\-1\\3}$ and $\vec{b}=\myvec{2\\-7\\1}$.
\item Find the area of a rectangle $ABCD$ with vertices
$\vec{A} = \myvec{-1\\\frac{1}{2}\\ 4},
 \vec{B} = \myvec{1\\\frac{1}{2}\\ 4},
\vec{C} = \myvec{1\\-\frac{1}{2}\\ 4},
\vec{D} = \myvec{-1\\-\frac{1}{2}\\ 4}.
$
\item The two adjacent sides of a parallelogram are \myvec{2\\ -4 \\ -5} and  \myvec{1\\-2\\ -3}. Find the unit vector parallel to its diagonal.  Also, find its area.
%
%
\end{enumerate}
%
