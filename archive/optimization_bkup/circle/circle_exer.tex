\renewcommand{\theequation}{\theenumi}
\begin{enumerate}[label=\arabic*.,ref=\thesubsection.\theenumi]
\numberwithin{equation}{enumi}
\item Draw a circle of diameter 6.1

\item With the same centre $\vec{O}$,  draw two circles of radii 4 and 2.5
\item Draw a circle of radius 3 and any two of its diameters.  draw the ends of these diameters. What figure do you get?
\item Let $\vec{A}$ and $\vec{B}$ be two circles of equal radii 3 such that each one of them passes through the centre of the other.  Let them intersect at $\vec{C}$ and $\vec{D}$.  Is $AB \perp CD$?

\item Construct a tangent to a circle of radius 4 units from a point on the concentric circle of radius 6 
units.
\\
\solution Take the centre of both circles to be at the origin.  
\item Draw a circle of radius 3 units. Take  two points $\vec{P}$ and $\vec{Q}$ on one of its extended 
diameter each at a distance of 7 units from its centre. Draw tangents to the circle from these two points 
$\vec{P}$ and $\vec{Q}$.
\\
\solution Take the diameter to be on the $x$-axis.
\item Draw a pair of tangents to a circle of radius 5 units which are inclined to each other at an angle of 
$60^{\degree}$.
\\
\solution The tangent is perpendicular to the radius.
\item Draw a line segment $AB$ of length 8 units. Taking $\vec{A}$ as centre, draw a circle of radius 4 units 
and taking $\vec{B}$ as centre, draw another circle of radius 3 units. Construct tangents to each circle from 
the centre of the other circle.
\\
\solution Let
\begin{align}
\vec{A} = \myvec{0 \\ 0}, \vec{B} = \myvec{8 \\ 0}.
\end{align}
\item Let ABC be a right triangle in which $a = 8, c = 6$ and $\angle B = 90^{\degree}$.  $BD$ is the 
perpendicular from $\vec{B}$ on $AC$ (altitude). The circle through $\vec{B}, \vec{C}, \vec{D}$ (circumcircle of $\triangle BCD$) is drawn.  Construct the 
tangents from $\vec{A}$ to this circle.
%\\
%\solution Since $\angle BDC = 90\degree$, $BC$ is the diameter of the circumcircle of $\triangle BCD$. Since $AB \perp BC$ and $BC$ is the diameter, $AB$ is a tangent to the circumcircle of $\triangle BCD$.  Let $\vec{O}$ be the centre of the circle.  The point of contact is obtained by rotating $\vec{B}$ by $\theta = 2\angle BAO$. Thus, if 
%\begin{align}
%\vec{B} &= \myvec{0 \\ 0}, \vec{C} = \myvec{a \\ 0},
%\\
%\vec{O} &= \frac{1}{2}\myvec{a \\ 0}
%\end{align}

\item Draw a circle with centre $\vec{C}$ and radius 3.4.  Draw any chord.  Construct the perpendicular bisector of the chord and examine if it passes through $\vec{C}$\end{enumerate}
%
\item Form the differential equation represeting the family of curves given by 
\begin{align}
\vec{x}^T \myvec{1 & 0 \\ 0 & 2} \vec{x} -\myvec{2a & 0}\vec{x} = 0,
\end{align}
%
where $a$ is an arbitrary constant.
%
\item Form the differntial equation of the family of circles in the first quadrant which touch the coordinate axes.
\end{enumerate}
